\subsubsection{Zusammenfassung}
\begin{itemize}
  \item $\tau\left( [a,b] \right)$ = Vektorraum der Treppenfunktionen auf $[a,b]$
  \item $\int: \tau[a;b]\to\mb{C}$ lineare Abbildung
  \item Eigenschaften:
    \begin{itemize}
      \item lineare Abbildung
      \item Monotonie: $f\leq g$ $\implies$ $\int^b_af\cdot \md x\leq \int^b_ag\cdot \md x$
      \item Beschränktheit: $\Abs{\int^b_af\cdot \md x}\leq \int^a_b\Abs{f(x)}\md x\leq (b-a)\Norm{f} = \sup_{x\in[a;b]}f$
    \end{itemize}
  \item Regelfunktionen: $R\left( [a,b] \right)$ = Vektor nach der Regel $f\supset\tau\left( [a;b] \right)$
  \item gleichmässige Konvergenz $f_n\to f\xLeftrightarrow{\text{def}}\Norm{f_n-f}\to 0$
\end{itemize}
\subsubsection{Vorgehen}
\begin{enumerate}
    \item Jede Regelfunktion kann man gleichmässig durch Treppenfunktionen approximieren.
    \item Damit kann man das Integral von Regelfunktionen definieren.
    \item Regenregeln (insbesondere Hauptsatz)
    \item Riemannsche Summen
\end{enumerate}
\begin{Sat}{Approximationssatz}
  \[f\in R{a;b}\Lra \exists \text{Folge} \phi_n\in \tau[a;b]:\phi_n\to f\text{gleichmässig}\]  
  ist per Definition äquivalent mit
  \[\exists \text{Folge}\phi_n\in \tau[a;b]: \Norm{\phi_n-f}\to 0\]
  wobei
  \[\Norm{\phi_n-f}=\sup_{x\in[a;b]}\Abs{\phi_n(x)-f(x)}\]
  Dieser Grenzwert ist wiederum äquivalent mit
  \[\forall\varepsilon>0 \exists\phi \in \tau[a;b]: \Norm{f-\phi}\leq \varepsilon\]
  (eine $\varepsilon$-approximierende Treppenfunktion)
\end{Sat}
\begin{Bew}{$\Ra$}
  d.h. $f\in R\implies \exists \varepsilon$-approx. Treppen. Widerspruchsbeweis:
  \begin{align*}
    f\in R[a;b]\\
    \exists \varepsilon >0: f \text{besitzt keine} \varepsilon \text{-approx. Treppenfunktion}
  \end{align*}
  Wir konstruieren eine Intervallschachtelung $I_n=[a_n;b_n]$ s.d. $\forall_n f|_{I_n}$ besitzt keine $\varepsilon$-approx.Treppenfunktion
  \[I_1=[a;b]\]
  rekursiv: $M=\frac{b_n-a_n}{2} +a_n$ Mittelpunkt
  \begin{align*}
    I_{n+1}:=\begin{cases}
      [a_n;M]&\text{falls} f|_{[a_n;M]}\text{keine} \varepsilon \text{-approx. Treppenfunktion bestzt}\\ [M,b_n]& \text{andernfalls}        
    \end{cases}
  \end{align*}
  Sei $\xi\in I_n \forall n$
  \begin{align*}
    c_e&:=&\lim_{x\uparrow \xi} f(x)\\
    c_r&:=&\lim_{x\downarrow \xi} f(x)
  \end{align*}
  $\implies$
  \begin{align*}
    \exists \delta:&\Abs{f(x)-c_e}<\varepsilon:&x\in\left[\xi-\delta;\xi\right)\\
    &\Abs{f(x)-c_r}<\varepsilon:&x\in\left(\xi;\xi+\delta\right]\\
  \end{align*}
  Auf $[\xi-\delta;\xi+\delta]$ definieren wir eine Treppenfunktion:
  \begin{align*}
    \phi(x):=\begin{cases}
      c_e&\xi-\delta\leq x< \xi\\
      f(\xi)&x=\xi\\
      c_r&\xi+\delta\geq x>\xi
    \end{cases}
  \end{align*}
  Fall 1 $\implies$ $\phi$ ist eine $\varepsilon$-approx. Treppenfunktion auf $[\xi-\delta],[\delta+\delta]$. Fall 2 $\implies$ $\phi$ ist eine $\varepsilon$-approx. Treppenfunktion auf $[\xi-\delta],[\delta+\delta]$, alle $I_n \subset[\xi+\delta;\xi+\delta]$\\
  $\blitza$
\end{Bew}
\begin{Bew}{$\La$}
  $f$ Regelfunktion $\La$ $f$ besitzt $\varepsilon$-approx. Treppenfunktion $\forall \varepsilon>0$. Sei $x_0\in\left[a;b\right)$. Zu zeigen: $\exists \lim_{x\downarrow x_0} f(x)$.
  \begin{align*}
    \forall \varepsilon>0\ \exists\phi\in \tau[a;b]:\Norm{f-\phi}<\frac{\varepsilon}{2}\\
  \end{align*}
  Sei $\beta>x_0:\phi$ konstant auf $(x_0,\beta)$
  \begin{align*}
    \forall x,x'\in(x_0;\beta)\\
    \Abs{f(x)-f(x')}&\leq& \Abs{f(x)-\phi(x)}+\Abs{\phi(x)^{(=\phi(x')}-f(x')}\\
    &\leq&\Norm{f-\phi}+\Norm{\phi-f}<\varepsilon
  \end{align*}
  $\forall \varepsilon>0$ $\exists\beta:$ Cauchyeigenschaft gilt auf $(x_0;R)$ $\implies$ $\exists \lim_{x\uparrow x_0}f(x)$. Ähnlich: $\exists\lim_{x\uparrow x_0}f(x)$ $\forall x_0\in\left(a;b\right]$.
\end{Bew}
\begin{Kor}
  \[f\in R[a;b]\ \Lra\ \exists \text{Folge}\Psi_b\in\tau[a;b]:\sum^\infty_{k=1}\phi_k=f\]
  konvergiert konstant
\end{Kor}
\begin{Kor}
  $f$ Regelfunktion auf $I$ $\implies$ $f$ fast überall stetig. d.h. $\exists A\subset I$ s.d.
  \begin{itemize}
    \item $f|_{I\setminus A}$ stetig
    \item $A$ höchstens abzählbar $x\in[a;b]$
  \end{itemize}
\end{Kor}
\begin{Bew}
  %\subparagraph{Fall 1: $I$ kompakt}
  \begin{align*}
    \Psi_k\in\tau [I]\\
    f=\sum\phi_k \text{normal}
  \end{align*}
  Ist $\phi_k$ stetig in $x\forall k$ $\implies$ $f$ stetig in $x$.\\
  Ist $x$ Unstetigkeitsstelle von $f$, $\exists k$: $\phi_k$ unstetig in $x$, höchstens abzählbare viele $k$.
  \begin{itemize}
    \item Eine Treppenfunktion hat endlich viele Unstetigkeitsstellen
  \end{itemize}  
  \{ Unstetigkeitsstellen von $f$\} $\subset$ (höchstens abzählbare Vereinigung von endlichen Mengen) $\implies$ höchstens abzählbar
  %\subparagraph{Fall 2: $I$ nicht kompakt}
  \[I=\overbrace{U_\alpha}^{\text{höchstens abzählbar}}\overbrace{I_\alpha}^{\text{kompakt}}\]
\end{Bew}
\begin{Sat}
  \[f\in R\left( [a:b] \right)\implies f \text{beschränkt auf} [a;b]\]
\end{Sat}
\begin{Bew}
  \begin{align*}
    \varepsilon =1\\
    \exists \overbrace{\phi}^{\text{\underline{beschränkt}}}\in\tau\left( [a;b] \right):\Norm{f-\phi}\leq 1\\
    \implies \Norm{f}=\Norm{f-\phi + \phi}\leq \Norm{f-\phi}+\Norm{\phi}=\leq 1+\Norm{\phi}
  \end{align*}
\end{Bew}
\begin{Def}{Integration von Regelfunktionen}
  \ldots auch bekannt als ``Regelintegral''\\
  Sei $f\in R[a;b]$
  \[\int_a^bf(x)fx:\Limi{n}\int^b_a\phi_n(x)\md x\]
  wobei $\phi_n$ eine approximierene Folge von Treppenfunktionen ist (d.h. $\Norm{\phi_n-f}\to 0$)
\end{Def}
\subparagraph{zu zeigen:}
\begin{enumerate}
  \item Die Folge $I_n:=\int^b_a\phi_n(x)\md x$ konvergiert $\forall \Norm{\phi_n-f}\to 0$
  \item Der Grenzwert ist von der Wahl der approximierenden Folge unabhängig
\end{enumerate}
\begin{Bew}{von 1}
  \[\Abs{I_n-I_m}=^{\text{Linearität}}\Abs{\int^b_a\left( \phi_n(x)-\phi_m(x) \right)\md x}\leq^{\text{beschränkt}} (b-a)\Norm{\phi_n-\phi_m}\]
  \[\Norm{\phi_n-f}\to 0 \xRightarrow{\text{Dreiecksungleichung}} \forall\varepsilon>0\ \exists N: \Norm{\phi_n-\phi_m}<\varepsilon\ \forall n,m> N\]
  $\implies$ $I_n$ Cauchyfolge $\implies$ $I_n$ konvergiert
\end{Bew}
\begin{Bew}{von 2}
  Seien $\phi_n, \psi_n\in \tau[a;b]$
  \begin{align*}
    \Norm{\phi_n-f}\to0\\
    \Norm{\psi_n-f}\to0
  \end{align*}
  \begin{align*}
    \{X_n\}=\psi_1,\phi_1,\psi_2,\phi_2,\psi_3,\phi_3,\cdots\\
    X_n:=\begin{cases}
      \phi_{\frac{n}{2}}& n \text{gerade}\\
      \psi_{\frac{n+1}{2}}& n \text{ungerade}
    \end{cases}
  \end{align*}
  $\implies$ $I_n(\phi)$ und $I_n(\psi)$ Teilfolgen von $I_n(X)$
  \begin{align*}
    \implies \Norm{X_n-f}\to 0\\
    I_n(x)=\int x_n\\ I_n(\phi)=\int \phi_n\\ I_n(\psi)=\int \psi_n\\
  \end{align*}
  $\implies$
  \begin{align*}
    \lim I_n(\phi)=\lim I_n(X)=\lim I_n(\psi)
  \end{align*}
\end{Bew}
\begin{Bsp}{Dirichlet}
  eine Funktion, die keine Regelfunktion ist.
  \begin{align*}
    f:[0;1]\to\mb{R}\\
    f(x)=\begin{cases}
      1 & x\in \mb{Q}\\
      0 & x\in \mb{R}\setminus\mb{Q}
    \end{cases}
  \end{align*}
  $f$ unstetig $\forall x$ intuitiv: $\int^1_0 f(x)fx =0$
\end{Bsp}
\begin{Bsp}{Riemann}
  sog. modifizierte Dirichlet-Funktion
  \begin{align*}
    g:[0;1]\to\mb{R}\\
    g(x)=\begin{cases}
      \frac{1}{q}  & x=\frac{p}{q}, p,q \text{teilerfremd}, q>0\\
      0 & x\in \mb{R}\setminus \mb{Q}
    \end{cases}
  \end{align*}
  $g\in R[0;1]$ und $int^b_ag(x)\md x=0$
\end{Bsp}
\subsubsection{Eigenschaften}
\begin{Sat}
  \[\forall f, g\in R [a;b] \forall \alpha,\beta\in \mb{C} \text{gelten}\]
  \begin{description}
    \item[Linearität] \[\int^b_a(\alpha f+ \beta g)\md x = \alpha \int^b_af\cdot \md x+\beta\int^b_a g\cdot \md x\]
    \item[Beschränktheit] \[\Abs{\int^b_a f(x)\cdot \md x}\leq \int^b_a\Abs{f(x)}\md x\leq (b-a)\Norm{f}\]
    \item[Monotonie] \[f\leq g \implies \int^b_af(x)\md x\leq \int^b_ag(x)\md x\]
  \end{description}
  ($f,g$ reellwertig $f(x)\leq g(x)\forall x$)
\end{Sat}
\begin{Sat}{Additivität}
  Sei $f\in R[a;b]$ und sei $c\in (a;b)$
  \[\int^b_af(x)\md x=\int^c_a f(x)\md x+\int^b_cf(x)\md x\]
\end{Sat}
