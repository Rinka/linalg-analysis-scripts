\begin{Bem}
  $\mb{K}=[a;b]$
  \begin{align*}
    f:X\times[a;b]&\to\mb{C}\\
    (x,y)&\mapsto f(x,y)
  \end{align*}
  \begin{align*}
    \forall x\in X: f_x:[a;b]&\to\mb{C}\\
    y&\mapsto f(x,y)
  \end{align*}
  stetig $\implies$ $\mathcal{R}$ auf $[a;b]$
  $f_x=f\circ i_x$
  \[i_x:\left\{ x \right\}\times[a;b]\to X\times[a;b]\]
  \begin{gather*}
    F(x):=\int^b_af(x,t)\md t
    F:X\to\mb{C}
  \end{gather*}
\end{Bem}
\begin{Sat}
  $F$ stetig
\end{Sat}
\begin{Bew}
  $\forall x_0\in X$ ist $F$ stetig in $x_0$. Sei $x_0\in X$
  \[\phi(x,t):=f(x,t)-f(x_0,t)\]
  stetig. Sei $\varepsilon>0$
  \[W:=\left\{ (x,t)\in X\times[a;b]:|\phi(x,t)<\frac{\varepsilon}{b-a} \right\}\]
  $\phi$ stetig $\implies$ $W$ offen
  \[\phi(x_0,t)=0\ \forall t\implies \left\{ x_0 \right\}\times[a;b]\subset W\]
  $\implies$ $\exists$ Umgebung $U$ von $x_0$ in $X$ mit
  \[U\times [a;b]\subset W\]
  $\forall x\in U$ gilt
  \begin{gather*}
    \Abs{F(x)-F(x_0)}=\Abs{\int^b_a\phi(x,t)\md t}\leq \int^b_A\Abs{\phi(x,t)}\md t < \int^b_a\frac{\varepsilon}{b-a}\md t=\varepsilon
  \end{gather*}
  Spezialfall $X=[c;d]$
  \[f:[c;d]\times[a;b]\to\mb{C}\]
  stetig
  \[F:[c;d]\to\mb{C}\ \text{stetig}\]
  $\implies$ Regelfunktion
  \[\int^d_cF(x)\md x=\underbrace{\int^d_c\left( \int^b_af(x,y)\md y \right)\md x}_{\text{interiertes Integral}}\]
\end{Bew}
\begin{Not}{interiertes Integral}
  \[\int_{[c;d]\times[a;b]}f(x,y)\md x\md y:=\int_c^dF(x)\md x\]
\end{Not}
\subsection{Zwischenwertsatz}
\begin{Def}{zusammenhängender metrischer Raum}
  Ein metrischer Raum $X$ heisst zusammenhängend, wenn es keine Zerlegung $X=U\cup V$ gibt, mit
  \begin{enumerate}
    \item $U,V$ disjunkt (d.h. $U\cap V=\varnothing$)
    \item $U,V$ offen
    \item $U,V$ nicht leer
  \end{enumerate}
\end{Def}
\begin{Def}
  Eine Teilmenge eines metrischern Raumes heisst zusammenhängend, wenn sie bezüglich der Spurtopologie so ist.
\end{Def}
\begin{Bsp}
  $\varnothing$ ist zusammenhängend
\end{Bsp}
\begin{Bsp}
  $\left\{ x \right\}\subset\mb{R}$ zusammenhängend
\end{Bsp}
\begin{Bsp}
  $\mb{Q}\in\mb{R}$ \underline{nicht} zusammenhängend
  \begin{align*}
    U&=\left\{ x\in\mb{Q}:x\leq 0\ \text{oder}\ x^2<2 \right\}=\mb{Q}\cap\left( -\infty,\sqrt{2} \right)\\
    V&=\left\{ x\in\mb{Q}:x>0\ \text{oder}\ x^2>2 \right\}=\mb{Q}\cap\left( \sqrt{2},+\infty \right)
  \end{align*}
  offen
\end{Bsp}
\begin{Sat}
  Sei $X\subset\mb{R}$ und besitze $X$ mindestens zwei verschiedene Punkte
  \[X\ \text{zusammenhängend}\ \Lra\ X\ \text{Intervall}\]
\end{Sat}
\begin{Bew}
  $\Ra$ Kontrapositionsbeweis: Sei $X$ kein Intervall
  \[\implies\exists u<s<v\ \text{mit} u,v\in X\ s\not\in X\]
  \[U:=X\cap\left( -\infty;s \right),\ V=X\cap\left( s;+\infty \right)\]
  $\implies$ $X$ nicht zusammenhängend\\
  $\La$ Widerspruchbeweis: Sei $X$ \underline{nicht} zusammenhängendes Intervall
  \begin{gather*}
    \exists U,V\ \text{offen in}\ I\\
  \end{gather*}
  \begin{align*}
    U,V&\neq \varnothing&\implies\exists u\in U, v\in V\\
    U\cap V&=\varnothing&\implies u\neq v\\
    U\cup V&=X&\\
  \end{align*}
  Annahme $u<v$:
  \[X\ \text{Intervall}\implies [u;v]\subset X\]
  Sei
  \[s=\sup\left( [u;v]\cap U \right)\]
  beschränkt $\subset [u,v]$
  \[\implies s\in [u;v]\ s\leq v\]
  $V$ offen
  \[U=X\setminus V\]
  \[\implies U\ \text{abgeschlossen} \implies s\in U\]
  \[U\cap V=\varnothing\implies s<v\]
  Sei $x\in X$ mit $x>s$ und $x\leq v$
  \[\xRightarrow{s\sup}\ x\in V\implies (s;v]\in V\]
  \[U\ \text{offen}\implies \exists \varepsilon>0\ \text{mid} \left( s-\varepsilon, s+\varepsilon \right)\cap X\in U\]
  $X$ Intervall $v\in X$, $s<v$
  \[\implies \lambda\in [s;s+\varepsilon)\cap X\in U\ \text{mit}\ \lambda\leq v\implies \lambda\in U\]
  $(s;v]\subset V$ $\implies \lambda\in V$ Widerspruch, da $U\cap V=\varnothing$
\end{Bew}
\begin{Sat}
  Sei $f:X\to Y$ stetig
  \[X\ \text{zusammenhängend}\implies Y\ \text{zusammenhängend}\]
\end{Sat}
\begin{Bew}
  Kontrapositionsbeweis: Sei $Y$ nicht zusammenhängend
  \[\implies Y=U\cup V\]
  \begin{align*}
    U,V&\neq\varnothing\\
    U\cap V&=\varnothing\\
    U,V&\ \text{offen}
  \end{align*}
  \[\tilde U:=f^{-1}(U),\tilde V:=f^{-1}(V)\]
  \[X=\tilde U\cup \tilde V\]
  \begin{align*}
    \tilde U, \tilde V&\neq \varnothing\\
    \tilde U\cap \tilde V=\varnothing\\
    \tilde U, \tilde V&\ \text{offen}
  \end{align*}
\end{Bew}
\begin{Sat}{Zwischenwertsatz}
  Sei $X$ zusammenhängend
  \[f:X\to\mb{R}\ \text{stetig}\]
  Für je zwei Punkte $a$ und $b$ $\in X$ nimmt $f$ alle Werte zwischen $f(a)$ und $f(b)$ an.
\end{Sat}
\begin{Bew}
  Fall 1: $f(a)=f(b)$ nichts zu zeigen\\
  Fall 2: $f(a)\neq f(b)$ und $f(x)$ zusammenhängend
  \[\implies f(x)\ \text{Intervall}\]
  $\implies$ $f(x)$ enthält alle Punkte zwischen $f(a)$ und $f(b)$
\end{Bew}
\begin{Def}{wegzusammenhängend}
  Ein metrischer Raum $X$ heisst wegzusammenhängend, wenn es $\forall a,b\in X$ eine stetige Kurve
  \[\gamma:[\alpha;\beta]\to X\]
  gibt mit $\gamma(\alpha)=a$ und $\gamma(\beta)=b$. Man sagt, $\gamma$ verbinde $a$ und $b$.
\end{Def}
\begin{Bsp}
  $\mb{R}\setminus\left\{ 0 \right\}$ ist nicht wegzusammenhängend. Beweis: Zwischenwertsatz.
\end{Bsp}
\begin{Def}{konvex}
  Sei $V$ Vektorraum, $X\subset V$ heisst konvex, wenn $\forall a,b\in X$
  \[\left\{ a+t(b-a):t\in [0;1] \right\}\subset X\]
  (Strecke, die $a$ und $b$ verbindet)
\end{Def}
\begin{Lem}
  Sei $V$ ein normierter Vektorraum, $X\subset V$
  \[X\ \text{konvex}\implies X\ \text{wegzusammenhängend}\]
\end{Lem}
\begin{Bew}
  Die Strecke ist eine stetige Kurve.
\end{Bew}
\begin{Sat}
  $\mb{R}\setminus\left\{ 0 \right\}$ und $S^{n-1}$ sind für $n\geq 2$ wegzusammenhängend.
\end{Sat}
\begin{Lem}
  \[X\ \text{wegzusammenhängend} \implies X\ \text{zusammenhängend}\]
\end{Lem}
\begin{Bew}
  Widerspruchsbeweis: Sei $X$ wegzusammenhängend, nicht zusammenhängend.
  \begin{align*}
    X&=U\cup V\\
    U,V&\ \text{offen}&\exists u\in U, v\in V\\
    U,V&\neq\varnothing&u\neq v\\
    U\cap V&=\varnothing
  \end{align*}
  $X$ wegzusammenhängend
  \[\implies \gamma:[\alpha;\beta]\to X\]
  $\gamma(\alpha)=u$, $\gamma(\beta)=v$
  \begin{gather*}
    \tilde U=\gamma^{-1}(U),\ \tilde V=\gamma^{-1}(V)\\
  \end{gather*}
  \begin{align*}
    [\alpha;\beta]&=\tilde U\cup \tilde V\\
    \tilde U,\tilde V&\ \text{offen}\\
    \tilde U,\tilde V&\neq\varnothing\\
    \tilde U\cap\tilde V&=\varnothing
  \end{align*}
  \[ [\alpha;\beta]\ \text{nicht zusammenhängend}\]
\end{Bew}
\begin{Kor}
  $\mb{R}\setminus\left\{ 0 \right\}$ und $S^{n-1}$ sind für $n\geq 2$ zusammenhängend.
\end{Kor}
\begin{Bew}
  $\mb{R}\setminus\left\{ 0 \right\}$ nicht zusammenhängend
\end{Bew}
\begin{Sat}
  Sei $V$ ein normierter Vektorraum, $X\subset V$ offen
  \[X\ \text{zusammenhängend}\implies X\ \text{wegzusammenhängend}\]
  Zusätzlich können je zwei Punkte in $X$ durch einen Streckenzug verbunden werden.
\end{Sat}
\begin{Bem}
  Sei 
  \[X:=\left\{ \left( x,\sin\frac{1}{x} \right),x>0 \right\}\cup \left\{ (0,y),y\in [-1;1] \right\}\subset\mb{R}\]
  \begin{itemize}
    \item $X$ ist nicht offen
    \item $X$ zusammenhängend
    \item $X$ nicht wegzusammenhängend
  \end{itemize}
\end{Bem}
\begin{Def}{Gebiet}
  Eine zusammenhängende offene Teilmenge normierten Vektorraumes heisst Gebiet.
\end{Def}
\begin{Sat}
  \[GL(n;\mb{R}=\left\{ A\in M(n\times n,\mb{R}), \det A\neq 0 \right\}\]
  ist nicht zusammenhängend.
  \[GL^+(n;\mb{R}=\left\{ A\in M(n\times n,\mb{R}), \det A> 0 \right\}\]
  ist zusammenhängend.
\end{Sat}
\begin{Bew}
  \[\det:GL(n,\mb{R})\to\mb{R}\setminus\left\{ 0 \right\}\ \text{stetig}\]
  Wäre $GL$ zusammenhängend, dann wäre auch $\mb{R}\setminus\left\{ 0 \right\}$ zusammenhängend.
\end{Bew}
\begin{Sat}
  Seien $X$ und $Y$ homöomorph. Dann
  \[X\ \text{zusammenhängend}\ \Lra\ Y\text{zusammenhängend}\]
\end{Sat}
\begin{Kor}
  $\mb{R}^n,n>1$ ist nicht homöomorph zu $\mb{R}$
\end{Kor}
\begin{Bew}
  $n>1$ $\mb{R}^n\setminus\left\{ 0 \right\}$ zusammenhängend.\\
  Widerspruchbeweis:
  \[\exists f:\mb{R}^n\to\mb{R}\ \text{Homöomorphismus}\]
  \[f|_{\mb{R}^n\setminus\left\{ 0 \right\}}:\underbrace{\mb{R}^n\setminus\left\{ 0 \right\}}_{\text{zusammenhängend}}\to\underbrace{\mb{R}\setminus f(0)}_{\text{nicht zusammenhängend}}\]
\end{Bew}
