\begin{Bsp}
  Sei $k\in\mb{Z}\setminus\{0\}$
  \begin{align*}
    \gamma_k:&\mb{R}\to\mb{C}\cong \mb{R}^2\\
    &t\mapsto e^{ikt}
  \end{align*}
  $\Abs{\gamma(t)}=1$ $\forall t$ $\Spur\gamma_k=S^1$
  $k>0$: Gegenuhrzeigersinn\\
  $k<0$: Uhrzeigersinn
\end{Bsp}
\begin{Bsp}{Schraubenlinie}
  $\gamma:\mb{R}\to\mb{R}^3$
  \[t\mapsto (r\cos t, r\sin t, h t)\]
\end{Bsp}
\begin{Def}{Tangentialvektor einer Kurve}
  Sei $\gamma:I\to\mb{R}^n$ differenzierbar.
  \[\dot{\gamma}:=(\dot{x}_1(t),\dot{x}_2(t),\dots)\]
  $\dot{\gamma}$ heisst der Tangentialvektor oder Geschwindigkeitsvektor zur Stelle $t$.
\end{Def}
\begin{Def}{Geschwindigkeit einer Kurve}
  $\Norm{\dot{\gamma}(t)}$ heisst Geschwindigkeit. Der Geschwindigkeitsvektor hängt vom Parameter ab, nicht von der Stelle in $\mb{R}^n$.
\end{Def}
\begin{Def}{reguläre Kurve}
  Eine stetig differenzierbare Kurve $\gamma:I\to\mb{R}^n$ heisst regulär an der Stelle $t_0\in I$, wenn $\dot{\gamma}(t_0)\neq 0$. Sie heisst regulär, wenn sie an allen STellen regulär ist.
\end{Def}
\begin{Bsp}
  $\gamma(t)=(t^3,t^3), t\in\mb{R}$ $\Spur \gamma= (y=x)$ $\dot{\gamma}(t)=(3t^2,3t^2)$ $\dot{\gamma}=(0,0)$ nicht regulär! Aber der Punkt $(0,0)$ ist nicht singulär.
\end{Bsp}
\begin{Def}{Tangentialeinheitsvektor}
  Ist $\gamma$ an der Stelle $t_0$ regulär, so definiert man
  \[T\gamma(t_0):=\frac{\dot{\gamma}(t_0)}{\Norm{\dot{\gamma}(t_0)}}\]
  als Tangentialeinheitsvektor. $\Norm{T_\gamma}=1$
\end{Def}
\begin{Def}{Parametrisierte Kurve}
  Sei $f:J\to\mb{R}$ stetig differenzierbar. Der parametrisierte Graph von $f$ ist die Kurve
  \begin{align*}
    \gamma_f:&J\to\mb{R}^2\\
    &t\mapsto (t,f(t))
  \end{align*}
  $\Spur(\gamma_f)=\Graph (f)$
  \[\dot{\gamma_f}(t)=(1,f'(t))\neq 0\ \forall t\]
\end{Def}
\begin{Eig}{parametrisierter Graph}
  Ein parametrisierter Graph ist regulär
\end{Eig}
\begin{Sat}
  Sei $\gamma:I\to\mb{R}^2$, $t\mapsto(x(t),y(t))$ stetig differenzierbar. Wenn $\dot{x}(t)$ keine Nullstennen hat, gibt es eine stetig differenzierbare Funktion
  \[f:J\to\mb{R}^2\]
  wobei
  \[J:=x(I)\]
  s.d.
  \[\Graph f=\Spur \gamma\]
\end{Sat}
\begin{Bem}
  $\dot{y}\neq 0$ $\rsa$ Graph von $x(y)$
\end{Bem}
\begin{Sat}
  Sei $t_0\in I$, $x_0:=x(t_0)$
  \[f'(x_0)=\frac{\dot{y(t_0)}}{\dot{x}(t_0)}\]
  \[y=\Diff{f}{x}=\frac{\Diff{y}{t}}{\Diff{x}{t}}\]
  Ist $\gamma$ w-mal stetig differenzierbar, so ist es $f$ auch und
  \[f''\underbrace{(x_0)}_{=x(x_0)}\frac{\dot{x}\ddot{y}-\ddot{x}\dot{y}}{\dot{x}^3}\]
\end{Sat}
\begin{Bew}
  $\dot{x}\neq 0$ $\implies$ $x(t)$ streng monoton $\implies$ invertierbar. $\exists$ Umkehrabbildung
  \begin{align*}
    \tau:J\to I\\
    \tau(x(t))=t\ \forall t\\
  \end{align*}
  stetig differenzierbar
  \[\tau=\frac{1}{\dot{x}}\]
  \begin{align*}
    \gamma(t)=(x(t),y(t))&=\left( x(t),y(\tau(x(t))) \right)\\
    &=\left( x(t),(y\circ \tau)(x(t))\right)\\
    &=(x(t),f\left( x(t) \right)\\
  \end{align*}
  \begin{align*}
    f:=y\circ \tau\\
    \gamma_f:x\mapsto(x,f(x))\\
    \Spur \gamma=\Spur \gamma_f=\Graph f
  \end{align*}
  \begin{align*}
    f'(x_0)=\dot{y}t(t_0)\tau'(x_0)=\dot{y}(t_0)\frac{1}{\dot{x}(t_0)}\\
f''=\left( \Diff{}{x}\dot{y} \right)\frac{1}{\dot{x}}+\dot{y}\Diff{}{x}\left( \frac{1}{\dot{x}} \right)=\\
    =\left( \ddot{y}\tau' \right)\frac{1}{\dot{x}}+\dot{y}\left( -\frac{1}{\dot{x}^2}\ddot{x}\tau' \right)=\\
    =\ddot{y}\frac{1}{\dot{x}}\frac{1}{\dot{x}}-\dot{y}\frac{1}{\dot{x}^2}\ddot{x}\frac{1}{\dot{x}}=\\
    =\frac{\dot{x}\ddot{y}-\ddot{x}\dot{y}}{\dot{x}^3}
  \end{align*}
\end{Bew}
\begin{Eig}
  \begin{align*}
    \dot{x}\neq 0 \rsa y=f(x)\\
    \dot{y}\neq 0 \rsa x=g(y)\\
    \gamma\text{regulär} \implies\ \forall t \exists \text{Umgebung $I$ von $t$ s.d.} \\
    \dot{x}(\tau)\neq 0\ \forall \tau\in I\\
    \dot{y}(\tau)\neq 0\ \forall \tau\in I
  \end{align*}
\end{Eig}
\subsection{Die Bogenlänge}
\begin{Def}
  Sei $\gamma:I\to\mb{R}^n$. Sei $Z=(t_0,t_1,\cdots,t_n)$ $t_i\in I$ $t_0<t_1<\cdots<t_n$ Länge des Sehnenpolygons.
  \[S(Z):=\sum^m_{i=1}\Norm{\gamma(t_i)-\gamma(t_{i-1})}\]
  Gilt $Z^*\supset Z$, dann $S(Z^*)\geq S(Z)$
  \[Z_1\subset Z^*, Z_2\subset Z^* \implies S(Z^*)\geq \max\left( S(Z_1),S(Z_2) \right)\]
  Idee: $s(\gamma):=\sup_ZS(2)$
\end{Def}
\begin{Def}{rektifizierbare Kurve}
  Eine Kurve $\gamma$ heisst rektifizierbar, wenn die Menge der Längen aller einbeschriebenen Sehnenpolygone beschränkt ist.
\end{Def}
\begin{Sat}
  Sei $\gamma:[a;b]\to\mb{R}^n$ fast überall stetig differenzierbar, (d.h. jede Komponente ist fast überall stetig differenzierbar). Dann ist $\gamma$ rektifizierbar (1) und
  \begin{align*}
    s(\gamma)=\int^b_a\Norm{\dot{\gamma}(t)}\md t\geq 0& & (2)
  \end{align*}
\end{Sat}
\begin{Bem}
  Ist $\gamma_f$ der pramametrisierte Graph von $f$
  \[\gamma_f(t)=(t,f(t))\]
  so ist
  \[\dot{\gamma}_f(t)=(1,f'(t))\]
  \[\Norm{\dot{\gamma}_f}=\sqrt{1+f'^2}\]
  \[s(\gamma_f)=\int^b_a\sqrt{1+f'(t)}\md t\]
\end{Bem}
\begin{Not}
  Sei $f=(f_1,\cdots,f_n)$ ein $n$-Tupel Funktionen
  \[\int f(x)\md x:=\left( \int f_1\md x, \int f_2\md x,\cdots,\int\ f_n\md x \right)\]
\end{Not}
\begin{Lem}
  \[\Norm{\int^b_af(x)\md x}\leq \int^b_a\Norm{f(x)}\md x\]
  Beweis
  \begin{enumerate}
    \item Lemma gilt für Treppenfunktionen
    \item Approximationssazu
  \end{enumerate}
\end{Lem}
\begin{Bew}
  Sei $Z=(t_0,\cdots,t_m)$ eine Zerlegung von $[a;b]$    
  \begin{align*}
    S(Z)&=\sum^m_{i=1}\Norm{\gamma(t_i)-\gamma(t_{i-1}}\\
    &=\sum\Norm{\int^{t_i}_{t_{i-1}}\dot{\gamma}(t)\md t}\\
    &\stackrel{\text{Lemma}}{\leq}\sum^m_{i=1}\int^{t_i}_{t_{i-1}}\Norm{\dot{\gamma}}\md t\\
    &=\int^b_a\Norm{\dot{\gamma}}\md t    
  \end{align*}
  ($\Norm{\dot{\gamma}}\in\mathcal{R}$  Diese Abschätzung gilt für alle Zerlegungen. $\implies$ $\gamma$ rektifizierbar.
  \[s(\gamma)\leq \int^b_a\Norm{\dot{\gamma}}\md t\]
  = für (2)
  \[\forall \varepsilon>0\ \exists Z: S(Z)\geq f(\Norm{\dot{\gamma}}-\varepsilon\]
  Treppenfunktionen + Approximationssatz
\end{Bew}
