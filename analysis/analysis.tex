
%= Document-Class ==================================================================================
\documentclass[10pt,a4paper]{article}

%= Packages ========================================================================================
\usepackage[utf8]{inputenc}
\usepackage{ngerman,amsmath,amssymb,amsfonts,mathrsfs}
\usepackage{bbm}
\usepackage{epic,eepic,pstricks,pst-node,pst-plot}
\usepackage{pstricks}
\usepackage{colortbl}
\usepackage{graphicx}
\usepackage{makeidx}
\usepackage{fancyhdr}
\usepackage{latexsym}
\usepackage{psfrag}
\usepackage{enumerate}
\usepackage{float}
%\usepackage{mathtext}
\usepackage[all, knot, poly]{xy}
\usepackage{dsfont}
\pagestyle{fancy}
\usepackage{multirow, bigdelim, bigstrut}
\usepackage{rotating}
\usepackage{ifthen}
\usepackage{boxedminipage}
\usepackage{mathtools}
% \usepackage{mathtools}

%= Seiten-Layout =========================================================================
\voffset-22mm \textheight715pt 

%Seitenbreite==============================================================

%\oddsidemargin=-0.2in
%\evensidemargin=-0.4in
%\textwidth=5.2in
%\headwidth=5.2in

%= Index-Befehle ========================================================================
\renewcommand{\indexname}{Stichwortverzeichnis}
\makeindex

%= Befehl-Overwriting =======================================================================
\makeatletter
\renewcommand{\section}{\@startsection {section}{1}{\z@}{-3.5ex \@plus -1ex \@minus -.2ex}{2.3ex \@plus.2ex}{\reset@font\Large\bfseries}}
\renewcommand{\subsection}{\setcounter{Prop}{0}\setcounter{Lem}{0}\setcounter{Sat}{0}\setcounter{Kor}{0}\setcounter{Def}{0}\@startsection{subsection}{2}{\z@}{-3.25ex\@plus -1ex \@minus -.2ex}{1.5ex \@plus .2ex}{\normalfont\large\bfseries}}
\newcommand\subnummer{\@arabic\c@section.\@arabic\c@subsection}
\makeatother

%= Strings ================================================================
\newcommand{\mainfold}{.}
\newcommand{\prefix}{A1-}

%= Eigene Befehle ==========================================================================
\DeclareMathOperator{\id}{Id}
\DeclareMathOperator{\arccot}{arccot}
\DeclareMathOperator{\arsinh}{arsinh}
\DeclareMathOperator{\arcosh}{arcosh}
\DeclareMathOperator{\artanh}{artanh}
\DeclareMathOperator{\md}{d}
\DeclareMathOperator{\Grad}{grad}

\newcommand{\Diff}[2]{\displaystyle\frac{\mathrm{d}#1}{\mathrm{d}#2}}
\newcommand{\End}{\hfill{\hbox{$\Box$}}\par\vspace{2mm}}
\newcommand{\eps}{\varepsilon}
\newcommand{\ePic}[1]{\input{\mainfold/graphics/\prefix#1.eepic}}
\newcommand{\pst}[1]{\input{\mainfold/graphics/\prefix#1.pst}}
\newcommand{\pic}[1]{\input{\mainfold/graphics/\prefix#1.pic}}
\newcommand{\Mx}[1]{\begin{pmatrix}#1\end{pmatrix}}
\newcommand{\im}[1]{\operatorname{Im}(#1)}
%\newcommand{\Include}[4]{\rhead{#2.#3.20#4}\input{\mainfold/lectures/#1-#4-#3-#2.tex}}
\newcommand{\Index}[1]{\emph{#1}\index{#1}}
\newcommand{\Int}[4]{\displaystyle\int\limits_{#1}^{#2}#3\,\mathrm{d}#4}
\newcommand{\diff}[1]{\operatorname{d}\!#1}
\newcommand{\Limi}[1]{\displaystyle\lim_{#1\rightarrow\infty}}
\newcommand{\Limo}[1]{\displaystyle\lim_{#1\rightarrow0}}
\newcommand{\mb}[1]{\mathbb{#1}}
\newcommand{\ds}{\displaystyle}
\newcommand{\ol}[1]{\overline{#1}}
\newcommand{\Part}[2]{\dfrac{\partial #1}{\partial #2}}
\newcommand{\QED}{\hfill{\hbox{(QED)}}\par\vspace{2mm}}
\newcommand{\re}[1]{\operatorname{Re}(#1)}
\newcommand{\s}{\hspace{2mm}}
\newcommand{\vsa}{\vspace{1mm} \\}
\newcommand{\vsb}{\vspace{2mm} \\}
\newcommand{\vsc}{\vspace{3mm} \\}
% \newcommand{\tr}[1]{\textrm{#1}}
\newcommand{\tr}[1]{\text{#1}}
\newcommand{\ra}{\rightarrow}
\newcommand{\Ra}{\Rightarrow}
\newcommand{\Lra}{\Leftrightarrow}
\newcommand{\La}{\Leftarrow}
\newcommand{\ul}[1]{\underline{#1}}
\newcommand{\rsa}{\rightsquigarrow}

%\newcommand{\detmx}{\left| \begin{array} #1 \end{array} \right|}

\newcommand{\grad}[1]{\Grad(#1)}
\newcommand{\fr}[2]{\displaystyle\frac{#1}{#2}} % fertiger bullshit, daf�r gibts \dfrac{}{}
\renewcommand{\Re}{\operatorname{Re}}
\renewcommand{\Im}{\operatorname{Im}}

% ---- DELIMITER PAIRS ----
\def\floor#1{\lfloor #1 \rfloor}
\def\ceil#1{\lceil #1 \rceil}
\def\seq#1{\langle #1 \rangle}
\def\set#1{\{ #1 \}}
\def\abs#1{\mathopen| #1 \mathclose|}	% use instead of $|x|$ 
\def\norm#1{\mathopen\| #1 \mathclose\|}% use instead of $\|x\|$ 

% --- Self-scaling delmiter pairs ---
\def\Floor#1{\left\lfloor #1 \right\rfloor}
\def\Ceil#1{\left\lceil #1 \right\rceil}
\def\Seq#1{\left\langle #1 \right\rangle}
\def\Set#1{\left\{ #1 \right\}}
\def\Abs#1{\left| #1 \right|}
\def\Norm#1{\left\| #1 \right\|}

%Adrians Abbildungs-Environment ==============================================

\newcommand{\Sidein}{\begin{rotate}{90}\small$\in$\end{rotate}}

\newcommand{\Abb}[5][]{\ensuremath{
    \begin{array}{lc}
      \ifthenelse{\equal{#1}{}}{}{#1:}\;\; & 
      \begin{xy}
        \xymatrixrowsep{1em}\xymatrixcolsep{2em}%
        \xymatrix{ #2 \ar[r] \ar@{}[d]^<<<<{\hspace{0.001em} \Sidein}
          & #3  \ar@{}[d]^<<<<{\hspace{0.001em} \Sidein} \\
          #4 \ar@{|->}[r] & #5} \end{xy}
    \end{array}
  }%
}

%= Environments ========================================================================
\def\thechapter{\Roman{chapter}}
\def\thesection{\arabic{section}}
\newenvironment{Bsp}{\paragraph{Beispiel:}\begin{quote}}{\end{quote}}
\newenvironment{Bem}{\paragraph{Bemerkung:}\begin{quote}}{\end{quote}}
\newenvironment{Bew}[1]{\paragraph{Beweis: #1}\begin{quote}}{$\hfill\Box$\par\vspace{2mm}\end{quote}}
\newcounter{Lem}
\newenvironment{Lem}[1]{\paragraph{Lemma \subnummer.\addtocounter{Lem}{1}\theLem: #1} \begin{quote}}{\end{quote}}
\newcounter{Kor}
\newenvironment{Kor}{\paragraph{Korollar \subnummer.\addtocounter{Kor}{1}\theKor:}\begin{quote}}{\end{quote}}
\newcounter{Def}
\newenvironment{Def}[1]{\paragraph{Definition \subnummer.\addtocounter{Def}{1}\theDef: #1} \begin{quote}}{\end{quote}}
\newcounter{Prop}
\newenvironment{Prop}{\paragraph{Proposition \subnummer.\addtocounter{prop}{1}\theProp:}\begin{quote}}{\end{quote}}
\newcounter{Sat}
\newenvironment{Sat}[1]{\paragraph{Satz \subnummer.\addtocounter{Sat}{1}\theSat: #1}\begin{quote}}{\end{quote}}
\newenvironment{parquote}[1]{\paragraph{#1}\begin{quote}}{\end{quote}}
%Der todo: Notation

\def\pstexInput#1{%
  \begin{center}
    \begin{picture}(0,0)%
      \special{psfile=\mainfold/graphics/A2-#1.pstex}%
    \end{picture}%
    \input{\mainfold/graphics/A2-#1.pstex_t}%
  \end{center}
}

%= Titelseite ===========================================================================
\begin{document}
\headheight15pt
\begin{titlepage}
\hfill
\vspace{20mm}
\pagenumbering{roman}
\begin{center}
{\LARGE Analysis I - Vorlesungs-Script} %\vskip 3em {\large Prof.
%Guido Mislin} \vskip 1.5em
%{\large Basisjahr 06/07}\vspace{30mm}\\
%{\large {\bf Mitschrift:} \vspace{2mm}\\
%Alexander Berthold van der Bourg}\vspace{5mm}\\ %30mm
%{\large {\bf Graphics:} \vspace{2mm}\\
%Pirmin Weigele }\vspace{30mm}\\ %30mm

\end{center}
\vfill

\end{titlepage}


%= Inhaltsverzeichnis ==========================================================================
\lhead{}
\rhead{}
\tableofcontents
\newpage
\pagenumbering{arabic}
\setcounter{page}{1}

%= Vorlesung-Skripts ==========================================================================
\cfoot{\thepage}
\fancyhead[L]{\nouppercase{\leftmark}}
\newpage

%= Analysis I & & II ==========================================================================

%Analysis I
\section{Integralrechnung}
\paragraph{Ziel} mathematisch präzise Formulierung des ``Flächeninhalts'' unter dem Graphen einer Funktion
\paragraph{Fragen}
\begin{itemize}
  \item Welche Funktionen sind zulässig?
  \item Wie definiert man das Integra für diese Funktionen?
\end{itemize}
\paragraph{Idee}
\begin{enumerate}
  \item def. Integral für spezielle Funktionen (Treppenfunktionen)
  \item betrachte Folgen von Treppenfunktionen und führe geeigneten Konvergenzbegriff ein (gleichmässige Konvergenz), $\to$ mögliche Limiten sind Regelfunktionen
  \item falls $f_n \xrightarrow{n\to\infty}f$ (Folge von Treppenfunktionen), setze $\int_a^b f \md x:= \Limi{n}\left( \int^b_a f_n \md x \right)$
    \begin{align*}
      f_n\to f \text{folgt} \left( \int^b_a f_n \md x \right)_{n\in\mb{N}} \text{konvergent}\\
      f_n \& g_n \to f \text{zwei Folgen} \implies \Limi{n} \left( \int^b_a f_n \md x \right)=\Limi{n} \left( \int^b_a g_n \right)
    \end{align*}
\end{enumerate}
\subsection{Treppenfunktionen}
\begin{itemize}
  \item $a<b, a,b\in\mb{R}$ $\{x_0,x_1,\cdots,x_n\}$ \underline{Zerlegung} von $[a,b] \Lra a=x_0<x_1<x_2<\cdots<x_{n-1}<x_n = b$
  \item $\phi[a,b]\to\mb{C}$ \underline{Treppenfunktion} (auf $[a,b]$) $\Lra$ $\exists$ Zerlegung $\{x_0,x_1,\cdots,x_n\}$ von $[a,b]$ so dass $\phi|_{(x_{n-1},x_n)}$ konstant $\forall k=1,\cdots,n$
\end{itemize}
\begin{Bem}
  \begin{itemize}
    \item keine Aussage über $\phi(x_0),\cdots,\phi(x_n)$
    \item nicht verboten zu feine Zerlegungen zu betrachten
  \end{itemize}
\end{Bem}
\begin{itemize}
  \item $\tau([a,b])$ (ein Vektorraum über $\mb{C}$, $\phi, \psi$ Treppenfunktionen) Menge aller Treppenfunktionen auf $[a,b]$
\end{itemize}
\begin{Def}{Integral von Treppenfunktionen} $\phi:[a,b]\to\mb{C}$ Teppenfunktion mit Zerlegung $\{x_0,x_1,\cdots,x_n\}$
  \begin{itemize}
    \item $c_K$ = Funktionswert von $\phi$ auf $(x_{k-1},x_k)$
    \item $\Delta x_k=x_k-x_{k-1}$
  \end{itemize}
  \[\int_a^b \phi(x)\md x=\sum^n_{k=1}\left( c_k\cdot\Delta x_k \right)\]
\end{Def}
\begin{Lem}{}
  Das Integral einer Treppenfunktion ist unabhängig von der gewählten Zerlegung
\end{Lem}
\begin{Bew}{}
  \begin{align*}
    Z=\{x_0,x_1,\cdots,x_n\} &\ \&\  Z'= \{y_0,y_1,\cdots,y_m\} & \text{Zerlegungen von} [a,b]\\
    \phi|_{(x_{k-1},x_k)} &\ \&\  \phi|_{(y_{k-1},y_k)} & \text{konstant}\\
    \rsa I(Z) & \rsa I(Z') &\ \ \leftarrow \text{Summen} \sum^n_{k=1}c_k\Delta x_k \ \&\ \sum^m_{k=1}c'_k\Delta y_k
  \end{align*}
  \subparagraph{Frage} $I(Z)=I(Z')$
  \subparagraph{Zeige} $I(Z)=I(Z\cup Z')=I(Z')$\\
  $Z\cup Z'$ entsteht aus $Z$ durch Hinzufügen von endlich vielen Punkten.\\
  Angenommen $Z\cup Z' = Z\cup\{y\}, y\not\in Z$. Leicht zu sehen: $I(Z)=I(Z\cup\{y\})$
  \begin{align*}
    I(Z)=I(Z\cup\{y\})\xRightarrow{\text{Ind}} I(Z) = I(Z\cup\{y_1\})=I(Z\cup \{y_1\}\cup\{y_2\}) = \cdots = I(Z\cup Z')
  \end{align*}
\end{Bew}
\begin{Lem}{}
  \begin{align*}
    \int_a^b \md x \tau([a,b])\to\mb{C}
  \end{align*}
  \begin{enumerate}
    \item $\int_a^b \md x$ ist linear, d.h.
      \[\forall \phi, \psi\in\tau([a,b]),\alpha, \beta, \in\mb{C}: \int_a^b\alpha\phi+\beta\psi \md x = \alpha\left( \int^b_a\phi \md x \right)+\beta\left( \int^b_a\phi \md x \right)\]
    \item \[\Abs{f^b_a\phi \md x}\leq \int^b_a\Abs{\phi} \md x \leq (b-a) \underbrace{\Norm{\phi}}_{\text{Supremum}}\]
    \item für $\phi,\psi:[a,b]\to\mb{R}$ mit $\phi(x)\leq\psi(x)\ \forall x\in[a,b] \implies$
      \[\int^b_a\psi \md x \leq \int^b_a \psi \md x\]
  \end{enumerate}
\end{Lem}
\begin{Bew}
  $\phi$ und $\psi$ Treppenfunktionen mit Zerlegung $Z$ bzw. $Z'$ $\implies$ $Z\cup Z'$ Zerlegung für $\phi$ und $\psi$
  \[\int^b_a\alpha\phi+\beta\psi \md x = (\alpha\phi)|_{(x_{k-1},x_k)}=\alpha(\phi|_{(x_{k-1},x_k)})\]
  wobei $\Delta x_k=x_k-x_{k-1}$.\\
  Wert von $\phi$ auf $(x_{k-1},x_k)$ =: $c_k$, Wert von $\psi$ auf $(x_{k-1},x_k)$ =: $\md_k$
  \[\sum^n_{i=1}(\alpha c_k+\beta \md_k)\Delta x_k=\alpha(\sum^n_{i=1})+\beta(\sum^n_{i=1}\md_k\Delta x_k)=\alpha\int^b_a \phi \md x+\beta \int^b_a \psi \md x\]
\end{Bew}
\begin{Bem}
  $\int^b_a \md x: \tau([a,b])\to\mb{C}$ linear, $\ker(\int_a^b \md x) \subset \tau([a,b])$ Untervektorraum
\end{Bem}
\begin{Bem}
  lineares erzeugendes System von $\tau([a,b])$ $A\subset\mb{R}$
  \[1_A(x) = \begin{cases}1&\text{für} x\in A\\0&\text{sonst}\end{cases}\]
  \{$1_{[c,d]}$ mit $a<c\leq d<b$ \} erzeugendes System
\end{Bem}
\subsection{Regelfunktionen}
\begin{Def}{Regelfunktionen}
  $f:[a,b] \to\mb{C}$ \underline{Regelfunktionen} (auf $[a,b]$) $\Lra$
  \begin{itemize}
    \item 
      \begin{align*}
        \forall y\in(a,b):\exists \lim_{x\searrow y}f(x)\ \&\ \lim_{x\nearrow y} f(x)\\
        (\text{nicht nötig:} \lim_{x\searrow y} f(x) = \lim_{x\nearrow y}f(x))
      \end{align*}
    \item \[\exists \lim_{x\swarrow y} f(x)\ \&\ \exists \lim_{x_\nearrow y} f(x)\]
  \end{itemize}
\end{Def}
\begin{Bem}
  \[\lim_{x\searrow y} f(x)=c:\ \Lra\ \forall \varepsilon > 0 \exists \rho\ \forall 0<x-y<\rho: \Abs{f(x)-c}<\varepsilon\]
  $\mathcal{R}([a,b])$ Menge aller Regelfunktionen auf $[a,b]$
  \begin{align*}
    \mathcal{R}([a,b])\ \text{Vektorraum über} \mb{C}\\
    \mathcal{T}([a,b])\subset \mathcal{R}([a,b]) \text{Untervektorraum}
  \end{align*}
  \subparagraph{Frage}$\mathcal{R}([a,b])/\mathcal{T}([a,b])$ Vektorraum über $\mb{C}$, Dimension?
\end{Bem}
\begin{Bsp}
  jede stetige Funktion ist eine Regelfunktion
\end{Bsp}
\begin{Bsp}
  jede monotone Funktion auf $[a,b]$ ist eine Regelfunktion (sehe Seite 78)
\end{Bsp}
\begin{Bem}
  \begin{align*}
    f,g\in \mathcal{R}([a,b])\implies \lambda f_{\lambda\in\mb{C}}, f+g, \Abs{f}, f\cdot g, \max(f,g), \min(f,g)
  \end{align*}
  sind in $\mathcal{R}([a,b])$
\end{Bem}
\begin{Def}{gleichmässige Konvergenz}
  $(f_n)_{n\in\mb{N}}$ Folge von Funktionen auf $D\subset \mathcal{R}, f$ Funktion auf $D$.\\
  $(f_n)_{n\in\mb{N}}$ \underline{konvergiert gleichmässig} gegen $f$ $\Lra$ $\Limi{n} \underbrace{\Norm{f-f_n}}_{\sup_{x=D}\Abs{f(x)-f_n(x)}}=0$
\end{Def}
\begin{Bem}
  falls $(f_n)_{n\in\mb{N}}$ konvergiert gleichmässig $\implies$ limes ist eindeutig
\end{Bem}
\begin{Bem}
  $(f_n)_{n\in\mb{N}}$ konvergiert gleichmässig gegen $f$ $\implies$ $f_n(x)\to f(x)\ \forall x\in D$
  \[(\Abs{f(x)-f_n(x)}\leq \sup_{x\in D} \Abs{f(x)-f_n(x)}\to 0)\]
\end{Bem}
\begin{Bem}
  Die Umkehrung gilt NICHT $D=(0,1]$
  \begin{align*}
    f=0, f_n(x)=\begin{cases}1-nx&0\leq x\leq \frac{1}{n}\\0&\frac{1}{n}\leq x \leq 1\end{cases}\\
    \forall x\in D: f_n(x)\xrightarrow{n\to\infty}0\\
    \Norm{f-f_n}=\sup_{x\in D} \Abs{f(x)-f_n(x)} =1\\
    \Limi{n}\Norm{f-f_n}=1
  \end{align*}
\end{Bem}



\newpage

%= Stichwortverzeichnis ======================================================================
\rhead{}
\addcontentsline{toc}{section}{Stichwortverzeichnis}
\printindex

\end{document}
