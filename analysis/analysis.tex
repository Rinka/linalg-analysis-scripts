% headers by Alexander Berthold van der Bourg / Pirmin Weigele 

%= Document-Class ==================================================================================
\documentclass[10pt,a4paper]{article}

%= Packages ========================================================================================
\usepackage[utf8]{inputenc}
\usepackage{ngerman,amsmath,amssymb,amsfonts,mathrsfs}
\usepackage{amsthm}
\usepackage{bbm}
\usepackage{ulsy}
\usepackage{epic,eepic,pstricks,pst-node,pst-plot}
\usepackage{pstricks}
\usepackage{colortbl}
\usepackage{graphicx}
\usepackage{makeidx}
\usepackage{fancyhdr}
\usepackage{latexsym}
\usepackage{psfrag}
\usepackage{enumerate}
\usepackage{float}
%\usepackage{mathtext}
\usepackage[all, knot, poly]{xy}
\usepackage{dsfont}
\pagestyle{fancy}
\usepackage{multirow, bigdelim, bigstrut}
\usepackage{rotating}
\usepackage{ifthen}
\usepackage{boxedminipage}
\usepackage{mathtools}
\usepackage{ulsy}
\usepackage{trfsigns}

%= Seiten-Layout =========================================================================
\voffset-22mm \textheight715pt 

%Seitenbreite==============================================================

%\oddsidemargin=-0.2in
%\evensidemargin=-0.4in
%\textwidth=5.2in
%\headwidth=5.2in

%= Index-Befehle ========================================================================
\renewcommand{\indexname}{Stichwortverzeichnis}
\makeindex

%= Befehl-Overwriting =======================================================================
\makeatletter
\makeatother

%= Strings ================================================================
\newcommand{\mainfold}{.}
\newcommand{\prefix}{A1-}

%= Eigene Befehle ==========================================================================
\DeclareMathOperator{\id}{Id}
\DeclareMathOperator{\arccot}{arccot}
\DeclareMathOperator{\arcsinh}{arcsinh}
\DeclareMathOperator{\arccosh}{arccosh}
\DeclareMathOperator{\arctanh}{arctanh}
\DeclareMathOperator{\md}{d}
\DeclareMathOperator{\Grad}{grad}
\DeclareMathOperator{\Spur}{Spur}
\DeclareMathOperator{\Graph}{Graph}
\DeclareMathOperator{\sign}{sign}

\newcommand{\Diff}[2]{\displaystyle\frac{\mathrm{d}#1}{\mathrm{d}#2}}
\newcommand{\End}{\hfill{\hbox{$\Box$}}\par\vspace{2mm}}
\newcommand{\eps}{\varepsilon}
\newcommand{\ePic}[1]{\input{\mainfold/graphics/\prefix#1.eepic}}
\newcommand{\pst}[1]{\input{\mainfold/graphics/\prefix#1.pst}}
\newcommand{\pic}[1]{\input{\mainfold/graphics/\prefix#1.pic}}
\newcommand{\Mx}[1]{\begin{pmatrix}#1\end{pmatrix}}
%\newcommand{\im}[1]{\operatorname{Im}(#1)}
%\newcommand{\Include}[4]{\rhead{#2.#3.20#4}\input{\mainfold/lectures/#1-#4-#3-#2.tex}}
\newcommand{\Index}[1]{\emph{#1}\index{#1}}
\newcommand{\Int}[4]{\displaystyle\int\limits_{#1}^{#2}#3\,\mathrm{d}#4}
\newcommand{\diff}[1]{\operatorname{d}\!#1}
\newcommand{\Limi}[1]{\displaystyle\lim_{#1\rightarrow\infty}}
\newcommand{\Limo}[1]{\displaystyle\lim_{#1\rightarrow0}}
\newcommand{\mb}[1]{\mathbb{#1}}
\newcommand{\ds}{\displaystyle}
\newcommand{\ol}[1]{\overline{#1}}
\newcommand{\Part}[2]{\dfrac{\partial #1}{\partial #2}}
\newcommand{\QED}{\hfill{\hbox{(QED)}}\par\vspace{2mm}}
\newcommand{\re}[1]{\operatorname{Re}(#1)}
\newcommand{\s}{\hspace{2mm}}
\newcommand{\vsa}{\vspace{1mm} \\}
\newcommand{\vsb}{\vspace{2mm} \\}
\newcommand{\vsc}{\vspace{3mm} \\}
% \newcommand{\tr}[1]{\textrm{#1}}
\newcommand{\tr}[1]{\text{#1}}
\newcommand{\ra}{\rightarrow}
\newcommand{\Ra}{\Rightarrow}
\newcommand{\Lra}{\Leftrightarrow}
\newcommand{\La}{\Leftarrow}
\newcommand{\ul}[1]{\underline{#1}}
\newcommand{\rsa}{\rightsquigarrow}
\newcommand{\ara}[2]{\autorightarrow{\ensuremath{#1}}{\ensuremath{#2}}}

%\newcommand{\detmx}{\left| \begin{array} #1 \end{array} \right|}

\newcommand{\grad}[1]{\Grad(#1)}
\newcommand{\fr}[2]{\displaystyle\frac{#1}{#2}} % fertiger bullshit, daf�r gibts \dfrac{}{}
\renewcommand{\Re}{\operatorname{Re}}
\renewcommand{\Im}{\operatorname{Im}}

% ---- DELIMITER PAIRS ----
\def\floor#1{\lfloor #1 \rfloor}
\def\ceil#1{\lceil #1 \rceil}
\def\seq#1{\langle #1 \rangle}
\def\set#1{\{ #1 \}}
\def\abs#1{\mathopen| #1 \mathclose|}	% use instead of $|x|$ 
\def\norm#1{\mathopen\| #1 \mathclose\|}% use instead of $\|x\|$ 

% --- Self-scaling delmiter pairs ---
\def\Floor#1{\left\lfloor #1 \right\rfloor}
\def\Ceil#1{\left\lceil #1 \right\rceil}
\def\Seq#1{\left\langle #1 \right\rangle}
\def\Set#1{\left\{ #1 \right\}}
\def\Abs#1{\left| #1 \right|}
\def\Norm#1{\left\| #1 \right\|}

%Adrians Abbildungs-Environment ==============================================

\newcommand{\Sidein}{\begin{rotate}{90}\small$\in$\end{rotate}}

\newcommand{\Abb}[5][]{\ensuremath{
    \begin{array}{lc}
      \ifthenelse{\equal{#1}{}}{}{#1:}\;\; & 
      \begin{xy}
        \xymatrixrowsep{1em}\xymatrixcolsep{2em}%
        \xymatrix{ #2 \ar[r] \ar@{}[d]^<<<<{\hspace{0.001em} \Sidein}
          & #3  \ar@{}[d]^<<<<{\hspace{0.001em} \Sidein} \\
          #4 \ar@{|->}[r] & #5} \end{xy}
    \end{array}
  }%
}

%= Environments ========================================================================
\def\thechapter{\Roman{chapter}}
\def\thesection{\arabic{section}}
\newtheorem{Bew}{Beweis}
\newtheorem{Lem}{Lemma}
\newtheorem{Kor}{Korollar}
\newtheorem{Sat}{Satz}
\newtheorem{Prop}{Proposition}
\theoremstyle{definition}
\newtheorem{Bsp}{Beispiel}
\newtheorem{Def}{Definition}
\newtheorem{Prob}{Problem}
\theoremstyle{remark}
\newtheorem{Bem}{Bemerkung}
\newtheorem{Eig}{Eigenschaften}
\newtheorem{Not}{Notation}

\def\pstexInput#1{%
  \begin{center}
    \begin{picture}(0,0)%
      \special{psfile=\mainfold/graphics/A2-#1.pstex}%
    \end{picture}%
    \input{\mainfold/graphics/A2-#1.pstex_t}%
  \end{center}
}

%= Titelseite ===========================================================================
\begin{document}
\headheight15pt
\begin{titlepage}
\hfill
\vspace{20mm}
\pagenumbering{roman}
\begin{center}
{\LARGE Analysis I - Vorlesungs-Script} \vskip 3em {\large Prof.
Alberto Cattaneo} \vskip 1.5em
{\large Basisjahr 08/09 Semester II}\vspace{30mm}\\
{\large {\bf Mitschrift:} \vspace{2mm}\\
Simon Hafner}\vspace{5mm}\\ %30mm
%{\large {\bf Graphics:} \vspace{2mm}\\
%Pirmin Weigele }\vspace{30mm}\\ %30mm
\author{Simon Hafner}

\end{center}
\vfill

\end{titlepage}


%= Inhaltsverzeichnis ==========================================================================
\lhead{}
\rhead{}
\tableofcontents
\newpage
\pagenumbering{arabic}
\setcounter{page}{1}

%= Vorlesung-Skripts ==========================================================================
\cfoot{\thepage}
\fancyhead[L]{\nouppercase{\leftmark}}
\newpage

%= Analysis I & & II ==========================================================================

%Analysis I
\section{Integralrechnung}
\paragraph{Ziel} mathematisch präzise Formulierung des ``Flächeninhalts'' unter dem Graphen einer Funktion
\paragraph{Fragen}
\begin{itemize}
  \item Welche Funktionen sind zulässig?
  \item Wie definiert man das Integra für diese Funktionen?
\end{itemize}
\paragraph{Idee}
\begin{enumerate}
  \item def. Integral für spezielle Funktionen (Treppenfunktionen)
  \item betrachte Folgen von Treppenfunktionen und führe geeigneten Konvergenzbegriff ein (gleichmässige Konvergenz), $\to$ mögliche Limiten sind Regelfunktionen
  \item falls $f_n \xrightarrow{n\to\infty}f$ (Folge von Treppenfunktionen), setze $\int_a^b f \md x:= \Limi{n}\left( \int^b_a f_n \md x \right)$
    \begin{align*}
      f_n\to f \text{folgt} \left( \int^b_a f_n \md x \right)_{n\in\mb{N}} \text{konvergent}\\
      f_n \& g_n \to f \text{zwei Folgen} \implies \Limi{n} \left( \int^b_a f_n \md x \right)=\Limi{n} \left( \int^b_a g_n \right)
    \end{align*}
\end{enumerate}
\subsection{Treppenfunktionen}
\begin{itemize}
  \item $a<b, a,b\in\mb{R}$ $\{x_0,x_1,\cdots,x_n\}$ \underline{Zerlegung} von $[a,b] \Lra a=x_0<x_1<x_2<\cdots<x_{n-1}<x_n = b$
  \item $\phi[a,b]\to\mb{C}$ \underline{Treppenfunktion} (auf $[a,b]$) $\Lra$ $\exists$ Zerlegung $\{x_0,x_1,\cdots,x_n\}$ von $[a,b]$ so dass $\phi|_{(x_{n-1},x_n)}$ konstant $\forall k=1,\cdots,n$
\end{itemize}
\begin{Bem}
  \begin{itemize}
    \item keine Aussage über $\phi(x_0),\cdots,\phi(x_n)$
    \item nicht verboten zu feine Zerlegungen zu betrachten
  \end{itemize}
\end{Bem}
\begin{itemize}
  \item $\tau([a,b])$ (ein Vektorraum über $\mb{C}$, $\phi, \psi$ Treppenfunktionen) Menge aller Treppenfunktionen auf $[a,b]$
\end{itemize}
\begin{Def}{Integral von Treppenfunktionen} $\phi:[a,b]\to\mb{C}$ Teppenfunktion mit Zerlegung $\{x_0,x_1,\cdots,x_n\}$
  \begin{itemize}
    \item $c_K$ = Funktionswert von $\phi$ auf $(x_{k-1},x_k)$
    \item $\Delta x_k=x_k-x_{k-1}$
  \end{itemize}
  \[\int_a^b \phi(x)\md x=\sum^n_{k=1}\left( c_k\cdot\Delta x_k \right)\]
\end{Def}
\begin{Lem}{}
  Das Integral einer Treppenfunktion ist unabhängig von der gewählten Zerlegung
\end{Lem}
\begin{Bew}{}
  \begin{align*}
    Z=\{x_0,x_1,\cdots,x_n\} &\ \&\  Z'= \{y_0,y_1,\cdots,y_m\} & \text{Zerlegungen von} [a,b]\\
    \phi|_{(x_{k-1},x_k)} &\ \&\  \phi|_{(y_{k-1},y_k)} & \text{konstant}\\
    \rsa I(Z) & \rsa I(Z') &\ \ \leftarrow \text{Summen} \sum^n_{k=1}c_k\Delta x_k \ \&\ \sum^m_{k=1}c'_k\Delta y_k
  \end{align*}
  \subparagraph{Frage} $I(Z)=I(Z')$
  \subparagraph{Zeige} $I(Z)=I(Z\cup Z')=I(Z')$\\
  $Z\cup Z'$ entsteht aus $Z$ durch Hinzufügen von endlich vielen Punkten.\\
  Angenommen $Z\cup Z' = Z\cup\{y\}, y\not\in Z$. Leicht zu sehen: $I(Z)=I(Z\cup\{y\})$
  \begin{align*}
    I(Z)=I(Z\cup\{y\})\xRightarrow{\text{Ind}} I(Z) = I(Z\cup\{y_1\})=I(Z\cup \{y_1\}\cup\{y_2\}) = \cdots = I(Z\cup Z')
  \end{align*}
\end{Bew}
\begin{Lem}{}
  \begin{align*}
    \int_a^b \md x \tau([a,b])\to\mb{C}
  \end{align*}
  \begin{enumerate}
    \item $\int_a^b \md x$ ist linear, d.h.
      \[\forall \phi, \psi\in\tau([a,b]),\alpha, \beta, \in\mb{C}: \int_a^b\alpha\phi+\beta\psi \md x = \alpha\left( \int^b_a\phi \md x \right)+\beta\left( \int^b_a\phi \md x \right)\]
    \item \[\Abs{f^b_a\phi \md x}\leq \int^b_a\Abs{\phi} \md x \leq (b-a) \underbrace{\Norm{\phi}}_{\text{Supremum}}\]
    \item für $\phi,\psi:[a,b]\to\mb{R}$ mit $\phi(x)\leq\psi(x)\ \forall x\in[a,b] \implies$
      \[\int^b_a\psi \md x \leq \int^b_a \psi \md x\]
  \end{enumerate}
\end{Lem}
\begin{Bew}
  $\phi$ und $\psi$ Treppenfunktionen mit Zerlegung $Z$ bzw. $Z'$ $\implies$ $Z\cup Z'$ Zerlegung für $\phi$ und $\psi$
  \[\int^b_a\alpha\phi+\beta\psi \md x = (\alpha\phi)|_{(x_{k-1},x_k)}=\alpha(\phi|_{(x_{k-1},x_k)})\]
  wobei $\Delta x_k=x_k-x_{k-1}$.\\
  Wert von $\phi$ auf $(x_{k-1},x_k)$ =: $c_k$, Wert von $\psi$ auf $(x_{k-1},x_k)$ =: $\md_k$
  \[\sum^n_{i=1}(\alpha c_k+\beta \md_k)\Delta x_k=\alpha(\sum^n_{i=1})+\beta(\sum^n_{i=1}\md_k\Delta x_k)=\alpha\int^b_a \phi \md x+\beta \int^b_a \psi \md x\]
\end{Bew}
\begin{Bem}
  $\int^b_a \md x: \tau([a,b])\to\mb{C}$ linear, $\ker(\int_a^b \md x) \subset \tau([a,b])$ Untervektorraum
\end{Bem}
\begin{Bem}
  lineares erzeugendes System von $\tau([a,b])$ $A\subset\mb{R}$
  \[1_A(x) = \begin{cases}1&\text{für} x\in A\\0&\text{sonst}\end{cases}\]
  \{$1_{[c,d]}$ mit $a<c\leq d<b$ \} erzeugendes System
\end{Bem}
\subsection{Regelfunktionen}
\begin{Def}{Regelfunktionen}
  $f:[a,b] \to\mb{C}$ \underline{Regelfunktionen} (auf $[a,b]$) $\Lra$
  \begin{itemize}
    \item 
      \begin{align*}
        \forall y\in(a,b):\exists \lim_{x\searrow y}f(x)\ \&\ \lim_{x\nearrow y} f(x)\\
        (\text{nicht nötig:} \lim_{x\searrow y} f(x) = \lim_{x\nearrow y}f(x))
      \end{align*}
    \item \[\exists \lim_{x\swarrow y} f(x)\ \&\ \exists \lim_{x_\nearrow y} f(x)\]
  \end{itemize}
\end{Def}
\begin{Bem}
  \[\lim_{x\searrow y} f(x)=c:\ \Lra\ \forall \varepsilon > 0 \exists \rho\ \forall 0<x-y<\rho: \Abs{f(x)-c}<\varepsilon\]
  $\mathcal{R}([a,b])$ Menge aller Regelfunktionen auf $[a,b]$
  \begin{align*}
    \mathcal{R}([a,b])\ \text{Vektorraum über} \mb{C}\\
    \mathcal{T}([a,b])\subset \mathcal{R}([a,b]) \text{Untervektorraum}
  \end{align*}
  \subparagraph{Frage}$\mathcal{R}([a,b])/\mathcal{T}([a,b])$ Vektorraum über $\mb{C}$, Dimension?
\end{Bem}
\begin{Bsp}
  jede stetige Funktion ist eine Regelfunktion
\end{Bsp}
\begin{Bsp}
  jede monotone Funktion auf $[a,b]$ ist eine Regelfunktion (sehe Seite 78)
\end{Bsp}
\begin{Bem}
  \begin{align*}
    f,g\in \mathcal{R}([a,b])\implies \lambda f_{\lambda\in\mb{C}}, f+g, \Abs{f}, f\cdot g, \max(f,g), \min(f,g)
  \end{align*}
  sind in $\mathcal{R}([a,b])$
\end{Bem}
\begin{Def}{gleichmässige Konvergenz}
  $(f_n)_{n\in\mb{N}}$ Folge von Funktionen auf $D\subset \mathcal{R}, f$ Funktion auf $D$.\\
  $(f_n)_{n\in\mb{N}}$ \underline{konvergiert gleichmässig} gegen $f$ $\Lra$ $\Limi{n} \underbrace{\Norm{f-f_n}}_{\sup_{x=D}\Abs{f(x)-f_n(x)}}=0$
\end{Def}
\begin{Bem}
  falls $(f_n)_{n\in\mb{N}}$ konvergiert gleichmässig $\implies$ limes ist eindeutig
\end{Bem}
\begin{Bem}
  $(f_n)_{n\in\mb{N}}$ konvergiert gleichmässig gegen $f$ $\implies$ $f_n(x)\to f(x)\ \forall x\in D$
  \[(\Abs{f(x)-f_n(x)}\leq \sup_{x\in D} \Abs{f(x)-f_n(x)}\to 0)\]
\end{Bem}
\begin{Bem}
  Die Umkehrung gilt NICHT $D=(0,1]$
  \begin{align*}
    f=0, f_n(x)=\begin{cases}1-nx&0\leq x\leq \frac{1}{n}\\0&\frac{1}{n}\leq x \leq 1\end{cases}\\
    \forall x\in D: f_n(x)\xrightarrow{n\to\infty}0\\
    \Norm{f-f_n}=\sup_{x\in D} \Abs{f(x)-f_n(x)} =1\\
    \Limi{n}\Norm{f-f_n}=1
  \end{align*}
\end{Bem}

\subsubsection{Zusammenfassung}
\begin{itemize}
  \item $\tau\left( [a,b] \right)$ = Vektorraum der Treppenfunktionen auf $[a,b]$
  \item $\int: \tau[a;b]\to\mb{C}$ lineare Abbildung
  \item Eigenschaften:
    \begin{itemize}
      \item lineare Abbildung
      \item Monotonie: $f\leq g$ $\implies$ $\int^b_af\cdot \md x\leq \int^b_ag\cdot \md x$
      \item Beschränktheit: $\Abs{\int^b_af\cdot \md x}\leq \int^a_b\Abs{f(x)}\md x\leq (b-a)\Norm{f} = \sup_{x\in[a;b]}f$
    \end{itemize}
  \item Regelfunktionen: $R\left( [a,b] \right)$ = Vektor nach der Regel $f\supset\tau\left( [a;b] \right)$
  \item gleichmässige Konvergenz $f_n\to f\xLeftrightarrow{\text{def}}\Norm{f_n-f}\to 0$
\end{itemize}
\subsubsection{Vorgehen}
\begin{enumerate}
    \item Jede Regelfunktion kann man gleichmässig durch Treppenfunktionen approximieren.
    \item Damit kann man das Integral von Regelfunktionen definieren.
    \item Regenregeln (insbesondere Hauptsatz)
    \item Riemannsche Summen
\end{enumerate}
\begin{Sat}{Approximationssatz}
  \[f\in R{a;b}\Lra \exists \text{Folge} \phi_n\in \tau[a;b]:\phi_n\to f\text{gleichmässig}\]  
  ist per Definition äquivalent mit
  \[\exists \text{Folge}\phi_n\in \tau[a;b]: \Norm{\phi_n-f}\to 0\]
  wobei
  \[\Norm{\phi_n-f}=\sup_{x\in[a;b]}\Abs{\phi_n(x)-f(x)}\]
  Dieser Grenzwert ist wiederum äquivalent mit
  \[\forall\varepsilon>0 \exists\phi \in \tau[a;b]: \Norm{f-\phi}\leq \varepsilon\]
  (eine $\varepsilon$-approximierende Treppenfunktion)
\end{Sat}
\begin{Bew}{$\Ra$}
  d.h. $f\in R\implies \exists \varepsilon$-approx. Treppen. Widerspruchsbeweis:
  \begin{align*}
    f\in R[a;b]\\
    \exists \varepsilon >0: f \text{besitzt keine} \varepsilon \text{-approx. Treppenfunktion}
  \end{align*}
  Wir konstruieren eine Intervallschachtelung $I_n=[a_n;b_n]$ s.d. $\forall_n f|_{I_n}$ besitzt keine $\varepsilon$-approx.Treppenfunktion
  \[I_1=[a;b]\]
  rekursiv: $M=\frac{b_n-a_n}{2} +a_n$ Mittelpunkt
  \begin{align*}
    I_{n+1}:=\begin{cases}
      [a_n;M]&\text{falls} f|_{[a_n;M]}\text{keine} \varepsilon \text{-approx. Treppenfunktion bestzt}\\ [M,b_n]& \text{andernfalls}        
    \end{cases}
  \end{align*}
  Sei $\xi\in I_n \forall n$
  \begin{align*}
    c_e&:=&\lim_{x\uparrow \xi} f(x)\\
    c_r&:=&\lim_{x\downarrow \xi} f(x)
  \end{align*}
  $\implies$
  \begin{align*}
    \exists \delta:&\Abs{f(x)-c_e}<\varepsilon:&x\in\left[\xi-\delta;\xi\right)\\
    &\Abs{f(x)-c_r}<\varepsilon:&x\in\left(\xi;\xi+\delta\right]\\
  \end{align*}
  Auf $[\xi-\delta;\xi+\delta]$ definieren wir eine Treppenfunktion:
  \begin{align*}
    \phi(x):=\begin{cases}
      c_e&\xi-\delta\leq x< \xi\\
      f(\xi)&x=\xi\\
      c_r&\xi+\delta\geq x>\xi
    \end{cases}
  \end{align*}
  Fall 1 $\implies$ $\phi$ ist eine $\varepsilon$-approx. Treppenfunktion auf $[\xi-\delta],[\delta+\delta]$. Fall 2 $\implies$ $\phi$ ist eine $\varepsilon$-approx. Treppenfunktion auf $[\xi-\delta],[\delta+\delta]$, alle $I_n \subset[\xi+\delta;\xi+\delta]$\\
  $\blitza$
\end{Bew}
\begin{Bew}{$\La$}
  $f$ Regelfunktion $\La$ $f$ besitzt $\varepsilon$-approx. Treppenfunktion $\forall \varepsilon>0$. Sei $x_0\in\left[a;b\right)$. Zu zeigen: $\exists \lim_{x\downarrow x_0} f(x)$.
  \begin{align*}
    \forall \varepsilon>0\ \exists\phi\in \tau[a;b]:\Norm{f-\phi}<\frac{\varepsilon}{2}\\
  \end{align*}
  Sei $\beta>x_0:\phi$ konstant auf $(x_0,\beta)$
  \begin{align*}
    \forall x,x'\in(x_0;\beta)\\
    \Abs{f(x)-f(x')}&\leq& \Abs{f(x)-\phi(x)}+\Abs{\phi(x)^{(=\phi(x')}-f(x')}\\
    &\leq&\Norm{f-\phi}+\Norm{\phi-f}<\varepsilon
  \end{align*}
  $\forall \varepsilon>0$ $\exists\beta:$ Cauchyeigenschaft gilt auf $(x_0;R)$ $\implies$ $\exists \lim_{x\uparrow x_0}f(x)$. Ähnlich: $\exists\lim_{x\uparrow x_0}f(x)$ $\forall x_0\in\left(a;b\right]$.
\end{Bew}
\begin{Kor}
  \[f\in R[a;b]\ \Lra\ \exists \text{Folge}\Psi_b\in\tau[a;b]:\sum^\infty_{k=1}\phi_k=f\]
  konvergiert konstant
\end{Kor}
\begin{Kor}
  $f$ Regelfunktion auf $I$ $\implies$ $f$ fast überall stetig. d.h. $\exists A\subset I$ s.d.
  \begin{itemize}
    \item $f|_{I\setminus A}$ stetig
    \item $A$ höchstens abzählbar $x\in[a;b]$
  \end{itemize}
\end{Kor}
\begin{Bew}
  %\subparagraph{Fall 1: $I$ kompakt}
  \begin{align*}
    \Psi_k\in\tau [I]\\
    f=\sum\phi_k \text{normal}
  \end{align*}
  Ist $\phi_k$ stetig in $x\forall k$ $\implies$ $f$ stetig in $x$.\\
  Ist $x$ Unstetigkeitsstelle von $f$, $\exists k$: $\phi_k$ unstetig in $x$, höchstens abzählbare viele $k$.
  \begin{itemize}
    \item Eine Treppenfunktion hat endlich viele Unstetigkeitsstellen
  \end{itemize}  
  \{ Unstetigkeitsstellen von $f$\} $\subset$ (höchstens abzählbare Vereinigung von endlichen Mengen) $\implies$ höchstens abzählbar
  %\subparagraph{Fall 2: $I$ nicht kompakt}
  \[I=\overbrace{U_\alpha}^{\text{höchstens abzählbar}}\overbrace{I_\alpha}^{\text{kompakt}}\]
\end{Bew}
\begin{Sat}
  \[f\in R\left( [a:b] \right)\implies f \text{beschränkt auf} [a;b]\]
\end{Sat}
\begin{Bew}
  \begin{align*}
    \varepsilon =1\\
    \exists \overbrace{\phi}^{\text{\underline{beschränkt}}}\in\tau\left( [a;b] \right):\Norm{f-\phi}\leq 1\\
    \implies \Norm{f}=\Norm{f-\phi + \phi}\leq \Norm{f-\phi}+\Norm{\phi}=\leq 1+\Norm{\phi}
  \end{align*}
\end{Bew}
\begin{Def}{Integration von Regelfunktionen}
  \ldots auch bekannt als ``Regelintegral''\\
  Sei $f\in R[a;b]$
  \[\int_a^bf(x)fx:\Limi{n}\int^b_a\phi_n(x)\md x\]
  wobei $\phi_n$ eine approximierene Folge von Treppenfunktionen ist (d.h. $\Norm{\phi_n-f}\to 0$)
\end{Def}
\subparagraph{zu zeigen:}
\begin{enumerate}
  \item Die Folge $I_n:=\int^b_a\phi_n(x)\md x$ konvergiert $\forall \Norm{\phi_n-f}\to 0$
  \item Der Grenzwert ist von der Wahl der approximierenden Folge unabhängig
\end{enumerate}
\begin{Bew}{von 1}
  \[\Abs{I_n-I_m}=^{\text{Linearität}}\Abs{\int^b_a\left( \phi_n(x)-\phi_m(x) \right)\md x}\leq^{\text{beschränkt}} (b-a)\Norm{\phi_n-\phi_m}\]
  \[\Norm{\phi_n-f}\to 0 \xRightarrow{\text{Dreiecksungleichung}} \forall\varepsilon>0\ \exists N: \Norm{\phi_n-\phi_m}<\varepsilon\ \forall n,m> N\]
  $\implies$ $I_n$ Cauchyfolge $\implies$ $I_n$ konvergiert
\end{Bew}
\begin{Bew}{von 2}
  Seien $\phi_n, \psi_n\in \tau[a;b]$
  \begin{align*}
    \Norm{\phi_n-f}\to0\\
    \Norm{\psi_n-f}\to0
  \end{align*}
  \begin{align*}
    \{X_n\}=\psi_1,\phi_1,\psi_2,\phi_2,\psi_3,\phi_3,\cdots\\
    X_n:=\begin{cases}
      \phi_{\frac{n}{2}}& n \text{gerade}\\
      \psi_{\frac{n+1}{2}}& n \text{ungerade}
    \end{cases}
  \end{align*}
  $\implies$ $I_n(\phi)$ und $I_n(\psi)$ Teilfolgen von $I_n(X)$
  \begin{align*}
    \implies \Norm{X_n-f}\to 0\\
    I_n(x)=\int x_n\\ I_n(\phi)=\int \phi_n\\ I_n(\psi)=\int \psi_n\\
  \end{align*}
  $\implies$
  \begin{align*}
    \lim I_n(\phi)=\lim I_n(X)=\lim I_n(\psi)
  \end{align*}
\end{Bew}
\begin{Bsp}{Dirichlet}
  eine Funktion, die keine Regelfunktion ist.
  \begin{align*}
    f:[0;1]\to\mb{R}\\
    f(x)=\begin{cases}
      1 & x\in \mb{Q}\\
      0 & x\in \mb{R}\setminus\mb{Q}
    \end{cases}
  \end{align*}
  $f$ unstetig $\forall x$ intuitiv: $\int^1_0 f(x)fx =0$
\end{Bsp}
\begin{Bsp}{Riemann}
  sog. modifizierte Dirichlet-Funktion
  \begin{align*}
    g:[0;1]\to\mb{R}\\
    g(x)=\begin{cases}
      \frac{1}{q}  & x=\frac{p}{q}, p,q \text{teilerfremd}, q>0\\
      0 & x\in \mb{R}\setminus \mb{Q}
    \end{cases}
  \end{align*}
  $g\in R[0;1]$ und $int^b_ag(x)\md x=0$
\end{Bsp}
\subsubsection{Eigenschaften}
\begin{Sat}
  \[\forall f, g\in R [a;b] \forall \alpha,\beta\in \mb{C} \text{gelten}\]
  \begin{description}
    \item[Linearität] \[\int^b_a(\alpha f+ \beta g)\md x = \alpha \int^b_af\cdot \md x+\beta\int^b_a g\cdot \md x\]
    \item[Beschränktheit] \[\Abs{\int^b_a f(x)\cdot \md x}\leq \int^b_a\Abs{f(x)}\md x\leq (b-a)\Norm{f}\]
    \item[Monotonie] \[f\leq g \implies \int^b_af(x)\md x\leq \int^b_ag(x)\md x\]
  \end{description}
  ($f,g$ reellwertig $f(x)\leq g(x)\forall x$)
\end{Sat}
\begin{Sat}{Additivität}
  Sei $f\in R[a;b]$ und sei $c\in (a;b)$
  \[\int^b_af(x)\md x=\int^c_a f(x)\md x+\int^b_cf(x)\md x\]
\end{Sat}

\begin{Bew}
  $f=\phi$ Treppenfunktion trivial
  \[f=\lim\phi_n\ \text{gleichmässig}\]
  \begin{align*}
    \phi_n\in \tau[a;c]
    \phi_n^l &:=& \phi_n|_{[a;b]}&\in\tau[a;b]\\
    \phi_n^r &:=& \phi_n|_{[b;c]}&\in\tau[b;c]\\
  \end{align*}
  \begin{align*}
    \int^c_a\phi_n(x)\md x = \int^b_a\phi_n^l(x)\md x+\int^c_b \int^r_n(x)\md x\\
    \Norm{\phi_n-f}\to0\\
    \Norm{\phi_n^l-f}_{[a;b]}\leq \Norm{\phi_n-f}\geq \Norm{\phi_b^+f}_{[b;c]}
  \end{align*}
  \begin{align*}
    \int^c_a\phi_n(x)\md x &=& \int^b_a\phi_n^l(x)\md x&+&\int^c_b \int^r_n(x)\md x\\
    =\int^c_a f \cdot \md x & &=\int^b_a f(x) \cdot \md x & & =\int^c_b f(x) \md x
  \end{align*}
  $\implies$
  \begin{align*}
    \phi_n^l&\to f|_{[a;b]}\\
    \phi_n^r&\to f|_{[b;c]}\\
  \end{align*}
\end{Bew}
\begin{Def}
  $f\in \mathcal{R}[a;b]$, $b>a$
  \[\int^a_bf(x)\md x := \int^b_af(x)\md x\]
  \[\int^a_af(x)\md x := 0 \]
\end{Def}
\begin{Sat}
  $f\in \mathcal{R}I(): \forall a,b,c \in I$
  \[\int^c_af(x)\md x = \int^b_af(x)\md x + \int^c_bf(x)\md x\]
\end{Sat}
\begin{Bem}
  \begin{description}
    \item[Linearität]
    \item[Beschränktheit]: \[\Abs{\int^b_af(x)\md x}\leq \Abs{\int^b_a\Abs{f(x)}\md x}\leq \Abs{b-a}\Norm{f}\]
    \item[Monotonie]
  \end{description}
  \begin{align*}
    f\leq g; b>a\\
    \int^b_a f(x)\md x \geq \int^b_ag(x)\md x
  \end{align*}
\end{Bem}
\begin{Bem}
  $f$ stetig ($[a;b]$) $\implies$ $\Norm{f} = \max\Abs{f}$\\
  reellwertig $\xRightarrow{\text{ZWS}}$ $f$ nimmt alle Werte zwischen $0$ und $\max\Abs{f}$ % zwischen <> und - nachprüfen
  \begin{align*}
    \exists\xi\in [a;b]:\\
    \int^b_af(x)\md x=(b-a)f(\xi)
  \end{align*}
\end{Bem}
\begin{Sat}{Mittelwertsatz}
  Sei $f:[a;b]\to \underline{\mb{R}}$ \underline{stetig}. Sei $p:[a;b]\to\mb{R}\in\mathcal{R}$ mit $p\geq 0$. Dann $\exists \xi\in [a;b]$ s.d.
  \[\int^b_a f(x)p(x)\md x=f(\xi)\int^b_a p(x)\md x\]
  Falls $\int p \neq 0$
  \begin{align*}
    \frac{\int f(x)p(x)\md x}{\int p(x)\md x}=f(\xi)=\int^b_a f(x)\tilde{p}(x)\md x\\
    \tilde{p}(x)=\frac{p(x)}{\int^b_ap(x)\md x}\\
    \implies \int^b_a \tilde{p}(x) \md x=1
  \end{align*}
\end{Sat}
\begin{Bew}
  $f$ besitzt ein Maximum $M$ und ein Minimum $m$
  \begin{align*}
    m\leq f(x) \leq M\ \forall x\in [a;b]\\
    m p(x)\leq f(x)p(x)\leq M p(x)\\
  \end{align*}
  $\xRightarrow{\text{Monotonie}}$
  \begin{align*}
    \int^b_am p(x)\md x &\leq& int^b_a f(x)p(x)\md x &\leq& \int^b_a M p(x)\md x\\
    = m\int^b_ap(x)\md x & & & &=M \int^b_a p(x)\md x
  \end{align*}
  $\implies \exists \mu\in [m;M]$:
  \begin{align*}
    \int^b_af(x)p(x)\md x = \mu \int^b_a p(x) \md x    
  \end{align*}
  ZWS $\implies$ $\exists \xi \in [a;b]$:
  \[\mu=f(\xi)\]
\end{Bew}
\begin{Sat}
  Sei $f:[a;b]\to \mb{R}\in\mathcal{R}$ mit $f\geq 0$ und $\int^b_af(x)\md x=0$. Dann ist $f(x_0)=0$ an jeder Stetigkeitsstelle $x_0$. Ferner gilt: $f=0$ fast überall.
\end{Sat}
\begin{Bew}{(Widerspruchsbeweis)}
  Sei $x_0$ eine Stetigkeitsstelle mit $f(x_0)>0$. $f$ stetig in $x_0$ $\implies$ $\exists x_0 \in [a:b]\subset[a:b]$ s.d.
  \[f(x)>\frac{1}{2}f(x_0)\ \forall x\in [\alpha:\beta]\]
  Sei
  \[\phi(x):=\begin{cases}
    \frac{1}{2}f(x_0)&x\in [\alpha;\beta]\\
    0 & x\not\in [\alpha;\beta]
  \end{cases}\]
  Treppenfunktion, deshalb Regelfunktion
  \[\implies f\geq \phi \implies \underbrace{\int^\beta_\alpha f(x)\md x}_{=0} \geq \int^\beta_\alpha\phi(x)\md x=\frac{\beta-\alpha}{2}f(x_0)>0\]
  $\blitza$
\end{Bew}
\begin{Sat}
  $f\in\mathcal{R}$ $\implies$ $f$ besitzt höchstens abzählbar viele Unstetigkeitsstellen $\implies$ $f=0$ fast überall
\end{Sat}
\begin{Kor}
  $f:[a;b]\to\mb{R}$ stetig, $f\geq 0$, $\int_a^bf(x)\md x=0$ $\implies$
  \begin{align*}
    f(x)=0\ \forall x\in [a;b]    
  \end{align*}
\end{Kor}
\subsection{Fundamentalsatz der Analysis}
\begin{Sat}
  Sei $f:I\to\mb{C}\in\mathcal{R}$ und sei $a\in I$. Für jedes $x\in I$ definiert man
  \[F(x):=\int_a^x f(t)\md t\ F:I\to\mb{C}\]
  Dann ist $F$ eine Stammfunktion zu $f$ (d.h. $F$ ist stetig und fast überall differenzierbar (und $F'=f$ fast überall)) mit
  \begin{align*}
    F_+'(x_0)=f_+(x_0)\\
    F_-'(x_0)=f_-(x_0)
  \end{align*}
  $\forall x_0 \in I$
\end{Sat}
\begin{Bew}
  $\forall x_1,x_2\in I$ gilt
  \begin{align*}
    F(x_2)-F(x_1)=\int_a^{x_2}f(t)\md t-int^{x_1}_a f(t)\md t =\\
    =\int^{x_2}_a+\int^a_{x_1}= \int^{x_2}_{x_1}f(t)\md t    
  \end{align*}
  Sei $\tau\subset I$ Teilintervall. $\forall x_1, x_2\in \tau$
  \begin{align*}
    \Abs{f(x_2)-F(x_1)}=\Abs{\int^{x_2}_{x_1}f(t)\md t}\leq^{\text{Bijektivität}} \Abs{x_2-x_1}\Norm{f}_\tau
  \end{align*}
  $\implies$ $F|_\tau$ Lipschitz-stetig $\implies$ $F|_\tau$ stetig $\forall \tau \implies$ \underline{$F$ stetig auf $I$}.\\
  Wir berechnen $F_+'(x_0)$. $f\in\mathcal{R}$ $\implies \exists f_+(x_0)$. $\forall \varepsilon>0 \exists \delta >0$
  \begin{align*}
    \Abs{f(x)-f_+(x_0)}<\varepsilon\ \forall x\in(x_0, x_0+\delta)\\
  \end{align*}
  Für $x\in (x_0, x_0+\delta)$
  \begin{align*}
    \Abs{ \frac{F(x)-F(x_0)}{x-x_0} -f_+(x_0) } = \\ \Abs{ \frac{1}{x-x_0} \int_{x_0}^x f(t)\md t-\frac{f_+(x_0)}{x-x_0} \int_{x_0}^x <Fehlt da nicht was?> \md t} = \\
    \Abs{\frac{1}{x-x_0}}\int_{x_0}^x\left( f(t)-f_+(x_0) \right)\md t \leq\\
    \frac{1}{\Abs{x-x_0}}\Abs{x-x_0}\Norm{f(x)-f_+(x_0)}_{x_0;x} \leq \varepsilon
  \end{align*}
\end{Bew}
\begin{Kor}
  Sei $f:I\to\mb{C}\mathcal{R}$ und sei $\Phi$ eine Stammfunktion zu $f$. Dann $\forall a,b\in I$
  \begin{align*}
    \int^b_af(x)\md x&=&\Phi(b)-\Phi(a)\\
    &=:&\Phi|^b_a
  \end{align*}
\end{Kor}
\begin{Bew}
  $\Phi$ und $F$ sind Stammfunktionen zu $f$, insbesondere $\Phi'=F'$ fast überall. Eindeutigkeitssatz $\implies \exists c$ konstant s.d.
  \[\Phi(x)=F(x)+c\ \forall x\in I\]
  \begin{align*}
    \int^b_af(x)\md x=F(b)=F(b)-\underbrace{F(a)}_{=0}=\\
    =\left( \Phi(b)-c \right) - \left( \Phi(a)-c \right) = \Phi(b)-\Phi(a)
  \end{align*}
\end{Bew}
\begin{Kor}
  Jede Regelfunktion beseitzt eine Stammfunktion  
\end{Kor}
\begin{Def}
  Eine Funktion heisst fast überall stetig differenzierbar, wenn sie die Stammfunktion zu einer Regelfunktion ist. (Wo sie nicht stetig differenzierbar ist, besitzt sie linke und Rechte Grenzwerte)
\end{Def}
\begin{Bsp}
  \[f(x)=\begin{cases}
    0& x=0\\
    x^2\sin\frac{1}{x}&x\neq 0
  \end{cases}\]
  $f$ ist in $\mb{R}\setminus\{0\}$ differenzierbar. $f'$ besitzt linke und rechte Grenzwerte, in 0 nicht. Also keine Regelfunktion.
\end{Bsp}
\begin{Bem}
  Mit dem Lebesgne-Integral kann man solche Funktionen aus einem Integral erhalten.
\end{Bem}
\begin{Eig}{Charakterisierung}
  $f$ fast überall stetig differenzierbar auf $I$ $\implies$ $\exists A\subset I$, $A$ höchstens abzählbar s.d.
  \begin{enumerate}
    \item $f$ ist auf $I\setminus A$ differenzierbar
    \item $f'$ ist auf $I\setminus A$ stetig
    \item $\forall x\in A$ existieren $f_+'(x)$ und $f_-'(x)$
  \end{enumerate}
\end{Eig}
\begin{Def}{unbestimmtes Integral}
  Das unbestimmte Integral der Regelfunktion $f$ ist die Gesamtheit aller Stammfunktionen zu $f$.
\end{Def}
\begin{Not}{unbestimmtes Integral}
  \[\int f(x)\md x\]
  In Tabellen wird oft
  \[\int x\md x = \frac{x^2}{2}\]
  geschrieben
\end{Not}
\begin{Bsp}
  \[\int x\md x = \frac{x^2}{2} + C\]
\end{Bsp}
\begin{Eig}
  \begin{align*}
    \int x^a\md x &=&  \frac{x^{a+1}}{a+1}\ a\in \mb{C}\setminus \{-1\}\\
    \int \frac{1}{x}\md x &=&  \ln\Abs{x}\\
    \int e^{cx} \md x &=&  \frac{1}{c}e^{cx},\ c\neq 0\\
    \int \sin x \cdot \md x &=& -\cos x\\
    \int \cos x \cdot \md x &=& \sin x
  \end{align*}
\end{Eig}
\begin{Sat}
  Seien $f_1$ und $f_2$ Regelfunktionen auf $I$
  \begin{align*}
    f_1=f_2 \text{f.ü.} \implies \int f_1 \md x=\int f_2 \md x
  \end{align*}
  Insbesondere $\forall a,b \in I$
  \begin{align*}
    \int^b_af(x)\md x = \int^b_a f_2(x)\md x
  \end{align*}
\end{Sat}
\begin{Bew}
  Sei $F_1$ / $F_2$ Stammfunktion zu $f_1$ / $f_2$
  \begin{align*}
    \implies F_1'=F_2'\ \text{f.ü.}\\
    \implies F_1=F_2+C
  \end{align*}
\end{Bew}
\begin{Bem}{Anwendung}
  \begin{align*}
    f(x)=\begin{cases}
      \frac{1}{q} & x=\frac{p}{q}, p,q \text{teilerfremd}\\
      0 & x\neq \mb{Q}
    \end{cases}\\
    \int^b_a f(x)\md x =0
  \end{align*}
\end{Bem}


\begin{Def}
  Sit $f$ eine fast überall differenzierbare Funktion, so bezeichnet $f'$ irgendeine Regelfunktion, die fast überall gleich zur Ableitung von $f$ ist.
\end{Def}
\begin{Sat}{Hauptsatz}
  Sei $f$ eine fast überall stetig differenzierbare Funktion auf $I$. Dann 
  \begin{align*}
    \int f'(x)\md x=f\\
    \int_a^bf'(x)=f(b)-f(a)\ a,b\in I
  \end{align*}
\end{Sat}
\begin{Not}{Leibnitz-Notation}
  \begin{align*}
    f'=\Diff{f}{x}\\
    \int\Diff{f}{x}\md x=f\\
    \int df=f\\
    \int^b_adf=\Delta F:=f(b)-f(a)
  \end{align*}
\end{Not}
\subsection{Integrationstechniken}
\begin{Eig}{Integrationstechniken}
  \begin{enumerate}
    \item Linearität
    \item Partielle Integration
    \item Substutionsregel
  \end{enumerate}
\end{Eig}
\subsubsection{Partielle Integration}
\begin{Sat}
  Seien $U$ und $V$ fast überall stetig differenzierbar Funktionen auf $I$, so ist auch $UV$ fast überall stetig differenzierbar und
  \begin{align*}
    \int uv'\md x=uv-\int u'v\md x\\
    \int^b_auv'\md x=(uv)|^b_a-\int^b_au'v\md x
  \end{align*}
\end{Sat}
\begin{Bew}
  $u,v$ stetig und $u,v\in\mathcal{R}$ $\implies$ $u'v+uv'\in\mathcal{R}$. Fast überall: $u'v+uv'=(uv)'$ Kettenregel.
  \begin{align*}
    \int(u'v+uv')\md x=\int(uv)'\md x=uv
  \end{align*}
\end{Bew}
\begin{Bsp}
  \begin{align*}
    \int\ln x\md x=\int 1 \cdot \ln x \md x=\int\Diff{x}{x}\ln x\md x=\\
    =x\ln x-\int x\Diff{\ln x}{x} \md x=x\ln x-\int x\frac{1}{x}\md x=x\ln x-x
  \end{align*}
\end{Bsp}
\begin{Bsp}
  \begin{align*}
    \int \cos^2x\md x=\int \cos x \cdot \cos x\md x = \int(\Diff{}{x}\sin x)\cos x \md x=\\
    =\sin x \cos x -\int \sin x \Diff{}{x}\cos x \md x = \sin x \cos x +\int \sin^2 x\\
    \int(\cos^2x-\sin^2x)\md x = \sin x \cos x\\
    \int(\cos^2x+\sin^2x)\md x = x\\
    \int \cos^2x\md x=\frac{\sin x\cos x+ x}{2}
  \end{align*}
\end{Bsp}
\begin{Bsp}
  \begin{align*}
    \int\sqrt{1+x^2}=\int\Diff{x}{x}\sqrt{1+x^2}\md x=x\sqrt{1+x^2}-\int x\frac{2x}{2\sqrt{1+x^2}}\md x=\\
    =x\sqrt{1+x^2} \int\frac{1+x^2}{\sqrt{1+x^2}}\md x+\int\frac{1}{\sqrt{1+x^2}}=\\
    =x\sqrt{1+x^2}-\int\sqrt{1+x^2}\md x+\arcsinh x\\
    \int\sqrt{1+x^2}\md x=\frac{x\sqrt{1+x^2}+\arcsinh x}{2}
  \end{align*}
\end{Bsp}
\subsubsection{Substitutionsregel}
\begin{Sat}{Substitutionsregel}
  Sei $f\in\mathcal{R}$ auf $I$, $F$ eine Stammfunktion zu $f$, $t:[a;b]\to I$ stetig differenzierbar und streng monoton. Dann ist $F\circ t$ eine Stammfunktion zu
  \[(f\circ t)t'\ \text{auf}\ [a;b]\]
  und
  \[\int^b_af(t(x))t'(x)\md x=\int^{t(b)}_{t(a)}f(t)\md t\]
  \[(I=[t(a);t(b)]\ \text{oder}\ [t(b);t(a))\]
\end{Sat}
\begin{Not}
  \[f\Diff{t}{x}\md x=\int f\md t\]
\end{Not}
\begin{Bew}
  Kettenregel:
  \begin{align*}
    \Diff{}{x}(F\circ t)=(F'\circ t)t'\stackrel{\text{f.ü.}}{=}(f\circ t)t'\\
    \int^b_af(t(x))t'(x)\md x=int^b_a\Diff{}{x}(F\circ t)\md x=F\circ t|^b_a=F(t(b))-F(t(a))\\
    \int^{t(b)}_{t(a)}f(t)\md t=F|^{t(b)}_{t(a)}=F(t(b))-F(t(a))
  \end{align*}
\end{Bew}
\begin{Bsp}
  \begin{align*}
    \int^b_af(x+c)\md x\stackrel{t(x)=x+c}{=}\int^b_af(x+c)t'\md x=\\
    =\int^{b+c}_{a+c}f(t)\md t
  \end{align*}
\end{Bsp}
\begin{Bsp}
  \begin{align*}
    \int^b_af(cx)\md x\stackrel{t(x)=cx}{=}\frac{1}{c}\int^b_af(cx)t'\md x=\frac{1}{c}\int^{cb}{ca}f(t)\md t
  \end{align*}
  $c=-1$
  \begin{align*}
    \int^b_af(-x)\md x=-\int^{-b}_{-a}f(x)\md x=\int^{-a}_{-b}f(x)\md x
  \end{align*}
\end{Bsp}
\begin{Kor}
  \begin{align*}
    f(-x)=-f(x)\\
    \int^a_{-a}f(x)=0
  \end{align*}
\end{Kor}
\begin{Bew}
  \begin{align*}
    \int^a_{-a}f(-x)\md x=-\int^a_{-a}f(x)\md x=\int^a_{-a}f(x)\md x
  \end{align*}
\end{Bew}
\begin{Bsp}
  \begin{align*}
    \int\frac{t'(x)}{t(x)}\md x\stackrel{f=\frac{1}{t}}{=}\int f(t)\md t=\\
    =\int\frac{1}{t}\md t=\ln\Abs{t}
  \end{align*}
\end{Bsp}
\subsubsection{Rationale Funktionen}
$\ra$ Pratialbruchzerlegung
\[\int\frac{\md x}{x+a}=\ln\Abs{x+a}\]
\[\int\frac{Bx+C}{x^2+2bx+c}\md x=\cdots\]
Wobei $x^2+2bx+c$ keine reelen Lösungen ergeben darf.
\begin{Sat}
  Eine rationale Funktion kann man mittels rationaler Funktionen, des Logarithmus sowie des Arcustangens integrieren.
\end{Sat}
\subsection{Reihenintegration}
\begin{Sat}
  Sei $f_n$ eine Folge Regelfunktionen auf $[a;b]$. Konvergiert die Reihe $\sum f_n$ normal, so ist 
  \[f:\sum^\infty_{n=1}f_n\]
  eine Regelfunktion und
  \[\int^b_af(x)\md x=\sum^\infty_{n=1}\int^b_af_n(x)\md x\]
  \[(\int\sum = \sum\int)\]
  Insbesondere gilt der Satz für Potenzreichen in ihren Konvergenzintervallen.
\end{Sat}
\begin{Bew}
  $\forall\varepsilon >0 \exists N$:
  \begin{align*}
    \sum^\infty_{n=N}\Norm{f_n}<\frac{\varepsilon}{2}
  \end{align*}
  $\forall p\geq N$
  \begin{align*}
    \Norm{f-\sum^p_{n=1}f_n}<\frac{\varepsilon}{2}
  \end{align*}
  $f_n\in\mathcal{R}$ $\implies$ $\sum^p_{n=1}f_n\in\mathcal{R}$ $\implies$ $\exists$ Treppenfunktion $\phi$ mit
 \begin{align*}
   \Norm{\sum^p_{n=1}f_n-\phi}<\frac{\varepsilon}{2}
 \end{align*}
 $\implies$
 \begin{align*}
   \Norm{f-\phi}\leq \Norm{f-\sum^p_{n=1}f_n}+\Norm{\sum^p_{n=1}f_n-\phi}<\varepsilon
 \end{align*}
 $\implies$ $f\in \mathcal{R}$
 \begin{align*}
   \Abs{\int_a^bf(x)\md x-\sum^p_{n=1}\int^b_af_n(x)\md x}\leq\\
   \leq \int^b_a\Abs{f(x)-\sum^p_{n=1}f_n(x)}\md x\leq \\
   \leq \Abs{b-a}\Norm{f-\sum^p_{n=1}f_n}<\\
   < \Abs{b-a}\frac{\varepsilon}{2}
 \end{align*}
\end{Bew}
\begin{Bsp}
  \begin{align*}
    \arctan x= \int^x_0\frac{1}{1+t^2}\md t=\int^x_0\sum^\infty_{n=0}(-1)^nt^{2n}\md t\stackrel{\Abs{x}<1}{=}\\
    \sum^\infty_{n=0}(-1)^n\frac{x^{2n+1}}{2n+1}
  \end{align*}
\end{Bsp}

\subsection{Reimannsche Summen}
\begin{itemize}
  \item alte Definition des Regelintegrals (äquivalent)
  \item Approximationstechnik
  \item Man kann Resultate über Summen erweitern (z.B. Höldersche Ungleichung, Cauchy-Schwarzsche Ungleichung)
\end{itemize}
\begin{Def}{Zerlegung}
  $[a;b]$ kompates Intervall\\
  Eine Zerlegung von $[a;b]$ ist die Wahl $x_0,x_1,x_2,\cdots,x_n$ s.d.
  \[a=x_0<x_1<x_2<\cdots<x_{n-1}x_n=b\]
\end{Def}
\begin{Not}
  $Z:=\{x_0,x_1,\cdots,x_n\}$
\end{Not}
\begin{Def}{Feinheit der Zerlegung}
  \begin{align*}
    \Delta x_k:=x_k -x_{k-1}
  \end{align*}
  Die \underline{Feinheit} der Zerlegung ist $\max \{\Delta x_1, \Delta x_2, \cdots , \Delta x_n\}$
\end{Def}
\begin{Def}
  Die Riemannsche Summe von $f$ bezüglich der Zerlegung $Z$ und der Wahl von Stützstellen $\xi=:\left( \xi_1,\cdots,x_n \right)$
  \[\xi_k\in \left[ x_{k-1};x_k \right]\]
  ist die Summe
  \[S(f;Z;\xi):=\sum^n_{k=1}f\left( \xi_k \right)\Delta x_k\]
\end{Def}
\begin{Sat}
  Sei $f:\left[ a;b \right]\to \mb{C}$ eine Regelfunktion. Dann gilt folgendes:
  \[\forall \varepsilon>0\ \exists \delta>0\]
  sd. für jede Zerlegung $Z$ der Feinheit $\leq \delta$ und für jede Wahl Stützstellen $\xi$ gilt
  \[\Abs{S\left( f;Z;\xi \right)-\int^b_af(x)\md x}<\varepsilon\]
\end{Sat}
\begin{Bew}
  (Idee)
  \begin{enumerate}
    \item Satz gilt, falls $f$ eine Treppenfunktion ist. Beweis durch Indunktion nach der Anzahl Sprungstellen
    \item $\exists \phi$ Treppenfunktion s.d.
      \[\Norm{f-\phi}<\frac{\varepsilon}{3(b-a)}\]
      1) $\implies$ $\exists Z,\xi$
      \[\Abs{S\left(\phi;Z;\xi\right) -\int^b_a\phi(x)\md x}<\frac{\varepsilon}{3}\]
      3-Ecks Ungleichung
  \end{enumerate}
\end{Bew}
\begin{Kor}
  Sei $f:[a;b]\to\mb{C}\in\mathcal{R}$. Sei $Z_1,Z_2,Z_3,\cdots$ Folge Zerlegungen von $[a;b]$ mit Feinheit $(Z_n)\to 0$. Für jede Wahl Stützstellen $\xi_m$ aus $Z_n$
  \[\Limi{n}S\left( f;Z_n;\xi_m \right)=\int^b_af(x)\md x\]
\end{Kor}
\begin{Def}{$p$-Norm}
  Sei $f[a;b]\to\mb{C}\in\mathcal{R}$. Die $p$-Norm von $f$ (mit $p\geq 1$)
  \[\Norm{f}_p:=\sqrt[p]{\int^b_a\Abs{f(x)}^p\md x}\]
\end{Def}
\begin{Sat}
  Seien $f,g:[a;b]\to\mb{C}\in\mathcal{R}$. Seien $p,q\geq 1$ mit $\frac{1}{p}+\frac{1}{q}=1$. Dann haben wir
  \[\int_a^b\Abs{f(x)g(x)}\md x\leq \Norm{f}_p\Norm{g}_q\]
  Höldersche Ungleichung\\
  Spezialfall: $p=q=2$ Cauchy-Schwarzsche Ungleichung
\end{Sat}
\begin{Bew}
  (Idee)
  \begin{enumerate}
    \item Man approximiert die 3 Integrale durch Riemannsche Summen
    \item Man benützt die Höldersche Ungleichung für Summen
    \item Man nimmt die Grenzwerte
  \end{enumerate}
\end{Bew}
\subsection{Das uneigentliche Integral}
\begin{Sat}
  Seien $a,b\in\bar{\mb{R}}$
  \[-\infty\leq a<b\leq +\infty\]
  Sei $I$ ein Intervall mit Randwerten $a$ und $b$ (z.B. $I=[a;b]$, $I=[a;b)$). Sei $f$ eine Regelfunktion auf $I$. Wir wollen $\int^b_af(x)\md x$ definieren, wenn möglich.
  \subparagraph{Fall 0}
  \begin{align*}
    a,b\in\mb{R},\ I=[a;b]\\
    \int^b_af(x)\md x \text{Regelintegral}
  \end{align*}
  \subparagraph{Fall 1}
  \begin{align*}
    b\in\bar{\mb{R}},\ I=[a;b)\\
    \int^b_af(x)\md x=\lim_{\beta\uparrow b}\int^\beta_af(x)\md x
  \end{align*}
  Falls der Grenzwert existiert.
  \subparagraph{Fall 2}
  \begin{align*}
    a\in\bar{\mb{R}}, b\in \mb{R}, b>a, I=(a;b]\\
    \int^b_af(x)\md x=\lim_{\alpha\downarrow a}\int^b_\alpha f(x)\md x
  \end{align*}
  Falls der Grenzwert existiert.
  \subparagraph{Fall 3}
  \begin{align*}
    a,b\in\bar{\mb{R}}, a<b, I=(a;b)\\
    \int^b_aF(x)\md x:=\overbrace{\int^c_af(x)\md x}^\text{Fall 2} + \overbrace{\int^b_cf(x)\md x}^\text{Fall 1}\\
  \end{align*}
  Sei $c\in(a;b)$ falls beide Integrale auf der rechten Seite existieren!
\end{Sat}
\begin{Def}{Wert eines Integrals}
  Existiert das uneigentliche Integral von $f$, so heisst $\int^b_af(x)\md x$ \underline{konvergent} so heisst der Grenzwert \underline{Wert} des Integrals
\end{Def}
\begin{Def}{absolut konvergentes Integral}
  Konvergiert das Integral von $\Abs{f}$, so heisst das Integrals \underline{absolut konvergent}
\end{Def}
\begin{Bsp}
  $I=(0;+\infty)$
  \begin{align*}
    F_s(x):=\int\frac{1}{x^s}\md x=\begin{cases}
      \ln x& s=1\\
      \frac{x^{1-s}}{1-s}&s\neq 1
    \end{cases}\\
    F_s(x)\xrightarrow{x\to\infty}0\ \Lra\ s>1, \text{divergiert sonst}\\
    F_s(x)\xrightarrow{x\to0}0\ \Lra\ s<1, \text{divergiert sonst}\\
  \end{align*}
  \[\int_a^{+\infty}\frac{1}{x^s}\md x\]
  existiert genau dann, wenn $a>0$ und $s>1$ und hat den Wert $\frac{a^{1-s}}{s-1}$
  \[\int_0^a\frac{1}{x^s}\md x\]
  existiert genau dann, wenn $s<1$ und hat den Wert $\frac{a^{1-s}}{1-s}$
\end{Bsp}
\begin{Bsp}
  $e^{-x}\in R(\mb{R})$
  \begin{align*}
    \int_0^{+\infty}e^{-x}\md x=\lim_{a\to+\infty}\int_0^ae^{-x}\md x=\\
    =\lim_{a\to+\infty}\left( e^{-x} \right)|^a_0=\lim_{a\to+\infty}\left[ -e^{-a}+e^0 \right]=1
  \end{align*}
\end{Bsp}
\begin{Bsp}
  $f(x)=\frac{x}{1+x^2}\in R(\mb{R})$
  \begin{align*}
    \int f(x)\md x=\frac{1}{2}\ln (1+x^2)
  \end{align*}
  divergiert $x\to\pm\infty$. Deshalb existieren
  \[\int_0^{+\infty}f(x)\md x\ \text{und}\ \int^0_{-\infty}f(x)\md x\]
  nicht. Aber:
  \[\int_{-R}^Rf(x)\md x=0\]
  \[\lim_{R\to+\infty}\int^R_{-R}f(x)\md x=0\]
\end{Bsp}
\begin{Bsp}
  Sei $F(x) = \begin{cases}
    x^2\sin\frac{1}{x}&x\neq 0\\
    0 & x=0
  \end{cases}$.
  Sei $f=F'\in R\left(\mb{R}\setminus \{0\}\right)$ aber keine Regelfunktion auf $\mb{R}$ $x>$
  \begin{align*}
    \int^\pi_0f(x)\md x=\lim_{\varepsilon\downarrow 0}\int^x_\varepsilon f(x)\md x=\\
    =\lim_{\varepsilon\downarrow 0}F(x)|^x_\varepsilon=\lim_{\varepsilon\to 0}\left( F(x)-F(\varepsilon) \right)=F(x)
  \end{align*}
\end{Bsp}
\subsection{Majorantenkriterium}
\begin{Sat}{Majorantenkriterium}
  Seien $f$ und $g$ Regelfunktionen $[a;b)$ mit $\Abs{f}\leq g$. Existiert $\int_a^b g(x)\md x$, so existiert auch $\int^b_a f(x)\md x$  
\end{Sat}
\begin{Bew}
  Sei
  \begin{align*}
    F(u)=\int^u_af(x)\md x\\
    G(u)=\int^u_ag(x)\md x\\
    \forall u,v\in [a;b)\\
    \Abs{F(u)-F(v)}=\Abs{\int^u_bf(x)\md x}\leq \Abs{f^u_v\Abs{f(x)}\md x}\leq\\
    \leq \Abs{\int^u_vg(x)\md x}=\Abs{G(u)-G(v)}
  \end{align*}
  $G(u)$ $u\to 0$ existiert $\implies$ $G$ erfüllt das Cauchykriterium. $\implies$ $F$ erfüöllt das Cauchykriterium $\implies$ $\lim_{n\to b}F(u)$ existiert
\end{Bew}
\section{Kurven (Kapitel 12)}
\begin{align*}
  \gamma:I&\to& \mb{R}^n\\
  \gamma:t&\mapsto&\left(x_1(t),x_2(t),x_3(t),\cdots,x_n(t)\right)
\end{align*}
$x_i:I\to\mb{R}$ \underline{Komponentenfunktionen}
\begin{Def}{parametrisierte Kurve}
  Eine parametrisierte Kurve (kurz: Kurve) ist eine Abbildung $\gamma:I\to\mb{R}^n$, deren Komponentenfunktionen stetig sind.
\end{Def}
\begin{Def}{differenzierbare Kurve}
  Eine Kurve heisst differenzierbar, wenn jede Komponentenfunktion differenzierbar ist. Analog für stetig differenzierbar.
\end{Def}
\begin{Def}{Spur}
  Das Bild $\gamma(I)\in\mb{R}^n$ heisst die Spur von $\gamma$.
  \[\text{Spur}(\gamma)\]
\end{Def}
\begin{Bem}
  Eine Kurve ist eine Abbildung und ihre Spur ist eine Teilmenge
\end{Bem}

\begin{Bsp}
  Sei $k\in\mb{Z}\setminus\{0\}$
  \begin{align*}
    \gamma_k:&\mb{R}\to\mb{C}\cong \mb{R}^2\\
    &t\mapsto e^{ikt}
  \end{align*}
  $\Abs{\gamma(t)}=1$ $\forall t$ $\Spur\gamma_k=S^1$
  $k>0$: Gegenuhrzeigersinn\\
  $k<0$: Uhrzeigersinn
\end{Bsp}
\begin{Bsp}{Schraubenlinie}
  $\gamma:\mb{R}\to\mb{R}^3$
  \[t\mapsto (r\cos t, r\sin t, h t)\]
\end{Bsp}
\begin{Def}{Tangentialvektor einer Kurve}
  Sei $\gamma:I\to\mb{R}^n$ differenzierbar.
  \[\dot{\gamma}:=(\dot{x}_1(t),\dot{x}_2(t),\dots)\]
  $\dot{\gamma}$ heisst der Tangentialvektor oder Geschwindigkeitsvektor zur Stelle $t$.
\end{Def}
\begin{Def}{Geschwindigkeit einer Kurve}
  $\Norm{\dot{\gamma}(t)}$ heisst Geschwindigkeit. Der Geschwindigkeitsvektor hängt vom Parameter ab, nicht von der Stelle in $\mb{R}^n$.
\end{Def}
\begin{Def}{reguläre Kurve}
  Eine stetig differenzierbare Kurve $\gamma:I\to\mb{R}^n$ heisst regulär an der Stelle $t_0\in I$, wenn $\dot{\gamma}(t_0)\neq 0$. Sie heisst regulär, wenn sie an allen STellen regulär ist.
\end{Def}
\begin{Bsp}
  $\gamma(t)=(t^3,t^3), t\in\mb{R}$ $\Spur \gamma= (y=x)$ $\dot{\gamma}(t)=(3t^2,3t^2)$ $\dot{\gamma}=(0,0)$ nicht regulär! Aber der Punkt $(0,0)$ ist nicht singulär.
\end{Bsp}
\begin{Def}{Tangentialeinheitsvektor}
  Ist $\gamma$ an der Stelle $t_0$ regulär, so definiert man
  \[T\gamma(t_0):=\frac{\dot{\gamma}(t_0)}{\Norm{\dot{\gamma}(t_0)}}\]
  als Tangentialeinheitsvektor. $\Norm{T_\gamma}=1$
\end{Def}
\begin{Def}{Parametrisierte Kurve}
  Sei $f:J\to\mb{R}$ stetig differenzierbar. Der parametrisierte Graph von $f$ ist die Kurve
  \begin{align*}
    \gamma_f:&J\to\mb{R}^2\\
    &t\mapsto (t,f(t))
  \end{align*}
  $\Spur(\gamma_f)=\Graph (f)$
  \[\dot{\gamma_f}(t)=(1,f'(t))\neq 0\ \forall t\]
\end{Def}
\begin{Eig}{parametrisierter Graph}
  Ein parametrisierter Graph ist regulär
\end{Eig}
\begin{Sat}
  Sei $\gamma:I\to\mb{R}^2$, $t\mapsto(x(t),y(t))$ stetig differenzierbar. Wenn $\dot{x}(t)$ keine Nullstennen hat, gibt es eine stetig differenzierbare Funktion
  \[f:J\to\mb{R}^2\]
  wobei
  \[J:=x(I)\]
  s.d.
  \[\Graph f=\Spur \gamma\]
\end{Sat}
\begin{Bem}
  $\dot{y}\neq 0$ $\rsa$ Graph von $x(y)$
\end{Bem}
\begin{Sat}
  Sei $t_0\in I$, $x_0:=x(t_0)$
  \[f'(x_0)=\frac{\dot{y(t_0)}}{\dot{x}(t_0)}\]
  \[y=\Diff{f}{x}=\frac{\Diff{y}{t}}{\Diff{x}{t}}\]
  Ist $\gamma$ w-mal stetig differenzierbar, so ist es $f$ auch und
  \[f''\underbrace{(x_0)}_{=x(x_0)}\frac{\dot{x}\ddot{y}-\ddot{x}\dot{y}}{\dot{x}^3}\]
\end{Sat}
\begin{Bew}
  $\dot{x}\neq 0$ $\implies$ $x(t)$ streng monoton $\implies$ invertierbar. $\exists$ Umkehrabbildung
  \begin{align*}
    \tau:J\to I\\
    \tau(x(t))=t\ \forall t\\
  \end{align*}
  stetig differenzierbar
  \[\tau=\frac{1}{\dot{x}}\]
  \begin{align*}
    \gamma(t)=(x(t),y(t))&=\left( x(t),y(\tau(x(t))) \right)\\
    &=\left( x(t),(y\circ \tau)(x(t))\right)\\
    &=(x(t),f\left( x(t) \right)\\
  \end{align*}
  \begin{align*}
    f:=y\circ \tau\\
    \gamma_f:x\mapsto(x,f(x))\\
    \Spur \gamma=\Spur \gamma_f=\Graph f
  \end{align*}
  \begin{align*}
    f'(x_0)=\dot{y}t(t_0)\tau'(x_0)=\dot{y}(t_0)\frac{1}{\dot{x}(t_0)}\\
f''=\left( \Diff{}{x}\dot{y} \right)\frac{1}{\dot{x}}+\dot{y}\Diff{}{x}\left( \frac{1}{\dot{x}} \right)=\\
    =\left( \ddot{y}\tau' \right)\frac{1}{\dot{x}}+\dot{y}\left( -\frac{1}{\dot{x}^2}\ddot{x}\tau' \right)=\\
    =\ddot{y}\frac{1}{\dot{x}}\frac{1}{\dot{x}}-\dot{y}\frac{1}{\dot{x}^2}\ddot{x}\frac{1}{\dot{x}}=\\
    =\frac{\dot{x}\ddot{y}-\ddot{x}\dot{y}}{\dot{x}^3}
  \end{align*}
\end{Bew}
\begin{Eig}
  \begin{align*}
    \dot{x}\neq 0 \rsa y=f(x)\\
    \dot{y}\neq 0 \rsa x=g(y)\\
    \gamma\text{regulär} \implies\ \forall t \exists \text{Umgebung $I$ von $t$ s.d.} \\
    \dot{x}(\tau)\neq 0\ \forall \tau\in I\\
    \dot{y}(\tau)\neq 0\ \forall \tau\in I
  \end{align*}
\end{Eig}
\subsection{Die Bogenlänge}
\begin{Def}
  Sei $\gamma:I\to\mb{R}^n$. Sei $Z=(t_0,t_1,\cdots,t_n)$ $t_i\in I$ $t_0<t_1<\cdots<t_n$ Länge des Sehnenpolygons.
  \[S(Z):=\sum^m_{i=1}\Norm{\gamma(t_i)-\gamma(t_{i-1})}\]
  Gilt $Z^*\supset Z$, dann $S(Z^*)\geq S(Z)$
  \[Z_1\subset Z^*, Z_2\subset Z^* \implies S(Z^*)\geq \max\left( S(Z_1),S(Z_2) \right)\]
  Idee: $s(\gamma):=\sup_ZS(2)$
\end{Def}
\begin{Def}{rektifizierbare Kurve}
  Eine Kurve $\gamma$ heisst rektifizierbar, wenn die Menge der Längen aller einbeschriebenen Sehnenpolygone beschränkt ist.
\end{Def}
\begin{Sat}
  Sei $\gamma:[a;b]\to\mb{R}^n$ fast überall stetig differenzierbar, (d.h. jede Komponente ist fast überall stetig differenzierbar). Dann ist $\gamma$ rektifizierbar (1) und
  \begin{align*}
    s(\gamma)=\int^b_a\Norm{\dot{\gamma}(t)}\md t\geq 0& & (2)
  \end{align*}
\end{Sat}
\begin{Bem}
  Ist $\gamma_f$ der pramametrisierte Graph von $f$
  \[\gamma_f(t)=(t,f(t))\]
  so ist
  \[\dot{\gamma}_f(t)=(1,f'(t))\]
  \[\Norm{\dot{\gamma}_f}=\sqrt{1+f'^2}\]
  \[s(\gamma_f)=\int^b_a\sqrt{1+f'(t)}\md t\]
\end{Bem}
\begin{Not}
  Sei $f=(f_1,\cdots,f_n)$ ein $n$-Tupel Funktionen
  \[\int f(x)\md x:=\left( \int f_1\md x, \int f_2\md x,\cdots,\int\ f_n\md x \right)\]
\end{Not}
\begin{Lem}
  \[\Norm{\int^b_af(x)\md x}\leq \int^b_a\Norm{f(x)}\md x\]
  Beweis
  \begin{enumerate}
    \item Lemma gilt für Treppenfunktionen
    \item Approximationssazu
  \end{enumerate}
\end{Lem}
\begin{Bew}
  Sei $Z=(t_0,\cdots,t_m)$ eine Zerlegung von $[a;b]$    
  \begin{align*}
    S(Z)&=\sum^m_{i=1}\Norm{\gamma(t_i)-\gamma(t_{i-1}}\\
    &=\sum\Norm{\int^{t_i}_{t_{i-1}}\dot{\gamma}(t)\md t}\\
    &\stackrel{\text{Lemma}}{\leq}\sum^m_{i=1}\int^{t_i}_{t_{i-1}}\Norm{\dot{\gamma}}\md t\\
    &=\int^b_a\Norm{\dot{\gamma}}\md t    
  \end{align*}
  ($\Norm{\dot{\gamma}}$ ist eine Regelfunktion) Diese Abschätzung gilt für alle Zerlegungen. $\implies$ $\gamma$ rektifizierbar.
  \[s(\gamma)\leq \int^b_a\Norm{\dot{\gamma}}\md t\]
  = für (2)
  \[\forall \varepsilon>0\ \exists Z: S(Z)\geq f(\Norm{\dot{\gamma}}-\varepsilon\]
  Treppenfunktionen + Approximationssatz
\end{Bew}

\begin{Bsp}{Länge des Kreisbogens}
  \begin{align*}
    \gamma:&[0,\phi]\to\mb{R}^2\\
    &t\mapsto \left( r \cos t, r\sin t \right) = \gamma(t)\\
  \end{align*}
  \begin{align*}
    \dot{\gamma}(t)=\left( -r\sin t, r\cos t \right)\\
    \Norm{\dot{\gamma}(t)}^2=r^2\sin^2t+r^2\cos^2t=r^2\\
    s(\gamma)=\int^\phi_0r\md t=rt|^\phi_0=r\phi
  \end{align*}
  \begin{align*}
    y=\sqrt{r^2-x^2}\\
  \end{align*}
  \begin{align*}
    \gamma:&[a;r]\to\mb{R}^2\\
    &x\mapsto \left( x,\sqrt{r^2-x^2} \right)
  \end{align*}
  \begin{align*}
    a:=r\cos\phi\\
    s(\gamma)=\int^r_a\sqrt{1+f'^2}\md x\\
    \sqrt{1+f'^2}=\sqrt{1+\frac{x^2}{r^2-x^2}}=\sqrt{\frac{r^2-x^2+x^2}{r^2-x^2}}=
    =\frac{1}{\sqrt{r^2-x^2}} = r\int_a^r\frac{\md x}{\sqrt{r^2-x^2}}\\
    \xi=\frac{x}{r}
    =r\int_{\frac{a}{r}}^1\frac{r\md \xi}{\sqrt{r^2-r^2\xi^2}}=r\int_{\frac{a}{r}}^1\frac{\md\xi}{\sqrt{1-\xi^2}}\\
    =-r\arccos\xi|^1_{\frac{1}{r}}=-r(\arccos 1- \arccos \cos \phi)= r\phi
  \end{align*}
\end{Bsp}
\subsection{Parameterwechsel}
\begin{Def}{$C^k$-Parametertransformation}
  Sei $k=0,1,2,\cdots,\infty$. Eine Abbildung $\sigma:I\to J$ heisst $C^k$-Parametertransforumation, wenn
  \begin{enumerate}
    \item $\sigma\in C^k(I;J)$
    \item $\sigma$ ist umkehrbar
    \item $\sigma^{-1}\in C^k(J;I)$
  \end{enumerate}
  Sei 
  \begin{align*}
    \gamma:&I\to\mb{R}^n\\
    \underbrace{\beta}_{\gamma\circ\sigma^{-1}}:&J\to\mb{R}^n
  \end{align*}
\end{Def}
\begin{Bsp}{Gegenbeispiel}
  \begin{align*}
    \sigma:&\mb{R}\to\mb{R}\\
    &x\mapsto x^3
  \end{align*}
  $\sigma$ umkehrbar, $\sigma\in C^1$. $\sigma\not\in C^1$ $\sigma$ ist eine $C^0$-Paramentertransformation, aber keine $C^0$-Parametertransforumation.
\end{Bsp}
\begin{Def}{Umparametrisierung}
  Sei 
  \begin{align*}
    \gamma:&I\to\mb{R}^n\\
    \underbrace{\beta}_{\gamma\circ\sigma^{-1}}:&J\to\mb{R}^n
  \end{align*}
  Ist $\gamma$ $C^k$-Kurve, $\sigma$ $C^k$ Parametertransformation, dann $\beta$ $C^k$-Kurve. $\beta$ heisst due Umparametrisierung von $\gamma$ mittels $\sigma$.
\end{Def}
\begin{Not}
  \begin{align*}
    \gamma:\underbrace{I}_{t\in}to\underbrace{\Sigma}_{\sigma \in}
  \end{align*}
\end{Not}
\begin{Bsp}
  \begin{align*}
    \gamma:&[0;\phi]\to\mb{R}^2\\
    &t\mapsto(r\cos t, r\sin t)
  \end{align*}
  \begin{align*}
    \sigma:&[0;\phi]\to[a;1]\\
    &t\mapsto r\cos t=: x
  \end{align*}
  \begin{align*}
    \beta(x)=\left( x;\sqrt{r^2-x^2} \right)
  \end{align*}
  orientierungsumkehrend
\end{Bsp}
\begin{Def}{orientierungstreu/-umkehrend}
  Eine Parametertransformation $\sigma:I\to J$ heisst orientierungstreu ($\dot\sigma>0$), wenn sie streng monoton wächst oder orientierungsumkehrend,($\dot\sigma<0$) wenn sie streng monoton fällt.
\end{Def}
\begin{Bem}
  Ist $\gamma$ reklifizierbar, so ist $\beta=\gamma\circ\sigma^{-1}$ und $S(\gamma)=S(\beta)$
\end{Bem}
\begin{Bew}
  $S(0)=\sup S(2)$ das hängt von der Parametrisierung nicht ab.
\end{Bew}
\begin{Bew}
  \begin{align*}
    S(\gamma)\int^b_a\Norm{\dot\gamma}\md t\\
    \dot\beta=\frac{\dot\gamma}{\dot\sigma}\\
    \sigma:[a;b]\to[c;d]\\
    \int^b_a\Norm{\dot\gamma}\md t=\int^d_c\Norm{\dot\Gamma}\frac{\md \sigma}{\dot\sigma}=\\
    \begin{cases}
      \int^d_c\Norm{\dot\beta}\md \sigma&\dot\sigma>0 (c>d)\\
      -\int^d_c\Norm{\dot\beta}\md \sigma&\dot\sigma<0 (\Abs{\dot\sigma}=-\dot\sigma) (d>c)
    \end{cases}\\
    \begin{cases}
      \int^d_c\Norm{\dot\beta}\md \sigma&\dot\sigma>0 (c>d)\\
      \int^c_d\Norm{\dot\beta}\md \sigma&\dot\sigma<0 (\Abs{\dot\sigma}=-\dot\sigma) (d>c)
    \end{cases}\\
    =S(\beta)
  \end{align*}
\end{Bew}
\begin{Def}{Umorientierung}
  \begin{align*}
    \sigma:&[a;b]\mapsto[-a;-b]\\
    &t\mapsto -t
  \end{align*}
\end{Def}
\begin{Not}
  \begin{align*}
    \gamma:[a;b]\to\mb{R}^n\\
    \gamma^-:[-a;-b]\to\mb{R}^n\\
    \gamma^-(t):=\gamma(-t)
  \end{align*}
\end{Not}
\begin{Def}{Umparametrisierung auf Bogenlänge}
  Sei $\gamma:I\to\mb{R}^n$ regulär und fast überall stetig differenzierbar. Sei $t_0\in I$
  \[S(t)=\int^t_{t_0}\Norm{\dot\gamma(\tau)}\md \tau, t\in I\]
  \begin{align*}
    S:I\to J=S(I)\\
    \dot S(T)=\Norm{\dot\varphi(t)}>0
  \end{align*}
  $\implies$ $s$ orientierungstreu.
  \begin{align*}
    \beta:=\gamma\circ s^{-1}\\
    \beta'(s)=\dot\gamma(t(s))\frac{1}{\dots(t(s))}=\frac{\dot\gamma}{\Norm{\dot\gamma}}(t(s))
  \end{align*}
  \[\Norm{\beta'(s)}=1\ \forall s\in J\]
\end{Def}
\subsection{Sektorfläche einer ebenen Kurve}
\begin{Def}{Sektorfläche}
  $\gamma:I\to\mb{R}^2$. $F_i$ = orientierte Fläche des $i$-ten Dreiecks.
  \[F(Z):=\sum_iF_i\]
\end{Def}
\begin{Lem}
  Seien $(0,0),(x,y),(\tilde x,\tilde y)$ die Ecken eines Dreiecks in $\mb{R}^2$. Die orientierte Fläche des Dreiecks ist
  \[F=\frac{1}{2}\left( x\tilde y-\tilde x y \right)\]
  \[=(x,y)\times(\tilde x, \tilde y)\]
  \[=\det\Mx{x&\tilde x\\y&\tilde y}=\det\Mx{x&y\\ \tilde x&y}\]
\end{Lem}
\begin{Bew}
  \begin{align*}
    \rho:=\Norm{(x,y)}\\
    \tilde\rho:=\Norm{(\tilde x, \tilde y)}\\
    F=\frac{1}{2}\rho h\\
    h=\tilde\rho\sin\psi\\
    F=\frac{1}\rho\tilde\rho\sin\psi
  \end{align*}
  \begin{align*}
    z=x+iy=\rho e^{i\phi}\\
    w=\tilde x+i\tilde y=\tilde\rho e^{i\tilde\phi}\\
    \psi=\tilde\phi-\phi\\
    \bar z w=\rho\tilde\rho e^{i(\tilde\phi-\psi)}\\
    \Im(\bar z w)=\rho\tilde\rho\sin\psi=2F\\
    \bar z w=(x-iy)(\tilde x+i \tilde y)=\\
    =(x\tilde{x}+<\tilde{y}+i(x\tilde{y}-\tilde{x}y)\\
    \Im \bar z w=x\tilde{y}-\tilde{x}y
  \end{align*}
\end{Bew}
\begin{Not}
  \begin{align*}
    \Delta x:=\tilde{x}-x\\
    \Delta y:=\tilde{y}-y\\
  \end{align*}
  \[F=\frac{1}{2}\left[ x(y+\Delta y)-(x+\Delta x)y\right]\]
  \[F=\frac{1}{2}\left( x\Delta y-y\Delta x \right)\]
\end{Not}
\begin{Bew}
  Sei $\gamma:[a;b]\to\mb{R}^2$ Kurve, $Z:=\underbrace{t_0}_{=a}<t_1<\cdots<\underbrace{t_n}_{=b}$ Zerlegung. $(x;y_i):=\gamma(t_i)$
  \begin{align*}
    \Delta x_i:=x_i-x_{i-1}\\
    \Delta y_i:=y_i-y_{i-1}
  \end{align*}
  $\implies$
  \begin{align*}
    F_i=\frac{x_{i-1}\Delta y_i-y_{i-1}\Delta x_i}{2}\\
    F(Z):=\sum^n_{i=1}F_i
  \end{align*}
\end{Bew}
\begin{Def}
  Der Fahrstrahl an die Kurve $\gamma$ überstreicht den orientierten Flächeninhalt $F(\gamma)$, wenn
  \[\forall \varepsilon>0\ \exists \delta>0\ \text{s.d.}\ \forall \text{Zerlegung} Z \text{des Fahrstrahls} \leq \delta\]
  gilt
  \[\Abs{F(Z)-F(\delta)}\leq \varepsilon\]
\end{Def}
\begin{Sat}{Sektorformel von Leibniz}
  Sei $\gamma:[a;b]\to\mb{R}^2$ fast überall stetig differenzierbar. Dann
  \[F(\gamma)=\frac{1}{2}\int^b_a(x\dot y-\dot xy)\md t\]
\end{Sat}
\begin{Bew}
  \begin{align*}
    \Delta x_i=x(t_i)-x(t_{i-1})=\int^{t_i}_{t_{i-1}}\dot x(t)\md t\\
    \Delta y_i=\int^{t_i}_{t_{i-1}}\dot y(t)\md t\\
    2F_i=\int^{t_i}_{t_{i-1}}\left( x_{i-1}\dot y-y_{i-1}\dot x \right)\md t\\
    \Abs{2F_i-\int^{t_i}_{t_{i-1}}\left( x\dot y-\dot x y\right)\md t}=\\
    =\Abs{\int^{t_i}_{t_{i-1}}\left[ (x_{i-1}-x)\dot y-(y_{i-1}-y)\dot x \right]\md t}\leq\\
    \leq \Abs{\int^{t_i}_{t_{i-1}}(x_{i-1}-x)\dot y\md t }+\Abs{\int^{t_i}_{t_{i-1}}(y_i-y)\dot x\md t}\\
  \end{align*}
  $\gamma$ fast überall stetig differenzierbar $\implies$ $\gamma$ stetig und fast überall differenzierbar $\xRightarrow{\text{verallgemeinerter Schrankensatz}}$ $\exists L:\Abs{\dot x}<L, \Abs{\dot y}<L$ fast überall und
  \begin{align*}
    \Abs{x(t)-x_{i-1}}=\\
    \Abs{x(t)-x(t_{i-1})}\leq L(t-t_{i-1})\\
    \Abs{y(t)-y_{i-1})}\leq L(t-t_{i-1})\\
    J_i\leq 2L^2\int^{t_i}_{t_{i-1}}(t-t_{i-1})\md t=\\
    =2L^2\frac{1}{2}(t-t_{i-1})^2|^{t_i}_{t_{i-1}}=\\
    =L^2(t_i-t_{i-1})^2
  \end{align*}
  Ist die Feinheit $\leq \delta$, so ist $t_i-t_{i-1}\leq \delta$
  \begin{align*}
    J_i\leq L^2\delta(t_i-t_{i-1})
  \end{align*}
  \begin{align*}
    \Abs{F(Z)-\frac{1}{2}\int^b_a(x\dot y-\dot xy)\md t}=\\
    =\Abs{\sum_{i=1}^nF_i(Z)-\sum\frac{1}{2}\int^{t_i}_{t_{i-1}}(x\dot y-\dot xy)\md t}\\
    \leq \frac{1}{2}\sum^n_{i-1}J_i\leq \frac{1}{2}\sum^n_{i=1}L^2\delta(t_i-t_{i-1})=\\
    =\frac{1}{2}L^2\delta(t_1-t_0+t_2-t_1+\cdots)=\\
    =\frac{1}{2}L^2\delta(b-a)\leq \varepsilon
  \end{align*}
  für
  \[\delta=\frac{2\varepsilon}{L^2(b-a)}\]
\end{Bew}
\begin{Bsp}
  \begin{align*}
    \gamma:&[0,\phi]\to\mb{R}^2\\
    t\mapsto(r\cos t,r\sin t)
  \end{align*}
  \begin{align*}
    \dot\gamma=(-r\sin t, r\cos t,r\cos t)\\
    F=\frac{1}{2}\int_0^\phi(r^2\cos^2t+r^2\sin^2t)\md t=\\
    =\frac{r^2}{2}\int^\phi_0\md t=\frac{r^2\phi}{2}\\
    \phi=2\phi \implies \pi r^2
  \end{align*}
\end{Bsp}

\begin{Eig}{Sektorformel}
  \begin{enumerate}
    \item Additivität: $c\in(a;b)$
      \[F(\gamma)=F\left(\gamma_{|_{[a;c]}}\right)+F\left( \gamma_{|_{[c;b]}} \right)\]
    \item Orientierungsumkehrung
      \[F(\gamma^-)=-F(\gamma)\]
      \[\gamma(t):=\gamma(-t)\]
    \item
      \[A:\mb{R}^2\to\mb{R}^2\]
      \[\Mx{e&f\\g&h}\Mx{x\\y}\mapsto \Mx{ex+fy\\gx+hy}\]
      \begin{gather*}
        (A\gamma)(t)=A\gamma(t)\\
        d(A\gamma)=A\dot\gamma\\
        x\dot y-y\dot x=\det\Mx{\gamma&|&\dot\gamma}=\det\Mx{x&\dot x\\y&\dot y}\\
        F(\gamma)=\frac{1}{2}\det\Mx{\gamma&|&\dot\gamma}\md t\\
        \det\Mx{A\gamma&|&A\dot\gamma}=\det\Mx{A(\gamma&|&\dot\gamma)}=\det A\det\Mx{\gamma&|&\dot\gamma}\\
        F(A\gamma)=\det A\cdot F(\gamma)\\
        \text{insbesondere}\\
        \det A=1(\text{d.h.}\ A\in SL(2;\mb{R}))\\
        F(A\gamma)=F(\gamma)
      \end{gather*}
  \end{enumerate}
\end{Eig}
\begin{Def}{Geschlossene Kurve}
  Eine Kurve $\gamma:[a;b]\to\mb{R}^n$ heisst geschlossen, wenn
  \[\gamma(a)=\gamma(b)\]
  gilt.
\end{Def}
\begin{Def}{umschlossener orienterierter Flächeninhalt}
  Sei $\gamma:[a;b]\to\mb{R}^n$ geschlossen und so dass $F(\gamma)$ existiert, so heisst $F(\gamma)$ der umschlossene orienterierte Flächeninhalt.
\end{Def}
\begin{Bem}
  $\gamma(a)=\gamma(b)$
  \begin{align*}
    \int_a^b\md (xy)\md t=(xy)|^b_a=0\\
    F(\gamma)=\int^b_ax\dot y\md t=-\int^b_a\dot x y\md t
  \end{align*}
  (wenn $\gamma$ geschlossen)
\end{Bem}
\begin{Bem}
  Polarkoordinaten
  \begin{gather*}
    (x,y)\in\mb{R}^2\\
    \rho e^{i\phi} = x+iy=:z\in\mb{C}\\
    \gamma:[a;b]\to\mb{R}^2\\
    \dot z= \dot \rho e^{i\phi}+i\rho\dot\phi e^{i\phi}
  \end{gather*}
  \begin{align*}
    t\mapsto& (x(t),y(t))\\
    t\mapsto&z(t)\\
    t\mapsto&\rho(t)e^{i\phi(t)}
  \end{align*}
  Man erlaubt $\rho(t)<0$
\end{Bem}
\begin{Bem}{Länge}
  \begin{gather*}
    L=\int_a^b\Norm{\dot \gamma}\md t=\int^b_a\sqrt{\dot\bar z\dot z}\md t\\
    \bar z=\rho e^{i-\phi},\ \dot\bar z=\dot\rho e^{-i\phi}-i\rho\dot\phi e^{-i\phi}\\
    z=x+iy,\ \bar z=x-iy,\\
    \dot z=\dot x+iy,\ \dot\bar z=\dot x-i\dot y\\
    \bar z\dot z=(x\dot x+y\dot y)+i(x\dot y-\dot xy)\\
    =\frac{1}{2}\int\Im(\bar z\dot z)\md t\\
    \bar z\dot z=\rho e^{-i\phi}\left(\dot \rho e^{i\phi}+ \rho\dot\phi e^{i\phi}\right) = \rho\dot\phi+i\rho^2\dot\phi=\\
    =\frac{1}{2}\int^b_a\rho^2\dot\phi\md t
  \end{gather*}
\end{Bem}
\begin{Bsp}
  \begin{align*}
    y:[0;2\pi]&\mapsto\mb{R}^2\\
    \phi&\mapsto a\cos(3\phi)e^{i\phi}\\
  \end{align*}
  \begin{align*}
    \rho(\phi)=a\cos(3\phi)
  \end{align*}
  $\rho$ kann auch negativ sein
  \begin{gather*}
    F(\gamma)=\frac{3}{2}\int_0^{\frac{\pi}{3}}a^2\cos^2(3\phi)\md\phi =\\
    =\frac{\not 3}{2}\int^{2\pi}a^2\cos^2(\phi)\frac{\md \phi}{\not 3}=\\
    \frac{a^2}{2}\int^{2\pi}_0\frac{(\cos^2\phi+\sin^2\phi)}{2}\md \phi=\frac{a^2}{4}2\pi=\frac{a^2\pi}{2}\\
    \int^{2\pi}_0\cos^2=\int_0^{2\pi}\sin^2
  \end{gather*}
\end{Bsp}
\section{Taylor [Kap 14]}
Wir wollen eine Funktion durch Polynom approximieren.
\begin{Def}
  Sei $f:I\to\mb{C}$ $n$-mal differenzierbar. Das $n$-te Taylorpolynom von $f$ im Punkt $a\in I$ ist das Polynom $T(x)$ des Grades $\leq n$ mit
  \begin{gather*}
    T(a)=f(a)\\
    T'(a)=f'(a)\\
    T''(a)=f''(a)\\
    \cdots
    T^{(n)}(a)=f^{(n)}(a)
  \end{gather*}
\end{Def}
\begin{Not}
  $I_n f(x;a)$
\end{Not}
\begin{Bsp}{$n=1$}
  \[T_1 f(x;a)=f(a)+f'(a)(x-a)\]
\end{Bsp}
\begin{Bem}
  Sei $I_n f(x;a)$ das $n$-te Taylorpolynom von $f$
  \[T(x)=I_n f(x;a)=\sum^n_{k=0}a_k(x-a)^k\]
  \begin{gather*}
    f(a)T'(x)=\sum^n_{k=1} k a_k(x-a)^{k-1}\\
    f(a)T''(x)=\sum^n_{k=2} k(k-1) a_k(x-a)^{k-2}\\
    f(a)T'''(x)=\sum^n_{k=3} k(k-1)(k-2) a_k(x-a)^{k-3}\\
    \cdots\\
    T(a)=a_0\\
    T'(a)=a_1\\
    T''(a)=2a_2\\
    T'''(a)=3\cdot 2a_3
  \end{gather*}
  Übung $l\leq n$ (Induktion)
  \[T^{(l)}_{(x)}=\sum^n_{k=l}k(k-1)(k-2)\cdots(k-l+1)a_k(x-a)^{k-l}\]
  \[T^{(l)}(a)=l!a_l=f^{(l)}(a)\]
  \[a_l=\frac{f^{(l)}(a)}{l!}\]
\end{Bem}
\begin{Eig}
  \[T_nf(x;a)=\sum^n_{k=0}\frac{f^{(k)}(a)}{k!}(x-a)^k\]
\end{Eig}
\begin{Def}{Fehler}
  \[R_{n+1}(x;a):=f(x)-T_nf(x;a)\]  
\end{Def}
\begin{Lem}
  \[\Limo{x}\frac{R_{n+1}(x;a)}{(x-a)^n}=0\]
  \begin{gather*}
    R_2=f(x)-T_1 f(x,a)\\
    T_1f(x;a)=f(a)+f'(a)(x-a)\\
    R_2= \frac{f(x)-f(a)-f'(a)(x-a)}{x-a}\xrightarrow{f\text{differenzierbar}} 0
  \end{gather*}
\end{Lem}
\begin{Bew}
  $T=T_nf$
  \begin{align*}
    &\lim_{x\to a}\frac{f(x)-T(x)}{(x-a)^n}=\\
    (L'Hopital)& = \lim_{x\to a}\frac{f'(x)-T'(x)}{n(x-a)^{n-1}}=\\
    (L'Hopital)& = \lim_{x\to a}\frac{f''(x)-T''(x)}{n(n-1)(x-a)^{n-2}}=\cdots\\
    \cdots&=\lim_{x\to a}\frac{f^{(n)}(x)-T^{(n)}(x)}{n!}=0
  \end{align*}
  denn $f^{(n)}(a)=T^{(n)}(a)$
\end{Bew}
\begin{Kor}{Qualitative Taylorformel}
  Sei $f:I\to\mb{C}$ stetig und $n$-mal differenzierbar. Dann
  \[\exists r;I\to\mb{C}\]
  stetig mit
  \[r(a)=0\]
  s.d.
  \[f(x)=I_nf(x;a)+(x-a)^nr(x)\]
\end{Kor}
\begin{Bew}
  \begin{gather*}
    r(x):=\frac{f(x)-I_nf(x;a)}{(x-a)^n}
  \end{gather*}
  $x\neq a$ stetig auf $I\setminus \left\{ a \right\}$
  \[\lim_{x\to a}r(x)\]
  Wir erweiter $r$ auf $I$ mit $r(a)=0$
\end{Bew}
\begin{Not}{Landan-Symbol}
  Seien $f$ und $g$ komplexe Funktionen in einer punktierten Umgebung von $a$. Man schreibt
  \[f=\circ(g),x\to a\]
  falls
  \[\lim_{x\to a}\frac{f(a)}{f(x)}=0\]
  Gilt zusätzlich
  \[\lim_{x\to a}g(x) =0\]
  so sagt man: $f$ geht für $x\to a$ schneller gegen 0 als $g$.\\
  $f:I\to\mb{C}$, $a\in I$ $n$-mal differenzierbar:
  \[f(x)=T_nf(x;a)+\circ\left( (x-a)^n \right),x\to a\]
\end{Not}
\begin{Bsp}
  $T_4(x;0)$
  \begin{align*}
    f(x)=\sin x& 0\\
    f'(x)=\cos x& 1\\
    f(x)=-\sin x& 0\\
    f'(x)=-\cos x& -1\\
    f(x)=\sin x& 0
  \end{align*}
  \[T_4f(x;0)=x-\frac{1}{3!}x^3\]
  \[\sin x=x-\frac{x^3}{6}+\circ(x^4)\]
\end{Bsp}
\begin{Bsp}
  \begin{gather*}
    \Limo{x}\frac{\sin x-x}{x^3}=\Limo{x}\frac{\frac{-x^3}{6}+\circ(x^4)}{x^3}=\\
    =-\frac{1}{6}\Limo{x}\frac{x^3}{x^3}+\Limo{x}\frac{x\circ(x^4)}{x^4}=\\
    =-\frac{1}{6}+0\cdot 0=-\frac{1}{6}
  \end{gather*}
\end{Bsp}
\begin{Sat}{Integralform von $R_{n+1}$}
  Sei $f\in \Phi^{n+1}(I,\mb{C})$ ($\Phi$ differnzierbare Funktion). Dann
  \[R_{n+1}(x)=\frac{1}{n!}\int_a^x(x-t)^nf^{(n+1)}(t)\md t\]
\end{Sat}
\begin{Bew}
  Durch Induktion
  \subparagraph{$n=0$}
  \begin{gather*}
    R_1(x)=f(x)-T_0f(x;a)\\
    T_0f(x;a)=f(a)\\
    R_1(x)=f(x)-f(a)\\
    \frac{1}{1!}\int^x_af'(t)\md t=f(x)-f(a)
  \end{gather*}
  \subparagraph{$n+1$}
  \begin{gather*}
    f-T_{n-1} f=R_n=\frac{1}{(n-1)!}\int^x_a(x-t)^{n-1}f^{(n)}(t)\md t\\
    =\frac{1}{(n-1)!}\int\Diff{}{t}\frac{(x-t)^n}{-n}f^{(n)}(t)\md t\\
    =-\frac{1}{n!}\left[ (x-1)^nf^{(n)}(t) \right]\Big|^x_a+\frac{1}{n!}\int(x-t)^nf^{(n+1)}(t)\md t\\
    =\frac{1}{n!}(x-a)^nf^{(n)}(a)+\frac{1}{n!}\int^x_a(x-t)^nf^{(n+1)}(t)\md t\\
    \implies f-T_nf=\frac{1}{n!}\int^x_a(x-t)^nf^{(n+1)}(t)\md t
  \end{gather*}
\end{Bew}
\begin{Kor}{Lagrange-Form für $R_{n+1}$}
  Sei $f\in\Phi^{n+1}(I;\mb{R})$ $a\in I$.
  \[\forall x\in I\ \exists \xi\in I: R_{n+1}(x)=\frac{f^{(n+1)}(\xi)}{(n+1)!}(x-a)^{n+1}\]
\end{Kor}

\begin{Bsp}
  \begin{gather*}
    f=\sin x\\
    T_n f(x;0)=x-\frac{x^3}{6}\\
    f^{(5)}(x)=\cos x\\
    \exists \xi: \sin x=x-\frac{x^3}{6}+\frac{1}{5!}\cos\xi x^5
  \end{gather*}
\end{Bsp}
\begin{Bew}
  $f\in \mathcal{R}^{n+1}(I:\mb{C})$
  \begin{gather*}
    R_{n+1}=\frac{1}{n!}\int^x_a (x-t)^nf^{(n+1)}(t)\md t = \sigma\int^x_ap(t)f^{(n+1)}(t)\md t = \cdots\\
    (t):=\frac{\Abs{x-t}^n}{n!}\geq 0\\
    \sigma =\begin{cases}
      1&a<x\\
      (-1)^n&a>x
    \end{cases}\\
    \cdots \stackrel{MWS}{=} \sigma f^{(n+1)}(\xi)\int^x_ap(t)\md t\\
    \int^x_ap(t)\md t=\sigma\frac{1}{n!}\int^x_a(x-t)^n\md t=\sigma\frac{1}{(n+1)!}(x-t)|^x_a=\sigma\frac{(x-a)^{n-1}}{(n+1)!}\\
    R_{n+1}\underbrace{\sigma^2}_{=1}f^{(n+1)}(\xi)\frac{(x-a)^{n+1}}{(n+1)!}
  \end{gather*}
\end{Bew}
\subsection{Lokale Extrema}
\begin{Sat}
  Sei $f\in\mathcal R^{n+1}(I,\mb{R})$. Sei $a\in I$ und es gelte
  \[f'(a)=f''(a)=\cdots=f^{(n)}(a)=0\]
  \[f^{(n+1)}(a)\neq 0\]
  Dann
  \begin{enumerate}
    \item $n$ gerade $\implies$ $f$ hat in $a$ kein Extrema
    \item $n$ ungerade, $f^{(n+1)}(a)>0$ $\implies$ $f$ hat in $a$ ein strenges lokale Minimumm
    \item $n$ ungerade, $f^{(n+1)}(a)<0$ $\implies$ $f$ hat in $a$ ein strenges lokale Maximum
  \end{enumerate}
  Hint: Beweis anschauen $>$ auswendig lernen
\end{Sat}
\begin{Bew}
  $T_nf(x;a)=f(a)$
  \begin{gather*}
    f(x)=T_nf(x;a)+R_{n+1}(x)\\
    =f(a)+\frac{f^{(n+1)}(\xi)}{(n+1)!}(x-a)^{n+1}\\
  \end{gather*}
  \begin{align*}
    f^{(n+1)} \text{stetig}& \implies & \exists \text{Umgebung von }a f^{(n+1)} \neq 0\\
    f^{(n+1)}(a)\neq 0
  \end{align*}
  Man ersetze $\neq$ durch $<$ und $>$.\\
  $n$ gerade $\implies$ $(n+1)$ ungerade. Das Vorzeichen $(x-a)^{n+1}$ verändert ishc \\
  $n$ ungerade $\implies$ $(n+1)$ gerade $(x-a)^{n+1}$ positiv
\end{Bew}
\subsection{Taylorreihen}
\begin{Def}{Taylorreihe}
  Sei $f\in \mathcal{R}^\infty(I,\mb{C})$. Man definiert
  \[Tf(x;a):=\sum^\infty_{k=0}\frac{f^{(k)}(a)}{k!}(x-a)^k\]
  Taylorreihe von $f$ im Punkt $a$
\end{Def}
\begin{Bem}
  \begin{enumerate}
    \item Es kann passieren, dass die Reihe nicht konvergiert
    \item Es kann auch passieren, dass die Reihe in einer Umgebung von $a$ konvegiert, aber nicht gegen $f$!
  \end{enumerate}
\end{Bem}
\begin{Bsp}
  \[f(x)=\begin{cases}
    0&x\leq 0\\
    e^{-\frac{1}{x}}&x>0
  \end{cases}\]
  \begin{gather*}
    f^{(k)}(0)=0 \ \forall k\\
    \implies Tf(x;0)=0\neq f(x)
  \end{gather*}
\end{Bsp}
\begin{Def}
  Konvergiert $Tf$ gegen $f$ in einer Umgebung $U$ von $a$, so sagt man: 
  \[\text{\underline{$f$ besitzt in $U$ eine Taylorentwicklung mit $a$ als Entwicklungspunkt.}}\]
  oder
  \[\text{\underline{$f$ ist reell analytisch in $U$}}\]
\end{Def}
\begin{Bew}
  Ist $f=\sum^\infty_{k=0}a_l(x-a)^k$ mit $\Abs{x-a}<R$ (Konvergenzradius)
  \begin{gather*}
    \Diff{}{x}\sum=\sum\Diff{}{x}\\
    f^{(k)}(a)=k!a_k\\
    \implies Tf=\sum a_k(x-a)^k
  \end{gather*}
\end{Bew}
\begin{Def}
  Sei $f:\overbrace{U}^{\in \mb{C}}\to\mb{C}$ Sei $a\in U$. Man sagt, $f$ ist analytisch in $a\in U$ wenn $\exists r>0$ mit $K_r(a)\subset U$ und $\exists$ Potenzreihe $\sum a_kz^k$ mit Konvergenradius $>r$ s.d.
  \[f(z)=\sum a_k(z-a)^k\ \forall z\in K_r(a)\]
\end{Def}
\begin{table}[htb]
  \centering
  \begin{tabular}{c|c|c}
    Struktur&Definitionsbereich&Zielmenge\\
    \hline
    stetige Funktionen & $U\subset \mb{R},\mb{C}$&$\mb{R},\mb{C}$\\
    differenzierbare Funktionen & $I\in \mb{R}$&$\mb{R},\mb{C}$\\
    itengierbare Funktionen & $I\in \mb{R}$&$\mb{R},\mb{C}$\\
    Kurven & $I\in \mb{R}$&$\mb{R}^n$\\
    \hline
    stetige Abbildungen & $U\in \mb{R}^m, \mb{C}^m$ & $\mb{R}^n, \mb{C}^n$\\
    & & Grenzwerte in $\mb{R}^m$\\
    differenzierbare Funktionen & $U\in\mb{R}^n$ & $\mb{R},\mb{C}$\\
    & & partielle Ableitung\\
    differenzierbare Abbildungen & $U\in\mb{R}^n$ & $\mb{R}^n,\mb{C}^n$\\
    \hline
    integrierbare Abbildungen & $U\in\mb{R}^n$ & $\mb{R}^n,\mb{C}^n$
  \end{tabular}
  \caption{Übersicht über Funktionen / Abbildungen}
\end{table}
\section{Elemente der Topologie [Band 2, Kap 1]}
Konvergenz, Abgeschlossenheit, Stetigkeit, Häufungspunkte
\begin{Def}{euklidische Norm}
  Die euklidische Norm auf $\mb{R}^n$ ist 
  \[\Norm{x}:=\sqrt{x_1^2+x_2^2+\cdots+x^2_n}\]
\end{Def}
\begin{Eig}
  \begin{gather}
    \Norm{x}>0\ \forall x\neq 0,\ \Norm{0}=0\\
    \Norm{\lambda x}=\Abs{\lambda}\Norm{x}\ \forall x\in\mb{R}^n,\ \lambda\in\mb{R}\\
    \Norm{x+y}\leq \Norm{x}+\Norm{y}\ \forall x,y\in\mb{R}^n
  \end{gather}
\end{Eig}
\begin{Def}{euklidischer Abstand}
  Der euklidische Abstand zweier Punkte $a,b\in\mb{R}^n$ ist
  \[d(a,b)=\Norm{b-a}\]
\end{Def}
\begin{Def}{offene Kugel}
  Die offene Kugel in $\mb{R}^n$ mit Mittelpunkt $a$ und Radius $r>0$ ist die Menge
  \[K_r(a):=\left\{ x\in\mb{R}:d(x,a)\le r \right\}\]
\end{Def}
\begin{Def}{Konvergenz}
  Eine Folge $(x_k)$ in $\mb{R}^n$ heisst konvergent, wenn $\exists a\in\mb{R}^n$
  \begin{gather*}
    \Limi{k} d(x_k,a)=0\\
    x_k\in\mb{R}^n\ \forall k\\
    x_k=(x_{k1},x_{k2},\cdots,x_{kn})\\
    x_{ki}\in\mb{R}
  \end{gather*}
  Ist das der Fall, so schreibt man
  \[\Limi{k}x_k=a\]
\end{Def}
\begin{Bem}
  (geometrisch)
  \begin{gather*}
    x_k\to a\ \Lra\ \forall\varepsilon>0
  \end{gather*}
  $k_\varepsilon(a)$ fast alle Folgenglieder enthält
\end{Bem}
\begin{Lem}
  \begin{align*}
    x_k\to a\in \mb{R}^n\ \Lra&\ \ x_{ki}\to a_i\ \forall i
    =(a_1,\cdots,a_n) & \\
    \text{Konvergenz}& \ \ \ \text{komponentenweise Konvergenz}
  \end{align*}
\end{Lem}
\begin{Bew}
  $\Ra$
  \begin{gather*}
    \forall i\ \Abs{x_{ki}-a_i}\leq \Norm{x_k-a}\to 0\\
    \implies x_{ki}\to a_i\ \forall i
  \end{gather*}
  $\La$
  \begin{gather*}
    \Norm{x_k-a}\leq \sum^n_{i=1}\Abs{x_{ki}-a_i}\to 0\\
    \implies \Norm{x_k-a}\to 0
  \end{gather*}
\end{Bew}
\begin{Def}
  Eine Folge $(x_k)\in\mb{R}^n$ heisst:
  \begin{description}
    \item[beschränkt] wenn $\exists r>0$ mit $x_k\in K_r(0)$ $\forall k$
    \item[Cauchyfolge] wenn $\forall \varepsilon>0$ $\exists N$
      \[\Norm{x_k-x_l}<\varepsilon\ \forall k,l>N\]
  \end{description}
\end{Def}
\begin{Sat}{Bolzano-Weierstrass}
  \begin{enumerate}
    \item Jede beschränkte Folge besitz eine konvergente Teilfolge
    \item Jede Cauchyfolge konvergiert
  \end{enumerate}
\end{Sat}
\begin{Bew}
  \begin{enumerate}
    \item durch Indunktion nach $n$\\
      $n=1$ Beweis in $\mb{R}$\\
      Annahme: Beweis gilt in $\mb{R}^n$ $(x_k)$ beschränkt in $\mb{R}^{n+1}$
      \begin{gather*}
        \implies (x_{k1},\cdots,x_{kn}) \text{ beschränkt in }\mb{R}\\
        \implies \exists l_k:(x_{k_l1},\cdots,x_{k_ln}) \text{ konvergiert}\\
        x_{k_ln+1} \text{ beschränkt in } \mb{R}\\
        \implies \exists l_m:x_{k_{l_m}n+1} \text{ konvergiert}\\
        \implies (x_{k_{l_m}}) \text{ konvergiert}
      \end{gather*}
    \item
      \begin{gather*}
        \Abs{x_{ki}-x_{li}}\leq \Norm{x_k-x_l}\ \forall i
      \end{gather*}
      $(x_k)$ Cauchy $\implies$ $x_{ki}$ Cauchy $\forall i$ $\implies$ $x_{ki}$ konvergiert $\implies$ $x_k$ konvergiert
  \end{enumerate}
\end{Bew}
\begin{Def}{Umgebungen}
  \begin{itemize}
    \item Die offene Kugel $K_\varepsilon(a), \varepsilon>0$ heisst $\varepsilon$-Umgebung von $a\in\mb{R}^n$
    \item Eine Menge $U\subset\mb{R}$ heisst Umgebung von $a\in\mb{R}^n$, wenn sie eine $\varepsilon$-Umgebung enthält.
  \end{itemize}
\end{Def}
\begin{Eig}{Umgebungen}
  \begin{enumerate}
    \item Seien $U,V$ Umgebungen von $a$ $\implies$ $U\cap V$ und $U\cup V$ sind Umgebungen von $a$
    \item $U$ Umgebung von $a$; $V \subset U$ $\implies$ $V$ Umgebung von $a$
    \item Hausdorffsche Trennungseigenschaft: $\forall a\neq b$ $\exists U$ von $a$ und $\exists V$ von $b$ mit $U\cap V=\varnothing$
  \end{enumerate}
\end{Eig}
\begin{Bsp}
  $U=K_\varepsilon(a)$, $V=K_\varepsilon(b)$ $\varepsilon=\frac{1}{3}\Norm{b-a}$\\
  Zu beweisen mit der Dreiecksungleichung
\end{Bsp}
\begin{Def}{offene Menge}
  Eine Menge $U\subset \mb{R}^n$ heisst hoffen, wenn sie eine Umgebung von $\forall x\in U$ ist. D.h.
  \[\forall x<in U\ \exists\varepsilon>0:\ K_\varepsilon(x) \subset U\]
\end{Def}
\begin{Bsp}
  \begin{enumerate}
    \item $\mb{R}^n$ ist offen
    \item $\varnothing\in\mb{R}^n$ ist offen
    \item $K_r(a)$ ($r>0$, $a\in \mb{R}^n$) ist offen
  \end{enumerate}
\end{Bsp}
\begin{Bem}{Rechenregeln}
  \begin{enumerate}
    \item Der Durchschnitt endlich vieler offener Menge ist offen.
    \item Die Vereinigung beliebig vieler offener Menge ist offen.
  \end{enumerate}
\end{Bem}

\begin{Def}{abgeschlossene Menge}
  Eine Menge $A\subset \mb{R}^n$ heisst abgeschlossen, wenn ihr Komplement offen ist.
\end{Def}
\begin{Bsp}
  \begin{itemize}
    \item $\varnothing$
    \item $\mb{R}^n$
    \item \[\overline{K_r(a)}:=\left\{ x\in\mb{R}^n:\Norm{x-a}\leq r \right\}\]
      \[K_r(a):=\left\{ \Norm{x-a}>r \right\}\ \text{offen}\]
  \end{itemize}
\end{Bsp}
\begin{Eig}
  \begin{itemize}
    \item Die Vereinigung endlich vieler abgeschlossener Mengen ist abgeschlossen.
    \item Der Durchschnitt beliebig vieler abgeschlossener Mengen ist abgeschlossen.
  \end{itemize}
\end{Eig}
\begin{Bsp}{Gegenbeispiel (wichtig!)}
  in $\mb{R}$ $\left( -\frac{1}{n},\frac{1}{n} \right)$ offen
  \[\cap_n\left( -\frac{1}{n},\frac{1}{n} \right)=\{0\}\ \text{abgeschlossen}\]
  Dabei erinnert man sich: $\cap$ endlich offen = offen
\end{Bsp}
\begin{Sat}
  $A\subset\mb{R}^n$\\
  $A$ abgeschlossen $\Lra$ $\forall$ konvergente Folge $(a_k)$ mit $a_k\in A$ $\forall k$ konvergiert gegen $a\in A$
\end{Sat}
\begin{Bew}
  $\Ra$ Widerspruchsbeweis\\
  Annahme: $A$ abgeschlossen, $(a_k)$, $a_k\subset A$ $\forall k$, $a_k\to a$, $a\not\in A$\\
  $A$ abgeschlossen $\implies$ $A^C=\mb{R}^b\setminus A$ offen\\
  $a\not\in A$ $\implies$ $a\in A^C$\\
  $\implies$ $A^C$ ist eine Umgebung von $a$ $\implies$ $X^C$ enthält unendlich viele $a_k$ Widerspruch, denn $a_k\not\in A^C$ $\forall k$\\
  $\La$ Kontrapositionsbeweis\\
  Sei $A$ nicht abgeschlossen, dann ist $A^C$ nicht offen.
  \[\implies a\in A^C:\ \forall \varepsilon >0\]
  \[K_\varepsilon(a)\not\subset A^C\]
  insbesondere $\varepsilon=\frac{1}{k}$ $k\in \mb{N}$\\
  Sei
  \[a_k\in K_\frac{1}{k}(a),\ a_k\not\in A^C\]
  \begin{enumerate}
    \item $a_k\in A$ $\forall k$
    \item $a_k\to a$ (da $\Norm{a_k a}<\frac{1}{k}$)
    \item $a\not\in A$
  \end{enumerate}
\end{Bew}
\begin{Def}{Randpunkt von $M$}
  Sei $M\subset\mb{R}^n$, $x\in\mb{R}^n$ $x$ heisst Randpunkt von $M$, wenn jede Umgebung von $x$ Punkte aus $M$ \underline{und} aus $M^C$ enthält.
\end{Def}
\begin{Not}{Randpunkte von $M$}
  \[\partial M:\left\{ \text{Randpunkte von } M \right\}\]
\end{Not}
\begin{Bem}
  \[\partial(M^C)=\partial M\]
\end{Bem}
\begin{Bsp}
  \[\partial K_r(a)=S_r(a):=\left\{ x\in\mb{R}^n:\Norm{x-a}=r \right\}=\partial\overline{K_r(a)}\]
  Übung: zeigen Sie das. Tipp: $x\in S_r(a)$ $K_\varepsilon(x), \varepsilon<r$
\end{Bsp}
\begin{Bsp}
  $\mb{Q}\subset\mb{R}$
  \[\partial\mb{Q}=\mb{R}\]
\end{Bsp}
\begin{Sat}
  Sei $M\in\mb{R}^n$
  \begin{enumerate}
    \item \begin{enumerate}
        \item $U\subset M$, $U$ offen $\implies$ $U\subset M\setminus \partial M$
        \item $M\setminus\partial M$ ist offen
      \end{enumerate}
    \item\begin{enumerate}
        \item $A\supset M$, $A$ angeschlossen $\implies A\supset M \cup \partial M$
        \item $M\cup \partial M$ abgeschlossen
      \end{enumerate}
    \item\begin{enumerate}
        \item $\partial M$ abgeschlossen
      \end{enumerate}
  \end{enumerate}
\end{Sat}
\begin{Bew}
  \begin{enumerate}
    \item \begin{enumerate}
        \item zu zeigen: $\partial M\cap U=\varnothing$. Widerspruchsbeweis: Sei $\partial M\cap U\neq \varnothing$ Sei $x\in \partial M\cap U$ $\implies$ $U$ Umgebung von $x$ und $x\in\partial M$ $\implies$ $U$ enthält aus $M^C$ Widerspruch, denn $U\subset M$
        \item Sei $a\in M\setminus \partial M$. Dann gibt es eine Umgebung $U$ von $a$ mit $U\subset M$ sonst wäre $a\in \partial M$ 1a $\implies$ $U\subset M\setminus \partial M$
      \end{enumerate}
    \item\begin{enumerate}
        \item Komplement
        \item Komplement
      \end{enumerate}
    \item\begin{enumerate}
        \item Durchschnitt zweier abgeschlossener Mengen
          \[\partial M=(M\cup\partial M)\cap(M^C\cup \partial M^C)\]
      \end{enumerate}
  \end{enumerate}
\end{Bew}
\begin{Kor}
  \[U\ \text{abgeschlossen}\ \Lra\ U\ \text{alle ihre Randpunkte enthält}\]
\end{Kor}
\begin{Not}{offener Kern/Innere, abgeschlossene Hülle}
  $M^0:=M\setminus\partial M$ der offene Kern von $M$ oder das $Innere$ von $M$. Die grösste offene Menge, die in $M$ liegt.\\
  $\overline{M}:=M\cup\partial M$ die abgeschlossene Hülle von $M$. Die kleinste abgeschlossene Menge, die $M$ umfasst.
\end{Not}
\begin{Def}{Häufungspunkt}
  Sei $M\subset \mb{R}^n$, $x\in\mb{R}^n$ $x$ heisst Häufungspunkt von $M$ wenn jede Umgebung von $x$ ein $y\in M$ enthält mit $y\neq x$.\\
  äquivalent: Jede punktierte Umgebung von $x$ enthält Punkte aus $M$
  \[\mathcal{H}(M):=\left\{ \text{Häufungspunkte} \right\}\]
  \[\mathcal{H}(Kr(a))=\mathcal{H}(\overline{Kr(a)})=S_r(a)=\partial K_r(a)\]
  im Allgemeinen: $\partial M\neq \mathcal{H}(M)$
\end{Def}
\begin{Bsp}
  $M=\mb{R}\in\mb{R}$
  \begin{align*}
    \mathcal{H}(\mb{R})=\mb{R} && \partial\mb{R}=\varnothing
  \end{align*}
\end{Bsp}
\begin{Bsp}
  $M=\{a\}\subset\mb{R}$
  \begin{align*}
    \mathcal{H}(\{a\})=\varnothing && \partial\{a\}=a
  \end{align*}
\end{Bsp}
\begin{Lem}
  Sei $M\subset\mb{R}^n$
  \[M\cup\mathcal{H}(M)=M\cup\partial M=\overline{M}\]
\end{Lem}
\begin{Bew}
  zu zeigen: 
  \begin{enumerate}
    \item $\mathcal{H}\setminus M\subset\partial M$
    \item $\partial M\setminus M\subset\mathcal{H}(M)$
  \end{enumerate}
  \begin{enumerate}
    \item Sei $x\in\mathcal{H}\setminus M$ $\implies$ Jede Umgebung von $x$ enthält ein $y$ mit $y\in M$, $x\neq y$
      \[U\ni x\in M^C\] %1. \in umkehren!
      $\implies$ Jede Umgebung von $x$ enthält Punkte in $M$ und aus $M^C$ $\implies$ $x\in \partial M$
    \item $x\in \partial M\setminus M$. Jede Umgebung von $x$ enthält ein $y\in M$
      \[x\in M^C\implies y\neq x\implies x\in\mathcal{H}(M)\]
  \end{enumerate}
\end{Bew}
\begin{Kor}
  $A$ abgeschlossen $\Lra$ $A$ enthält alle ihre Häufungspunkte.
\end{Kor}
\subsection{Verallgemeinerung: Normierte Räume}
\begin{Def}{Norm}
  Sei $\mb{K}=\mb{R}$ oder $\mb{C}$ als Körper.\\
  Sei $V$ ein Vektorraum über $\mb{K}$ Eine Norm auf $V$ ist eine Abbildung
  \[\Norm{\ }:V\to\mb{R}\]
  s.d.
  \begin{enumerate}
    \item \[\Norm{0}=0,\ \Norm{x}>0, \forall x\in V\setminus\{0\}\]
    \item \[\Norm{\lambda x}=\Abs{\lambda}\Norm{x}\ \forall \lambda\in\mb{K}\ \forall x\in V\]
    \item \[\Norm{x+y}\leq \Norm{x}+\Norm{y}\ \forall x,y\in V\]
  \end{enumerate}
\end{Def}
\begin{Def}{normierter Raum}
  Das Paar $(V,\Norm{\ })$ heisst normierter Raum.
\end{Def}
\begin{Bsp}
  $\mb{R}^n$ mit der euklidischen Norm
\end{Bsp}
\begin{Bsp}{$p$-Norm}
  $\mb{K}^n$ mit der $p$-Norm $p\geq 1$
  \[\Norm{x}_p:=\sqrt[p]{\sum^n_{i=1}\Abs{x_i}^p}\]
  ($p=2$ euklidisch)
\end{Bsp}
\begin{Bsp}{Maximumsnorm}
    $\mb{K}^n$ mit der Maximumnorm
    \[\Norm{x}_\infty:=\max\left\{ \Abs{x_1},\abs{x_2},\dots,\Abs{x_n} \right\}\]
    Lemma: \[\Norm{x}_\infty=\Limi{p}\Norm{x}_p\]
\end{Bsp}
\begin{Bsp}{$L^p$-Norm}
  $\mathcal{C}^0([a;b],\mb{K})$ mit der $L^p-Norm$, $p\geq 1$
  \[\Norm{f}_p=\sqrt[p]{\int_a^b\Abs{f(x)}^p\md x}\]
  $p=2$ ist für die Quantenmechanik interessant
\end{Bsp}
\begin{Bsp}{Supremumsnorm}
  $\mathcal{C}([a;b],\mb{K})$ mit der Supremumsnorm
  \[\Norm{f}_\infty:=\sup\left\{ \Abs{f(x)},x\in [a;b] \right\}\]
\end{Bsp}
\begin{Bsp}
  Sei $\left\langle , \right\rangle$ ein Skalarprodukt auf $V$. Dann ist
  \[\Norm{x}=\sqrt{\left\langle x,x \right\rangle }\]
  eine Norm.
\end{Bsp}
\begin{Bem}
  Alles was wir bisher bewiesen haben, gilt auf beliebigen normierte Räumen.
\end{Bem}
\subsection{Verallgemeinerung: Metrische Räume}
\begin{Def}{Abstand, metrischer Raum}
  Sei $X$ eine Menge. Eine Metrik auf $X$ ist eine Abbildung
  \[d:X\times X\to\mb{R}\]
  s.d.
  \begin{enumerate}
    \item $d(x,x)=0$, $d(x,y)>0$ $\forall x,y\in x$ mit $x\neq y$
    \item $d(x,y)=d(y,x)$
    \item $d(x,y)\leq d(x,z)+d(z,y)$ $\forall x,y,z\in x$
  \end{enumerate}
  \begin{itemize}
    \item Die Zahl $d(x,y)$ heisst Abstand der Punkte $x$ und $y$.
    \item Das Paar $(X,d)$ heisst metrischer Raum.
  \end{itemize}
\end{Def}
\begin{Bsp}
  $(V,\Norm{\ })$ normierter Raum
  \[d(x,y):=\Norm{x-y}\]
\end{Bsp}

\begin{Bsp}
  $M$ nicht leere Menge
  \[d(x,y):=\begin{cases}
    0&x=y\\1&x\neq y
  \end{cases}\]
\end{Bsp}
\begin{Bsp}
  Sei $\gamma:I\to\mb{R}^n$ fast überall differenzierbar und regulär
  \[M=I,\ d(x,y):=\Abs{\int^y_x\Norm{\dot\gamma(t)}\md t}\]
  Länge zwischen $\gamma(x)$ und $\gamma(y)$.
\end{Bsp}
\begin{Def}
  \[K_r(a):=\left\{ x\in M:d(x,a)<r \right\}\]
  offene Kugel
  \begin{itemize}
    \item $K_\varepsilon(a)$ $\varepsilon$-Umgebungen von $a$
    \item Umgebungen
    \item \ldots
  \end{itemize}
\end{Def}
\begin{Bem}
  Es gelten die gleichen Rechenregeln für offene Mengen
\end{Bem}
\begin{Def}{Durch $d$ erzeugte Topologie}
  $U\subset X$ heisst offen, wenn $U$ eine Umgebung von jedem $x\in U$ ist.
  \[\mathcal{O}(d):=\left\{ \text{offene Mengen von $X$ bez. $d$} \right\}\subset P(x)\]
  die durch $d$ erzeugte Topologie.
\end{Def}
\begin{Def}
  $A$ ist abgeschlossen, wenn $A^C$ offen ist.
\end{Def}
\begin{Def}
  Eine Folge $(x_k)$ in $(X,d)$ heisst konvergent, wenn $\exists x\in X$ mit
  \[\Limi{k}d(x_k,x)=0\]
\end{Def}
\begin{Lem}
  Eine Folge besitzt höchstens einen Grenzwert.
\end{Lem}
\begin{Bew}
  Seien $x,x'\in X$
  \begin{gather*}
    \lim d(x,x_k)=0\\
    \lim d(x',x_k)=0\\
    0\leq d(x,x')\leq d(x,x_k)+d(x_k,x')\\
    \implies x=x'
  \end{gather*}
\end{Bew}
\begin{Sat}
  $A\subset X,d$\\
  $A$ abgeschlossen $\Lra$ $\forall$ konvergente Folge $(x_k)$ mit $x_k\in A\forall k$ gegen ein Element von $A$ konvergiert
\end{Sat}
\subsection{Teilraumtopologie}
\begin{Def}{induzierte Metrix / Spurmetrik}
  Sei $(X,d)$ metrischer Raum. Sei $X_0\subset X$. Man definiert
  \[d_0:=X_0\times X_0\to\mb{R}\]
  \[d_0:=d|_{X_0\times X_0}\]
  $\forall x,y\in X_0$
  \[d_0(x,y)=d(x,y)\]
\end{Def}
\begin{Lem}
  $d_0$ ist eine Metrik.
\end{Lem}
\begin{Def}{Spurtopologie}
  $a\in X_0$
  \[K_r^{d_0}(a):=\left\{ x\in X_0:d_0(x,a)<r \right\}\]
  \[K_r^{d_0}(a):=K_r(a)\cap X_0\]
  $\implies$
  \[\mathcal{O}d_0=\left\{ U\cap X_0,U\in \mathcal{O}(d) \right\}\]
\end{Def}
\begin{Not}
  $X_0$-offen bedeutet $X_0$ bezüglich der Spurtopologie
\end{Not}
\begin{Bem}
  $X_0$-offen $\not\implies$ offen in $X$
\end{Bem}
\begin{Bsp}
  $X=\mb{R}$ (euklidisch) ,$X_0=\mb{Q}$. $\mb{Q}\subset X_0$ ist $X_0$-offen. $\mb{Q}$ ist $\mb{Q}$-offen, aber nicht offen in $\mb{R}$.
\end{Bsp}
\begin{Lem}
  $U\subset X_0$ Ist $X_0$ offen in $X$, dann $U$ ist $X_0$-offen $\Lra$ $U$ offen in $X$
\end{Lem}
\begin{Bew}
  $U$ $X_0$ offen $\implies \exists V\subset X$, offen s.d. $U=V\cap X_0$ $\implies$ $U$ offen
\end{Bew}
\subsection{Produkttopologie}
\begin{Def}{Produkttopologie}
  $(X,d_x)$, $(Y,d_y)$ metrische Räume. Man definiert $d$ auf $X\times Y$
  \[d:(X\times Y)\times(X\times Y)\]
  \[d\left( (x_1,y_1),(x_2,y_2) \right):=\max\left\{ d_x(x_1,x_2),d_y(y_1,y_2) \right\}\]
  $x_1,x_2\in X$ $y_1,y_2\in Y$
\end{Def}
\begin{Lem}
  $d$ ist eine Metrik auf $X\times Y$
\end{Lem}
\begin{Bem}{offene Kugeln}
  \[K_r^d\left( (x,y) \right):=\left\{ (\tilde x,\tilde y)\in X\times Y:\max\left\{ d_x(\tilde x,x),d_y(\tilde y,y) \right\} \right\}<r\]
  \[=\left\{ (\tilde x,\tilde y)\in X\times Y: \stackrel{d_x(\tilde x,x)<r}{d_y(\tilde y,y)<r} \right\}\]
  \[K_r^d\left( (x,y) \right)=K_r^{d_x}(x)\times K_r^{d_y}\]
  $\implies$
  \[W\subset X\times Y\ \text{offen}\ \Lra\ \forall (x,y)\in W\ \exists\]
  Umgebung $U$ von $x$ in $X$ und Umgebung $V$ von $y$ in $Y$ s.d
  \[W\subset U\times V\]
\end{Bem}
\begin{Bsp}
  Sind $U\subset X$ und $V\subset Y$ offen, dann ist $U\times V$ offen in $X\times Y$
\end{Bsp}
\subsection{Äquivalenz Metriken und Normen}
\begin{Def}{äquivalente Metriken}
  Seien $d$ und $d^*$ Metriken auf $X$. Sie heissen äquivalent, wenn
  \[\mathcal{O}(d)=\mathcal{O}(d^*)\]
\end{Def}
\begin{Lem}
  Zwei Metriken $d$ und $d^*$ sind genau dann äquivalent, wenn jede $d$-Kugel eine $d^*$-Kugel enthält mit demselben Mittelpunkt und umgekehrt.
  \label{top:equi}
\end{Lem}
\begin{Bew}
  $\Ra$ trivial\\
  $\La$ Eine $d$-Umgebung $U$ enthält eine $d$-Kugel, deshalb enthält sie eine $d^*$-Kugel, deshalb ist sie eine $d^*$-Umgebung
\end{Bew}
\begin{Def}{äquivalente Normen}
  Zwei Normen $\Norm{\ }$ und $\Norm{\ }^*$ auf $V$ heissen äquivalent, wenn sie äquivalente Metriken erzeugen.
\end{Def}
\begin{Lem}
  $\Norm{\ }$ und $\Norm{\ }^*$ sind genau dann äquivalent, wenn
  \[\exists c>0 \text{ und } C>0 \text{ s.d. }\ \forall x\in V\]
  \[c\Norm{x}\leq \Norm{x}^*\leq C\Norm{x}\]
\end{Lem}
\begin{Not}
  $K$ offene Kugel bezüglich $\Norm{\ }$ 
  $K^*$ offene Kugel bezüglich $\Norm{\ }^*$ 
\end{Not}
\begin{Bew}
  $\Ra$ $\Norm{\ }$ und $\Norm{\ }^*$ äquivalent. Lemma \ref{top:equi} $\implies$ $K_1(0)$ enthält eine Kugel $K_r^*(0), r>0$
  \begin{gather*}
    x=0\\
    x\neq 0, y:=\frac{rx}{2\Norm{x}^*}\\
    \Norm{y}^*=\frac{r}{2}<r\\
    \implies y\in K_r^*(0)\implies y\in K_1(0)\implies \Norm{y}<1\\
    \Norm{y}=\frac{r}{2}\frac{\Norm{x}}{\Norm{x}^*}\\
    \Norm{x}^*>\frac{r}{2}\Norm{x}\\
    c:=\frac{r}{2}
  \end{gather*}
  $\La$ \[K^*_{cr}(a)\subset K_r(a)\subset K_{Cr}^*(a)\]
  $\implies$ Metriken sind äquivalent.
\end{Bew}
\begin{Sat}
  Je zwei Normen auf einem endlichdimensionalen $\mb{K}$-Vektorraum sind äquivalent.
\end{Sat}
\begin{Bew}
   Sei $V=\mb{R}^n$ mit Norm $\Norm{\ }$. wir zeigen, $\Norm{\ }$ äquivalent zu $\Norm{\ }_2$ euklidisch.
  \begin{enumerate}
    \item 
      \[\exists C>0:\Norm{x}\leq C\Norm{x}_2\]
      Sei $\left\{ e_y \right\}_{\nu=1,\cdots,n}$ Standardbasis von $\mb{R}^n$
      \[x\in V:x=\sum^n_{\nu=1}x_\nu e_\nu,\ x_\nu\in \mb{R}\]
      \[\Norm{x}\leq \sum^n_{\nu=1}\Abs{x_\nu}\Norm{e_\nu}\]
      \[\Abs{x_\nu}\leq \Norm{x}_2\]
      $\implies$
      \[\Norm{x}\leq\Norm{x_2}\underbrace{\sum^n_{\nu=1}\Norm{e_\nu}}_{=:C}\]
    \item
      \[\exists c>0:-c\Norm{x}_2\leq \Norm{x}\]
      Sei $S:=\left\{ x\in\mb{R}^n:\Norm{x}_2=1 \right\}$ (euklidische Einheitssphäre)
      \[c:=\inf\left\{ \Norm{x}:x\in S \right\}\]
      $x\neq 0$, $y:=\frac{x}{\Norm{x}_2}$ $\implies y\in S$
      \[\implies c\leq \Norm{y}=\frac{\Norm{x}}{\Norm{x}_2}\]
      \[\implies c\Norm{x}_2\leq\Norm{x}\]
      zu zeigen: $c>0$
  \end{enumerate}
\end{Bew}
\begin{Lem}
  $c>0$
\end{Lem}
\begin{Bew}{Widerspruchsbeweis}
  Annahme $c=0$
  \[\implies \exists (x_k),\ x_k\in S< \Norm{x_k}\xrightarrow{k\to\infty}0\]
  $(x_k)$ beschränkt bezüglich $\Norm{\ }_2$ BW $\implies$ $\exists$ bez. $\Norm{\ }_2$ konvergente Teilfolge $x_{k_l}$ d.h. $\exists a\in \mb{R}^n$
  \[\Limi{l}\Norm{x_{k_l}-a}_2=0\]
  \[\implies a_\nu=\Limi{l}x_{k_l,\nu}\]
  Konvergenz bez. $\Norm{\ }_2$ $\Lra$ komponentenweise Konvergenz
  \[\Norm{a}_2^2\sum_{\nu}a_\nu^2=\Limi{l}\sum_\nu\left( x_{k_l,\nu} \right)^2=1\]
  $\implies a\in S$
  \[\Norm{a}\leq \Norm{a-x_{k_l}}+\Norm{x_{k_l}}\]
  \[\stackrel{a}{\leq}\underbrace{C\Norm{a-x_{k_l}}_2}_{\to 0}+\underbrace{\Norm{x_{k_l}}}_{\to 0}\]
  \[\implies \Norm{a}=0\implies a=0\]
  Widerspruch, denn $0\not\in S$
\end{Bew}
\begin{Bew}
  Sei $(V,\Norm{\ })$ normierter endlichdimensionaler Vektorraum $\dim V=n$
  \[\exists\phi:\mb{R}^n\to V\ \text{Isomorpisnus}\]
  \[\Norm{x}_\phi:=\Norm{\phi(x)}\]
  $\Norm{\ }^*$ eine zweite Norm auf $V$
  \[\Norm{x}^*_\phi:=\Norm{\phi(x)}^*\]
\end{Bew}
\begin{Kor}
  Für jede $\Norm{\ }$ auf $\mb{R}^n$
  \[\Norm{x_k-a}\to 0\ \Lra\ x_{k,\nu}\to a_\nu \forall \nu\]
\end{Kor}

\section{Stetigkeit}
\begin{Def}{setig}
  Seien $(X,\md_x)$ und $(Y,\md_y)$ metrische Räume. Eine Abbildung $f:X\to Y$ heisst stetig im Punkt $a\in X$, wenn
  \[\forall \varepsilon>0\exists\delta>0:\md_y(f(x),f(a))<\varepsilon\ \forall x\in X\ \text{mit}\ \md_x(x,a)<\delta\]
\end{Def}
\begin{Not}
  Sind $X\subset\mb{R}$ und $Y\subset\mb{R}$ dann sind die durch irgend eine Norm erzeugten Symmetriken zu nehmen.
\end{Not}
\begin{Def}{Lipschitz-Stetigkeit} $f:X\to Y$ heisst Lipschitz-stetig, wenn
  \[\exists L\geq 0:\forall x,x'\in X:\md_y(f(x),f(x))\leq L\md_x(x,x')\]
\end{Def}
\begin{Lem}
  Lipschitz-stetig $\implies$ stetig.
\end{Lem}
\begin{Bsp}
  Folgende Abbildungen sind Lipschitz-stetig und deshalb stetig.
  \begin{enumerate}
    \item $f:V\to W$ $V,W$ normierte Vektorräume, $f$ linear und $V$ endlich dimensional
    \item $\Norm{\ }:V\to\mb{R}$
    \item Abstandfunktion: Sei $(x,d)$ metrischer Raum $\varnothing\neq A\subset A$, $x\in X$ Abstand zwischen $x$ und $A$:
      \[d_A(x):=\inf\left\{ d(x,a):a\in A \right\}\]
      $d_A:x\to\mb{R}$ ist Lipschitz-stetig.
  \end{enumerate}
  \begin{enumerate}
    \item Sei $\left\{ e_1,\cdots,e_n \right\}$ eine Basis von $V$, seien $x,y\in V$
      \[x=\sum^n_{i=0}x_ie_i,\ y=\sum^n_{i=0}y_ie_i,\]
      \begin{gather*}
        f(x)-f(y)\stackrel{\text{linear}}{=} f(x-y)= \sum^n_{i=1} (x_i-y_i) f(e_i) \\
        \Norm{f(x)-f(y)}_W\leq\sum^n_{i=0}\Abs{x_i-y_i}\Norm{f(e_i)}_W\\
        M:=\max\left\{ \Norm{f(e_i)}_W,\cdots,\Norm{f(e_n)} \right\}\\
        \Norm{f(x)-f(y)}_W\leq M\sum^n_{i=1}\Abs{x_i-y_i}\\
        \Norm{y}^*_V:=\sum^n_{i=1}\Abs{y_i}\ \text{eine Norm auf $V$}\\
        \Norm{f(x)-f(y)}_W\leq M\Norm{x-y}^*_V
      \end{gather*}
      Je zwei Normen auf einem endlich dimensionalen Vektorraum sind äquivalent.
      \begin{gather*}
        \implies \exists C>0: \Norm{y}_v^*\leq C\Norm{y}_V\\
        \implies \Norm{f(x)-f(y)}_W\leq L\Norm{x-y}_V\\
        L=MC
      \end{gather*}
      $\qed$
  \end{enumerate}
\end{Bsp}
\begin{Def}{Folgenstetigkeit}
  $f:X\to Y$ metrischer Räume heisst folgenstetig in $x\in X$, wenn
  \[x_k\to x\implies f(x_k)\to f(x)\]
\end{Def}
\begin{Lem}
  $f$ stetig $\Lra$ $f$ folgenstetig.
\end{Lem}
\begin{Bsp}{Gegenbeispiel}
  Sei $V=\mathcal{C}^1([a;b],\mb{R})$, $W=\mb{R}$, $a<0<b$
  \begin{align*}
    D:&V\to W\\
    &f\mapsto f'(0)
  \end{align*}
  $D$ ist linear, aber nicht stetig. eigentlich $D$ nicht folgenstetig. Sei
  \begin{gather*}
    f_n=\frac{1}{n}\sin(nx)\in V\ \forall n\\
    \Norm{f_n}=\sup\Abs{f_n}\leq \frac{1}{n}\to 0\\
    \implies f_n\to 0\\
    D f_n=\cos(nx)|_{x=0}=1\\
    D f_n\not\to D 0=0
  \end{gather*}
\end{Bsp}
\begin{Sat}{(Königsberger, 1.3.V)}
  Seien $V,W$ normierte Vektorräume, $f:V\to W$ linear
  \[f\ \text{stetig}\ \Lra\ \exists C:\Norm{f(x)}_W\leq C\Norm{x}_V\ \forall x\in V\]
  $f$ heisst beschränkt.
\end{Sat}
\begin{Bem}
  Ist $V$ endlichdimensional, dann ist $f$ automatisch beschränkt.
  \[\Norm{f(x)}_W=\Norm{f\left(\sum x_i e_i\right)}\leq \sum \Abs{x_i}\Norm{f(e_i)}_W\leq M\sum\Abs{x_i}=M\Norm{x_i}^*_V\leq MC\Norm{x}_V\]
\end{Bem}
\begin{Bew}
  $\Ra$ $f$ stetig $\implies$ $f$ stetig in 0
  \begin{gather*}
    \forall \varepsilon>0\exists \delta>0:\Norm{f(\xi)-f(0)}\leq \varepsilon\\
    \Norm{\xi -0}_W<1
  \end{gather*}
  insbesondere
  \begin{gather*}
    \varepsilon=1\ \exists\delta:\Norm{f(\xi)}_W<1\ \forall \xi \Norm{\xi}\leq \delta
  \end{gather*}
  Sei $x\in V\setminus \left\{ 0 \right\}$, $y:=\delta\frac{x}{\Norm{x}_V}$
  \begin{gather*}
    \Norm{y}_V=\delta\implies \Norm{f(y)}_W\leq 1\\
    \Norm{f(y)}_W=\Norm{\frac{\delta}{x}f(x)}_W=\frac{\delta}{\Norm{x}_V}\Norm{f(x)}_W\\
    \implies \Norm{f(x)}_W\leq \frac{1}{\delta}\Norm{x}_V,\ C=\frac{1}{\delta}
  \end{gather*}
  $\La$
  \[\Norm{f(x)-f(y)}_W=\Norm{f(x-y)}_W\leq C\Norm{x-y}_V\implies f\ \text{Lipschitzstetig} \implies f \ \text{stetig}\]
\end{Bew}
\begin{Bem}{Rechenregel I}
  Seien $f_1,f_2:a\in X\to W$ $X$ metrischer Raum und $W$ normierter Vektorraum. Sind $f_1$ und $f_2$ stetig in $a$, so ist $f_1+f_2$ stetig in $a$.
  \begin{enumerate}
    \item Ist zusätzlich $W=\mb{R}$, $f_1,f_2$ stetig in $a$ $\implies$ $f_1\cdot f_2$ stetig in $a$.
    \item Ist zusätzlich $f_2(a)\neq 0$, dann $\frac{f_1}{f_2}$ stetig in $a$
  \end{enumerate}
\end{Bem}
\begin{Def}{Polynomfunktion}
  Eine Funktion $f:\mb{R}^n\to\mb{R}$ heisst Polynomfunktion, wenn sie durch endliche Addition und Multiplikation der Koordinaten erzeugt wird. Eine Polynomfunktion ist immer stetig.
\end{Def}
\begin{Def}{rationale Funktion}
  $f:\mb{K}_{\subset\mb{R}^n}\to\mb{R}$ heisst rational, wenn sie als Quotient von Polynomfunktionen geschrieben werden kann.
\end{Def}
\begin{Kor}
  Jede rationale Funktion ist ihrem Definitionsbereich stetig.
\end{Kor}
\begin{Bem}{Rechenregel II}
  Seien $f:X\to Y$ und $g:Y\to Z$ Sei $f$ stetig in $a\in X$ und $g$ stetig in $f(a)\in Y$, dann ist $g\circ f$ stetig in $a$  
\end{Bem}
\begin{Bem}{Rechenregel III}
  Seien $f_1:X\to Y_1$ und $f_2:X\to Y_2$ und $X,Y_1,Y_2$ metrische Räume. Man definiert
  \begin{align*}
    f:=f_1\times f_2:&X\to Y_1\times Y_2\\
    &x\mapsto (f_1(x),f_2(x))
  \end{align*}
  \[f\ \text{stetig in}\ a\in X\ \Lra\ f_1\ \text{und}\ f_2\ \text{stetig in}\ a\in X\]
\end{Bem}
\begin{Kor}
  $f:X\to\mb{R}^n$ stetig in $a$ $\Lra$ Alle Komponentenfunktionen $f_1,\cdots,f_n$ stetig in $a$
\end{Kor}
\begin{Bsp}
  Kurven $I\to\mb{R}^n$  
\end{Bsp}
\begin{Bem}{(wichtig!)}
  Sei $f:\mb{R}^n\to\mb{R}^n$ Die Stetigkeit aller Einschränkung von $f$ auf den Koordinatenachsen impliziert die Stetigkeit von $f$ \underline{nicht}
\end{Bem}
\begin{Bsp}
  $f:\mb{R}^2\to\mb{R}$
  \[f(x,y)=\begin{cases}
    \frac{2xy}{x^2+y^2}&(x,y)\neq(0,0)\\ 0&(x,y)=0
  \end{cases}\]
  $f$ ist nicht stetig in $0$
  \[f(t,t)=\frac{2t^2}{t^2+t^2}=1\]
  $t\neq 0$, $x=y=t$
  \[\Norm{f(t,t)-f(0,0)}=1\ \forall t\neq 0\]
  Sei \[\left( \frac{1}{k},\frac{1}{k} \right)\to 0\ \neq\ f\left(\frac{1}{k},\frac{1}{k}\right)\to 1\] $\implies$ $f$ nicht stetig.\\
  $f(x,0)=0 \forall x$, $f(y,0)=0 \forall y$ sind stetig\\
  $c\in \mb{R}$
  \begin{gather*}
    f_c(x):=f(x,c)\\
    \tilde f_c(y):=f(c,y)
  \end{gather*}
  $\forall c$ $f_c,\tilde f_c:\mb{R}\to\mb{R}$ stetig $\forall c$
\end{Bsp}
\begin{Bsp}
  \begin{gather*}
    x=r\cos\phi,\ y=r\sin\phi\\
    f(x,y)=\frac{2r^2\sin\phi\cos\phi}{r^2}\\
    f(x,y)=\sin2\phi,\ (x,y)\neq 0
  \end{gather*}
  In jeder Umgebung von 0 nimmt die Funktion all seine Werte an.
\end{Bsp}
\begin{Sat}
  Seien $X,Y$ metriche Räume $f:X\to Y$, $f$ stetig in $a$ $\Lra$ $\forall$ Umgebung $V$ von $f(a)$ $\exists$ Umgebung $U$ von $a$ mit $f(U)\subset V$
\end{Sat}
\begin{Kor}
  $f:X\to Y$ metrische Räume. Dann sind folgende Aussagen äquivalent:
  \begin{enumerate}
    \item $f$ ist stetig auf $X$
    \item das Urbild jeder offenen Menge aus $Y$ ist offen in $X$
    \item das Urbild jeder abgeschlossenen Menge aus $Y$ ist abgeschlossen in $X$
  \end{enumerate}
\end{Kor}
\begin{Kor}
  $f:x\to\mb{R}$ stetig, sei $c\in\mb{R}$
  \[U:=\left\{ x\in y: f(x)<c \right\}\ \text{ist offen}\]
  \[A:=\left\{ x\in y: f(x)\leq c \right\}\ \text{ist abgeschlossen}\]
\end{Kor}
\begin{Bem}
  Das \underline{Bild} einer offenen Menge kann nicht offen sein.
\end{Bem}
\begin{Bsp}
  $\sin(0;2\pi)\to\mb{R}$ $\sin(0;2\pi)=[-1;1]$
\end{Bsp}
\begin{Bem}
  Die Umkehrung einer stetigen Funktion ist im Allgemeinen nicht stetig.
\end{Bem}
\begin{Bsp}
  \begin{align*}
    f:&[0;2\pi)\to S^1\\
    x\mapsto e^{ix}
  \end{align*}
  bijektiv und stetig.
  \[g:S^1\to[0;2\pi)\]
  ist nicht stetig.
  \[g(e^{ix})=x\ e^{ix}\neq 1\ g(1)=0\]
  \begin{gather*}
    x_k=e^{\left( 2\pi-\frac{1}{k} \right)i}\\
    x_k\to 1\in S^1\\
    g(x_k)=2\pi-\frac{1}{k}\\
    g(x_k)\not\to 0=g(1)
  \end{gather*}
\end{Bsp}

\begin{Def}{Homöomorphismus}
  $f:X\to Y$ heisst
  \begin{enumerate}
    \item $f$ stetig
    \item $f$ ist umkehrbar
    \item $f^{-1}Y\to X$stetig
  \end{enumerate}
\end{Def}
\begin{Eig}{Homöomorphismus}
  In diesem Falle sind auch die Bilder offener Mengen offen.  
\end{Eig}
\begin{Bsp}
  $V,W$ endlich dimensionale Vektorräume, $f:V\to W$ linear umkehrbar $\implies$ $f$ Homöomorphismus
\end{Bsp}
\begin{Def}{homöomorphe Räume}
  Zwei metrische Räume $X,Y$ heissen homöomorph, wenn es einen Homöomorphismus $X\to Y$ gibt.
\end{Def}
\begin{Bem}
  $\phi:X\to Y$ und $\psi:Y\to Z$ Homöomorphismus $\psi\circ\phi$ Homöomorphismus.
\end{Bem}
\begin{Bsp}
  Zwei endlich dimensionale Vektorräume der \underline{gleichen} Dimension sind homöomorph.
\end{Bsp}
\begin{Bem}
  Man kann zeigen: Zwei endlich dimensionale Vektorräume sind genau dann homöomorph, wenn sie die gleiche Dimension haben. Im Allgemeinen $\not\exists$ Homöomorphismus $\mb{R}^m\to\mb{R}^n$ $m\neq n$
\end{Bem}
\begin{Bsp}
  $K_1(0)\in\mb{R}^n$ sind Homöomorph.
  \begin{align*}
    f:&K_1(0)\to\mb{R}^n\\
    &x\mapsto \frac{x}{1-\Norm{x}}\\
    g:&\mb{R}^n\to K_1(0)\\
    &y\mapsto \frac{y}{1+\Norm{y}}
  \end{align*}
\end{Bsp}
\begin{Bem}
  Polarkoordinaten
  \begin{gather*}
    \mb{R}^2\to\mb{R}^2\\
    (r,\phi)\mapsto \left( r\cos \phi, r\sin \phi \right)
  \end{gather*}
  stetig, nicht bijektiv\\
  $r>0$, $\phi\in\left( -\pi, \pi \right)$, Bild: $\mb{R}^2\setminus S$, $S=\left\{ (x,0)\in\mb{R}^2, x\leq 0 \right\}$
  \begin{align*}
    P_2:&\mb{R}^+_*\times (-\pi,\pi)\to\mb{R}^2\setminus S\\
    &(r,\phi)\mapsto\left( r\cos \phi, r\sin \phi \right)
  \end{align*}
  Homöomorphismus\\
  Umkehrabbildung: $r=\sqrt{x^2+y^2}$, $\phi \sign(y)\arccos\frac{x}{\sqrt{x^2+y^2}}$
\end{Bem}
\begin{Bem}
  3d
  \begin{align*}
    \begin{cases}
      x_1=r\cos \phi_1\cos\phi_2\\
      x_2=r\sin\phi_1\cos\phi_2\\
      x_3=r\sin\phi_2
    \end{cases}\\
    r>0\\
    \phi_1\in\left( -\pi,\pi \right)\\
    \phi_2\in\left( -\frac{\pi}{2},\frac{\pi}{2} \right)
  \end{align*}
  Bild $\mb{R}^3\setminus(S\times \mb{R})$
  \[\Mx{x_1\\x_2}=\Mx{r\cos\phi_1\\r\sin\phi_1}\cos\phi_2=P_2(r,\phi_1)\cos\phi_2\]
  Im allgemeinen definiert man Polarkoordinaten rekursiv
  \begin{gather*}
    P_n:\mb{R}^+_*\times\prod_{n-1}\to\mb{R}^n\setminus(S\times \mb{R}^{n-2})\\
    \prod_{n-1}=(-\pi,\pi)\times\left( -\frac{\pi}{2},\frac{\pi}{2} \right)^{n-2}\\
  \end{gather*}
  \begin{gather*}
    P_n(r,\phi_1,\phi_2,\cdots,\phi_{n-1})=\Mx{P_{n-1}(r,\phi_1,\cdots,\phi_{n-2})\cos\phi_{n-1}\\ r\sin\phi_{n-1}}
  \end{gather*}
\end{Bem}
\begin{Def}{stetige Erweiterung/Grenzwert}
  Seien $X,Y$ metrische Räume, $D\in X$, $f:D\to Y$, $a\in X$ Häufungspunkt $D$, $b\in Y$. Man definiert
  \begin{gather*}
    F:D\cup \left\{ a \right\}\to Y\\
    F(x):=\begin{cases}
      f(x)&x\in D\setminus \left\{ a \right\}\\
      b&x=a
    \end{cases}
  \end{gather*}
  \begin{itemize}
    \item $F$ heisst die stetige Erweiterung von $f$ in Punkt $a$ wenn $F$ stetig in $a$ ist.
    \item In diesem Falle heisst $b$ die Grenzwert von $f$ in Punkt $a$
  \end{itemize}
\end{Def}
\begin{Not}
  \[b=\lim_{x\to a}f(x)\]
\end{Not}
\begin{Bem}
  \begin{itemize}
    \item Die stetige Erweiterung ist eindeutig bestimmt, wenn sie exisitiert.
    \item Der Grenzwert ist eindeutig bestimmt, wenn er existiert.
  \end{itemize}
\end{Bem}
\begin{Lem}
  \[\lim_{x\to a}f(x)=b\ \Lra\ \forall \varepsilon>0\ \exists \delta>0:d_y(f(x),b)<\varepsilon\ \forall x\in D\setminus \left\{ a \right\}, d_x(x,a)<\delta\]
\end{Lem}
\begin{Def}
  Sei $x$ ein metrischer Raum, $(x_k)$ Folge in $x$. $(x_k)$ heisst Cauchyfolge wenn
  \[\forall \varepsilon>0\ \exists N:d_x(x_k,x_l)<\varepsilon\ \forall x,l\geq N\]
\end{Def}
\begin{Def}{vollständiger Raum}
  Ein metrischer Raum $X$ heisst vollständig, wenn jede Cauchyfolge in $X$ konvergiert.
\end{Def}
\begin{Bsp}
  $\mb{R}^n$ mit einer Norm vollständig
\end{Bsp}
\begin{Bem}
  \begin{itemize}
    \item Wir haben die Aussage für die euklidische Norm bewiesen
    \item Aber je zwei Normen auf $\mb{R}^n$ sind äquivalent
  \end{itemize}
\end{Bem}
\begin{Bsp}
  Jeder endlich dimensionaler, normierter Raum ist vollständig.
\end{Bsp}
\begin{Lem}
  Sei $(X,d)$ vollständig, $M\subset X$
  \[M\ \text{vollständig}\ \Lra\ M\ \text{abgeschlossen}\]
  (bezüglich der Spurmetrik)
\end{Lem}
\begin{Bsp}
  $[a;b]\subset\mb{R}$ ist vollständig
\end{Bsp}
\begin{Bew}
  $\Ra$ Sei $M$ vollständig. Sei $(x_k)$ konvergente Folge in $X$ mit $x_k\in M\forall k$. $(a_k)$ konvergiert $\implies$ $(x_k)$ Cauchy $\xRightarrow{M\ \text{vollständig}}$ $x_k\to x\in M$ $\implies$ $M$ abgeschlossen.\\
  $\La$ Sei $M$ abgeschlossen. Sei $(x_k)$ eine Cauchyfolge in $M$ $\implies$ $(x_k)$ eine Cauchyfolge in $X$ $\xRightarrow{X\ \text{vollständig}}$ $x_k\to x\in X$. $x_k\in M,(x_n)$ konvergiert $\implies$ $x\in M$ $\implies$ $M$ vollständig.
\end{Bew}
\begin{Kor}
  Jede abgeschlossene Teilmenge von $\mb{R}^b$ ist vollständig.
\end{Kor}
\begin{Bem}{Vereinbarung}
  $\mb{R}^n$ ist für nur innen als normierter Raum betrachtet.
\end{Bem}
\begin{Def}{Barnachraum}
  Ein normierter $\mb{K}$-Vektorraum heisst Barnachraum, wenn er vollständig ist.
\end{Def}
\begin{Bsp}
  Jeder endlich dimensionaler Vektorraum ist ein Barnachraum.
\end{Bsp}
\begin{Bem}
  Nicht jeder unendlich dimensionaler Vektorraum ist ein Barnachraum.
\end{Bem}
\begin{Bsp}
  Sei $V=\mathcal{C}^0\left( [a;b],\mb{R} \right)$ mit $L^1$-Norm: $\Norm{f}=\int^b_a\Abs{f}\md x$ $V$ ist nicht vollständig.
\end{Bsp}
\begin{Bsp}
  $a=0$, $b=2$
  \[f_n(x):=\begin{cases}
    x^n&0\leq x<1\\
    1&1\leq x \leq 2
  \end{cases}\]
  $f_n$ stetig $\forall n$, $(f_n)$ Folge in $V$, $(f_n)$ Cauchy $n>m$
  \[\Norm{f_m-f_n}=\int^1_0(x^m-x^n)\md x=\frac{1}{n+1}-\frac{1}{n+1}<\frac{1}{m+1}\]
  $f_n$ konvergiert in $V$ nicht
  \[f_n(x)\to f(x) =\begin{cases}
    0&0\leq x<1\\
    1&1\leq x\leq 2
  \end{cases}\]
  $f\not\in V$
\end{Bsp}
\begin{Bsp}
  $\mathcal{C}^0\left( [a;b];\mb{R} \right)$ mit Supremum-Norm ist vollständig.
\end{Bsp}
\subsection{Vervollständigung}
\begin{Sat}
  Jeder normierte Vektorraum $(V,\Norm{\ }_V)$ kann vervollstänigt werden. D.h.
  \begin{gather*}
    \exists \text{Barnachraum}(W,\Norm{\ }_W)\\
    i:V\hookrightarrow W\ \text{lineare Inklusion}\\
    \Norm{i(v)}_W=\Norm{v}_V\ \forall v\in V\ \text{s.d.}\ \overline{i(V)}=W
  \end{gather*}
\end{Sat}
\begin{Bew}
  (Konstruktion)
  \begin{gather*}
    W:=\left\{ \text{Cauchyfolgen in $V$} \right\}\setminus\left\{ \text{Nullfolgen in $V$} \right\}\\
    \Norm{\left[ (x_k) \right]}_W:=\Limi{k}\Norm{x_k}_V\\
  \end{gather*}
  \begin{align*}
    i:&V\hookrightarrow W\\
    &v\mapsto\left[ \text{konstante Folge}\ x_k=v\ \forall k \right]
  \end{align*}
\end{Bew}
\begin{Def}{Hilbertraum}
  Ein Vektorraum mit einem Skalarprodukt der bezüglich der induzierten Norm vollständig ist, heisst Hilbertraum.
\end{Def}
\begin{Bsp}
  \[l^2:=\left\{ (x_k)\ \text{in}\ \mb{C}:\sum\Abs{x_k}^2<\infty \right\}\]
  \[(x,y):=\sum \overline{x_k}y_k\]
\end{Bsp}
\begin{Bsp}
  \[\mathcal{C}^0\left( [a;b],\mb{C} \right), L^2\ \text{Norm}\]
  \[\left\langle f,g \right\rangle =\int^b_a\bar f(x)g(x)\md x\]
  nicht vollständig\\
  Vervollständigung: $L^2\left( [a;b] \right)$ für die Quantenmechanik
\end{Bsp}

\begin{Def}{folgenkompakt}
  Ein metrischer Raum $X$ heisst folgenkompakt, wenn jede Folge in $X$ eine konvergente Teilfolge besitzt. [Bolzano-Weierstrass-Eigenschaft]
\end{Def}
\begin{Def}
  Eine Teilmenge eines metrischen Raumes heisst folgenkompakt, wenn sie bezüglich der Spurmetrik folgenkompakt ist.
\end{Def}
\begin{Bsp}
    $\mb{R}$ ist nicht folgenkompakt (Folgen, die gegen $\infty$ konvergieren)
\end{Bsp}
\begin{Bsp}
    $[a;b]$ ist folgenkompakt (Satz von Bolzano-Weierstrass)
\end{Bsp}
\subsection{Überdeckung}
\begin{Def}{Überdeckung}
  Sei $X$ ein Menge, sei $I$ eine Indexmenge und sei $\left\{ U_i \right\}_{i\in I}$ Familie von Teilmengen von $X$. $\left\{ U_i \right\}_{i\in I}$ heisst Überdeckung von $X$, wenn $X=\bigcup_{i\in I}U_i$ d.h.
  \[\forall x\in X\ \exists i\in I:x\in U_i\]
  Sei $X$ ein metrischer Raum. Dann heisst eine Überdeckung $\left\{ U_i \right\}_{i\in I}$ offen, wenn $U_i$ offen $\forall i$ ist.
\end{Def}
\begin{Bsp}
  $x=[0;1]$
  \[\left\{ \left[0;\frac{2}{3}\right),\left(\frac{1}{3};1\right] \right\}\ \text{Überdeckung}\]
  offen bezüglich der Spurtopologie
\end{Bsp}
\begin{Bsp}
  $x=(0;1)$
  \[\left\{ \left( \frac{1}{n};1 \right) \right\}_{n\in\mb{N}_x}\ \text{offen Überdeckt}\]
\end{Bsp}
\begin{Bsp}
  $x=[0;1]$
  \begin{align*}
    U_n:=\left( \frac{1}{n};1 \right]&& n>0\\
    U_0:=\left[ 0;\frac{1}{2} \right]
  \end{align*}
  $\left\{ U_n \right\}_{n\geq 0}$ offene Überdeckung von $X$
\end{Bsp}
\begin{Def}{endliche Überdeckung}
  eine Überdeckung $\left\{ U_i \right\}_{i\in I}$ heisst endlich, wenn $I$ eine endliche Menge ist.
\end{Def}
\begin{Def}{kompakter metrischer Raum}
  Ein metrischer Raum $X$ heisst kompakt , wenn aus \underline{jeder} offenen Überdeckung von $X$ eine endliche Überdeckung ausgewählt werden kann. d.h.
  \begin{align*}
  \forall \left\{ U_i \right\}_{i\in I}\ x=\bigcup_{i\in I}U_i\ \text{offen}\\
  \exists n\in \mb{N}\ \text{und}\ \exists i_1,i_2,\cdots,i_n\in I\\
  \text{s.d.}\ X=U_{i_1}\cup U_{i_2}\cup\cdots\cup U_{i_n}=\cup_{j=1}U_{i_j}
  \end{align*}
\end{Def}
\begin{Def}{kompakte Teilmenge}
  Eine Teilmenge eines metrischen Raumes heisst kompakt, wenn sie bezüglich der Spurmetrik so ist.
\end{Def}
\begin{Sat}
  \[X\ \text{kompakt}\ \Lra\ X\ \text{folgenkompakt}\]
\end{Sat}
\begin{Bew}
  $\Ra$ Sei $(a_k)$ Folge in $X$. Zu zeigen: $(a_k)$ besitzt eine konvergente Teilfolge.
  \[A:=\left\{ a_k,k\in\mb{N} \right\}\]
  \subparagraph{Fall 1}$A$ ist endlich $\implies$ $(a_k)$ besitzt eine konstante Teilfolge.
  \subparagraph{Fall 2}$A$ unendlich
  \begin{Lem}
    $A$ besitzt einen Häufungspunkt.\\
    $A$ besitzt keinen Häufungspunkt.
    \[\forall x\in X\ \exists U(x)\ \text{Umgebung von $x$}\]
    s.d.
    \[U(x)\cap A=\begin{cases}
      \varnothing&x\not\in A\\
      \left\{ x \right\}&x\in A
    \end{cases}\]
    Zudem:
    \begin{gather*}
    \forall x\in U(x)\ \bigcup_{x\in A}U(x)=X\\
    \left\{ U(x) \right\}_{x\in X}\ \text{ist eine offene Überdeckung von $X$}\\
    \xRightarrow{\text{$X$ kompakt}}\\
    \exists n:\exists x_1,\cdots,x_n\in X\ \text{s.d.}\ X=U(x_1)\cup\cdots\cup U(x_n)\\
    A=X\cap A=\left( U(x_1)\cup\cdots\cup U(x_n) \right)\cap A=\left\{ x_i:x_i\in A \right\}\subset \left\{ x_i \right\}\\
    \implies A\ \text{endlich}\implies \text{Widerspruch!}
    \end{gather*}
  \end{Lem}
  Sei $a$ Häufungspunkt von $A$ $\implies$
  \begin{gather*}
    \forall\mu\in \mb{N}:K_{\frac{1}{\mu}}(a)\ni a_{k_\mu}\in A\setminus\left\{ a \right\}\\
    (a_{k_\mu})\ \text{Teilfolge}, (a_{k_\mu})\in K \implies \Limi{\mu}a_{k_\mu}=a
  \end{gather*}
\end{Bew}
\begin{Def}{beschränkt}
  Sei $X$ ein metrischer Raum, $\mb{K}\subset X$. $\mb{K}$ heisst beschränkt, wenn 
  \[\exists x\in X\ \exists r>0:\mb{K}\subset K_r(x)\]
\end{Def}
\begin{Lem}
  Sei $X$ ein metrischer Raum, $\mb{K}\subset X$
  \[\mb{K}\ \text{folgenkompakt} \implies\text{$\mb{K}$ beschränkt und abgeschlossen}\]
\end{Lem}
\begin{Bew}
  Sei $\mb{K}$ nicht beschränkt oder nicht abgeschlossen.
  \subparagraph{Fall 1}$\mb{K}$ nicht beschränkt\\
  Sei $x\in \mb{K}$. Da $\mb{K}$ nicht beschränkt
  \[\forall k\exists x_k\in \mb{K}:d(x_k,x)>k\]
  (sonst wäre $\mb{K}\subset K_k(x)$)
  $(x_k)$ besitzt keine konvergente Teilfolge. Sonst:
  \[x_{k_i}\xrightarrow{i\to\infty}x\implies d(x_k,x)\to 0\]
  (was aber nicht möglich ist, da der Abstand immer grösser wird)
  \subparagraph{Fall 2}$\mb{K}$ nicht abgeschlossen
  \[\exists (x_k),x_k\in \mb{K}\forall k\ \text{und}\ x_k\in x\not\in X\]
  $\implies$ jede Teilfolge von $(x_k)$ konvergiert gegen $x\in X$.
\end{Bew}
\begin{Bem}
  $\mb{K}$ folgenkompakt $\implies$ $\mb{K}$ abgeschlossen und beschränkt.\\
  Im allgemeinen $\not\La$
\end{Bem}
\begin{Bsp}
  $X=\mathcal{C}\left( [0;\pi],\mb{C} \right)$ mit Supremumsnorm
  \[\mb{K}=K_1(0)=\left\{ f\in X:\overbrace{\Norm{f}}^{\sup\Abs{f}}\leq 1 \right\}\]
  $\mb{K}$ ist abgeschlossen
  \[\overline{K_1(0)}\subset K_2(0)\]
  \ldots und beschränkt.
  \begin{align*}
    e_k(x):=e^{ikx}&&\\
    e_k\in \mb{K}\ \forall k&&\\
    \Norm{e_k-e_l}=2\ \forall k,l
  \end{align*}
  \begin{Bew}
    \begin{gather*}
      \Abs{e_k(x)-e_l(x)}^2=\left( e^{-ikx}-e^{ilx} \right)\left( e^{ikx}-e^{ilx} \right)=\\
      =1-e^{i(l-k)x}-e^{i(k-l)x}+1 = 2\left( 1-cos(k-l) \right)
    \end{gather*}
    Maximum 4 wenn $\cos = -1$, $\sup\Abs{e_k-e_l}=2$ $\implies$ jede Teilfolge $e_k$
    \[\Norm{e_{ki}-e_{kj}}=2\ \forall i,j\]
    keine Cauchyfolge. Keine Teilfolge ist Cauchy. $\implies$ keine Teilfolge konvergiert
  \end{Bew}
\end{Bsp}
\begin{Sat}
  Sei $V$ ein \underline{endlichdimensionaler} normierter Vektorraum, sei $\mb{K}\subset V$. Dann sind folgende Aussagen equivalent:
  \begin{enumerate}
    \item $\mb{K}$ ist beschränkt und abgeschlossen
    \item $\mb{K}$ kompakt
    \item $\mb{K}$ ist folgenkompakt
  \end{enumerate}
  zu zeigen: $1.\implies 2.$
\end{Sat}
\begin{Sat}
  Sei $X$ kompakt und $A\subset X$ abgeschlossen. Dann ist $A$ kompakt.
\end{Sat}
\begin{Bew}
  Sei $\left\{ U_i \right\}_{i\in I}$ offene Überdeckung von $A$.
  \begin{gather*}
    U_i\ \text{offen in}\ A\implies \exists V_i\subset X\ text{offen, mit}\ U_i=A\cap U_i\\
    \bigcup_{i\in I}U_i=A \implies \bigcup_{i\in A}V_i\supset A\\
    X=X\setminus A\cup \bigcup_{i\in I}V_i\\
    X\setminus A,V_i\ \text{Überdeckung von $X$}\\
    A\ \text{abgeschlossen}\implies X\setminus A\ \text{offen}\\
    X\setminus A,V_i\ \text{offene Überdeckung}\\
    X\ \text{kompakt}\implies\ \exists n:i_1,\cdots,i_n: X=X\setminus A\cup V_{i_1}\cup V_{i_1}\cup\cdots\cup V_{i_n}\\
    \implies U_{i_1},\cdots,U_{i_n}\ \text{Überdeckung von $A$}
  \end{gather*}
\end{Bew}
\subsection{Existenz von Maxima und Minima}
\begin{Sat}
  Sei $f:X\to Y$ stetig ($X,Y$ metrische Räume)
  \[X\ \text{kompakt}\implies f(x)\ \text{kompakt}\]
\end{Sat}
\begin{Bew}
  Sei $\left\{ U_i \right\}_{i\in I}$ eine offene Überdeckung von $f(x)$ $V_i:=f^{-1}(U_i)$ $\implies$ $\left\{ V_i \right\}_{i\in I}$ offene Überdeckung von $X$.
  \[\implies\exists n:i_1,\cdots,i_n\in I\ X=v_{i_1}\cup\cdots\cup V_{i_n}\implies f(x)=U_{i_1}\cup\cdots\cup U_{i_n}\]
\end{Bew}
\begin{Sat}{von Maxima und Minima}
  Sei $f:x\to\mb{R}$ stetig und $X$ kompakt. Dann nimmt $f$ ein Maximum und ein Minimum an.
\end{Sat}
\begin{Bew}
  $f$ stetig und $X$ kompakt $\implies$ $f(x)\subset\mb{R}$ kompakt. $\implies$ $f(x)$ beschränkt und abgeschlossen.\\
  beschränkt $\implies$ $f(x)$ besitzt ein Supremum und ein Infimum\\
  abgeschlossen $\implies$ $\sup, \inf f\in f(x)$
\end{Bew}
\begin{Def}{gleichmässig stetig}
  $f:X\to Y$, ($X,Y$ metrische Räume) heisst gleichmässig stetig, wenn
  \[\forall\varepsilon>0\ \exists \delta>0:\forall x_1,x_2\subset X\ \text{mit}\ d_x(x_1,x_2)<\delta\]
  gilt
  \[d_y\left( f(x_1),f(x_2) \right)<\varepsilon\]
\end{Def}
\begin{Bem}
  $f$ gleichmässig stetig $\implies$ $f$ stetig
\end{Bem}
\begin{Sat}
  Sei $f:X\to Y$ $X$ kompakt
  \[f\ \text{stetig}\implies f\ \text{gleichmässig stetig}\]
\end{Sat}
\begin{Bew}
  Wie im Falle $X\subset\mb{R}$  
\end{Bew}
\begin{Lem}{Tubenlemma}
  Sei $X$ ein metrischer Raum, $\mb{K}$ ein kompakter Raum, $x_0\in X$, $W\subset X\times \mb{K}$ offen mit $\left\{ x_0 \right\}times\mb{K}\subset W$.\\
  Dann $\exists$ Umgebung von $x_0$ in $X$ s.d. \[U\times\mb{K}\subset W\]
\end{Lem}
\begin{Bew}
  $W$ offen in der Produkttopologie.
  \[\forall y,x\in \mb{K}, \left( x_0,y \right)\in W\]
  $\exists$ Umgebung von $U_y$ von $x_0$ in $X$
  $\exists$ Umgebung von $V_y$ von $x_0$ in $\mb{K}$
  mit $U_y\times V_y\subset W$
  \begin{gather*}
    \bigcup_{y\in\mb{K}}V_j=\mb{K}\\
    y\in V_y\ \forall y\\
    \left\{ V_y \right\}_{y\in \mb{K}}\ \text{offene Überdecktung von $\mb{K}$}\ \text{$\mb{K}$ kompakt}\\
    \implies \forall n,u_1,\cdots,u_n\in\mb{K},\ \mb{K}=V_{y_1}\cup\cdots\cup V_{y_n}\\
    U:=U_{y_1}\cap U_{y_2}\cap\cdots\cap U_{y_n}\\
    U\ni x_0\\
    U\times\mb{K}\subset W\\
    U\ \text{\underline{offen}}
  \end{gather*}
\end{Bew}
\begin{Kor}
  $\mb{K}$ kompakt und $L$ kompakt $\implies$ $\mb{K}\times L$ kompakt
\end{Kor}

\begin{Bem}
  $\mb{K}=[a;b]$
  \begin{align*}
    f:X\times[a;b]&\to\mb{C}\\
    (x,y)&\mapsto f(x,y)
  \end{align*}
  \begin{align*}
    \forall x\in X: f_x:[a;b]&\to\mb{C}\\
    y&\mapsto f(x,y)
  \end{align*}
  stetig $\implies$ $\mathcal{R}$ auf $[a;b]$
  $f_x=f\circ i_x$
  \[i_x:\left\{ x \right\}\times[a;b]\to X\times[a;b]\]
  \begin{gather*}
    F(x):=\int^b_af(x,t)\md t
    F:X\to\mb{C}
  \end{gather*}
\end{Bem}
\begin{Sat}
  $F$ stetig
\end{Sat}
\begin{Bew}
  $\forall x_0\in X$ ist $F$ stetig in $x_0$. Sei $x_0\in X$
  \[\phi(x,t):=f(x,t)-f(x_0,t)\]
  stetig. Sei $\varepsilon>0$
  \[W:=\left\{ (x,t)\in X\times[a;b]:|\phi(x,t)<\frac{\varepsilon}{b-a} \right\}\]
  $\phi$ stetig $\implies$ $W$ offen
  \[\phi(x_0,t)=0\ \forall t\implies \left\{ x_0 \right\}\times[a;b]\subset W\]
  $\implies$ $\exists$ Umgebung $U$ von $x_0$ in $X$ mit
  \[U\times [a;b]\subset W\]
  $\forall x\in U$ gilt
  \begin{gather*}
    \Abs{F(x)-F(x_0)}=\Abs{\int^b_a\phi(x,t)\md t}\leq \int^b_A\Abs{\phi(x,t)}\md t < \int^b_a\frac{\varepsilon}{b-a}\md t=\varepsilon
  \end{gather*}
  Spezialfall $X=[c;d]$
  \[f:[c;d]\times[a;b]\to\mb{C}\]
  stetig
  \[F:[c;d]\to\mb{C}\ \text{stetig}\]
  $\implies$ Regelfunktion
  \[\int^d_cF(x)\md x=\underbrace{\int^d_c\left( \int^b_af(x,y)\md y \right)\md x}_{\text{interiertes Integral}}\]
\end{Bew}
\begin{Not}{interiertes Integral}
  \[\int_{[c;d]\times[a;b]}f(x,y)\md x\md y:=\int_c^dF(x)\md x\]
\end{Not}
\subsection{Zwischenwertsatz}
\begin{Def}{zusammenhängender metrischer Raum}
  Ein metrischer Raum $X$ heisst zusammenhängend, wenn es keine Zerlegung $X=U\cup V$ gibt, mit
  \begin{enumerate}
    \item $U,V$ disjunkt (d.h. $U\cap V=\varnothing$)
    \item $U,V$ offen
    \item $U,V$ nicht leer
  \end{enumerate}
\end{Def}
\begin{Def}
  Eine Teilmenge eines metrischern Raumes heisst zusammenhängend, wenn sie bezüglich der Spurtopologie so ist.
\end{Def}
\begin{Bsp}
  $\varnothing$ ist zusammenhängend
\end{Bsp}
\begin{Bsp}
  $\left\{ x \right\}\subset\mb{R}$ zusammenhängend
\end{Bsp}
\begin{Bsp}
  $\mb{Q}\in\mb{R}$ \underline{nicht} zusammenhängend
  \begin{align*}
    U&=\left\{ x\in\mb{Q}:x\leq 0\ \text{oder}\ x^2<2 \right\}=\mb{Q}\cap\left( -\infty,\sqrt{2} \right)\\
    V&=\left\{ x\in\mb{Q}:x>0\ \text{oder}\ x^2>2 \right\}=\mb{Q}\cap\left( \sqrt{2},+\infty \right)
  \end{align*}
  offen
\end{Bsp}
\begin{Sat}
  Sei $X\subset\mb{R}$ und besitze $X$ mindestens zwei verschiedene Punkte
  \[X\ \text{zusammenhängend}\ \Lra\ X\ \text{Intervall}\]
\end{Sat}
\begin{Bew}
  $\Ra$ Kontrapositionsbeweis: Sei $X$ kein Intervall
  \[\implies\exists u<s<v\ \text{mit} u,v\in X\ s\not\in X\]
  \[U:=X\cap\left( -\infty;s \right),\ V=X\cap\left( s;+\infty \right)\]
  $\implies$ $X$ nicht zusammenhängend\\
  $\La$ Widerspruchbeweis: Sei $X$ \underline{nicht} zusammenhängendes Intervall
  \begin{gather*}
    \exists U,V\ \text{offen in}\ I\\
  \end{gather*}
  \begin{align*}
    U,V&\neq \varnothing&\implies\exists u\in U, v\in V\\
    U\cap V&=\varnothing&\implies u\neq v\\
    U\cup V&=X&\\
  \end{align*}
  Annahme $u<v$:
  \[X\ \text{Intervall}\implies [u;v]\subset X\]
  Sei
  \[s=\sup\left( [u;v]\cap U \right)\]
  beschränkt $\subset [u,v]$
  \[\implies s\in [u;v]\ s\leq v\]
  $V$ offen
  \[U=X\setminus V\]
  \[\implies U\ \text{abgeschlossen} \implies s\in U\]
  \[U\cap V=\varnothing\implies s<v\]
  Sei $x\in X$ mit $x>s$ und $x\leq v$
  \[\xRightarrow{s\sup}\ x\in V\implies (s;v]\in V\]
  \[U\ \text{offen}\implies \exists \varepsilon>0\ \text{mid} \left( s-\varepsilon, s+\varepsilon \right)\cap X\in U\]
  $X$ Intervall $v\in X$, $s<v$
  \[\implies \lambda\in [s;s+\varepsilon)\cap X\in U\ \text{mit}\ \lambda\leq v\implies \lambda\in U\]
  $(s;v]\subset V$ $\implies \lambda\in V$ Widerspruch, da $U\cap V=\varnothing$
\end{Bew}
\begin{Sat}
  Sei $f:X\to Y$ stetig
  \[X\ \text{zusammenhängend}\implies Y\ \text{zusammenhängend}\]
\end{Sat}
\begin{Bew}
  Kontrapositionsbeweis: Sei $Y$ nicht zusammenhängend
  \[\implies Y=U\cup V\]
  \begin{align*}
    U,V&\neq\varnothing\\
    U\cap V&=\varnothing\\
    U,V&\ \text{offen}
  \end{align*}
  \[\tilde U:=f^{-1}(U),\tilde V:=f^{-1}(V)\]
  \[X=\tilde U\cup \tilde V\]
  \begin{align*}
    \tilde U, \tilde V&\neq \varnothing\\
    \tilde U\cap \tilde V=\varnothing\\
    \tilde U, \tilde V&\ \text{offen}
  \end{align*}
\end{Bew}
\begin{Sat}{Zwischenwertsatz}
  Sei $X$ zusammenhängend
  \[f:X\to\mb{R}\ \text{stetig}\]
  Für je zwei Punkte $a$ und $b$ $\in X$ nimmt $f$ alle Werte zwischen $f(a)$ und $f(b)$ an.
\end{Sat}
\begin{Bew}
  Fall 1: $f(a)=f(b)$ nichts zu zeigen\\
  Fall 2: $f(a)\neq f(b)$ und $f(x)$ zusammenhängend
  \[\implies f(x)\ \text{Intervall}\]
  $\implies$ $f(x)$ enthält alle Punkte zwischen $f(a)$ und $f(b)$
\end{Bew}
\begin{Def}{wegzusammenhängend}
  Ein metrischer Raum $X$ heisst wegzusammenhängend, wenn es $\forall a,b\in X$ eine stetige Kurve
  \[\gamma:[\alpha;\beta]\to X\]
  gibt mit $\gamma(\alpha)=a$ und $\gamma(\beta)=b$. Man sagt, $\gamma$ verbinde $a$ und $b$.
\end{Def}
\begin{Bsp}
  $\mb{R}\setminus\left\{ 0 \right\}$ ist nicht wegzusammenhängend. Beweis: Zwischenwertsatz.
\end{Bsp}
\begin{Def}{konvex}
  Sei $V$ Vektorraum, $X\subset V$ heisst konvex, wenn $\forall a,b\in X$
  \[\left\{ a+t(b-a):t\in [0;1] \right\}\subset X\]
  (Strecke, die $a$ und $b$ verbindet)
\end{Def}
\begin{Lem}
  Sei $V$ ein normierter Vektorraum, $X\subset V$
  \[X\ \text{konvex}\implies X\ \text{wegzusammenhängend}\]
\end{Lem}
\begin{Bew}
  Die Strecke ist eine stetige Kurve.
\end{Bew}
\begin{Sat}
  $\mb{R}\setminus\left\{ 0 \right\}$ und $S^{n-1}$ sind für $n\geq 2$ wegzusammenhängend.
\end{Sat}
\begin{Lem}
  \[X\ \text{wegzusammenhängend} \implies X\ \text{zusammenhängend}\]
\end{Lem}
\begin{Bew}
  Widerspruchsbeweis: Sei $X$ wegzusammenhängend, nicht zusammenhängend.
  \begin{align*}
    X&=U\cup V\\
    U,V&\ \text{offen}&\exists u\in U, v\in V\\
    U,V&\neq\varnothing&u\neq v\\
    U\cap V&=\varnothing
  \end{align*}
  $X$ wegzusammenhängend
  \[\implies \gamma:[\alpha;\beta]\to X\]
  $\gamma(\alpha)=u$, $\gamma(\beta)=v$
  \begin{gather*}
    \tilde U=\gamma^{-1}(U),\ \tilde V=\gamma^{-1}(V)\\
  \end{gather*}
  \begin{align*}
    [\alpha;\beta]&=\tilde U\cup \tilde V\\
    \tilde U,\tilde V&\ \text{offen}\\
    \tilde U,\tilde V&\neq\varnothing\\
    \tilde U\cap\tilde V&=\varnothing
  \end{align*}
  \[ [\alpha;\beta]\ \text{nicht zusammenhängend}\]
\end{Bew}
\begin{Kor}
  $\mb{R}\setminus\left\{ 0 \right\}$ und $S^{n-1}$ sind für $n\geq 2$ zusammenhängend.
\end{Kor}
\begin{Bew}
  $\mb{R}\setminus\left\{ 0 \right\}$ nicht zusammenhängend
\end{Bew}
\begin{Sat}
  Sei $V$ ein normierter Vektorraum, $X\subset V$ offen
  \[X\ \text{zusammenhängend}\implies X\ \text{wegzusammenhängend}\]
  Zusätzlich können je zwei Punkte in $X$ durch einen Streckenzug verbunden werden.
\end{Sat}
\begin{Bem}
  Sei 
  \[X:=\left\{ \left( x,\sin\frac{1}{x} \right),x>0 \right\}\cup \left\{ (0,y),y\in [-1;1] \right\}\subset\mb{R}\]
  \begin{itemize}
    \item $X$ ist nicht offen
    \item $X$ zusammenhängend
    \item $X$ nicht wegzusammenhängend
  \end{itemize}
\end{Bem}
\begin{Def}{Gebiet}
  Eine zusammenhängende offene Teilmenge normierten Vektorraumes heisst Gebiet.
\end{Def}
\begin{Sat}
  \[GL(n;\mb{R}=\left\{ A\in M(n\times n,\mb{R}), \det A\neq 0 \right\}\]
  ist nicht zusammenhängend.
  \[GL^+(n;\mb{R}=\left\{ A\in M(n\times n,\mb{R}), \det A> 0 \right\}\]
  ist zusammenhängend.
\end{Sat}
\begin{Bew}
  \[\det:GL(n,\mb{R})\to\mb{R}\setminus\left\{ 0 \right\}\ \text{stetig}\]
  Wäre $GL$ zusammenhängend, dann wäre auch $\mb{R}\setminus\left\{ 0 \right\}$ zusammenhängend.
\end{Bew}
\begin{Sat}
  Seien $X$ und $Y$ homöomorph. Dann
  \[X\ \text{zusammenhängend}\ \Lra\ Y\text{zusammenhängend}\]
\end{Sat}
\begin{Kor}
  $\mb{R}^n,n>1$ ist nicht homöomorph zu $\mb{R}$
\end{Kor}
\begin{Bew}
  $n>1$ $\mb{R}^n\setminus\left\{ 0 \right\}$ zusammenhängend.\\
  Widerspruchbeweis:
  \[\exists f:\mb{R}^n\to\mb{R}\ \text{Homöomorphismus}\]
  \[f|_{\mb{R}^n\setminus\left\{ 0 \right\}}:\underbrace{\mb{R}^n\setminus\left\{ 0 \right\}}_{\text{zusammenhängend}}\to\underbrace{\mb{R}\setminus f(0)}_{\text{nicht zusammenhängend}}\]
\end{Bew}


\newpage

%= Stichwortverzeichnis ======================================================================
\rhead{}
\addcontentsline{toc}{section}{Stichwortverzeichnis}
\printindex

\end{document}
