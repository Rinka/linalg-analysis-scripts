
%= Document-Class ==================================================================================
\documentclass[10pt,a4paper]{article}

%= Packages ========================================================================================
\usepackage[utf8]{inputenc}
\usepackage{ngerman,amsmath,amssymb,amsfonts,mathrsfs}
\usepackage{amsthm}
\usepackage{bbm}
\usepackage{epic,eepic,pstricks,pst-node,pst-plot}
\usepackage{pstricks}
\usepackage{colortbl}
\usepackage{graphicx}
\usepackage{makeidx}
\usepackage{fancyhdr}
\usepackage{latexsym}
\usepackage{psfrag}
\usepackage{enumerate}
\usepackage{float}
%\usepackage{mathtext}
\usepackage[all, knot, poly]{xy}
\usepackage{dsfont}
\pagestyle{fancy}
\usepackage{multirow, bigdelim, bigstrut}
\usepackage{rotating}
\usepackage{ifthen}
\usepackage{boxedminipage}
\usepackage{mathtools}
\usepackage{ulsy}
\usepackage{trfsigns}

%= Seiten-Layout =========================================================================
\voffset-22mm \textheight715pt 

%Seitenbreite==============================================================

%\oddsidemargin=-0.2in
%\evensidemargin=-0.4in
%\textwidth=5.2in
%\headwidth=5.2in

%= Index-Befehle ========================================================================
\renewcommand{\indexname}{Stichwortverzeichnis}
\makeindex

%= Befehl-Overwriting =======================================================================
\makeatletter
\renewcommand{\section}{\@startsection {section}{1}{\z@}{-3.5ex \@plus -1ex \@minus -.2ex}{2.3ex \@plus.2ex}{\reset@font\Large\bfseries}}
\renewcommand{\subsection}{\setcounter{Prop}{0}\setcounter{Lem}{0}\setcounter{Sat}{0}\setcounter{Kor}{0}\setcounter{Def}{0}\@startsection{subsection}{2}{\z@}{-3.25ex\@plus -1ex \@minus -.2ex}{1.5ex \@plus .2ex}{\normalfont\large\bfseries}}
\newcommand\subnummer{\@arabic\c@section.\@arabic\c@subsection}
\makeatother

%= Strings ================================================================
\newcommand{\mainfold}{.}
\newcommand{\prefix}{A1-}

%= Eigene Befehle ==========================================================================
\DeclareMathOperator{\id}{Id}
\DeclareMathOperator{\arccot}{arccot}
\DeclareMathOperator{\arsinh}{arsinh}
\DeclareMathOperator{\arcosh}{arcosh}
\DeclareMathOperator{\artanh}{artanh}
\DeclareMathOperator{\md}{d}
\DeclareMathOperator{\Grad}{grad}

\newcommand{\Diff}[2]{\displaystyle\frac{\mathrm{d}#1}{\mathrm{d}#2}}
\newcommand{\End}{\hfill{\hbox{$\Box$}}\par\vspace{2mm}}
\newcommand{\eps}{\varepsilon}
\newcommand{\ePic}[1]{\input{\mainfold/graphics/\prefix#1.eepic}}
\newcommand{\pst}[1]{\input{\mainfold/graphics/\prefix#1.pst}}
\newcommand{\pic}[1]{\input{\mainfold/graphics/\prefix#1.pic}}
\newcommand{\Mx}[1]{\begin{pmatrix}#1\end{pmatrix}}
%\newcommand{\im}[1]{\operatorname{Im}(#1)}
%\newcommand{\Include}[4]{\rhead{#2.#3.20#4}\input{\mainfold/lectures/#1-#4-#3-#2.tex}}
\newcommand{\Index}[1]{\emph{#1}\index{#1}}
\newcommand{\Int}[4]{\displaystyle\int\limits_{#1}^{#2}#3\,\mathrm{d}#4}
\newcommand{\diff}[1]{\operatorname{d}\!#1}
\newcommand{\Limi}[1]{\displaystyle\lim_{#1\rightarrow\infty}}
\newcommand{\Limo}[1]{\displaystyle\lim_{#1\rightarrow0}}
\newcommand{\mb}[1]{\mathbb{#1}}
\newcommand{\ds}{\displaystyle}
\newcommand{\ol}[1]{\overline{#1}}
\newcommand{\Part}[2]{\dfrac{\partial #1}{\partial #2}}
\newcommand{\QED}{\hfill{\hbox{(QED)}}\par\vspace{2mm}}
\newcommand{\re}[1]{\operatorname{Re}(#1)}
\newcommand{\s}{\hspace{2mm}}
\newcommand{\vsa}{\vspace{1mm} \\}
\newcommand{\vsb}{\vspace{2mm} \\}
\newcommand{\vsc}{\vspace{3mm} \\}
% \newcommand{\tr}[1]{\textrm{#1}}
\newcommand{\tr}[1]{\text{#1}}
\newcommand{\ra}{\rightarrow}
\newcommand{\Ra}{\Rightarrow}
\newcommand{\Lra}{\Leftrightarrow}
\newcommand{\La}{\Leftarrow}
\newcommand{\ul}[1]{\underline{#1}}
\newcommand{\rsa}{\rightsquigarrow}
\newcommand{\ara}[2]{\autorightarrow{\ensuremath{#1}}{\ensuremath{#2}}}

%\newcommand{\detmx}{\left| \begin{array} #1 \end{array} \right|}

\newcommand{\grad}[1]{\Grad(#1)}
\newcommand{\fr}[2]{\displaystyle\frac{#1}{#2}} % fertiger bullshit, daf�r gibts \dfrac{}{}
\renewcommand{\Re}{\operatorname{Re}}
\renewcommand{\Im}{\operatorname{Im}}

% ---- DELIMITER PAIRS ----
\def\floor#1{\lfloor #1 \rfloor}
\def\ceil#1{\lceil #1 \rceil}
\def\seq#1{\langle #1 \rangle}
\def\set#1{\{ #1 \}}
\def\abs#1{\mathopen| #1 \mathclose|}	% use instead of $|x|$ 
\def\norm#1{\mathopen\| #1 \mathclose\|}% use instead of $\|x\|$ 

% --- Self-scaling delmiter pairs ---
\def\Floor#1{\left\lfloor #1 \right\rfloor}
\def\Ceil#1{\left\lceil #1 \right\rceil}
\def\Seq#1{\left\langle #1 \right\rangle}
\def\Set#1{\left\{ #1 \right\}}
\def\Abs#1{\left| #1 \right|}
\def\Norm#1{\left\| #1 \right\|}

%Adrians Abbildungs-Environment ==============================================

\newcommand{\Sidein}{\begin{rotate}{90}\small$\in$\end{rotate}}

\newcommand{\Abb}[5][]{\ensuremath{
    \begin{array}{lc}
      \ifthenelse{\equal{#1}{}}{}{#1:}\;\; & 
      \begin{xy}
        \xymatrixrowsep{1em}\xymatrixcolsep{2em}%
        \xymatrix{ #2 \ar[r] \ar@{}[d]^<<<<{\hspace{0.001em} \Sidein}
          & #3  \ar@{}[d]^<<<<{\hspace{0.001em} \Sidein} \\
          #4 \ar@{|->}[r] & #5} \end{xy}
    \end{array}
  }%
}

%= Environments ========================================================================
\def\thechapter{\Roman{chapter}}
\def\thesection{\arabic{section}}
\newtheorem{Bew}{Beweis}
\newtheorem{Lem}{Lemma}
\newtheorem{Kor}{Korollar}
\newtheorem{Sat}{Satz}
\newtheorem{Prop}{Proposition}
\theoremstyle{definition}
\newtheorem{Bsp}{Beispiel}
\newtheorem{Def}{Definition}
\newtheorem{Prob}{Problem}
\theoremstyle{remark}
\newtheorem{Bem}{Bemerkung}
\newtheorem{Eig}{Eigenschaften}
\newtheorem{Not}{Notation}

\def\pstexInput#1{%
  \begin{center}
    \begin{picture}(0,0)%
      \special{psfile=\mainfold/graphics/A2-#1.pstex}%
    \end{picture}%
    \input{\mainfold/graphics/A2-#1.pstex_t}%
  \end{center}
}

%= Titelseite ===========================================================================
\begin{document}
\headheight15pt
\begin{titlepage}
\hfill
\vspace{20mm}
\pagenumbering{roman}
\begin{center}
{\LARGE Analysis I - Vorlesungs-Script} %\vskip 3em {\large Prof.
%Guido Mislin} \vskip 1.5em
%{\large Basisjahr 06/07}\vspace{30mm}\\
%{\large {\bf Mitschrift:} \vspace{2mm}\\
%Alexander Berthold van der Bourg}\vspace{5mm}\\ %30mm
%{\large {\bf Graphics:} \vspace{2mm}\\
%Pirmin Weigele }\vspace{30mm}\\ %30mm

\end{center}
\vfill

\end{titlepage}


%= Inhaltsverzeichnis ==========================================================================
\lhead{}
\rhead{}
\tableofcontents
\newpage
\pagenumbering{arabic}
\setcounter{page}{1}

%= Vorlesung-Skripts ==========================================================================
\cfoot{\thepage}
\fancyhead[L]{\nouppercase{\leftmark}}
\newpage

%= Analysis I & & II ==========================================================================

%Analysis I
\section{Integralrechnung}
\paragraph{Ziel} mathematisch präzise Formulierung des ``Flächeninhalts'' unter dem Graphen einer Funktion
\paragraph{Fragen}
\begin{itemize}
  \item Welche Funktionen sind zulässig?
  \item Wie definiert man das Integra für diese Funktionen?
\end{itemize}
\paragraph{Idee}
\begin{enumerate}
  \item def. Integral für spezielle Funktionen (Treppenfunktionen)
  \item betrachte Folgen von Treppenfunktionen und führe geeigneten Konvergenzbegriff ein (gleichmässige Konvergenz), $\to$ mögliche Limiten sind Regelfunktionen
  \item falls $f_n \xrightarrow{n\to\infty}f$ (Folge von Treppenfunktionen), setze $\int_a^b f \md x:= \Limi{n}\left( \int^b_a f_n \md x \right)$
    \begin{align*}
      f_n\to f \text{folgt} \left( \int^b_a f_n \md x \right)_{n\in\mb{N}} \text{konvergent}\\
      f_n \& g_n \to f \text{zwei Folgen} \implies \Limi{n} \left( \int^b_a f_n \md x \right)=\Limi{n} \left( \int^b_a g_n \right)
    \end{align*}
\end{enumerate}
\subsection{Treppenfunktionen}
\begin{itemize}
  \item $a<b, a,b\in\mb{R}$ $\{x_0,x_1,\cdots,x_n\}$ \underline{Zerlegung} von $[a,b] \Lra a=x_0<x_1<x_2<\cdots<x_{n-1}<x_n = b$
  \item $\phi[a,b]\to\mb{C}$ \underline{Treppenfunktion} (auf $[a,b]$) $\Lra$ $\exists$ Zerlegung $\{x_0,x_1,\cdots,x_n\}$ von $[a,b]$ so dass $\phi|_{(x_{n-1},x_n)}$ konstant $\forall k=1,\cdots,n$
\end{itemize}
\begin{Bem}
  \begin{itemize}
    \item keine Aussage über $\phi(x_0),\cdots,\phi(x_n)$
    \item nicht verboten zu feine Zerlegungen zu betrachten
  \end{itemize}
\end{Bem}
\begin{itemize}
  \item $\tau([a,b])$ (ein Vektorraum über $\mb{C}$, $\phi, \psi$ Treppenfunktionen) Menge aller Treppenfunktionen auf $[a,b]$
\end{itemize}
\begin{Def}{Integral von Treppenfunktionen} $\phi:[a,b]\to\mb{C}$ Teppenfunktion mit Zerlegung $\{x_0,x_1,\cdots,x_n\}$
  \begin{itemize}
    \item $c_K$ = Funktionswert von $\phi$ auf $(x_{k-1},x_k)$
    \item $\Delta x_k=x_k-x_{k-1}$
  \end{itemize}
  \[\int_a^b \phi(x)\md x=\sum^n_{k=1}\left( c_k\cdot\Delta x_k \right)\]
\end{Def}
\begin{Lem}{}
  Das Integral einer Treppenfunktion ist unabhängig von der gewählten Zerlegung
\end{Lem}
\begin{Bew}{}
  \begin{align*}
    Z=\{x_0,x_1,\cdots,x_n\} &\ \&\  Z'= \{y_0,y_1,\cdots,y_m\} & \text{Zerlegungen von} [a,b]\\
    \phi|_{(x_{k-1},x_k)} &\ \&\  \phi|_{(y_{k-1},y_k)} & \text{konstant}\\
    \rsa I(Z) & \rsa I(Z') &\ \ \leftarrow \text{Summen} \sum^n_{k=1}c_k\Delta x_k \ \&\ \sum^m_{k=1}c'_k\Delta y_k
  \end{align*}
  \subparagraph{Frage} $I(Z)=I(Z')$
  \subparagraph{Zeige} $I(Z)=I(Z\cup Z')=I(Z')$\\
  $Z\cup Z'$ entsteht aus $Z$ durch Hinzufügen von endlich vielen Punkten.\\
  Angenommen $Z\cup Z' = Z\cup\{y\}, y\not\in Z$. Leicht zu sehen: $I(Z)=I(Z\cup\{y\})$
  \begin{align*}
    I(Z)=I(Z\cup\{y\})\xRightarrow{\text{Ind}} I(Z) = I(Z\cup\{y_1\})=I(Z\cup \{y_1\}\cup\{y_2\}) = \cdots = I(Z\cup Z')
  \end{align*}
\end{Bew}
\begin{Lem}{}
  \begin{align*}
    \int_a^b \md x \tau([a,b])\to\mb{C}
  \end{align*}
  \begin{enumerate}
    \item $\int_a^b \md x$ ist linear, d.h.
      \[\forall \phi, \psi\in\tau([a,b]),\alpha, \beta, \in\mb{C}: \int_a^b\alpha\phi+\beta\psi \md x = \alpha\left( \int^b_a\phi \md x \right)+\beta\left( \int^b_a\phi \md x \right)\]
    \item \[\Abs{f^b_a\phi \md x}\leq \int^b_a\Abs{\phi} \md x \leq (b-a) \underbrace{\Norm{\phi}}_{\text{Supremum}}\]
    \item für $\phi,\psi:[a,b]\to\mb{R}$ mit $\phi(x)\leq\psi(x)\ \forall x\in[a,b] \implies$
      \[\int^b_a\psi \md x \leq \int^b_a \psi \md x\]
  \end{enumerate}
\end{Lem}
\begin{Bew}
  $\phi$ und $\psi$ Treppenfunktionen mit Zerlegung $Z$ bzw. $Z'$ $\implies$ $Z\cup Z'$ Zerlegung für $\phi$ und $\psi$
  \[\int^b_a\alpha\phi+\beta\psi \md x = (\alpha\phi)|_{(x_{k-1},x_k)}=\alpha(\phi|_{(x_{k-1},x_k)})\]
  wobei $\Delta x_k=x_k-x_{k-1}$.\\
  Wert von $\phi$ auf $(x_{k-1},x_k)$ =: $c_k$, Wert von $\psi$ auf $(x_{k-1},x_k)$ =: $\md_k$
  \[\sum^n_{i=1}(\alpha c_k+\beta \md_k)\Delta x_k=\alpha(\sum^n_{i=1})+\beta(\sum^n_{i=1}\md_k\Delta x_k)=\alpha\int^b_a \phi \md x+\beta \int^b_a \psi \md x\]
\end{Bew}
\begin{Bem}
  $\int^b_a \md x: \tau([a,b])\to\mb{C}$ linear, $\ker(\int_a^b \md x) \subset \tau([a,b])$ Untervektorraum
\end{Bem}
\begin{Bem}
  lineares erzeugendes System von $\tau([a,b])$ $A\subset\mb{R}$
  \[1_A(x) = \begin{cases}1&\text{für} x\in A\\0&\text{sonst}\end{cases}\]
  \{$1_{[c,d]}$ mit $a<c\leq d<b$ \} erzeugendes System
\end{Bem}
\subsection{Regelfunktionen}
\begin{Def}{Regelfunktionen}
  $f:[a,b] \to\mb{C}$ \underline{Regelfunktionen} (auf $[a,b]$) $\Lra$
  \begin{itemize}
    \item 
      \begin{align*}
        \forall y\in(a,b):\exists \lim_{x\searrow y}f(x)\ \&\ \lim_{x\nearrow y} f(x)\\
        (\text{nicht nötig:} \lim_{x\searrow y} f(x) = \lim_{x\nearrow y}f(x))
      \end{align*}
    \item \[\exists \lim_{x\swarrow y} f(x)\ \&\ \exists \lim_{x_\nearrow y} f(x)\]
  \end{itemize}
\end{Def}
\begin{Bem}
  \[\lim_{x\searrow y} f(x)=c:\ \Lra\ \forall \varepsilon > 0 \exists \rho\ \forall 0<x-y<\rho: \Abs{f(x)-c}<\varepsilon\]
  $\mathcal{R}([a,b])$ Menge aller Regelfunktionen auf $[a,b]$
  \begin{align*}
    \mathcal{R}([a,b])\ \text{Vektorraum über} \mb{C}\\
    \mathcal{T}([a,b])\subset \mathcal{R}([a,b]) \text{Untervektorraum}
  \end{align*}
  \subparagraph{Frage}$\mathcal{R}([a,b])/\mathcal{T}([a,b])$ Vektorraum über $\mb{C}$, Dimension?
\end{Bem}
\begin{Bsp}
  jede stetige Funktion ist eine Regelfunktion
\end{Bsp}
\begin{Bsp}
  jede monotone Funktion auf $[a,b]$ ist eine Regelfunktion (sehe Seite 78)
\end{Bsp}
\begin{Bem}
  \begin{align*}
    f,g\in \mathcal{R}([a,b])\implies \lambda f_{\lambda\in\mb{C}}, f+g, \Abs{f}, f\cdot g, \max(f,g), \min(f,g)
  \end{align*}
  sind in $\mathcal{R}([a,b])$
\end{Bem}
\begin{Def}{gleichmässige Konvergenz}
  $(f_n)_{n\in\mb{N}}$ Folge von Funktionen auf $D\subset \mathcal{R}, f$ Funktion auf $D$.\\
  $(f_n)_{n\in\mb{N}}$ \underline{konvergiert gleichmässig} gegen $f$ $\Lra$ $\Limi{n} \underbrace{\Norm{f-f_n}}_{\sup_{x=D}\Abs{f(x)-f_n(x)}}=0$
\end{Def}
\begin{Bem}
  falls $(f_n)_{n\in\mb{N}}$ konvergiert gleichmässig $\implies$ limes ist eindeutig
\end{Bem}
\begin{Bem}
  $(f_n)_{n\in\mb{N}}$ konvergiert gleichmässig gegen $f$ $\implies$ $f_n(x)\to f(x)\ \forall x\in D$
  \[(\Abs{f(x)-f_n(x)}\leq \sup_{x\in D} \Abs{f(x)-f_n(x)}\to 0)\]
\end{Bem}
\begin{Bem}
  Die Umkehrung gilt NICHT $D=(0,1]$
  \begin{align*}
    f=0, f_n(x)=\begin{cases}1-nx&0\leq x\leq \frac{1}{n}\\0&\frac{1}{n}\leq x \leq 1\end{cases}\\
    \forall x\in D: f_n(x)\xrightarrow{n\to\infty}0\\
    \Norm{f-f_n}=\sup_{x\in D} \Abs{f(x)-f_n(x)} =1\\
    \Limi{n}\Norm{f-f_n}=1
  \end{align*}
\end{Bem}

\subsubsection{Zusammenfassung}
\begin{itemize}
  \item $\tau\left( [a,b] \right)$ = Vektorraum der Treppenfunktionen auf $[a,b]$
  \item $\int: \tau[a;b]\to\mb{C}$ lineare Abbildung
  \item Eigenschaften:
    \begin{itemize}
      \item lineare Abbildung
      \item Monotonie: $f\leq g$ $\implies$ $\int^b_af\cdot \md x\leq \int^b_ag\cdot \md x$
      \item Beschränktheit: $\Abs{\int^b_af\cdot \md x}\leq \int^a_b\Abs{f(x)}\md x\leq (b-a)\Norm{f} = \sup_{x\in[a;b]}f$
    \end{itemize}
  \item Regelfunktionen: $R\left( [a,b] \right)$ = Vektor nach der Regel $f\supset\tau\left( [a;b] \right)$
  \item gleichmässige Konvergenz $f_n\to f\xLeftrightarrow{\text{def}}\Norm{f_n-f}\to 0$
\end{itemize}
\subsubsection{Vorgehen}
\begin{enumerate}
    \item Jede Regelfunktion kann man gleichmässig durch Treppenfunktionen approximieren.
    \item Damit kann man das Integral von Regelfunktionen definieren.
    \item Regenregeln (insbesondere Hauptsatz)
    \item Riemannsche Summen
\end{enumerate}
\begin{Sat}{Approximationssatz}
  \[f\in R{a;b}\Lra \exists \text{Folge} \phi_n\in \tau[a;b]:\phi_n\to f\text{gleichmässig}\]  
  ist per Definition äquivalent mit
  \[\exists \text{Folge}\phi_n\in \tau[a;b]: \Norm{\phi_n-f}\to 0\]
  wobei
  \[\Norm{\phi_n-f}=\sup_{x\in[a;b]}\Abs{\phi_n(x)-f(x)}\]
  Dieser Grenzwert ist wiederum äquivalent mit
  \[\forall\varepsilon>0 \exists\phi \in \tau[a;b]: \Norm{f-\phi}\leq \varepsilon\]
  (eine $\varepsilon$-approximierende Treppenfunktion)
\end{Sat}
\begin{Bew}{$\Ra$}
  d.h. $f\in R\implies \exists \varepsilon$-approx. Treppen. Widerspruchsbeweis:
  \begin{align*}
    f\in R[a;b]\\
    \exists \varepsilon >0: f \text{besitzt keine} \varepsilon \text{-approx. Treppenfunktion}
  \end{align*}
  Wir konstruieren eine Intervallschachtelung $I_n=[a_n;b_n]$ s.d. $\forall_n f|_{I_n}$ besitzt keine $\varepsilon$-approx.Treppenfunktion
  \[I_1=[a;b]\]
  rekursiv: $M=\frac{b_n-a_n}{2} +a_n$ Mittelpunkt
  \begin{align*}
    I_{n+1}:=\begin{cases}
      [a_n;M]&\text{falls} f|_{[a_n;M]}\text{keine} \varepsilon \text{-approx. Treppenfunktion bestzt}\\ [M,b_n]& \text{andernfalls}        
    \end{cases}
  \end{align*}
  Sei $\xi\in I_n \forall n$
  \begin{align*}
    c_e&:=&\lim_{x\uparrow \xi} f(x)\\
    c_r&:=&\lim_{x\downarrow \xi} f(x)
  \end{align*}
  $\implies$
  \begin{align*}
    \exists \delta:&\Abs{f(x)-c_e}<\varepsilon:&x\in\left[\xi-\delta;\xi\right)\\
    &\Abs{f(x)-c_r}<\varepsilon:&x\in\left(\xi;\xi+\delta\right]\\
  \end{align*}
  Auf $[\xi-\delta;\xi+\delta]$ definieren wir eine Treppenfunktion:
  \begin{align*}
    \phi(x):=\begin{cases}
      c_e&\xi-\delta\leq x< \xi\\
      f(\xi)&x=\xi\\
      c_r&\xi+\delta\geq x>\xi
    \end{cases}
  \end{align*}
  Fall 1 $\implies$ $\phi$ ist eine $\varepsilon$-approx. Treppenfunktion auf $[\xi-\delta],[\delta+\delta]$. Fall 2 $\implies$ $\phi$ ist eine $\varepsilon$-approx. Treppenfunktion auf $[\xi-\delta],[\delta+\delta]$, alle $I_n \subset[\xi+\delta;\xi+\delta]$\\
  $\blitza$
\end{Bew}
\begin{Bew}{$\La$}
  $f$ Regelfunktion $\La$ $f$ besitzt $\varepsilon$-approx. Treppenfunktion $\forall \varepsilon>0$. Sei $x_0\in\left[a;b\right)$. Zu zeigen: $\exists \lim_{x\downarrow x_0} f(x)$.
  \begin{align*}
    \forall \varepsilon>0\ \exists\phi\in \tau[a;b]:\Norm{f-\phi}<\frac{\varepsilon}{2}\\
  \end{align*}
  Sei $\beta>x_0:\phi$ konstant auf $(x_0,\beta)$
  \begin{align*}
    \forall x,x'\in(x_0;\beta)\\
    \Abs{f(x)-f(x')}&\leq& \Abs{f(x)-\phi(x)}+\Abs{\phi(x)^{(=\phi(x')}-f(x')}\\
    &\leq&\Norm{f-\phi}+\Norm{\phi-f}<\varepsilon
  \end{align*}
  $\forall \varepsilon>0$ $\exists\beta:$ Cauchyeigenschaft gilt auf $(x_0;R)$ $\implies$ $\exists \lim_{x\uparrow x_0}f(x)$. Ähnlich: $\exists\lim_{x\uparrow x_0}f(x)$ $\forall x_0\in\left(a;b\right]$.
\end{Bew}
\begin{Kor}
  \[f\in R[a;b]\ \Lra\ \exists \text{Folge}\Psi_b\in\tau[a;b]:\sum^\infty_{k=1}\phi_k=f\]
  konvergiert konstant
\end{Kor}
\begin{Kor}
  $f$ Regelfunktion auf $I$ $\implies$ $f$ fast überall stetig. d.h. $\exists A\subset I$ s.d.
  \begin{itemize}
    \item $f|_{I\setminus A}$ stetig
    \item $A$ höchstens abzählbar $x\in[a;b]$
  \end{itemize}
\end{Kor}
\begin{Bew}
  %\subparagraph{Fall 1: $I$ kompakt}
  \begin{align*}
    \Psi_k\in\tau [I]\\
    f=\sum\phi_k \text{normal}
  \end{align*}
  Ist $\phi_k$ stetig in $x\forall k$ $\implies$ $f$ stetig in $x$.\\
  Ist $x$ Unstetigkeitsstelle von $f$, $\exists k$: $\phi_k$ unstetig in $x$, höchstens abzählbare viele $k$.
  \begin{itemize}
    \item Eine Treppenfunktion hat endlich viele Unstetigkeitsstellen
  \end{itemize}  
  \{ Unstetigkeitsstellen von $f$\} $\subset$ (höchstens abzählbare Vereinigung von endlichen Mengen) $\implies$ höchstens abzählbar
  %\subparagraph{Fall 2: $I$ nicht kompakt}
  \[I=\overbrace{U_\alpha}^{\text{höchstens abzählbar}}\overbrace{I_\alpha}^{\text{kompakt}}\]
\end{Bew}
\begin{Sat}
  \[f\in R\left( [a:b] \right)\implies f \text{beschränkt auf} [a;b]\]
\end{Sat}
\begin{Bew}
  \begin{align*}
    \varepsilon =1\\
    \exists \overbrace{\phi}^{\text{\underline{beschränkt}}}\in\tau\left( [a;b] \right):\Norm{f-\phi}\leq 1\\
    \implies \Norm{f}=\Norm{f-\phi + \phi}\leq \Norm{f-\phi}+\Norm{\phi}=\leq 1+\Norm{\phi}
  \end{align*}
\end{Bew}
\begin{Def}{Integration von Regelfunktionen}
  \ldots auch bekannt als ``Regelintegral''\\
  Sei $f\in R[a;b]$
  \[\int_a^bf(x)fx:\Limi{n}\int^b_a\phi_n(x)\md x\]
  wobei $\phi_n$ eine approximierene Folge von Treppenfunktionen ist (d.h. $\Norm{\phi_n-f}\to 0$)
\end{Def}
\subparagraph{zu zeigen:}
\begin{enumerate}
  \item Die Folge $I_n:=\int^b_a\phi_n(x)\md x$ konvergiert $\forall \Norm{\phi_n-f}\to 0$
  \item Der Grenzwert ist von der Wahl der approximierenden Folge unabhängig
\end{enumerate}
\begin{Bew}{von 1}
  \[\Abs{I_n-I_m}=^{\text{Linearität}}\Abs{\int^b_a\left( \phi_n(x)-\phi_m(x) \right)\md x}\leq^{\text{beschränkt}} (b-a)\Norm{\phi_n-\phi_m}\]
  \[\Norm{\phi_n-f}\to 0 \xRightarrow{\text{Dreiecksungleichung}} \forall\varepsilon>0\ \exists N: \Norm{\phi_n-\phi_m}<\varepsilon\ \forall n,m> N\]
  $\implies$ $I_n$ Cauchyfolge $\implies$ $I_n$ konvergiert
\end{Bew}
\begin{Bew}{von 2}
  Seien $\phi_n, \psi_n\in \tau[a;b]$
  \begin{align*}
    \Norm{\phi_n-f}\to0\\
    \Norm{\psi_n-f}\to0
  \end{align*}
  \begin{align*}
    \{X_n\}=\psi_1,\phi_1,\psi_2,\phi_2,\psi_3,\phi_3,\cdots\\
    X_n:=\begin{cases}
      \phi_{\frac{n}{2}}& n \text{gerade}\\
      \psi_{\frac{n+1}{2}}& n \text{ungerade}
    \end{cases}
  \end{align*}
  $\implies$ $I_n(\phi)$ und $I_n(\psi)$ Teilfolgen von $I_n(X)$
  \begin{align*}
    \implies \Norm{X_n-f}\to 0\\
    I_n(x)=\int x_n\\ I_n(\phi)=\int \phi_n\\ I_n(\psi)=\int \psi_n\\
  \end{align*}
  $\implies$
  \begin{align*}
    \lim I_n(\phi)=\lim I_n(X)=\lim I_n(\psi)
  \end{align*}
\end{Bew}
\begin{Bsp}{Dirichlet}
  eine Funktion, die keine Regelfunktion ist.
  \begin{align*}
    f:[0;1]\to\mb{R}\\
    f(x)=\begin{cases}
      1 & x\in \mb{Q}\\
      0 & x\in \mb{R}\setminus\mb{Q}
    \end{cases}
  \end{align*}
  $f$ unstetig $\forall x$ intuitiv: $\int^1_0 f(x)fx =0$
\end{Bsp}
\begin{Bsp}{Riemann}
  sog. modifizierte Dirichlet-Funktion
  \begin{align*}
    g:[0;1]\to\mb{R}\\
    g(x)=\begin{cases}
      \frac{1}{q}  & x=\frac{p}{q}, p,q \text{teilerfremd}, q>0\\
      0 & x\in \mb{R}\setminus \mb{Q}
    \end{cases}
  \end{align*}
  $g\in R[0;1]$ und $int^b_ag(x)\md x=0$
\end{Bsp}
\subsubsection{Eigenschaften}
\begin{Sat}
  \[\forall f, g\in R [a;b] \forall \alpha,\beta\in \mb{C} \text{gelten}\]
  \begin{description}
    \item[Linearität] \[\int^b_a(\alpha f+ \beta g)\md x = \alpha \int^b_af\cdot \md x+\beta\int^b_a g\cdot \md x\]
    \item[Beschränktheit] \[\Abs{\int^b_a f(x)\cdot \md x}\leq \int^b_a\Abs{f(x)}\md x\leq (b-a)\Norm{f}\]
    \item[Monotonie] \[f\leq g \implies \int^b_af(x)\md x\leq \int^b_ag(x)\md x\]
  \end{description}
  ($f,g$ reellwertig $f(x)\leq g(x)\forall x$)
\end{Sat}
\begin{Sat}{Additivität}
  Sei $f\in R[a;b]$ und sei $c\in (a;b)$
  \[\int^b_af(x)\md x=\int^c_a f(x)\md x+\int^b_cf(x)\md x\]
\end{Sat}

\begin{Bew}
  $f=\phi$ Treppenfunktion trivial
  \[f=\lim\phi_n\ \text{gleichmässig}\]
  \begin{align*}
    \phi_n\in \tau[a;c]
    \phi_n^l &:=& \phi_n|_{[a;b]}&\in\tau[a;b]\\
    \phi_n^r &:=& \phi_n|_{[b;c]}&\in\tau[b;c]\\
  \end{align*}
  \begin{align*}
    \int^c_a\phi_n(x)\md x = \int^b_a\phi_n^l(x)\md x+\int^c_b \int^r_n(x)\md x\\
    \Norm{\phi_n-f}\to0\\
    \Norm{\phi_n^l-f}_{[a;b]}\leq \Norm{\phi_n-f}\geq \Norm{\phi_b^+f}_{[b;c]}
  \end{align*}
  \begin{align*}
    \int^c_a\phi_n(x)\md x &=& \int^b_a\phi_n^l(x)\md x&+&\int^c_b \int^r_n(x)\md x\\
    =\int^c_a f \cdot \md x & &=\int^b_a f(x) \cdot \md x & & =\int^c_b f(x) \md x
  \end{align*}
  $\implies$
  \begin{align*}
    \phi_n^l&\to f|_{[a;b]}\\
    \phi_n^r&\to f|_{[b;c]}\\
  \end{align*}
\end{Bew}
\begin{Def}
  $f\in \mathcal{R}[a;b]$, $b>a$
  \[\int^a_bf(x)\md x := \int^b_af(x)\md x\]
  \[\int^a_af(x)\md x := 0 \]
\end{Def}
\begin{Sat}
  $f\in \mathcal{R}I(): \forall a,b,c \in I$
  \[\int^c_af(x)\md x = \int^b_af(x)\md x + \int^c_bf(x)\md x\]
\end{Sat}
\begin{Bem}
  \begin{description}
    \item[Linearität]
    \item[Beschränktheit]: \[\Abs{\int^b_af(x)\md x}\leq \Abs{\int^b_a\Abs{f(x)}\md x}\leq \Abs{b-a}\Norm{f}\]
    \item[Monotonie]
  \end{description}
  \begin{align*}
    f\leq g; b>a\\
    \int^b_a f(x)\md x \geq \int^b_ag(x)\md x
  \end{align*}
\end{Bem}
\begin{Bem}
  $f$ stetig ($[a;b]$) $\implies$ $\Norm{f} = \max\Abs{f}$\\
  reellwertig $\xRightarrow{\text{ZWS}}$ $f$ nimmt alle Werte zwischen $0$ und $\max\Abs{f}$ % zwischen <> und - nachprüfen
  \begin{align*}
    \exists\xi\in [a;b]:\\
    \int^b_af(x)\md x=(b-a)f(\xi)
  \end{align*}
\end{Bem}
\begin{Sat}{Mittelwertsatz}
  Sei $f:[a;b]\to \underline{\mb{R}}$ \underline{stetig}. Sei $p:[a;b]\to\mb{R}\in\mathcal{R}$ mit $p\geq 0$. Dann $\exists \xi\in [a;b]$ s.d.
  \[\int^b_a f(x)p(x)\md x=f(\xi)\int^b_a p(x)\md x\]
  Falls $\int p \neq 0$
  \begin{align*}
    \frac{\int f(x)p(x)\md x}{\int p(x)\md x}=f(\xi)=\int^b_a f(x)\tilde{p}(x)\md x\\
    \tilde{p}(x)=\frac{p(x)}{\int^b_ap(x)\md x}\\
    \implies \int^b_a \tilde{p}(x) \md x=1
  \end{align*}
\end{Sat}
\begin{Bew}
  $f$ besitzt ein Maximum $M$ und ein Minimum $m$
  \begin{align*}
    m\leq f(x) \leq M\ \forall x\in [a;b]\\
    m p(x)\leq f(x)p(x)\leq M p(x)\\
  \end{align*}
  $\xRightarrow{\text{Monotonie}}$
  \begin{align*}
    \int^b_am p(x)\md x &\leq& int^b_a f(x)p(x)\md x &\leq& \int^b_a M p(x)\md x\\
    = m\int^b_ap(x)\md x & & & &=M \int^b_a p(x)\md x
  \end{align*}
  $\implies \exists \mu\in [m;M]$:
  \begin{align*}
    \int^b_af(x)p(x)\md x = \mu \int^b_a p(x) \md x    
  \end{align*}
  ZWS $\implies$ $\exists \xi \in [a;b]$:
  \[\mu=f(\xi)\]
\end{Bew}
\begin{Sat}
  Sei $f:[a;b]\to \mb{R}\in\mathcal{R}$ mit $f\geq 0$ und $\int^b_af(x)\md x=0$. Dann ist $f(x_0)=0$ an jeder Stetigkeitsstelle $x_0$. Ferner gilt: $f=0$ fast überall.
\end{Sat}
\begin{Bew}{(Widerspruchsbeweis)}
  Sei $x_0$ eine Stetigkeitsstelle mit $f(x_0)>0$. $f$ stetig in $x_0$ $\implies$ $\exists x_0 \in [a:b]\subset[a:b]$ s.d.
  \[f(x)>\frac{1}{2}f(x_0)\ \forall x\in [\alpha:\beta]\]
  Sei
  \[\phi(x):=\begin{cases}
    \frac{1}{2}f(x_0)&x\in [\alpha;\beta]\\
    0 & x\not\in [\alpha;\beta]
  \end{cases}\]
  Treppenfunktion, deshalb Regelfunktion
  \[\implies f\geq \phi \implies \underbrace{\int^\beta_\alpha f(x)\md x}_{=0} \geq \int^\beta_\alpha\phi(x)\md x=\frac{\beta-\alpha}{2}f(x_0)>0\]
  $\blitza$
\end{Bew}
\begin{Sat}
  $f\in\mathcal{R}$ $\implies$ $f$ besitzt höchstens abzählbar viele Unstetigkeitsstellen $\implies$ $f=0$ fast überall
\end{Sat}
\begin{Kor}
  $f:[a;b]\to\mb{R}$ stetig, $f\geq 0$, $\int_a^bf(x)\md x=0$ $\implies$
  \begin{align*}
    f(x)=0\ \forall x\in [a;b]    
  \end{align*}
\end{Kor}
\subsection{Fundamentalsatz der Analysis}
\begin{Sat}
  Sei $f:I\to\mb{C}\in\mathcal{R}$ und sei $a\in I$. Für jedes $x\in I$ definiert man
  \[F(x):=\int_a^x f(t)\md t\ F:I\to\mb{C}\]
  Dann ist $F$ eine Stammfunktion zu $f$ (d.h. $F$ ist stetig und fast überall differenzierbar (und $F'=f$ fast überall)) mit
  \begin{align*}
    F_+'(x_0)=f_+(x_0)\\
    F_-'(x_0)=f_-(x_0)
  \end{align*}
  $\forall x_0 \in I$
\end{Sat}
\begin{Bew}
  $\forall x_1,x_2\in I$ gilt
  \begin{align*}
    F(x_2)-F(x_1)=\int_a^{x_2}f(t)\md t-int^{x_1}_a f(t)\md t =\\
    =\int^{x_2}_a+\int^a_{x_1}= \int^{x_2}_{x_1}f(t)\md t    
  \end{align*}
  Sei $\tau\subset I$ Teilintervall. $\forall x_1, x_2\in \tau$
  \begin{align*}
    \Abs{f(x_2)-F(x_1)}=\Abs{\int^{x_2}_{x_1}f(t)\md t}\leq^{\text{Bijektivität}} \Abs{x_2-x_1}\Norm{f}_\tau
  \end{align*}
  $\implies$ $F|_\tau$ Lipschitz-stetig $\implies$ $F|_\tau$ stetig $\forall \tau \implies$ \underline{$F$ stetig auf $I$}.\\
  Wir berechnen $F_+'(x_0)$. $f\in\mathcal{R}$ $\implies \exists f_+(x_0)$. $\forall \varepsilon>0 \exists \delta >0$
  \begin{align*}
    \Abs{f(x)-f_+(x_0)}<\varepsilon\ \forall x\in(x_0, x_0+\delta)\\
  \end{align*}
  Für $x\in (x_0, x_0+\delta)$
  \begin{align*}
    \Abs{ \frac{F(x)-F(x_0)}{x-x_0} -f_+(x_0) } = \\ \Abs{ \frac{1}{x-x_0} \int_{x_0}^x f(t)\md t-\frac{f_+(x_0)}{x-x_0} \int_{x_0}^x <Fehlt da nicht was?> \md t} = \\
    \Abs{\frac{1}{x-x_0}}\int_{x_0}^x\left( f(t)-f_+(x_0) \right)\md t \leq\\
    \frac{1}{\Abs{x-x_0}}\Abs{x-x_0}\Norm{f(x)-f_+(x_0)}_{x_0;x} \leq \varepsilon
  \end{align*}
\end{Bew}
\begin{Kor}
  Sei $f:I\to\mb{C}\mathcal{R}$ und sei $\Phi$ eine Stammfunktion zu $f$. Dann $\forall a,b\in I$
  \begin{align*}
    \int^b_af(x)\md x&=&\Phi(b)-\Phi(a)\\
    &=:&\Phi|^b_a
  \end{align*}
\end{Kor}
\begin{Bew}
  $\Phi$ und $F$ sind Stammfunktionen zu $f$, insbesondere $\Phi'=F'$ fast überall. Eindeutigkeitssatz $\implies \exists c$ konstant s.d.
  \[\Phi(x)=F(x)+c\ \forall x\in I\]
  \begin{align*}
    \int^b_af(x)\md x=F(b)=F(b)-\underbrace{F(a)}_{=0}=\\
    =\left( \Phi(b)-c \right) - \left( \Phi(a)-c \right) = \Phi(b)-\Phi(a)
  \end{align*}
\end{Bew}
\begin{Kor}
  Jede Regelfunktion beseitzt eine Stammfunktion  
\end{Kor}
\begin{Def}
  Eine Funktion heisst fast überall stetig differenzierbar, wenn sie die Stammfunktion zu einer Regelfunktion ist. (Wo sie nicht stetig differenzierbar ist, besitzt sie linke und Rechte Grenzwerte)
\end{Def}
\begin{Bsp}
  \[f(x)=\begin{cases}
    0& x=0\\
    x^2\sin\frac{1}{x}&x\neq 0
  \end{cases}\]
  $f$ ist in $\mb{R}\setminus\{0\}$ differenzierbar. $f'$ besitzt linke und rechte Grenzwerte, in 0 nicht. Also keine Regelfunktion.
\end{Bsp}
\begin{Bem}
  Mit dem Lebesgne-Integral kann man solche Funktionen aus einem Integral erhalten.
\end{Bem}
\begin{Eig}{Charakterisierung}
  $f$ fast überall stetig differenzierbar auf $I$ $\implies$ $\exists A\subset I$, $A$ höchstens abzählbar s.d.
  \begin{enumerate}
    \item $f$ ist auf $I\setminus A$ differenzierbar
    \item $f'$ ist auf $I\setminus A$ stetig
    \item $\forall x\in A$ existieren $f_+'(x)$ und $f_-'(x)$
  \end{enumerate}
\end{Eig}
\begin{Def}{unbestimmtes Integral}
  Das unbestimmte Integral der Regelfunktion $f$ ist die Gesamtheit aller Stammfunktionen zu $f$.
\end{Def}
\begin{Not}{unbestimmtes Integral}
  \[\int f(x)\md x\]
  In Tabellen wird oft
  \[\int x\md x = \frac{x^2}{2}\]
  geschrieben
\end{Not}
\begin{Bsp}
  \[\int x\md x = \frac{x^2}{2} + C\]
\end{Bsp}
\begin{Eig}
  \begin{align*}
    \int x^a\md x &=&  \frac{x^{a+1}}{a+1}\ a\in \mb{C}\setminus \{-1\}\\
    \int \frac{1}{x}\md x &=&  \ln\Abs{x}\\
    \int e^{cx} \md x &=&  \frac{1}{c}e^{cx},\ c\neq 0\\
    \int \sin x \cdot \md x &=& -\cos x\\
    \int \cos x \cdot \md x &=& \sin x
  \end{align*}
\end{Eig}
\begin{Sat}
  Seien $f_1$ und $f_2$ Regelfunktionen auf $I$
  \begin{align*}
    f_1=f_2 \text{f.ü.} \implies \int f_1 \md x=\int f_2 \md x
  \end{align*}
  Insbesondere $\forall a,b \in I$
  \begin{align*}
    \int^b_af(x)\md x = \int^b_a f_2(x)\md x
  \end{align*}
\end{Sat}
\begin{Bew}
  Sei $F_1$ / $F_2$ Stammfunktion zu $f_1$ / $f_2$
  \begin{align*}
    \implies F_1'=F_2'\ \text{f.ü.}\\
    \implies F_1=F_2+C
  \end{align*}
\end{Bew}
\begin{Bem}{Anwendung}
  \begin{align*}
    f(x)=\begin{cases}
      \frac{1}{q} & x=\frac{p}{q}, p,q \text{teilerfremd}\\
      0 & x\neq \mb{Q}
    \end{cases}\\
    \int^b_a f(x)\md x =0
  \end{align*}
\end{Bem}



\newpage

%= Stichwortverzeichnis ======================================================================
\rhead{}
\addcontentsline{toc}{section}{Stichwortverzeichnis}
\printindex

\end{document}
