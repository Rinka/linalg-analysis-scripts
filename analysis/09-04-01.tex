\begin{Def}{Homöomorphismus}
  $f:X\to Y$ heisst
  \begin{enumerate}
    \item $f$ stetig
    \item $f$ ist umkehrbar
    \item $f^{-1}Y\to X$stetig
  \end{enumerate}
\end{Def}
\begin{Eig}{Homöomorphismus}
  In diesem Falle sind auch die Bilder offener Mengen offen.  
\end{Eig}
\begin{Bsp}
  $V,W$ endlich dimensionale Vektorräume, $f:V\to W$ linear umkehrbar $\implies$ $f$ Homöomorphismus
\end{Bsp}
\begin{Def}{homöomorphe Räume}
  Zwei metrische Räume $X,Y$ heissen homöomorph, wenn es einen Homöomorphismus $X\to Y$ gibt.
\end{Def}
\begin{Bem}
  $\phi:X\to Y$ und $\psi:Y\to Z$ Homöomorphismus $\psi\circ\phi$ Homöomorphismus.
\end{Bem}
\begin{Bsp}
  Zwei endlich dimensionale Vektorräume der \underline{gleichen} Dimension sind homöomorph.
\end{Bsp}
\begin{Bem}
  Man kann zeigen: Zwei endlich dimensionale Vektorräume sind genau dann homöomorph, wenn sie die gleiche Dimension haben. Im Allgemeinen $\not\exists$ Homöomorphismus $\mb{R}^m\to\mb{R}^n$ $m\neq n$
\end{Bem}
\begin{Bsp}
  $K_1(0)\in\mb{R}^n$ sind Homöomorph.
  \begin{align*}
    f:&K_1(0)\to\mb{R}^n\\
    &x\mapsto \frac{x}{1-\Norm{x}}\\
    g:&\mb{R}^n\to K_1(0)\\
    &y\mapsto \frac{y}{1+\Norm{y}}
  \end{align*}
\end{Bsp}
\begin{Bem}
  Polarkoordinaten
  \begin{gather*}
    \mb{R}^2\to\mb{R}^2\\
    (r,\phi)\mapsto \left( r\cos \phi, r\sin \phi \right)
  \end{gather*}
  stetig, nicht bijektiv\\
  $r>0$, $\phi\in\left( -\pi, \pi \right)$, Bild: $\mb{R}^2\setminus S$, $S=\left\{ (x,0)\in\mb{R}^2, x\leq 0 \right\}$
  \begin{align*}
    P_2:&\mb{R}^+_*\times (-\pi,\pi)\to\mb{R}^2\setminus S\\
    &(r,\phi)\mapsto\left( r\cos \phi, r\sin \phi \right)
  \end{align*}
  Homöomorphismus\\
  Umkehrabbildung: $r=\sqrt{x^2+y^2}$, $\phi \sign(y)\arccos\frac{x}{\sqrt{x^2+y^2}}$
\end{Bem}
\begin{Bem}
  3d
  \begin{align*}
    \begin{cases}
      x_1=r\cos \phi_1\cos\phi_2\\
      x_2=r\sin\phi_1\cos\phi_2\\
      x_3=r\sin\phi_2
    \end{cases}\\
    r>0\\
    \phi_1\in\left( -\pi,\pi \right)\\
    \phi_2\in\left( -\frac{\pi}{2},\frac{\pi}{2} \right)
  \end{align*}
  Bild $\mb{R}^3\setminus(S\times \mb{R})$
  \[\Mx{x_1\\x_2}=\Mx{r\cos\phi_1\\r\sin\phi_1}\cos\phi_2=P_2(r,\phi_1)\cos\phi_2\]
  Im allgemeinen definiert man Polarkoordinaten rekursiv
  \begin{gather*}
    P_n:\mb{R}^+_*\times\prod_{n-1}\to\mb{R}^n\setminus(S\times \mb{R}^{n-2})\\
    \prod_{n-1}=(-\pi,\pi)\times\left( -\frac{\pi}{2},\frac{\pi}{2} \right)^{n-2}\\
  \end{gather*}
  \begin{gather*}
    P_n(r,\phi_1,\phi_2,\cdots,\phi_{n-1})=\Mx{P_{n-1}(r,\phi_1,\cdots,\phi_{n-2})\cos\phi_{n-1}\\ r\sin\phi_{n-1}}
  \end{gather*}
\end{Bem}
\begin{Def}{stetige Erweiterung/Grenzwert}
  Seien $X,Y$ metrische Räume, $D\in X$, $f:D\to Y$, $a\in X$ Häufungspunkt $D$, $b\in Y$. Man definiert
  \begin{gather*}
    F:D\cup \left\{ a \right\}\to Y\\
    F(x):=\begin{cases}
      f(x)&x\in D\setminus \left\{ a \right\}\\
      b&x=a
    \end{cases}
  \end{gather*}
  \begin{itemize}
    \item $F$ heisst die stetige Erweiterung von $f$ in Punkt $a$ wenn $F$ stetig in $a$ ist.
    \item In diesem Falle heisst $b$ die Grenzwert von $f$ in Punkt $a$
  \end{itemize}
\end{Def}
\begin{Not}
  \[b=\lim_{x\to a}f(x)\]
\end{Not}
\begin{Bem}
  \begin{itemize}
    \item Die stetige Erweiterung ist eindeutig bestimmt, wenn sie exisitiert.
    \item Der Grenzwert ist eindeutig bestimmt, wenn er existiert.
  \end{itemize}
\end{Bem}
\begin{Lem}
  \[\lim_{x\to a}f(x)=b\ \Lra\ \forall \varepsilon>0\ \exists \delta>0:d_y(f(x),b)<\varepsilon\ \forall x\in D\setminus \left\{ a \right\}, d_x(x,a)<\delta\]
\end{Lem}
\begin{Def}
  Sei $x$ ein metrischer Raum, $(x_k)$ Folge in $x$. $(x_k)$ heisst Cauchyfolge wenn
  \[\forall \varepsilon>0\ \exists N:d_x(x_k,x_l)<\varepsilon\ \forall x,l\geq N\]
\end{Def}
\begin{Def}{vollständiger Raum}
  Ein metrischer Raum $X$ heisst vollständig, wenn jede Cauchyfolge in $X$ konvergiert.
\end{Def}
\begin{Bsp}
  $\mb{R}^n$ mit einer Norm vollständig
\end{Bsp}
\begin{Bem}
  \begin{itemize}
    \item Wir haben die Aussage für die euklidische Norm bewiesen
    \item Aber je zwei Normen auf $\mb{R}^n$ sind äquivalent
  \end{itemize}
\end{Bem}
\begin{Bsp}
  Jeder endlich dimensionaler, normierter Raum ist vollständig.
\end{Bsp}
\begin{Lem}
  Sei $(X,d)$ vollständig, $M\subset X$
  \[M\ \text{vollständig}\ \Lra\ M\ \text{abgeschlossen}\]
  (bezüglich der Spurmetrik)
\end{Lem}
\begin{Bsp}
  $[a;b]\subset\mb{R}$ ist vollständig
\end{Bsp}
\begin{Bew}
  $\Ra$ Sei $M$ vollständig. Sei $(x_k)$ konvergente Folge in $X$ mit $x_k\in M\forall k$. $(a_k)$ konvergiert $\implies$ $(x_k)$ Cauchy $\xRightarrow{M\ \text{vollständig}}$ $x_k\to x\in M$ $\implies$ $M$ abgeschlossen.\\
  $\La$ Sei $M$ abgeschlossen. Sei $(x_k)$ eine Cauchyfolge in $M$ $\implies$ $(x_k)$ eine Cauchyfolge in $X$ $\xRightarrow{X\ \text{vollständig}}$ $x_k\to x\in X$. $x_k\in M,(x_n)$ konvergiert $\implies$ $x\in M$ $\implies$ $M$ vollständig.
\end{Bew}
\begin{Kor}
  Jede abgeschlossene Teilmenge von $\mb{R}^b$ ist vollständig.
\end{Kor}
\begin{Bem}{Vereinbarung}
  $\mb{R}^n$ ist für nur innen als normierter Raum betrachtet.
\end{Bem}
\begin{Def}{Barnachraum}
  Ein normierter $\mb{K}$-Vektorraum heisst Barnachraum, wenn er vollständig ist.
\end{Def}
\begin{Bsp}
  Jeder endlich dimensionaler Vektorraum ist ein Barnachraum.
\end{Bsp}
\begin{Bem}
  Nicht jeder unendlich dimensionaler Vektorraum ist ein Barnachraum.
\end{Bem}
\begin{Bsp}
  Sei $V=\mathcal{C}^0\left( [a;b],\mb{R} \right)$ mit $L^1$-Norm: $\Norm{f}=\int^b_a\Abs{f}\md x$ $V$ ist nicht vollständig.
\end{Bsp}
\begin{Bsp}
  $a=0$, $b=2$
  \[f_n(x):=\begin{cases}
    x^n&0\leq x<1\\
    1&1\leq x \leq 2
  \end{cases}\]
  $f_n$ stetig $\forall n$, $(f_n)$ Folge in $V$, $(f_n)$ Cauchy $n>m$
  \[\Norm{f_m-f_n}=\int^1_0(x^m-x^n)\md x=\frac{1}{n+1}-\frac{1}{n+1}<\frac{1}{m+1}\]
  $f_n$ konvergiert in $V$ nicht
  \[f_n(x)\to f(x) =\begin{cases}
    0&0\leq x<1\\
    1&1\leq x\leq 2
  \end{cases}\]
  $f\not\in V$
\end{Bsp}
\begin{Bsp}
  $\mathcal{C}^0\left( [a;b];\mb{R} \right)$ mit Supremum-Norm ist vollständig.
\end{Bsp}
\subsection{Vervollständigung}
\begin{Sat}
  Jeder normierte Vektorraum $(V,\Norm{\ }_V)$ kann vervollstänigt werden. D.h.
  \begin{gather*}
    \exists \text{Barnachraum}(W,\Norm{\ }_W)\\
    i:V\hookrightarrow W\ \text{lineare Inklusion}\\
    \Norm{i(v)}_W=\Norm{v}_V\ \forall v\in V\ \text{s.d.}\ \overline{i(V)}=W
  \end{gather*}
\end{Sat}
\begin{Bew}
  (Konstruktion)
  \begin{gather*}
    W:=\left\{ \text{Cauchyfolgen in $V$} \right\}\setminus\left\{ \text{Nullfolgen in $V$} \right\}\\
    \Norm{\left[ (x_k) \right]}_W:=\Limi{k}\Norm{x_k}_V\\
  \end{gather*}
  \begin{align*}
    i:&V\hookrightarrow W\\
    &v\mapsto\left[ \text{konstante Folge}\ x_k=v\ \forall k \right]
  \end{align*}
\end{Bew}
\begin{Def}{Hilbertraum}
  Ein Vektorraum mit einem Skalarprodukt der bezüglich der induzierten Norm vollständig ist, heisst Hilbertraum.
\end{Def}
\begin{Bsp}
  \[l^2:=\left\{ (x_k)\ \text{in}\ \mb{C}:\sum\Abs{x_k}^2<\infty \right\}\]
  \[(x,y):=\sum \overline{x_k}y_k\]
\end{Bsp}
\begin{Bsp}
  \[\mathcal{C}^0\left( [a;b],\mb{C} \right), L^2\ \text{Norm}\]
  \[\left\langle f,g \right\rangle =\int^b_a\bar f(x)g(x)\md x\]
  nicht vollständig\\
  Vervollständigung: $L^2\left( [a;b] \right)$ für die Quantenmechanik
\end{Bsp}
