\begin{Eig}{Sektorformel}
  \begin{enumerate}
    \item Additivität: $c\in(a;b)$
      \[F(\gamma)=F\left(\gamma_{|_{[a;c]}}\right)+F\left( \gamma_{|_{[c;b]}} \right)\]
    \item Orientierungsumkehrung
      \[F(\gamma^-)=-F(\gamma)\]
      \[\gamma(t):=\gamma(-t)\]
    \item
      \[A:\mb{R}^2\to\mb{R}^2\]
      \[\Mx{e&f\\g&h}\Mx{x\\y}\mapsto \Mx{ex+fy\\gx+hy}\]
      \begin{gather*}
        (A\gamma)(t)=A\gamma(t)\\
        d(A\gamma)=A\dot\gamma\\
        x\dot y-y\dot x=\det\Mx{\gamma&|&\dot\gamma}=\det\Mx{x&\dot x\\y&\dot y}\\
        F(\gamma)=\frac{1}{2}\det\Mx{\gamma&|&\dot\gamma}\md t\\
        \det\Mx{A\gamma&|&A\dot\gamma}=\det\Mx{A(\gamma&|&\dot\gamma)}=\det A\det\Mx{\gamma&|&\dot\gamma}\\
        F(A\gamma)=\det A\cdot F(\gamma)\\
        \text{insbesondere}\\
        \det A=1(\text{d.h.}\ A\in SL(2;\mb{R}))\\
        F(A\gamma)=F(\gamma)
      \end{gather*}
  \end{enumerate}
\end{Eig}
\begin{Def}{Geschlossene Kurve}
  Eine Kurve $\gamma:[a;b]\to\mb{R}^n$ heisst geschlossen, wenn
  \[\gamma(a)=\gamma(b)\]
  gilt.
\end{Def}
\begin{Def}{umschlossener orienterierter Flächeninhalt}
  Sei $\gamma:[a;b]\to\mb{R}^n$ geschlossen und so dass $F(\gamma)$ existiert, so heisst $F(\gamma)$ der umschlossene orienterierte Flächeninhalt.
\end{Def}
\begin{Bem}
  $\gamma(a)=\gamma(b)$
  \begin{align*}
    \int_a^b\md (xy)\md t=(xy)|^b_a=0\\
    F(\gamma)=\int^b_ax\dot y\md t=-\int^b_a\dot x y\md t
  \end{align*}
  (wenn $\gamma$ geschlossen)
\end{Bem}
\begin{Bem}
  Polarkoordinaten
  \begin{gather*}
    (x,y)\in\mb{R}^2\\
    \rho e^{i\phi} = x+iy=:z\in\mb{C}\\
    \gamma:[a;b]\to\mb{R}^2\\
    \dot z= \dot \rho e^{i\phi}+i\rho\dot\phi e^{i\phi}
  \end{gather*}
  \begin{align*}
    t\mapsto& (x(t),y(t))\\
    t\mapsto&z(t)\\
    t\mapsto&\rho(t)e^{i\phi(t)}
  \end{align*}
  Man erlaubt $\rho(t)<0$
\end{Bem}
\begin{Bem}{Länge}
  \begin{gather*}
    L=\int_a^b\Norm{\dot \gamma}\md t=\int^b_a\sqrt{\dot\bar z\dot z}\md t\\
    \bar z=\rho e^{i-\phi},\ \dot\bar z=\dot\rho e^{-i\phi}-i\rho\dot\phi e^{-i\phi}\\
    z=x+iy,\ \bar z=x-iy,\\
    \dot z=\dot x+iy,\ \dot\bar z=\dot x-i\dot y\\
    \bar z\dot z=(x\dot x+y\dot y)+i(x\dot y-\dot xy)\\
    =\frac{1}{2}\int\Im(\bar z\dot z)\md t\\
    \bar z\dot z=\rho e^{-i\phi}\left(\dot \rho e^{i\phi}+ \rho\dot\phi e^{i\phi}\right) = \rho\dot\phi+i\rho^2\dot\phi=\\
    =\frac{1}{2}\int^b_a\rho^2\dot\phi\md t
  \end{gather*}
\end{Bem}
\begin{Bsp}
  \begin{align*}
    y:[0;2\pi]&\mapsto\mb{R}^2\\
    \phi&\mapsto a\cos(3\phi)e^{i\phi}\\
  \end{align*}
  \begin{align*}
    \rho(\phi)=a\cos(3\phi)
  \end{align*}
  $\rho$ kann auch negativ sein
  \begin{gather*}
    F(\gamma)=\frac{3}{2}\int_0^{\frac{\pi}{3}}a^2\cos^2(3\phi)\md\phi =\\
    =\frac{\not 3}{2}\int^{2\pi}a^2\cos^2(\phi)\frac{\md \phi}{\not 3}=\\
    \frac{a^2}{2}\int^{2\pi}_0\frac{(\cos^2\phi+\sin^2\phi)}{2}\md \phi=\frac{a^2}{4}2\pi=\frac{a^2\pi}{2}\\
    \int^{2\pi}_0\cos^2=\int_0^{2\pi}\sin^2
  \end{gather*}
\end{Bsp}
\section{Taylor [Kap 14]}
Wir wollen eine Funktion durch Polynom approximieren.
\begin{Def}
  Sei $f:I\to\mb{C}$ $n$-mal differenzierbar. Das $n$-te Taylorpolynom von $f$ im Punkt $a\in I$ ist das Polynom $T(x)$ des Grades $\leq n$ mit
  \begin{gather*}
    T(a)=f(a)\\
    T'(a)=f'(a)\\
    T''(a)=f''(a)\\
    \cdots
    T^{(n)}(a)=f^{(n)}(a)
  \end{gather*}
\end{Def}
\begin{Not}
  $I_n f(x;a)$
\end{Not}
\begin{Bsp}{$n=1$}
  \[T_1 f(x;a)=f(a)+f'(a)(x-a)\]
\end{Bsp}
\begin{Bem}
  Sei $I_n f(x;a)$ das $n$-te Taylorpolynom von $f$
  \[T(x)=I_n f(x;a)=\sum^n_{k=0}a_k(x-a)^k\]
  \begin{gather*}
    f(a)T'(x)=\sum^n_{k=1} k a_k(x-a)^{k-1}\\
    f(a)T''(x)=\sum^n_{k=2} k(k-1) a_k(x-a)^{k-2}\\
    f(a)T'''(x)=\sum^n_{k=3} k(k-1)(k-2) a_k(x-a)^{k-3}\\
    \cdots\\
    T(a)=a_0\\
    T'(a)=a_1\\
    T''(a)=2a_2\\
    T'''(a)=3\cdot 2a_3
  \end{gather*}
  Übung $l\leq n$ (Induktion)
  \[T^{(l)}_{(x)}=\sum^n_{k=l}k(k-1)(k-2)\cdots(k-l+1)a_k(x-a)^{k-l}\]
  \[T^{(l)}(a)=l!a_l=f^{(l)}(a)\]
  \[a_l=\frac{f^{(l)}(a)}{l!}\]
\end{Bem}
\begin{Eig}
  \[T_nf(x;a)=\sum^n_{k=0}\frac{f^{(k)}(a)}{k!}(x-a)^k\]
\end{Eig}
\begin{Def}{Fehler}
  \[R_{n+1}(x;a):=f(x)-T_nf(x;a)\]  
\end{Def}
\begin{Lem}
  \[\Limo{x}\frac{R_{n+1}(x;a)}{(x-a)^n}=0\]
  \begin{gather*}
    R_2=f(x)-T_1 f(x,a)\\
    T_1f(x;a)=f(a)+f'(a)(x-a)\\
    R_2= \frac{f(x)-f(a)-f'(a)(x-a)}{x-a}\xrightarrow{f\text{differenzierbar}} 0
  \end{gather*}
\end{Lem}
\begin{Bew}
  $T=T_nf$
  \begin{align*}
    &\lim_{x\to a}\frac{f(x)-T(x)}{(x-a)^n}=\\
    (L'Hopital)& = \lim_{x\to a}\frac{f'(x)-T'(x)}{n(x-a)^{n-1}}=\\
    (L'Hopital)& = \lim_{x\to a}\frac{f''(x)-T''(x)}{n(n-1)(x-a)^{n-2}}=\cdots\\
    \cdots&=\lim_{x\to a}\frac{f^{(n)}(x)-T^{(n)}(x)}{n!}=0
  \end{align*}
  denn $f^{(n)}(a)=T^{(n)}(a)$
\end{Bew}
\begin{Kor}{Qualitative Taylorformel}
  Sei $f:I\to\mb{C}$ stetig und $n$-mal differenzierbar. Dann
  \[\exists r;I\to\mb{C}\]
  stetig mit
  \[r(a)=0\]
  s.d.
  \[f(x)=I_nf(x;a)+(x-a)^nr(x)\]
\end{Kor}
\begin{Bew}
  \begin{gather*}
    r(x):=\frac{f(x)-I_nf(x;a)}{(x-a)^n}
  \end{gather*}
  $x\neq a$ stetig auf $I\setminus \left\{ a \right\}$
  \[\lim_{x\to a}r(x)\]
  Wir erweiter $r$ auf $I$ mit $r(a)=0$
\end{Bew}
\begin{Not}{Landan-Symbol}
  Seien $f$ und $g$ komplexe Funktionen in einer punktierten Umgebung von $a$. Man schreibt
  \[f=\circ(g),x\to a\]
  falls
  \[\lim_{x\to a}\frac{f(a)}{f(x)}=0\]
  Gilt zusätzlich
  \[\lim_{x\to a}g(x) =0\]
  so sagt man: $f$ geht für $x\to a$ schneller gegen 0 als $g$.\\
  $f:I\to\mb{C}$, $a\in I$ $n$-mal differenzierbar:
  \[f(x)=T_nf(x;a)+\circ\left( (x-a)^n \right),x\to a\]
\end{Not}
\begin{Bsp}
  $T_4(x;0)$
  \begin{align*}
    f(x)=\sin x& 0\\
    f'(x)=\cos x& 1\\
    f(x)=-\sin x& 0\\
    f'(x)=-\cos x& -1\\
    f(x)=\sin x& 0
  \end{align*}
  \[T_4f(x;0)=x-\frac{1}{3!}x^3\]
  \[\sin x=x-\frac{x^3}{6}+\circ(x^4)\]
\end{Bsp}
\begin{Bsp}
  \begin{gather*}
    \Limo{x}\frac{\sin x-x}{x^3}=\Limo{x}\frac{\frac{-x^3}{6}+\circ(x^4)}{x^3}=\\
    =-\frac{1}{6}\Limo{x}\frac{x^3}{x^3}+\Limo{x}\frac{x\circ(x^4)}{x^4}=\\
    =-\frac{1}{6}+0\cdot 0=-\frac{1}{6}
  \end{gather*}
\end{Bsp}
\begin{Sat}{Integralform von $R_{n+1}$}
  Sei $f\in \Phi^{n+1}(I,\mb{C})$ ($\Phi$ differnzierbare Funktion). Dann
  \[R_{n+1}(x)=\frac{1}{n!}\int_a^x(x-t)^nf^{(n+1)}(t)\md t\]
\end{Sat}
\begin{Bew}
  Durch Induktion
  \subparagraph{$n=0$}
  \begin{gather*}
    R_1(x)=f(x)-T_0f(x;a)\\
    T_0f(x;a)=f(a)\\
    R_1(x)=f(x)-f(a)\\
    \frac{1}{1!}\int^x_af'(t)\md t=f(x)-f(a)
  \end{gather*}
  \subparagraph{$n+1$}
  \begin{gather*}
    f-T_{n-1} f=R_n=\frac{1}{(n-1)!}\int^x_a(x-t)^{n-1}f^{(n)}(t)\md t\\
    =\frac{1}{(n-1)!}\int\Diff{}{t}\frac{(x-t)^n}{-n}f^{(n)}(t)\md t\\
    =-\frac{1}{n!}\left[ (x-1)^nf^{(n)}(t) \right]\Big|^x_a+\frac{1}{n!}\int(x-t)^nf^{(n+1)}(t)\md t\\
    =\frac{1}{n!}(x-a)^nf^{(n)}(a)+\frac{1}{n!}\int^x_a(x-t)^nf^{(n+1)}(t)\md t\\
    \implies f-T_nf=\frac{1}{n!}\int^x_a(x-t)^nf^{(n+1)}(t)\md t
  \end{gather*}
\end{Bew}
\begin{Kor}{Lagrange-Form für $R_{n+1}$}
  Sei $f\in\Phi^{n+1}(I;\mb{R})$ $a\in I$.
  \[\forall x\in I\ \exists \xi\in I: R_{n+1}(x)=\frac{f^{(n+1)}(\xi)}{(n+1)!}(x-a)^{n+1}\]
\end{Kor}
