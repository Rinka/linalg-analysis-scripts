\section{Bilinearformen}
Das kanonische Skalarprodukt (oder: Standardskalarprodukt) von $\mb{R}^n$ ist die Abbildung
\begin{align*}<,>: & \mb{R}^nx\mb{R}^n & \to & \mb{R}\\
 & (x,y) & \mapsto & <x,y>\in\mb{R}
\end{align*}
gegeben durch
\[<x,y>:=x_1y_1+\cdots+x_ny_n\]
falls $x=(x_1,\cdots,x_n)$ und $y=(y_1,\cdots,y_n)$ sind.
\begin{Def}[Konvention]
eine $1x1$ Matrix wird mit Eintrag indentifiziert
\[(x)\in{}M(1\times 1,K) \lra x\in{}K\]
Dann können wir schreiben:
\[<x,y>=(x^t)(y)\]
\[x=\Mx{x_1,\vdots,x_n}, y=\Mx{y_1,\vdots,}\]
\[(x_1,\cdots,x_n)\Mx{y_1,\vdots,y_n}=x_1y_1+\cdots+x_ny_n)\]
\end{Def}
\begin{Bem}[$<,>$ ist bilinear]
\begin{align*}
  <x+x',y>&=<x,y>+<x',y>\\
  <\lambda{}x,y>&=\lambda<x,y>\\
  <x,y+y'>&=<x,y>+<x,y'>\\
  <x,\lambda{}y>&=\lambda<x,y>
\end{align*}
symmetrisch:
\[<x,y>=<y,x>\]
positiv definit:
\begin{align*}
  <x,x> &\geq& 0 \\
  <x,x> &=&= \Lra x=0\in\mb{R}^n
\end{align*}
\[\text{für} \forall x,y,x',y'\in\mb{R}^n, \lambda\in\mb{R}\]
\end{Bem}
\begin{Bem}[Hintergrund: euklidische Geometrie]
\begin{align*}
  \Norm{x}&=&=\sqrt{<x,x>}
  &=&=\sqrt{x^2_1+\cdots+x_n^2}
\end{align*}
\end{Bem}
\begin{Bem}[Eigenschaften von $\Norm{.}$]
\begin{align*}
  \Norm{x}\geq0, \text{mit} \Norm{x}=0 \Lra x=0\\
  \Norm{\lambda{}x}=\Abs{\lambda}\Norm{x}\\
  \Norm{x+y}\leq\Norm{x}+\Norm{y}
\end{align*}
\end{Bem}
Dann definieren wir den Abstand von $x,y\in\mb{R}^n$:
\[d(x,y)\in\mb{R}\]
\[d(x,y):=\Norm{y-x}\]
\begin{Bem}{Eigenschaften}
\begin{align*}
  d(x,y)\geq0, \text{mit}d(x,y)=0 \Lra x=y\\
  d(y,x)=d(x,y)\\
  d(x,z)\leq d(x,y)+d(y,z)
\end{align*}
\[\text{für} x,y,z\in\mb{R}^n\]
\end{Bem}
Wir sind motiviert, Struktiren zu definieren, basierend auf diesen Eigenschaften, so z.B.
\begin{itemize}
  \item Bilineare Formen (symetrisch, positiv definit)
  \item Norme
  \item Metriken
\end{itemize}
\begin{proof} $\Norm{.}$ und $d$: die Dreiecksungleichung folgt aus der Cauchy-Schwarzschen Ungleichung
\begin{align*}
  \Norm{x+y}^2&=&=<x+y,x+y>\\
  &=&=\Norm{x}^2+\Norm{y}^2+2<x,y> \leq (?) \left(\Norm{x}+\Norm{y}\right)^2\\
  &&\Lra <x,y> \leq \Norm{x}\Norm{y}
\end{align*}
Cauchy-Schwarz'sche Ungleichung: für $x,y\in\mb{R}^n$
\[<x,y>^2\leq<x,x><y,y>\]
mit Gleichheit genau dann, wenn $x$ und $y$ linear abhängig sind.
\[\Lra\]
\[A=\Mx{---&x&---\\---&y&---}\in M(2\times n,\mb{R})\]
$A$ hat Rang $\leq 1$
\end{proof}
\begin{proof}
\begin{align*}
  A\cdot{}A^t&=&=\Mx{<x,x>&<x,y>\\<x,y>&<y,y>}\in M(2\times 2),\mb{R}\\
  \det(A\cdot{}A^t)&=&=<x,x><y,y>-<x,y>^2
\end{align*}
Es gibt eine Gleichung von Determinanten:
\begin{align*}
  A,B\in M(k\times n,K)\\
  \det(A\cdot B^t)=\sum_{1\leq s_1 <s_2<\cdots<s_k\leq n} \det(A^{s_1,\cdots,s_k})\det(B^{s_1,\cdots,s_k})\\
  \text{wobei } A^{s_1,\cdots,s_k}:=(a_i,s_j)_{1\leq i,j\leq k}, B^{s_1,\cdots,s_k}=(b_i,s_j)_{1\leq i,j\leq k}\\
\end{align*}
Beweis-Skizze: Reduktion zum Fall, dass die Zeilen von $A$ und $B$ Standardbasiselemente sind; direkte Berechnung in diesem Fall.\\
Es folgt:
\begin{align*}
  \det(A\cdot A^t)=\sum_{1\leq i < j \leq n} \det(A^{i,j})^2 \geq 0
\end{align*}
und ist $=0$ $\Lra$ alle $2\times 2$ Minoren von $A$ sind 0 $\Lra$ $rang(A)\leq 1$
\end{proof}
\begin{Kor} Wir können definieren
\begin{align*}
  \angle(x,y):=\cos^{-1}\underbrace{\frac{<x,y}{\Norm{x}\Norm{y}}}_{\in [-1,1]\in\mb{R}} \in [0,\pi]\in\mb{R}\\
  \text{für}\\
  0\neq x\in\mb{R}^n\\
  0\neq y\in\mb{R}^n
\end{align*}
\end{Kor}
\begin{Kor}
$x,y$ Vektoren, $\theta$ Winkel zwischen den beiden
\[<x,y>=\frac{1}{2}\left(\Norm{x}^2+\Norm{y}^2-\Norm{y-x}^2\right)\]
und deshalb:
\[\cos\theta=\frac{\Norm{x}^2+\Norm{y}^2-\Norm{y-x}^2}{2\Norm{x}\Norm{y}}\]
$\implies$ Winkel eines Dreiecks ist nur von den Seitenlängen abhängig.
\end{Kor}
\begin{Bsp}
\[\angle(x,y)=\frac{\pi}{2}\Lra <x,y>=0\]
\[\underbrace{\{y|<x,y>=0\}={0}}_{\text{Untervektorraum}}\cup\{0\neq y\in\mb{R}^n|\angle(x,y)=\frac{\pi}{2}\}\]
Man nennt $x$ und $y$ \underline{senkrecht} falls $<x,y>=0$
\end{Bsp}
