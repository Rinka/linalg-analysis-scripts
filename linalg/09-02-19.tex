\begin{Faz}
  \begin{align*}
    <.,.>& \mb{R}^n\times\mb{R}^n\to\mb{R}^n&\text{bilinear form}\\
    \Norm{.}& \mb{R}^n\to \mb{R}_{\geq 0}& \text{Norm}\\
    d(.,.)&\mb{R}^n\times\mb{R}^n\to\mb{R}_{>0}&\text{Metrik}
  \end{align*}
  \begin{align*}
    \Norm{x}&=\sqrt{<x,x>}\\
    d(x,y)&=\Norm{y-x}\\
    <x,y>&=\frac{\Norm{x}^2+\Norm{y}^2-\Norm{y-x}^2}{2}
  \end{align*}
\end{Faz}
\subsection{Vektorprodukt in $\mb{R}^3$}
  \begin{align*}
    \mb{R}^3\times\mb{R}^3&\to&\mb{R}^3\\
    (x,y)&\mapsto&x\times y
  \end{align*}
  für $y=(y_1,y_2,y_3)$ und $y=(y_1,y_2,y_3)$ ist 
  \[x\times y=(x_2y_3-x_3y_2, x_3y_1-x_1y_2,x_1y_2-x_2y_1)\]
  oder:
  \[x\times y=\det\Mx{e_1&e_2&e_3\\x_1&x_2&x_3\\y_1&y_2&y_3}\]
  wobei $(e_1,e_2,e_3)$ die Standardbasis ist. Es ist deshalb klar, dass
  \[0=\det\Mx{x_1&x_2&x_3\\x_1&x_2&x_3\\y_1&y_2&y_3}=<x,x\times y>\]
  \[0=\det\Mx{y_1&y_2&y_3\\x_1&x_2&x_3\\y_1&y_2&y_3}=<y,x\times y>\]
  $x\times y$ liegt auf der Gerade von Vektoren senkrecht zu $x$ und $y$.
  weiter:
  \[\det\Mx{w_1&w_2&w_3\\x_1&x_2&x_3\\y_1&y_2&y_3}=<x\times y,x\times y>\]
  \[=\Norm{x\times y}^2=(x_2y_3-x_3y_2)^2+(x_3y_1-x_1y_3)^2+(x_1y_2-x_2y_1)^2\]
  \[=\Norm{x}^2\Norm{y}^2-<x,y>^2=\Norm{x}^2\Norm{y}^2\left( 1-\frac{<x,y>^2}{\Norm{x}^2\Norm{y}^2} \right)\]
  \[=\Norm{x}^2\Norm{y}^2(1-\cos^2\angle (x,y) = \Norm{x}^2\Norm{y}^2\sin^2\angle (x,y)\]
\begin{Faz}
  Wenn das Ergebnis $=0$, folgt daraus, dass $x$ und $y$ linear abhängig sind. Falls $x$ und $y$ linear unabhängig sind, dann folgt dass $(x\times y,x,y)$ zu derselben Orientierungsklasse gehört wie $(e_1,e_2,e_3)$. Insgesamt bedeutet dies, dass $x\times y$ folgende Eigenschaften hat:
  \begin{itemize}
    \item ist senkrecht zu $x$ und $y$
    \item ist 0 $\Lra$ $x$ und $y$ sind linear abhängig
    \item hat Länge $\Norm{x}\Norm{y}\sin \angle (x,y)$
    \item und hat die Richtung, die mit $x$ und $y$ die gleiche Orientierungsklassse wie die Standardbasis hat.
  \end{itemize}
\end{Faz}
\subsection{Skalarprodukt über $\mb{C}^n$}
Sei $z=(z_1,\cdots,z_n)$ und $w=(w_1,\cdots,w_n)\in\mb{C}^n$
\begin{Bem}
  Der Ausdruck macht Sinn.
  \begin{align*}
    <z,w>&:=z_1w_1+\cdots+z_nw_n\\
    <z,z>&:=z_1^2+\cdots+z_n^2\\
  \end{align*}
  Dann kann die Länge nicht mehr interpretiert werden, z.B. für $z=(1,i,0,\cdots,0)$ haben wir $<z,z>=1^2+i^2=0$. Isotropische Untervektorräume von $\mb{C}^n$ werden nicht in in diesem Kurs behandelt. ($V\subset\mb{C}^n$ s.d. $<v,w>=0 \ \forall v,w\in V$). Für die Physik, die Geometrie usw. ist eine Interpretation in Zusammenhang mit Länge wichtig, deshalb brauchen wir eine neue Definition.
\end{Bem}
\begin{Def}[Das kanonische Skalarprodukt]
  von $\mb{C}^n$ ist gegeben durch
  \begin{align*}
    <.,.>_c&:\mb{C}^n\mb{C}^n&\to&\mb{C}\\
    & (z,w)&\mapsto&z_1\bar{w_1}+\cdots+z_n\bar{w_n}
  \end{align*}
\end{Def}
\begin{Eig}[von $<.,>_c$]
  \begin{align*}
    <z+z',w>&=& <z,w>_c+<z',w>_c\\
    <\lambda z,w>_c&=& \lambda<z,w>_c\\
    <z,w+w'>_c&=& <z,w>_c+<z,w'>_c\\
    <z,\lambda w>_c&=& \bar{\lambda}<z,w>_c
  \end{align*}
  für $z,z',w,w'\in\mb{C}^n$, $\lambda\in\mb{C}$\\
  $<.,.>_c$ ist sesquilinear
  \begin{align*}
    <w,z>_c&=& \overline{<z,w>_c}& \text{hermitisch}\\
    <z,z>_c&\in&\mb{R}_{\geq 0}& \text{positiv definit}\\
    <z,z>=0 &\Lra& z=0
  \end{align*}
\end{Eig}
\begin{Faz}
  $<.,.>_c$ ist sesquilinear, hermitisch und positiv definit.
\end{Faz}
\begin{proof}
  Bei Bedarf sonstwo nachschauen (Zu viele Zeichen und zu wenig Sinn). Es läuft auf eine Sammlung von Quadraten heraus.
\end{proof}
\begin{Def}[Norm von $\mb{C}^n$]
  \[\Norm{z}=\sqrt{<z,z>_c}\]
\end{Def}
\begin{Bem}
  Sei $w=(x_1'+xy_1',\cdots,x_n'+iy_n')$. Dann:
  \begin{align*}
    <z,w>_c=(x_1+iy_1)(x_1'-iy_1')+\cdots+(x_n+iy_n)(x_n'-iy_n')\\
    =(x_1x_1'+y_1y_1'+\cdots+x_nx_n'+y_ny_n')+i(x_1'y_1-x_1y_1'+\cdots+x_n'x_y-x_ny_n')
  \end{align*}
  Auf diese Weise ist $<.,.>_c$ eine Erweiterung von reellen Skalarprodukt.
  \begin{align*}
    \mb{R}^{2n}&\xrightarrow{~}&\mb{C}^n&\mb{R}\text{-linear}\\
    e_1&\mapsto&(1,0,\cdots,0)\\
    e_2&\mapsto&(i,0,\cdots,0\\
    \cdots\\
    e_{2n}&\mapsto&(0,\cdots,0,i)
  \end{align*}
  \[<.,.>_c=\left( <.,.> \text{von} \mb{R}^{2n} \right) + i(\text{neues})\]
  $\Re<.,.>_c=<.,.>$ von $\mb{R}^{2n}$ unter diesem Isomorpismus.\\
  Sei $\omega:=\Im<.,.>$:
  \begin{align*}
    \omega:&\mb{C}^n\times\mb{C}^n&\to&\mb{R}\\
    \text{oder}\mb{R}^{2n}\times\mb{R}^{2n}&\to&\mb{R}
  \end{align*}
\end{Bem}
\begin{Eig}[von $\omega$ (Imaginärteil des kanonischen Skalarproduktes)]
  \begin{description}
    \item[bilinear]
    \item[schiefsymmetrisch] $\omega(w,z)=-\omega(z,w)$
    \item[] $\omega(z,z)=0\ \forall z\in\mb{C}^n$ (oder $\mb{R}^{2n}$)
  \end{description}
\end{Eig}
\subsection{Bilinearform}
Sei $K$ ein Körper und $V$ ein $K$-Vektorraum.
\begin{Def}[Bilinearform]
  Eine bilineare Form auf $V$ ist eine Abbildung
  \[s:V\times V\to K\]
  so dass:
  \begin{align*}
    s(v+v',w)&=&s(v,w)+s(v',w)\\
    s(\lambda v,w)&=&\lambda s(v,w)\\
    s(v,w+w')&=&s(v,w)+s(v,w')\\
    s(v,\lambda w)&=& \lambda s(v,w)
  \end{align*}
  $\forall v,v',w,w'\in V, \lambda \in K$\\
  Und: $s$ heisst \underline{symmetrisch}, falls $s(w,v)=s(v,w)$ und \underline{schiefsymmetrisch}, falls $s(w,v)=-s(v,w)$.
\end{Def}
\begin{Bsp}
  \begin{itemize}
    \item $<.,.>:=\mb{R}^n\times \mb{R}^n\to\mb{R}$ ist eine symmetrische bilineare Form
    \item $\omega$ ist eine schiefsymmetrisch bilineare Form
    \item ($<.,.>_c$ nicht)
    \item $V=\{\text{stetige Abbildung} [0,1] \to\mb{R}\}$ über $\mb{R}$: $f,g\in V$
      \[s(f,g)=\int^1_0 f(x)g(x)dx\]
      ist eine symmetrisch bilineare Form auf $V$
  \end{itemize}
\end{Bsp}
