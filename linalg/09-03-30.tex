\begin{Faz}
  $\mb{K}$: Körper, $\text{char}(\mb{K})\neq 2$\\
  $V$:endlichdimensionaler $\mb{K}$-Vektorraum\\
  $s$: $V\times V\to\mb{K}$ symmetrische Bilinearform
  \begin{description}
    \item[Rang] Rang < $\dim V$ $\Lra$ $s$ ist ausgeartet. $U$:=Ausartungsraum. $\bar s$ induzierte Bilinearform auf $\bar V:=V/U$ (nicht ausgeartet)
    \item[Diskriminiante] für $s$ nicht ausgeartet: $\disc(s)\in\mb{K}^*/(K^*)^2$
  \end{description}
\end{Faz}
\begin{Bsp}
  $\mb{K}=\mb{Q}$, $V=\mb{Q}^2$ Bilinear entsprechend zu $\Mx{1&0\\0&1}$ und $\Mx{3&0\\0&3}$
  \begin{itemize}
    \item Beide: Rang 2, Diskriminante 1
    \item nicht äquivalent
  \end{itemize}
\end{Bsp}
\begin{Bem}
  Die Frage, ob eine nicht ausgeartete symmetrische Bilinearform auf $V:=\mb{Q}^2$ der Diskriminante 1 äquivalent zum Standardskalarprodukt ist, können wir nur beantworten mittels einem Resultat aus der Zahlentheorie.
\end{Bem}
\begin{Sat}
  $s$ nicht ausgeartete symmetrische Bilinearform auf $\mb{Q}^2$ $\disc(s)=1$ $\implies$ $\exists B$ Basis mit
  \[M_B(s)=\Mx{a&0\\0&a},\ a\in\mb{Z},a\neq 0\]
  Dann: $s$ ist äquivalent zum Standardskalarprodukt $\Lra$
  \[\exists v\in V,\ s(v,v)=1\text{ d.h. }\exists x,y\in\mb{Q}:\ ax^2+ay^2=1\]
  $\Lra$
  \[\exists x,y\in\mb{Q}\text{ mit }x^2+y^2=a\]
\end{Sat}
\begin{Bem}
  Ein Resultat aus der Zahlentheorie gibt uns eine Charakterisierung von Summen zweier Quadrate in $\mb{Q}$: für $a\in\mb{Z},a\neq 0$:
  \[\exists x,y\in\mb{Q}:x^2+y^2=a\ \Lra x,y\in\mb{Z}:x^2+y^2=a\]
  $\Lra$ $a>0$ und jede Primzahl $p=4k+3$ $(k\in\mb{N})$ kommt mit gerader Vielfachheit in der Primzahlzerlegun von $a$ vorkommen. Der Beweis nutzt
  \[(x_1x_1'-x_2x_2')^2+(x_1x_2'+x_2x_1')^2=(x_1^2+x_2^2)(x_1'^2+x_2'^2)\]
  Satz von Fermat: $p$ Primzahl 
  \[\exists x,y\in\mb{Z},x^2+y^2=p\ \Lra\ p=2\vee 4|(p-1)\]
  Argument vom letzten Mal (auszuschliessen $a=3$)
\end{Bem}
\begin{Faz}
  Zurück zum Fall $\mb{K}=\mb{R}$\\
  Wir wissen: eine symmetrische Bilinearform $s:V\times V\to\mb{R}$ $(\dim_\mb{R}V<\infty)$ ist durch die Signatur $(r_+,r_-,r_0)$ charaktertisiert.
  \[A:M_B(s)\ P_A(t)=\prod^n_{i=1}(t-\lambda_i)\]
  \begin{gather*}
    r_+=\#\left\{ i|\lambda_i>0 \right\}\\
    r_-=\#\left\{ i|\lambda_i<0 \right\}\\ 
    r_0=\#\left\{ i|\lambda_i=0 \right\}
  \end{gather*}
  und
  \[s>0\ \Lra\ \text{Signatur } (n,0,0)\ \Lra\ \lambda_i>0\forall i\ \Lra\ \text{Koeff. }P_A(t)\ \text{hat alternierende Vorzeichen }\]
\end{Faz}
\begin{Def}{Hauptminor}
  Sei $A=(a_{ij})\in\Mat{K}$. Wir schreiben $A_k$ für die Teilmatrix $(A_{ij})_{1\leq i,j\leq k}$, für $1\leq k\leq n$. Der $k$-te Hauptminor von $A$ ist $\det (A_k)$
\end{Def}
\begin{Bem}
  $\dim_\mb{R}<\infty$, $s:V\times V\to\mb{R}$ symmetrische Bilinearform. Wann ist $s$ positiv? Es ist notwendig, aber nicht hinreichend, dass $\det(A)>0$ für eine darstellende Matrix $A$.
\end{Bem}
\begin{Prop}{Hauptminorenkriterium von Jacobi-Sylvester}
  Sei $V$ ein endlichdimensionaler reeller Vektorraum, $s:V\times V\to\mb{R}$ eine symmetrische Bilinearform und $A$ eine darstellende Matrix. Dann ist $s$ positiv definit $\Lra$ $\det(A_K)>0,\ k=1,\cdots,n$ $n=\dim V$
\end{Prop}
\begin{Bew}
  $\Ra$ Ist $s>0$, so ist $s|_W>0$ für alle Untervektorräume $S\subset V$. Sei $B=(v_1,\cdots,v_n)$ eine Basis mit $A=M_B(s)$. Sei $V_k:=\Span(v_1,\cdots,v_k)$ für $1\leq k\leq n$. Dann haben wir:
  \[M_{(v_1,\cdots,v_k)}\left( s|_{V_k} \right)=\left( s(v_i,v_j) \right)_{1\leq i,j\leq k}=\left( A_K \right)\]
  Weil $s|_{V_k}>0$, folgt: $\det(A_k)>0$.\\
  $\La$ Durch eine Induktion nach $n$:\\
  IA: $n=1$\\
  IS: Wir nehmen das Resultat an für einen Vektorraum der Dimension $n-1$. Seien $B=(v_1,\cdots,v_n)$ eine Basis von $V$ und $A=(a_{ij})_{i\leq i,j\leq n}$ $A=M_B(s)$. Wir haben
  \begin{itemize}
    \item aus $\det(A)>0$ folgt: $s$ ist nicht ausgeartet
    \item aus $\det(A_1)>0$ folgt $a_{11}>0$
  \end{itemize}
  Wir haben $V=\Span(v_1)\oplus V^\perp_1$. Es genügt zu zeigen, dass $s|_{v^\perp_1}$ positiv definit ist. Eine Basis von $v^\perp_1$ sieht so aus: Sei $c_i:=\frac{a_{1i}}{a_{11}}$ und $\tilde v_i:=v_i-c_iv_1$ für $i=2,\cdots,n$.
  \begin{gather*}
  s(v_1,\tilde v_1)=s(v_1,v_i)-c_is(v_1,v_1)=a_{1i}-c_ia_{11}=0\\
  (\tilde v_2,\cdots,\tilde v_n)\ \Lra\ \Mx{-c_2&1&&&\\-c_3&&1&&\\&&&\ddots&\\-c_n&&&&1}\\
  (\tilde a_{ij})_{2\leq i,j\leq n}\ \tilde a_{ij}=s(\tilde v_i,\tilde v_j)=a_{ij}-c_ia_{i1}+c_ic_ja_{11}=a_{ij}-c_ia_{1j}\\
  \text{weil }c_ic_ja_{11}=\frac{a_{1i}}{a_{11}}a_{1j}=c_ja_{i1}
  \end{gather*}
  Wir haben für $s\leq k\leq n$:
  \begin{gather*}
    \Mx{-c_2&1&&&\\-c_3&&1&&\\&&&\ddots&\\-c_k&&&&1}\Mx{a_{11}&a_{12}&\cdots&a_{1k}\\a_{21}&&&\\\vdots&&\ddots&\\a_{k1}&&&a_{kk}}=\Mx{a_{11}&a_{12}&\cdots&a_{1k}\\0&a_{22}-c_2a_{12}&\cdots&a_{2k}-c_2a_{1k}\\\vdots&\vdots&\vdots&\vdots\\0&a_{2k}-c_ka_{12}&\cdots&a_{kk}-c_ka_{1k}}\\
    = \Mx{a_{11}&a_{12}&\cdots&a_{1k}\\0&\tilde a_{22}&\cdots&\tilde a_{2k}\\\vdots&\vdots&\vdots&\vdots\\0&\tilde a_{k2}&\cdots&\tilde a_{kk}}\\
    \implies \det(A_k)=a_{11}\det(\tilde a_{ij})_{2\leq i,j\leq k}
  \end{gather*}
  Aus $\det(A_k)>0$ und $a_{11}>0$ folgt:
  \[\det(\tilde a_{ij})_{2\leq i,j\leq k}>0,\ \text{für}\ k=2,\cdots,n\]
  Aus der Induktionsvoraussetzung folgt $s|_{v_1^\perp}>0$
\end{Bew}
