% headers by Alexander Berthold van der Bourg / Pirmin Weigele 

%= Document-Class ==================================================================================
\documentclass[10pt,a4paper]{article}

%= Packages ========================================================================================
\usepackage[utf8]{inputenc}
\usepackage{ngerman,amsmath,amssymb,amsfonts,mathrsfs}
\usepackage{amsthm}
\usepackage{bbm}
\usepackage{ulsy}
\usepackage{epic,eepic,pstricks,pst-node,pst-plot}
\usepackage{pstricks}
\usepackage{colortbl}
\usepackage{graphicx}
\usepackage{makeidx}
\usepackage{fancyhdr}
\usepackage{latexsym}
\usepackage{psfrag}
\usepackage{enumerate}
\usepackage{float}
%\usepackage{mathtext}
\usepackage[all, knot, poly]{xy}
\usepackage{dsfont}
\pagestyle{fancy}
\usepackage{multirow, bigdelim, bigstrut}
\usepackage{rotating}
\usepackage{ifthen}
\usepackage{boxedminipage}
\usepackage{chemarrow,stmaryrd}
\usepackage{mathtools}

%= Seiten-Layout =========================================================================
\voffset-22mm \textheight715pt 

%Seitenbreite==============================================================

%\oddsidemargin=0.4in
%\evensidemargin=0.2in
%\textwidth=5.2in
%\headwidth=5.2in

%= Index-Befehle ========================================================================
\renewcommand{\indexname}{Stichwortverzeichnis}
\makeindex

%= Befehl-Overwriting =======================================================================
\makeatletter
\makeatother

%= Strings ================================================================
\newcommand{\mainfold}{.}

%= Eigene Befehle ==========================================================================
\DeclareMathOperator{\id}{Id}
\DeclareMathOperator{\arccot}{arccot}
\DeclareMathOperator{\arsinh}{arsinh}
\DeclareMathOperator{\arcosh}{arcosh}
\DeclareMathOperator{\artanh}{artanh}
\DeclareMathOperator{\md}{d}
\DeclareMathOperator{\Grad}{grad}
\DeclareMathOperator{\Ker}{Ker}
\DeclareMathOperator{\Span}{span}
\DeclareMathOperator{\eig}{Eig}
\DeclareMathOperator{\diag}{diag}
\DeclareMathOperator{\disc}{disc}

\newcommand{\Diff}[2]{\displaystyle\frac{\mathrm{d}#1}{\mathrm{d}#2}}
\newcommand{\End}{\hfill{\hbox{$\Box$}}\par\vspace{2mm}}
\newcommand{\eps}{\varepsilon}
\newcommand{\ePic}[1]{\input{\mainfold/graphics/\prefix#1.eepic}}
\newcommand{\pst}[1]{\input{\mainfold/graphics/\prefix#1.pst}}
\newcommand{\pic}[1]{\input{\mainfold/graphics/\prefix#1.pic}}
\newcommand{\Mx}[1]{\begin{pmatrix}#1\end{pmatrix}}
\newcommand{\im}[1]{\operatorname{Im}(#1)}
%\newcommand{\Include}[4]{\rhead{#2.#3.20#4}\input{\mainfold/lectures/#1-#4-#3-#2.tex}}
\newcommand{\Index}[1]{\emph{#1}\index{#1}}
\newcommand{\Int}[4]{\displaystyle\int\limits_{#1}^{#2}#3\,\mathrm{d}#4}
\newcommand{\diff}[1]{\operatorname{d}\!#1}
\newcommand{\Limi}[1]{\displaystyle\lim_{#1\rightarrow\infty}}
\newcommand{\Limo}[1]{\displaystyle\lim_{#1\rightarrow0}}
\newcommand{\mb}[1]{\mathbb{#1}}
\newcommand{\ds}{\displaystyle}
\newcommand{\ol}[1]{\overline{#1}}
\newcommand{\Part}[2]{\dfrac{\partial #1}{\partial #2}}
\newcommand{\QED}{\hfill{\hbox{(QED)}}\par\vspace{2mm}}
\newcommand{\re}[1]{\operatorname{Re}(#1)}
\newcommand{\s}{\hspace{2mm}}
\newcommand{\vsa}{\vspace{1mm} \\}
\newcommand{\vsb}{\vspace{2mm} \\}
\newcommand{\vsc}{\vspace{3mm} \\}
% \newcommand{\tr}[1]{\textrm{#1}}
\newcommand{\tr}[1]{\text{#1}}
\newcommand{\ra}{\rightarrow}
\newcommand{\Ra}{\Rightarrow}
\newcommand{\Lra}{\Leftrightarrow}
\newcommand{\lra}{\leftrightarrow}
\newcommand{\La}{\Leftarrow}
\newcommand{\ul}[1]{\underline{#1}}
\newcommand{\rsa}{\rightsquigarrow}
\newcommand{\ara}[2]{\autorightarrow{\ensuremath{#1}}{\ensuremath{#2}}} 
\newcommand{\Mat}[1]{\ensuremath{(n\times n, \mb{#1})}}

%\newcommand{\detmx}{\left| \begin{array} #1 \end{array} \right|}

\newcommand{\grad}[1]{\Grad(#1)}
\newcommand{\fr}[2]{\displaystyle\frac{#1}{#2}} % fertiger bullshit, daf�r gibts \dfrac{}{}
\renewcommand{\Re}{\operatorname{Re}}
\renewcommand{\Im}{\operatorname{Im}}

% ---- DELIMITER PAIRS ----
\def\floor#1{\lfloor #1 \rfloor}
\def\ceil#1{\lceil #1 \rceil}
\def\seq#1{\langle #1 \rangle}
\def\set#1{\{ #1 \}}
\def\abs#1{\mathopen| #1 \mathclose|}	% use instead of $|x|$ 
\def\norm#1{\mathopen\| #1 \mathclose\|}% use instead of $\|x\|$ 

% --- Self-scaling delmiter pairs ---
\def\Floor#1{\left\lfloor #1 \right\rfloor}
\def\Ceil#1{\left\lceil #1 \right\rceil}
\def\Seq#1{\left\langle #1 \right\rangle}
\def\Set#1{\left\{ #1 \right\}}
\def\Abs#1{\left| #1 \right|}
\def\Norm#1{\left\| #1 \right\|}

%Adrians Abbildungs-Environment ==============================================

\newcommand{\Sidein}{\begin{rotate}{90}\small$\in$\end{rotate}}

\newcommand{\Abb}[5][]{\ensuremath{
    \begin{array}{lc}
      \ifthenelse{\equal{#1}{}}{}{#1:}\;\; & 
      \begin{xy}
        \xymatrixrowsep{1em}\xymatrixcolsep{2em}%
        \xymatrix{ #2 \ar[r] \ar@{}[d]^<<<<{\hspace{0.001em} \Sidein}
          & #3  \ar@{}[d]^<<<<{\hspace{0.001em} \Sidein} \\
          #4 \ar@{|->}[r] & #5} \end{xy}
    \end{array}
  }%
}

%= Environments ========================================================================
\def\thechapter{\Roman{chapter}}
\def\thesection{\arabic{section}}
\newtheorem{Bew}{Beweis}
\newtheorem{Lem}{Lemma}
\newtheorem{Kor}{Korollar}
\newtheorem{Sat}{Satz}
\newtheorem{Prop}{Proposition}
\theoremstyle{definition}
\newtheorem{Bsp}{Beispiel}
\newtheorem{Def}{Definition}
\newtheorem{Prob}{Problem}
\theoremstyle{remark}
\newtheorem{Bem}{Bemerkung}
\newtheorem{Eig}{Eigenschaften}
\newtheorem{Faz}{Fazit}
\newtheorem{Not}{Not}

\def\pstexInput#1{%
  \begin{center}
    \begin{picture}(0,0)%
      \special{psfile=\mainfold/graphics/A2-#1.pstex}%
    \end{picture}%
    \input{\mainfold/graphics/A2-#1.pstex_t}%
  \end{center}
}

%= Titelseite ===========================================================================
\begin{document}
\headheight15pt
\begin{titlepage}
\hfill
\vspace{20mm}
\pagenumbering{roman}
\begin{center}
{\LARGE Lineare Algebra I - Vorlesungs-Script} \vskip 3em {\large Prof. Andrew Kresch} \vskip 1.5em
{\large Basisjahr 08/09 Semester II}\vspace{30mm}\\
{\large {\bf Mitschrift:} \vspace{2mm}\\
Simon Hafner}\vspace{5mm}\\ %30mm
%{\large {\bf Graphics:} \vspace{2mm}\\
%Pirmin Weigele }\vspace{30mm}\\ %30mm
\end{center}
\vfill

\end{titlepage}


%= Inhaltsverzeichnis ==========================================================================
\lhead{}
\rhead{}
\tableofcontents
\newpage
\pagenumbering{arabic}
\setcounter{page}{1}

%= Vorlesung-Skripts ==========================================================================
\cfoot{\thepage}
\fancyhead[L]{\nouppercase{\leftmark}}
\newpage

%= LinAlg I & & II ==========================================================================

%LinAlg I
\section{Bilinearformen}
Das kanonische Skalarprodukt (oder: Standardskalarprodukt) von $\mb{R}^n$ ist die Abbildung
\begin{align*}\Seq{,}: & \mb{R}^nx\mb{R}^n & \to & \mb{R}\\
  & (x,y) & \mapsto & \Seq{x,y}\in\mb{R}
\end{align*}
gegeben durch
\[\Seq{x,y}:=x_1y_1+\cdots+x_ny_n\]
falls $x=(x_1,\cdots,x_n)$ und $y=(y_1,\cdots,y_n)$ sind.
\begin{Def}[Konvention]
eine $1x1$ Matrix wird mit Eintrag indentifiziert
\[(x)\in{}M(1\times 1,K) \lra x\in{}K\]
Dann können wir schreiben:
\[\Seq{x,y}=(x^t)(y)\]
\[x=\Mx{x_1,\vdots,x_n}, y=\Mx{y_1,\vdots,}\]
\[(x_1,\cdots,x_n)\Mx{y_1,\vdots,y_n}=(x_1y_1+\cdots+x_ny_n)\]
\end{Def}
\begin{Bem}[$\Seq{,}$ ist bilinear]
\begin{align*}
  \Seq{x+x',y}&=\Seq{x,y}+\Seq{x',y}\\
  \Seq{\lambda{}x,y}&=\lambda\Seq{x,y}\\
  \Seq{x,y+y'}&=\Seq{x,y}+\Seq{x,y'}\\
  \Seq{x,\lambda{}y}&=\lambda\Seq{x,y}
\end{align*}
symmetrisch:
\[\Seq{x,y}=\Seq{y,x}\]
positiv definit:
\begin{align*}
  \Seq{x,x} &\geq 0 \\
  \Seq{x,x} &= \Lra x=0\in\mb{R}^n
\end{align*}
\[\text{für} \forall x,y,x',y'\in\mb{R}^n, \lambda\in\mb{R}\]
\end{Bem}
\begin{Bem}[Hintergrund: euklidische Geometrie]
\begin{gather*}
  \Norm{x}=\sqrt{\Seq{x,x}}
  =\sqrt{x^2_1+\cdots+x_n^2}
\end{gather*}
\end{Bem}
\begin{Bem}[Eigenschaften von $\Norm{.}$]
\begin{align*}
  \Norm{x}\geq0, \text{mit} \Norm{x}=0 \Lra x=0\\
  \Norm{\lambda{}x}=\Abs{\lambda}\Norm{x}\\
  \Norm{x+y}\leq\Norm{x}+\Norm{y}
\end{align*}
\end{Bem}
Dann definieren wir den Abstand von $x,y\in\mb{R}^n$:
\[d(x,y)\in\mb{R}\]
\[d(x,y):=\Norm{y-x}\]
\begin{Bem}{Eigenschaften}
\begin{align*}
  d(x,y)\geq0, \text{mit}d(x,y)=0 \Lra x=y\\
  d(y,x)=d(x,y)\\
  d(x,z)\leq d(x,y)+d(y,z)
\end{align*}
\[\text{für} x,y,z\in\mb{R}^n\]
\end{Bem}
Wir sind motiviert, Strukturen zu definieren, basierend auf diesen Eigenschaften, so z.B.
\begin{itemize}
  \item Bilineare Formen (symetrisch, positiv definit)
  \item Norme
  \item Metriken
\end{itemize}
\begin{Bew} $\Norm{.}$ und $d$: die Dreiecksungleichung folgt aus der Cauchy-Schwarzschen Ungleichung
\begin{align*}
  \Norm{x+y}^2&=\Seq{x+y,x+y}\\
  &=\Norm{x}^2+\Norm{y}^2+2\Seq{x,y} \leq (?) \left(\Norm{x}+\Norm{y}\right)^2\\
  &\Lra \Seq{x,y} \leq \Norm{x}\Norm{y}
\end{align*}
Cauchy-Schwarz'sche Ungleichung: für $x,y\in\mb{R}^n$
\[\Seq{x,y}^2\leq\Seq{x,x}\Seq{y,y}\]
mit Gleichheit genau dann, wenn $x$ und $y$ linear abhängig sind.
\[\Lra\]
\[A=\Mx{---&x&---\\---&y&---}\in M(2\times n,\mb{R})\]
$A$ hat Rang $\leq 1$
\end{Bew}
\begin{Bew}
\begin{align*}
  A\cdot{}A^t&=&=\Mx{\Seq{x,x}&\Seq{x,y}\\\Seq{x,y}&\Seq{y,y}}\in M(2\times 2),\mb{R}\\
  \det(A\cdot{}A^t)&=&=\Seq{x,x}\Seq{y,y}-\Seq{x,y}^2
\end{align*}
Es gibt eine Gleichung von Determinanten:
\begin{align*}
  A,B\in M(k\times n,K)\\
  \det(A\cdot B^t)=\sum_{1\leq s_1 <s_2<\cdots<s_k\leq n} \det(A^{s_1,\cdots,s_k})\det(B^{s_1,\cdots,s_k})\\
  \text{wobei } A^{s_1,\cdots,s_k}:=(a_i,s_j)_{1\leq i,j\leq k}, B^{s_1,\cdots,s_k}=(b_i,s_j)_{1\leq i,j\leq k}\\
\end{align*}
Beweis-Skizze: Reduktion zum Fall, dass die Zeilen von $A$ und $B$ Standardbasiselemente sind; direkte Berechnung in diesem Fall.\\
Es folgt:
\begin{align*}
  \det(A\cdot A^t)=\sum_{1\leq i < j \leq n} \det(A^{i,j})^2 \geq 0
\end{align*}
und ist $=0$ $\Lra$ alle $2\times 2$ Minoren von $A$ sind 0 $\Lra$ $rang(A)\leq 1$
\end{Bew}
\begin{Kor} Wir können definieren
\begin{gather*}
  \angle(x,y):=\cos^{-1}\underbrace{\frac{\Seq{x,y}}{\Norm{x}\Norm{y}}}_{\in [-1,1]\in\mb{R}} \in [0,\pi]\in\mb{R}\\
  \text{für}\\
  0\neq x\in\mb{R}^n\\
  0\neq y\in\mb{R}^n
\end{gather*}
\end{Kor}
\begin{Kor}
$x,y$ Vektoren, $\theta$ Winkel zwischen den beiden
\[\Seq{x,y}=\frac{1}{2}\left(\Norm{x}^2+\Norm{y}^2-\Norm{y-x}^2\right)\]
und deshalb:
\[\cos\theta=\frac{\Norm{x}^2+\Norm{y}^2-\Norm{y-x}^2}{2\Norm{x}\Norm{y}}\]
$\implies$ Winkel eines Dreiecks ist nur von den Seitenlängen abhängig.
\end{Kor}
\begin{Bsp}
\[\angle(x,y)=\frac{\pi}{2}\Lra \Seq{x,y}=0\]
\[\underbrace{\{y|\Seq{x,y}=0\}={0}}_{\text{Untervektorraum}}\cup\{0\neq y\in\mb{R}^n|\angle(x,y)=\frac{\pi}{2}\}\]
Man nennt $x$ und $y$ \underline{senkrecht} falls $\Seq{x,y}=0$
\end{Bsp}

\begin{Faz}
  \begin{align*}
    <.,.>& \mb{R}^n\times\mb{R}^n\to\mb{R}^n&\text{bilinear form}\\
    \Norm{.}& \mb{R}^n\to \mb{R}_{\geq 0}& \text{Norm}\\
    d(.,.)&\mb{R}^n\times\mb{R}^n\to\mb{R}_{>0}&\text{Metrik}
  \end{align*}
  \begin{align*}
    \Norm{x}&=\sqrt{<x,x>}\\
    d(x,y)&=\Norm{y-x}\\
    <x,y>&=\frac{\Norm{x}^2+\Norm{y}^2-\Norm{y-x}^2}{2}
  \end{align*}
\end{Faz}
\subsection{Vektorprodukt in $\mb{R}^3$}
  \begin{align*}
    \mb{R}^3\times\mb{R}^3&\to&\mb{R}^3\\
    (x,y)&\mapsto&x\times y
  \end{align*}
  für $y=(y_1,y_2,y_3)$ und $y=(y_1,y_2,y_3)$ ist 
  \[x\times y=(x_2y_3-x_3y_2, x_3y_1-x_1y_2,x_1y_2-x_2y_1)\]
  oder:
  \[x\times y=\det\Mx{e_1&e_2&e_3\\x_1&x_2&x_3\\y_1&y_2&y_3}\]
  wobei $(e_1,e_2,e_3)$ die Standardbasis ist. Es ist deshalb klar, dass
  \[0=\det\Mx{x_1&x_2&x_3\\x_1&x_2&x_3\\y_1&y_2&y_3}=<x,x\times y>\]
  \[0=\det\Mx{y_1&y_2&y_3\\x_1&x_2&x_3\\y_1&y_2&y_3}=<y,x\times y>\]
  $x\times y$ liegt auf der Gerade von Vektoren senkrecht zu $x$ und $y$.
  weiter:
  \[\det\Mx{w_1&w_2&w_3\\x_1&x_2&x_3\\y_1&y_2&y_3}=<x\times y,x\times y>\]
  \[=\Norm{x\times y}^2=(x_2y_3-x_3y_2)^2+(x_3y_1-x_1y_3)^2+(x_1y_2-x_2y_1)^2\]
  \[=\Norm{x}^2\Norm{y}^2-<x,y>^2=\Norm{x}^2\Norm{y}^2\left( 1-\frac{<x,y>^2}{\Norm{x}^2\Norm{y}^2} \right)\]
  \[=\Norm{x}^2\Norm{y}^2(1-\cos^2\angle (x,y) = \Norm{x}^2\Norm{y}^2\sin^2\angle (x,y)\]
\begin{Faz}
  Wenn das Ergebnis $=0$, folgt daraus, dass $x$ und $y$ linear abhängig sind. Falls $x$ und $y$ linear unabhängig sind, dann folgt dass $(x\times y,x,y)$ zu derselben Orientierungsklasse gehört wie $(e_1,e_2,e_3)$. Insgesamt bedeutet dies, dass $x\times y$ folgende Eigenschaften hat:
  \begin{itemize}
    \item ist senkrecht zu $x$ und $y$
    \item ist 0 $\Lra$ $x$ und $y$ sind linear abhängig
    \item hat Länge $\Norm{x}\Norm{y}\sin \angle (x,y)$
    \item und hat die Richtung, die mit $x$ und $y$ die gleiche Orientierungsklassse wie die Standardbasis hat.
  \end{itemize}
\end{Faz}
\subsection{Skalarprodukt über $\mb{C}^n$}
Sei $z=(z_1,\cdots,z_n)$ und $w=(w_1,\cdots,w_n)\in\mb{C}^n$
\begin{Bem}
  Der Ausdruck macht Sinn.
  \begin{align*}
    <z,w>&:=z_1w_1+\cdots+z_nw_n\\
    <z,z>&:=z_1^2+\cdots+z_n^2\\
  \end{align*}
  Dann kann die Länge nicht mehr interpretiert werden, z.B. für $z=(1,i,0,\cdots,0)$ haben wir $<z,z>=1^2+i^2=0$. Isotropische Untervektorräume von $\mb{C}^n$ werden nicht in in diesem Kurs behandelt. ($V\subset\mb{C}^n$ s.d. $<v,w>=0 \ \forall v,w\in V$). Für die Physik, die Geometrie usw. ist eine Interpretation in Zusammenhang mit Länge wichtig, deshalb brauchen wir eine neue Definition.
\end{Bem}
\begin{Def}[Das kanonische Skalarprodukt]
  von $\mb{C}^n$ ist gegeben durch
  \begin{align*}
    <.,.>_c&:\mb{C}^n\mb{C}^n&\to&\mb{C}\\
    & (z,w)&\mapsto&z_1\bar{w_1}+\cdots+z_n\bar{w_n}
  \end{align*}
\end{Def}
\begin{Eig}[von $<.,>_c$]
  \begin{align*}
    <z+z',w>&=& <z,w>_c+<z',w>_c\\
    <\lambda z,w>_c&=& \lambda<z,w>_c\\
    <z,w+w'>_c&=& <z,w>_c+<z,w'>_c\\
    <z,\lambda w>_c&=& \bar{\lambda}<z,w>_c
  \end{align*}
  für $z,z',w,w'\in\mb{C}^n$, $\lambda\in\mb{C}$\\
  $<.,.>_c$ ist sesquilinear
  \begin{align*}
    <w,z>_c&=& \overline{<z,w>_c}& \text{hermitisch}\\
    <z,z>_c&\in&\mb{R}_{\geq 0}& \text{positiv definit}\\
    <z,z>=0 &\Lra& z=0
  \end{align*}
\end{Eig}
\begin{Faz}
  $<.,.>_c$ ist sesquilinear, hermitisch und positiv definit.
\end{Faz}
\begin{proof}
  Bei Bedarf sonstwo nachschauen (Zu viele Zeichen und zu wenig Sinn). Es läuft auf eine Sammlung von Quadraten heraus.
\end{proof}
\begin{Def}[Norm von $\mb{C}^n$]
  \[\Norm{z}=\sqrt{<z,z>_c}\]
\end{Def}
\begin{Bem}
  Sei $w=(x_1'+xy_1',\cdots,x_n'+iy_n')$. Dann:
  \begin{align*}
    <z,w>_c=(x_1+iy_1)(x_1'-iy_1')+\cdots+(x_n+iy_n)(x_n'-iy_n')\\
    =(x_1x_1'+y_1y_1'+\cdots+x_nx_n'+y_ny_n')+i(x_1'y_1-x_1y_1'+\cdots+x_n'x_y-x_ny_n')
  \end{align*}
  Auf diese Weise ist $<.,.>_c$ eine Erweiterung von reellen Skalarprodukt.
  \begin{align*}
    \mb{R}^{2n}&\xrightarrow{~}&\mb{C}^n&\mb{R}\text{-linear}\\
    e_1&\mapsto&(1,0,\cdots,0)\\
    e_2&\mapsto&(i,0,\cdots,0\\
    \cdots\\
    e_{2n}&\mapsto&(0,\cdots,0,i)
  \end{align*}
  \[<.,.>_c=\left( <.,.> \text{von} \mb{R}^{2n} \right) + i(\text{neues})\]
  $\Re<.,.>_c=<.,.>$ von $\mb{R}^{2n}$ unter diesem Isomorpismus.\\
  Sei $\omega:=\Im<.,.>$:
  \begin{align*}
    \omega:&\mb{C}^n\times\mb{C}^n&\to&\mb{R}\\
    \text{oder}\mb{R}^{2n}\times\mb{R}^{2n}&\to&\mb{R}
  \end{align*}
\end{Bem}
\begin{Eig}[von $\omega$ (Imaginärteil des kanonischen Skalarproduktes)]
  \begin{description}
    \item[bilinear]
    \item[schiefsymmetrisch] $\omega(w,z)=-\omega(z,w)$
    \item[] $\omega(z,z)=0\ \forall z\in\mb{C}^n$ (oder $\mb{R}^{2n}$)
  \end{description}
\end{Eig}
\subsection{Bilinearform}
Sei $K$ ein Körper und $V$ ein $K$-Vektorraum.
\begin{Def}[Bilinearform]
  Eine bilineare Form auf $V$ ist eine Abbildung
  \[s:V\times V\to K\]
  so dass:
  \begin{align*}
    s(v+v',w)&=&s(v,w)+s(v',w)\\
    s(\lambda v,w)&=&\lambda s(v,w)\\
    s(v,w+w')&=&s(v,w)+s(v,w')\\
    s(v,\lambda w)&=& \lambda s(v,w)
  \end{align*}
  $\forall v,v',w,w'\in V, \lambda \in K$\\
  Und: $s$ heisst \underline{symmetrisch}, falls $s(w,v)=s(v,w)$ und \underline{schiefsymmetrisch}, falls $s(w,v)=-s(v,w)$.
\end{Def}
\begin{Bsp}
  \begin{itemize}
    \item $<.,.>:=\mb{R}^n\times \mb{R}^n\to\mb{R}$ ist eine symmetrische bilineare Form
    \item $\omega$ ist eine schiefsymmetrisch bilineare Form
    \item ($<.,.>_c$ nicht)
    \item $V=\{\text{stetige Abbildung} [0,1] \to\mb{R}\}$ über $\mb{R}$: $f,g\in V$
      \[s(f,g)=\int^1_0 f(x)g(x)dx\]
      ist eine symmetrisch bilineare Form auf $V$
  \end{itemize}
\end{Bsp}

Sei $K$ ein Körper, $V$ ein $K$-Vektorraum, mit $\dim_KV<\infty$, und $s:V\times V\to K$ eine bilineare Form.
\begin{Def}
  Ist $B=(v_i)_{1\leq i \leq n}$ eine Basis von $V$, so setzen wir 
  \[M_B(s):=\left( s(v_i,v_j) \right)_{1\leq i, j\leq n} \in M(n\times n,K)\]
  die \underline{darstellende Matrix}
\end{Def}
\begin{Kor}
  für $x,y\in V$
  \begin{align*}
    x&=& x_1v_1+\cdots+x_nv_n\\
    y&=& y_1v_1,+\cdots+y_nv_n
  \end{align*}
  und
  \[M_B(s)=(a_{ij})_{1\leq i, j\leq n}, \text{d.h.} a_{ij}=s(v_i,v_j)\]
  haben wir:
  \begin{align*}
    s(x,y)&=& \sum^n_{i,j=1}x_iy_ja_{ij}\\
    &=& (x_1\cdots x_n)\cdot\Mx{a_{11}&\cdots&a_{1n}\\ \vdots&&\vdots\\a_{n1}&\cdots&a_{nn}}\cdot \Mx{y_1\\ \vdots\\ y_n}\\
    &=& x^tM_B(s)\cdot y
  \end{align*}
\end{Kor}
\begin{Prop}
  Sei $V$ ein endlich-dim. Vektorraum über $K$ mit Basis $B=(v_i)_{1\leq i \leq n}$. Es gibt eine Bijektion zwischen der Menge von Bilinearformen und $M(n\times n,K)$, gegeben durch
  \[\left( s:V\times V\to K \right)\mapsto M_B(s)\]
\end{Prop}
\begin{Bew}
  Wir schreiben einen Vektor $x\in V$ als $(x_1,\cdots,x_n)$ falls $x=x_1v_1+\cdots+x_nv_n$. Ähnlich für $y$. Dann ist
  \begin{align*}
   A\in M(n\times n,K)\mapsto & V\times V \to K\\
   & (x,y)\mapsto x^t\cdot A \cdot y
  \end{align*}
  inverses zu der obigen Abbildung.
\end{Bew}
\begin{Bem}
  Sei $\left( s:V\times V\to K \right)$ eine bilineare Forum und $A=(a_{ij})_{1\leq i,j\leq n}$ die darstellende Matrix. Wir erinnern uns an die Notation
  \begin{align*}
    \Phi_B:&K^n\to V\\
    &e_1 \mapsto v_i
  \end{align*}
  Dann:
  % use diagram package
  \begin{align*}
    K^n\times K^n&\ara{\Phi_B\times\Phi_B}{}V\times V\ara{s}{}&K\\
    \text{ist gegeben durch}\\
    (x,y)&\mapsto&t_xA\cdot y
  \end{align*}
  Sei $A=(u_i)_{1\leq i\leq n}$ eine andere Basis.  
  \begin{align*}
    K^n&\ara{\Phi_A}{}&V\\
    \ara{T}{}=\Phi^{-1}_B\circ \Phi_A&\\
    K^n&\ara{\Phi_B}{}&
  \end{align*}
\end{Bem}
\begin{Prop}{Transforationsformel}
  Mit dieser Notation haben wir:
  \[M_A(s)=T^t\cdot M_B(s)\cdot T\]
\end{Prop}
\begin{Bew}
  \begin{align*}
    K^n\times K^n &\ara{\Phi_B\times\Phi_B}{} V\times V\ara{s}{}&K\\
    (x,y)&\mapsto&t_x\cdot M_B(s)\cdot y
  \end{align*}
  Es folgt: (eine Bastelei\ldots)
  \begin{align*}
    K^n\times K^n&\ara{\Phi_A\times\Phi_A}{}&V\times V&\ara{s}{}&K\\
    \ara{T\times T}{}&K^n\times K^n&\ara{\Phi_B\Phi_B}{}\uparrow&&
  \end{align*}
  \begin{align*}
    K^n\times K^n (x,y)&\mapsto& K(x^tM_a(s)\cdot y)=t_x^tM_B(s)T_y = (T_x)^tM_B(s)(T_y)\\
    \mapsto&K^n\times K^n(T_x,T_y)&\xmapsto{(T_x)^tM_B(s)(T_y)}\uparrow
  \end{align*}
  Es folgt aus der oberen Proposition (Vor der Transf.):
  \[T^tM_B(s)T=M_a(s)\]
\end{Bew}
\begin{Bsp}
  $V=K^n$, mit Standardskalaprodukt $<.,.>$. Ist $B=(e_1,\cdots,e_n)$, so ist
  \[\Mx{1&&0\\&\ddots&\\0&&1}=M_{\text{Standardbasis}}(<.,.>)\]
  Sei
  \begin{align*}
    A&=& (e_1,&e_2-e_1,&e_3-e_2,\cdots,&e_n-e_{n-1})\\
    &=:& (u_1,&u_2,&\cdots,&u_n
  \end{align*}
  Direkt aus der Definition:
  \[<u_i,u_j>=\begin{cases} 1& i=j=1 \\ 2&i=j>1\\ -1& \Abs{i-j}=1 \\ 0&\text{sonst}\end{cases} \]
  oder mit der Transformationsformel
  \[T=\Mx{1&1&&0\\\cdots\\\cdots&&&1\\0&&&1} \text{und} T^tE_N T''\]
\end{Bsp}
\begin{Bem}
  Ist $A$ die darstellende Matrix bezügloich einer Basis, so haben wir:
  \begin{itemize}
    \item symmetrisch $\Lra$ $A=A^t$
    \item schiefsymmetrisch $\Lra$ $A=-A^t$
  \end{itemize}
  Das stimmt überein mit (vgl. Übungblatt 3): $A\in M(n\times n)$ ist symmetrisch $\Lra$ $A=A^t$. $A$ ist schiefsymmetrisch oder antisymmetrisch (oder alternierend wenn $\text{char}(K)\neq 2$) $\Lra$ $A=-A^t$
\end{Bem}
\subsection{Bilineare und quadratische Formen}
Eine quadratische Form $V\to K$ wird zu einer Bilinearform assoziert. Falls $\dim_KV<\infty$: ``quadratische Form'' bedeutet $q:V\to K$ bezüglich einem Koordinatensystem gegeben als homogenes quadratisches Polynom. Ist $s:V\times V\to K$ eine bilineare Form, dann heisst
\begin{align*}
  q:&V&\to&K\\
  &v& \mapsto &q(v)=s(v,v)
\end{align*}
die zu $s$ gehörige quadratische Form.
\begin{Bsp}
  $<v,v>=v_1^2+\cdots+v_n^2$ für $v\in K^n$\\
  Für $A=(a_{ij})_{1\leq i, j\leq n}$ eine symmetrische Matrix mit $s:V\times V\to K$, $(x,y)\mapsto x^tAy$, haben wir
  \begin{align*}
    s(x,x)&=& x^tAx\\
    &=& \Mx{x_1&\cdots&x_n}\Mx{a_{11}&\cdots&a_{1n}\\ \vdots&&\vdots\\ a_{n1}&\cdots&a_{nn}}\Mx{x_1\\ \vdots \\ s_n}\\
    &=& \sum^n_{i,j=1}a_{ij}x_ix_j\\
    &=& \sum^n_{i=1}a_{ii}x_i^2+2\sum_{1\leq i<j\leq n}a_{ij}x_ix_j
  \end{align*}
\end{Bsp}
Ist $\text{char}(K)\neq 2$, so haben wir:
\begin{align*}
  \{\text{symm. bilineare Formen in $K^n$}\}&\lra&\{\text{quadr. Formen auf} K^n\}\\
  s&\mapsto &q(v):=s(v,v)\\
  &\mapsfrom (\text{Polarisierungsformel}) &
\end{align*}
\subsubsection{Polarisierungsformel}
Ist $s$ eine symmetrische Bilinearform und $q$ die zu $s$ gehörende quadratische Form über einem Vektorraum $V$ über $K$ mit $\text{char}(K)\neq 2$, dann gilt: 
\begin{align*}
  s(v,w)&=& \frac{1}{2}\left( q(v+w) - q(v) - q(w) \right)\\
  &=& \frac{1}{2}\left( q(v)+q(w)-q(v+w) \right)\\
  &=& \frac{1}{4}\left( q(v+w)-q(v-w) \right)
\end{align*}
\subsection{Sesquilineare Form}
\begin{Def}
  Sei $V$ ein komplexer Vektorraum. Eine Abbildung
  \[s:V\times V\to \mb{C}\]
  heisst sesquilinear falls:
  \begin{align*}
    s(v+v',w)&=& s(v,w)+s(v',w)\\
    s(\lambda v,w)&=& \lambda s(v,w)\\
    s(v,w+w')&=& s(v,w)+s(v,w')\\
    s(v,\lambda w)&=& \bar{\lambda} s(v,w)
  \end{align*}
  für $v,v',w,w'\in V$, $\lambda\in\mb{C}$
\end{Def}
\begin{Bsp}
  $<.,.>$ auf $\mb{C}^n$
  \begin{align*}
    s(f,g)&=& \int^1_0f(x)g(x)dx\\
    \text{auf} V&:=& \{\text{stetige Abb.}\}[0,1]\to\mb{C}\}
  \end{align*}
\end{Bsp}

\begin{Def}
  Eine sesquilineare Form heisst \underline{hermitesch}, falls
  \[s(w,v)=s(v,w)\ \forall v,w\in V\]
\end{Def}
\begin{Bsp}
  $<.,.>_c$ auf $\mb{C}^n$ ist hermitesch.
\end{Bsp}
\begin{Bem}
  Man spricht von \underline{hermiteschen Form}, diese sind immer sesquilinear
\end{Bem}
\begin{Def}
  Sei $\dim_\mb{C} V < \infty$, und $B:=(V_i)_{1\leq i \leq n}$ eine Basis. Ist $s$ eine sesquilineare Form, so definieren wir
  \[M_B(s):=\left( s(v_i,v_j) \right)_{1\leq i \leq n}\]
  die \underline{darstellende Matrix}. Sind $z,w\in V$
  \begin{align*}
    z&=& z_1v_1+\cdots+z_nv_n\\
    w&=& w_1v_1+\cdots+w_nv_n
  \end{align*}
  dann haben wir
  \begin{align*}
    s(z,w)&=&\sum^n_{i,j=1}z_i\bar{w_j}a_{ij} \text{wobei} a_{ij} = s(v_i,v_j)\\
    &=& (z_1\cdots z_n)\Mx{a_{11}&\cdots&a_{1n}\\ \vdots & & \vdots \\ a_{n1} & \cdots & a_{nn}} \Mx{\bar{w_1} \\ \vdots \\ \bar{w_n}}
    &=&  z^t M_B(s)\cdot \bar{w}
  \end{align*}
\end{Def}
\begin{Prop}
  Sei $V$ ein endlich dim. $\mb{C}$ Vektorraum und $B=(v_i)_{1\leq i \leq n}$. Wir haben eine Bijektion
  \[\{\text{sesquilineare Form auf $V$}\}\ \lra\ M(n\times n, \mb{C})\]
  Unter dieser Bijektion haben wir:
  \[\{\text{hermitesche Formen}\}\ \lra\ \{A\in M(n\times n, \mb{C}): A^t=\bar{A}\}\]
  Man sagt: eine Matrix $A\in M(n\times n, \mb{C})$ mit $A^t=\bar{A}$ ist \underline{hermitesch}.
\end{Prop}
\begin{Sat}{Transformationsformel}
  Sei $A=(u_1,\cdots,u_n)$ eine andere Basis mit Transformationsmatrix $T$:
  \[\text{TODO: hier einfügen}\]
  Dann gilt: 
  \[M_A(s)=T^t\cdot M_B(s)\cdot \bar{T}\]
  Mit $g(v):=s(v,v)$ gilt die Polarisierungsformel
  \begin{align*}
    s(v,w)=\frac{1}{4}\left( q(v+w)-q(v-w)+iq(v+iw)-iq(v-iw) \right)
  \end{align*}
\end{Sat}
\begin{Def}
  Sei $K=\mb{R}$ oder $\mb{C}$, $V$ ein $K$-Vektorraum und $s:V\times V\to K$ eine Billinearform $\begin{cases}
    \text{symmetrisch} & K= \mb{R} \\
    \text{hermitesch} & K=\mb{C}
  \end{cases}$
  heisst \underline{positiv definit}, falls $s(v;v)>0$ $\forall 0 \neq v\in V$
\end{Def}
\begin{Bsp}
  $<.,.>$ ist positiv definit auf $\mb{R}^n$\\
  $<.,.>_c$ ist psoitiv definit auf $\mb{C}^n$
\end{Bsp}
\begin{Def}
  Ein \underline{Skalarprodukt} ist $ \begin{cases}
    \text{positiv definite symetrische bilineare Form} & K=\mb{R}\\
    \text{eine positiv definite hermetische Form} & K=\mb{C}
  \end{cases}$ 
\end{Def}
\begin{Def}
  Skalarprodukt oft $<.,.>$, Norm $\Norm{v} := \sqrt{<v,v>}$
\end{Def}
\begin{Def}{Euklidischer Vektorraum}
  Vektorraum über $\mb{R}$ mit Skalarprodukt
\end{Def}
\begin{Def}{Untärer Vektorraum}
  Vektorraum über $\mb{C}$ mit Skalarprodukt
\end{Def}
\begin{Bsp}
  \begin{align*}
    V=\left\{ f:[0,1]\to\mb{R} \text{stetig} \right\} \text{mit} <f,g>=\int_0^1f(x)g(x)dx
    V=\left\{ f:[0,1]\to\mb{C} \text{stetig} \right\} \text{mit} <f,g>=\int_0^1f(x)\overline{g(x)}dx
  \end{align*}
  in beiden Fällen
  \[\Norm{f}=\sqrt{\int_0^1\Abs{f(x)}^2dx}\]
  ``$L^2$-Norm''
\end{Bsp}
\begin{Bem}
  In einem beliebigen euklidischen bzw. unitären Vektorraum gilt die Cauchy-Schwarz'sche Ungleichtung
  \[\Abs{<v,w>}\leq \Norm{v}\Norm{w}\ \forall v,w\in V\]
  mit $=$ genau dann, wenn $v$ und $w$ linear abhängig sind.
\end{Bem}
\begin{Bew}
  (Skizze) klar falls $v=0$ oder $w=0$, also nehmen wir an, dass $v\neq 0$ und $w\neq 0$
  1. Reduktion: zum Fall $\Norm{v}=\Norm{w}=1$.
  \begin{align*}
    v_1:=\frac{v}{\Norm{v}} & w_1:=\frac{w}{\Norm{w}}\\
    \Norm{v_1}=1 & \Norm{w_1}=1
  \end{align*}
  2. Reduktion:  Es reicht aus, zu zeigen: $\Re <v,w>\leq 1$ = genau dann wenn $V=W$
  \begin{align*}
    \Abs{<v,w>}&=& \mu<v,w> & \mu\in \mb{C}, \Abs{\mu}=1 \\
    &=& <\mu v,w>\in \mb{R}_\geq \\
    &=& \Re <v',w> \text{wobei} v':=\mu v
  \end{align*}
  Cauchy-Schwarz'sche Ungleichung $\leq$, Gleicheit: $v,w$ linear unabhängig $\implies$ $v',w$ linear unabhängig $\implies$ $v'\neq w$
\end{Bew}
\begin{Eig}
  \begin{align*}
     & = \text{falls} \\
    <v-w,v-w> \geq 0 & v-w=0\\
    <v,v> - <v,w> - <w,v> + <w,w> \geq 0 & v=w\\
    1-<v,v>-\overline{<v,w>} + 1 \geq 0 & v=w
  \end{align*}
\end{Eig}
\begin{Bsp}
  Ist $T:V\to\mb{R}^n$ oder $T:V\to\mb{C}^n$ ein Isomorphismus, dann ist $s:V\times V\to\mb{R}$ (bzw. $s:V\times V\to\mb{C}$) gegeben durch
  \[s(x,y)=<T_x,T_y>\]
  bzw.
  \[s(x,y)=<T_x,T_y>_c\]
  ein Skalarprodukt.
\end{Bsp}
\begin{Def}
  Sei $V$ ein exklusiver, bzw. unitärer Vektorraum
  \begin{itemize}
    \item $v,w\in V$ heisst orthogonal, falls $<v,w>=0$
    \item $U,W\subset V$ heissen orthogonal (geschrieben $U\perp V$) falls $U\perp W$ $\forall u\in U$, $w\in W$
    \item $U\subset W$ das orthagonale Koplement ist $U^\perp=\left\{ v\in V: u\perp v \forall u\in U \right\}$
    \item $v_1,\cdots, v_n$ sind orthogonal, falls $v_i\perp v_j$ $\forall i\neq j$
    \item $v_1,\cdots, v_n$ sind orthonormal, falls $v_i\perp v_j$ $\forall i\neq j$ und $\Norm{v_i}=1$ $\forall i$
    \item $V$ ist orthagonale direkte Summe von Untervektorräumen $V_1,\cdots,V_r$ falls 
      \begin{align*}
        V=V_1\bigoplus \cdots \bigoplus V_r\\
        V_i\perp V_j \forall i\neq j
      \end{align*}
  \end{itemize}
  \[C([-1,1],\mb{R}):=\left\{ f:[-1;1]\to \mb{R} \text{stetig} \right\}\]
  dann ist $C([-1,1]\mb{R})$ die orthogonale direkte Summe von $C([-1,1]\mb{R})_\text{gerade}$ und $C([-1,1]\mb{R})_\text{ungerade}$. gerade: $f(-x)=f(x)$ und ungerade: $f(-x)=-f(x)$
  \[f(x)=\underbrace{\frac{f(x)+f(-x)}{2}}_\text{gerader Teil} + \underbrace{\frac{f(x)-f(-x)}{2}}_\text{ungerader Teil}\]
  $g$ gerade, $h$ ungerade $\implies$ $gh$ ungerade $\implies$ $<g,h> = \int_{-1}^1g(x)h(x)=0$
\end{Def}

\begin{Bem}
  Ist $v_1,\cdots,v_n$ eine orthonormale Familie mit $v_i \neq 0\forall i$, so gilt 
  \begin{enumerate}
    \item $(i)(v_1,\cdots,v_n)$ ist linear unabhängig ($c_1v_1+\cdots+c_nv_n=0$ $\implies$ $c_i <v_i,v-i>+\cdots+c_i<v_i,v_i>+\cdots+c_n<v_n,v_i>=0$ $\implies$ $c_i\Norm{v_i}^2=0$ $\implies$ $c_i=0$)
    \item $\left(\frac{v_1}{\Norm{v_i}},\cdots,\frac{v_n}{\Norm{v_i}}\right)$ ist orthonormal
  \end{enumerate}
\end{Bem}
\begin{Sat}
  Ist $(v_1,\cdots,v_n)$ eine orthonormale Basis von $V$, so gilt folgendes für beliebiges $v\in V$
  \[v=\sum^n_{i=1}<v_i,v_j>v_i\]
  \begin{align*}
    v&=&\sum^b_{i=1}c_iv_i\\
    <v,v_j>&=& \sum^n_{i=1}c_i<v_i,v_j>\\
    &=& c_j<v_i,v_j>
    &=& c_j
  \end{align*}
\end{Sat}
\begin{Prop}
  Sei $K$ $=\mb{R}$ oder $\mb{C}$ und $(V,<.,.>)$ ein euklidischer bzw. unitärer Vektorraum über $K$
  \begin{enumerate}
    \item Ist $n:=\dim_KV<\infty$ und $(v_1,\cdots,v_d)$ eine orthonormale Familie von Vektoren von $V$, so existieren $v_{d+1},\cdots,v_n$, so dass $(v_1,\cdots,v_n)$ eine orthonormale Basis von $V$ ist.
    \item Ist $U\subset V$ ein endlichdimensionaler Untervektorraum, so gilt $V=U\bigoplus U^\perp$, orthonormal direkte Summe.
  \end{enumerate}
\end{Prop}
\begin{Bew}
  Es gibt triviale Fälle: $d=n$ in 1., $U=0$ in 2. Auch: der Fall $(d=0)$ in 1. $\La$ $d=1$: $0\neq v\in V$ beliebiger Vektor, wir nehmen $b_1=\frac{v}{\Norm{v}}$\\
  Beweis durch Induktion nach $N$ mit Indunktionsannahme 1. gilt für $n\leq N$ $N=1$ okay.
  \subparagraph{Plan:}Wir zeigen IA $\implies$ 2. für $\dim_KU\leq N$ und IA $\implies$ 1. für $n\leq N+1$.\\
  IA $\xRightarrow{\dim U \leq N}$ $\exists$ orthonormale Basis $(u_1,\cdots,u_d)$ von $U$ $d:=\dim U$. Für beliebiges $v\in V$ gilt:
  \[v-\sum^n_{i=1}<v_i,v_j>u_i\in U^\perp\]
  denn
  \[\left<v-\sum^n_{i=1}<v_i,u_i>u_i,u_j\right>=<v,u_j>-\sum^n_{i=1}<v,u_j><u_i,u_j>=0\]
  Und: 1. für $\dim V\leq N+1$ folgt aus IA und 2. für $U\leq N$
  \begin{align*}
    1\leq d<n=\dim V \geq N+1\\
    \implies 1\leq d \leq N \text{und} 1\leq \dim V-d \geq N\\
  \end{align*}
  Sei $U:= span(v_1,\cdots,v_d)$ Aus 2. haben wir $V=U\bigoplus U^\perp$ Nach IA, $\exists$ orthonormale Basis $(v_{d+1},\cdots,v_n)$ von $U^\perp$ Es folgt, dass $(v_1,\cdots,v_n)$ ist eine orthonormale Basis von $V$.
\end{Bew}
\begin{Eig}
  Praktisches Verfahren zu testen ob ein symmetrisch bilineare bzw. hermetische Form ein Skalaprodukt ist (falls $\dim_KV<\infty$). Verfahren:
  \begin{itemize}
    \item wählen $U\subset V$ nicht trivialer Untervektorraum (z.B. $U=span(v), 0 \neq v\in V$)
    \item Berechnen $U^\perp$
    \item Testen:
      \begin{itemize}
        \item Ist $V=U\bigoplus U^\perp$?
        \item Ist die Einschränkung von der Form auf $U$ ein Skalarprodukt? 
        \item Ist die Einschränkung von der Form auf $U^\perp$ ein Skalarprodukt?
      \end{itemize}
  \end{itemize}
\end{Eig}
\begin{Bsp}
  $\mb{R}^3\times\mb{R}^3\to\mb{R}$
  \begin{align*}
    M:=\Mx{3&1&2\\
    1&3&-1\\
    2&-1&2}
  \end{align*}
  Wir betrachten die entsprechende Bilinearform
  \[(x,y)\mapsto t_x\cdot M \cdot y\]
  \begin{align*}
    U&=& \span(e_1)\\
    U^\perp&=& \left\{ (x_1,x_2,x_3 :3x_1+x_2+2x_3=0\right\}\\
    &=& \span\left( (1,-3,0),(0,2,-1) \right)
  \end{align*}
  \begin{table}
    \centering
    \begin{tabular}{c|c}
      $U:$ 1-dimensional & $U^\perp$: 2-dim\\
      $s(e_1,e_1)=3$ & =24
    \end{tabular}
  \end{table}
  Die darstellende Matrix:
  \[s|_{U^\perp}\rsa \Mx{24&-21\\-21&18}\]
  \begin{align*}
    U &\cong& \mb{R}^2\\
    W&=&\span(e_1)\in\mb{R}^2\\
    W^\perp&=&\left\{ (x_1,x_2)|24x_1-21x_2=0 \right\}\\
    &=& \span(21,24)\\
  \end{align*}
  $W$ 1-dim, $W^\perp$ 1-dim\\
  \begin{align*}
    (s|_{U^\perp})(e_1,e_2)=24\\
    \Mx{21&24}\Mx{24&-21\\-21&18}\Mx{21&18}=-216
  \end{align*}
  $\implies$ kein Skalarprodukt
\end{Bsp}
\subsection{Volumen}
\begin{Def}{Volumen}
  \begin{tabular}{ccccc}
    Skalarprodukt & $\rsa$ & Norm & $\rsa$ &Metrik\\
    $<.,.>$ & & $\Norm{v}:=\sqrt{<v,v>}$ & & $d(x,y)=\Norm{y,x}$
  \end{tabular}
  $K=\mb{R}$ $\rsa$ Volumen ($\dim V < \infty$)
\end{Def}
\subsubsection{Spat}
\begin{Def}{Spat}
  $u_1,\cdots,u_n$ orthonormale Basis. Dann ist der von $(u_1,\cdots,u_n)$ aufgespannte Spat definiert als
  \begin{align*}
    \left\{ \sum^n_{i=1}c_iu_i|0\leq c_i\leq 1\ \forall i \right\}
  \end{align*}
  Vol(Spat):=1\\
  Falls $v_1,\cdots,v_n\in V$ beliebig sind, dann hat der von $(v_1,\cdots,v_n)$ aufgespannte Spat
  \[\text{Vol} = \Abs{\det\Mx{a_{11}&\cdots&a_{1n}\\ \vdots & & \vdots \\ a_{n1}&\cdots&a_{nn}}}\ v_i=\sum^n_{j=1}a_{ij}u_i\]
  Sei $b_ij:=<v_i,v_j>$ und $B:=\Mx{b_{11}&\cdots&b_{1n}\\ \vdots & & \vdots \\ b_{n1}&\cdots&b_{nn}}$ Wir haben $b_{ij}=\sum^n_{k=1}a_{ik}a_{jk}$, also $B=A\cdot A^t$. Es folgt:
  \[\text{Vol}=\sqrt{(\det A)^2}=\sqrt{\det B}\]
  Vorteile:
  \begin{itemize}
    \item keine Wahl von orthonormaler Basis nötig
    \item auch sinnvoll für eine Kollektion $v_1,\cdots,v_m$ evzl. $m\neq n$
  \end{itemize}
\end{Def}
\begin{Bsp}{$m=1$}
  \begin{align*}
    \det B &=& \Norm{v_1}^2\\
    \sqrt{\det B}&=& \Norm{v_1}
  \end{align*}
\end{Bsp}
\begin{Def}{Grammsche Determinante}
  Im $m$-dim Volumen $:=\sqrt{G(v_1,\cdots,v_m)}$ wobei
  \[G(v_1,\cdots,v_m):=\det\Mx{<v_i,v_j>}_{1\leq i,j\leq m}\]
  die sogenannte \underline{Grammsche Determinante} ist.
\end{Def}
\begin{Bem}
  Es gilt $G(v_1,\cdots,v_m)=0$ $\Lra$ $v_1,\cdots,v_m$ lineare abhängig, weil
  \[A=(a_{ij})\in M(m\times n,\mb{R})\]
  mit
  \[G(v_1,\cdots,v_m)=\det (A,A^t)=\sum(m\times m \text{Minor})\]
\end{Bem}
\begin{Bem}
  \[\text{Vol}(v_1,\cdots,v_m):=\sqrt{G(v_1,\cdots,v_m)}\]
  ist 0 falls $\exists i:v_i=0$\\
  sonst:
  \[\text{Vol}(v_1,\cdots,v_m)=\Norm{v_1}\cdots\Norm{v_m}\text{Vol}(\frac{v_1}{\Norm{v_1}},\cdots,\frac{v_m}{\Norm{v_m}}\]
\end{Bem}

\begin{Sat}{Hadamard'sche Ungleichung}
  \[Vol(v_1,\cdots,v_m)\leq \Norm{v_1}\cdots\Norm{v_m}\]
  für $0\neq v_i\in V$, $i=1,\cdots,m$. Mit Gleichheit genau dann wenn $v_1,\cdots,v_m$ orthogonal sind.
\end{Sat}
\begin{Bew}
  Durch fallende Induktion nach
  \[\max\left\{ \Abs{I}:I\subset\left\{ 1,\cdots,m \right\}|(v_i)_{i\in I}\ \text{orthogonal} \right\}\]
  \subparagraph{Fall $\max\left\{ \cdots \right\}=m$} das bedeutet, $v_1,\cdots,v_m$ sind orthogonal. Dann:
  \[G(v_1,\cdots,v_m)=\det\Mx{\Norm{v_1}^2& &0\\ & \ddots & \\ 0 & & \Norm{v_m}^2}=\Norm{v_1}^2\cdots\Norm{v_m}^2\]
  Die ist der Induktionsanfang.\\
  Sei $r\in\mb{N}$, $1\leq r <m$. Induktionsanahme: Ungleichung für den Fall
  \[\max\left\{ \Abs{I}:(v_i)_{i\in I}\ \text{orthogonal} \right\}>r\]
  Sei $v_1,\cdots,v_m$, so dass $\max\left\{ \cdots \right\}=r$. o.B.d.A: $v_1,\cdots,v_r$ orthagonal. Wir schreiben:
  \begin{align*}
    v_m & =\underbrace{v_m-\sum^r_{i=1}\frac{\langle v_m,v_i\rangle }{\langle v_i,v_i\rangle }v_i}_{\tilde{v}_m \in \Span (v_1,\cdots,v_r)^\perp} +\underbrace{\sum^r_{i=1}\frac{\langle v_m,v_i\rangle }{\langle v_i,v_i\rangle }v_i}_{\tilde{\tilde{v}}_m\in \Span (v_1,\cdots,v_r)}\\
    & < \tilde{v_m},\tilde{\tilde{v_m}}=0
  \end{align*}
  \begin{itemize}
    \item $v=\tilde{v}+\tilde{\tilde{v}}$
    \item $<\tilde{v},\tilde{\tilde{v}}>=0$
    \item $\Norm{v}^2=\Norm{\tilde{v}}^2+\Norm{\tilde{\tilde{v}}}^2$
  \end{itemize}
  Das ist eine \underline{Orthogonale Projektion}
  Wir haben 
  \[G(v_1,\cdots,v_m)=G(v_1,\cdots,v_{m-1},\tilde{v_m})\]
  weil (Spalten- und Zeilenumforumgen\ldots). Es folgt:
  \begin{align*}
    Vol(v_1,\cdots,v_m)=Vol(v_1,\cdots,v_{m-1},\tilde{v_m}) &\leq \Norm{v_1}\cdots\Norm{v_{m-1}}\Norm{\tilde{v_m}}\\
    & <\Norm{v_1}\cdots\Norm{v_{m-1}}\Norm{\tilde{v_m}}
  \end{align*}
\end{Bew}
\begin{Def}{Gram-Schmidt-Orthagonalisierungsverfahren}
  \[\tilde{v_r}:=v_r-\sum^{r-1}_{i=1}\frac{<v_r,\tilde{v_i}>}{<\tilde{v_i},\tilde{v_i}>}\tilde{v_i}, \text{für} 1,2,\cdots\]
  gegeben: eine Kollektion $(v_1,\cdots,v_n)$ oder abzählbar unendlich $(v_1,v_2,\cdots)$. Das Verfahren produziert $(\tilde{v_1},\tilde{v_2},\cdots)$, mit:
  \begin{align*}
    \span(\tilde{v_1},\tilde{v_2},\cdots&=&\span(v_1,v_2,\cdots)\\
    \span(\tilde{v_1},\cdots,\tilde{v_m})&=& \span(v_1,\cdots,v_m)\ \forall m\\
    (\tilde{v_1},\tilde{v_2},\cdots) & & \text{sind orthogonal}
  \end{align*}
\end{Def}
\begin{Bsp}
  $C([-1,1],\mb{R})$ mit $<f,g>=\int^1_{-1}f(x)g(x)\md x$
  \begin{align*}
    (1,x,x^2,\cdots)\\
    \xrightarrow{\text{GS}}& \frac{<x^2,1>}{<1,1>}=\frac{2/3}{2}\\
    (1,x,x^2-\frac{1}{3},x^3-\frac{3}{5}\cdots)
  \end{align*}
  Bis auf Normalisierung bekommen wir die Legendre-Polynome.
\end{Bsp}
\begin{Faz}
  \begin{tabular}[htbp]{rcrcr}
    Bilinearform & $\xrightarrow{\text{+ def, symm}}$& Norm &$\xrightarrow{\text{}}$& Metrik\\
    Sesquilinearform & $\xrightarrow{\text{+ def, hermitesch}}$ & \\
  \end{tabular}
  Norm:
  \begin{align*}
    \Norm{.}:V\to\mb{R}_{\geq 0}\\
    \Norm{x}=0\ \Lra \ x=0\\
    \Norm{\lambda x}=\Abs{\lambda}\Norm{x}\\
    \Norm{x+y}\leq \Norm{x} +\Norm{y}\\
  \end{align*}
  Metrik:
  \begin{align*}
    d:V\times V\to\mb{R}_{\leq 0}\\
    d(x,<)=0\ \Lra\ x=y\\
    d(x,y)=d(y,x)\\
    d(x,z)\leq d(y,y) + d(y,z)
  \end{align*}
  Aber: nicht jede Metrik, nicht einmal jede transinvariante Metrik kommt von einer Norm.
\end{Faz}
\begin{Bem}
  Eine Norm kommt von einer +def, symm Bilinearform
  \[\Lra\ \Norm{x+y}^2+\Norm{x-y}^2=2\left(\Norm{x}^2+\Norm{y}^2\right)\ \forall x,y\in V\]
\end{Bem}
\begin{Def}{ausgeartete Bilinearform}
  Eine Bilinearform $s:V\times V\to K$ ist ausgeartet (oder: entartet), falls eine oder beide der induzierten Abbildungen $V\to V^*$ \underline{nicht} injektiv ist.
  \begin{align*}
    v \mapsto & \left( w\mapsto s(v,w) \right)\\
    v \mapsto & \left( w\mapsto s(w,v) \right)\\
  \end{align*}
\end{Def}
\begin{Bem}
  Falls $\dim_K V <\infty$, dann:
  % Hier diagram\ldots
  \begin{align*}
    v\mapsto & \left(w\mapsto s(v,w)\right) & \text{injektiv}\\
    &\updownarrow \\
    v\mapsto & \left(w\mapsto s(w,v)\right) & \text{injektiv}\\
    &\updownarrow \\
    &s \text{ist nicht ausgeartet}\\
    &\updownarrow \\
    &\text{die darstennelde Matrix ist invertierbar}
  \end{align*}
  \begin{align*}
    s(v,w) & = v^t\cdot A \cdot w\\
    & = \left( A^t\cdot v \right)^t
  \end{align*}<++>
\end{Bem}
\begin{Sat}
  Sei $V$ ein $K$-Vektorraum, $s:V\times V\to K$ eine symmetrische oder schiefsymmetrische Bilinearform. Für $U\subset V$ Untervektorraum, schreiben wir noch
  \[U^\perp := \left\{ v\in V:s(u,v)=0\ \forall u\in U \right\}\]
  ($s(v,u)=0\ \Lra\ s(u,v)=0$ weil $s$ symm. bzw. schiefsymm.)
\end{Sat}
\begin{Prop}
  Sei $V$ ein endlich dimensionaler $K$-Vektorraum und $s:V\times V\to K$ eine nicht ausgeartete symmetrische oder schiefsymmetrische Bilinearform. Sei $U\subset V$ ein Untervektorraum. Dann gilt:
  \[\dim U+\dim U^\perp=\dim V\]
\end{Prop}
\begin{Bew}
  Sei $(v_i)_{i=1,\cdots n}$ eine Basis mit $n:=\dim V$, und $A$ die darstellende Matrix von $s$ bzw. $(v_i)$. Wir haben dann:
  \[s(x,y)=x^t\cdot A \cdot y\]
  und $A^t=\pm A$, $\det A\neq 0$
  \begin{align*}
    U^\perp &= \left\{ x\in V_i| x^t\cdot A \cdot y=0\ \forall y\in U \right\}
    &= \left\{ x\in V_i| (x\cdot A)^t \cdot y=0\ \forall y\in U \right\}
  \end{align*}
  Sei $F:V\to V$ lin.Abb. $\lra$ $A$. Dann:
  \begin{align*}
    F(U^\perp) &=\left\{ Ax|(Ax)^ty=0\ \forall y\in U \right\}
    &=\left\{ Ax|\tilde{x}^ty=0\ \forall y\in U \right\}
  \end{align*}
  Es folgt: mit
  \[B:=\Mx{| & & |\\ u_1 & \cdots & u_d \\ | & & |}\]
  $(u_1,\cdots,u_d)$ Basis von $U$, dann ist $F(U^\perp)=\Ker B$. Jetzt:
  \[\dim U^\perp = \dim F(U^\perp)=\dim \Ker B=n-\dim U\]
\end{Bew}
\begin{Kor}
  $\dim U<\infty$, $s:V\times V\to K$ nicht ausgeartet, (schief-) symm.
  \[U\subset \implies \left( U^\perp \right)^\perp = U\]
\end{Kor}<++>

\begin{Bem}
  Es ist \underline{nicht} immer der Fall, dass $V=U\bigoplus U'$, weil es ist möglich, dass $U\cup U^\perp\neq 0$. 2 Extremfälle:
  \begin{itemize}
    \item $U$ ist isotropisch ($s|_{U'}$ ist trivial) $\Lra$ $U\subset \underbrace{U^\perp}_{\dim V-\dim U}$
    \item $s|_U$ ist auch nicht ausgeartet $\Lra$ $U\cup U^\perp=0$ $\Lra$ $V=U\bigoplus U^\perp$
  \end{itemize}
  Aus 1. ist klar:
  \[\dim U \leq \frac{1}{2}\dim V\ \forall \text{isotrop} U\subset V\]
\end{Bem}
\subsection{Orthogonale und unitäre Endomorphismen}
$K=\mb{R}$ oder $\mb{C}$
\begin{Def}{orthogonaler bzw. unitärer Endomorphismus}
  Sei $V, \langle, \rangle$ ein ortho. bzw. unitärer Vektorraum. Ein Endomprhismus $F:V\to V$ heisst \underline{orthogonal} bzw. \underline{unitär} falls
  \[\left\langle F(v), F(w) \right\rangle = \left\langle v, w \right\rangle\ \forall v,w\in V\]
\end{Def}
\begin{Bem}
  Das ist äquivalent zu
  \[\Norm{F(v)}=\Norm{v}\ \forall v\in V\]
\end{Bem}
\begin{Eig}{orthogonaler bzw. unitärer Endomorphismus}
  Sei $F$ ein ortho. bzw. unitärer Endomorphismus. Dann:
  \begin{itemize}
    \item $F$ ist injektiv
    \item Falls $\dim_KV<\infty$, $F$ ist bijektiv, und $F'$ ist auch ortho. bzw. unitär
    \item Für jeden Eigenwert $\lambda\in K$ gilt $\Abs{\lambda}=1$. Eigenvektor $v$:
      \[\Norm{v}=\Norm{F(v)}=\Norm{\lambda v}=\Abs{\lambda}\Norm{v}\]
  \end{itemize}
  Falls $V=\mb{R}^n$ oder $\mb{C}^n$ mit Standardskalarprodukt
  \begin{align*}
    \left\langle v,w \right\rangle = v^tw&\text{bzw}&\left\langle v,w \right\rangle_c=v^t\bar{w}    
  \end{align*}
  Ist $F$ zur Matrix $A$ entsprechend, dann
  \begin{align*}
    \left\langle F(v),F(w) \right\rangle=\left\langle v,w \right\rangle\ &\Lra& (Av)^tAw=v^tw\\
    & &\text{bzw.} (Av)^t\overline{Aw}=v^tw\\
    \Lra\ v^tA^tAw=v^tw&\Lra&A^tA=E_n\\
    \text{bzw}\Lra\ v^tA^t\bar{A}\bar{w}=v^t\bar{w}&\Lra&A^t\bar{A}=E_n
  \end{align*}
\end{Eig}
\begin{Def}{ortho. bzw. unitäre Matrix}
  $O(n):= A\in GL_n(\mb{R})$ heisst \underline{orthogonal} falls $A^tA=E_n$\\
  $U(n):= A\in GL_n(\mb{C})$ heisst \underline{unitär} falls $A^t\bar{A}=E_n$
\end{Def}
\begin{Not}
  \[O_n:=\left\{ A\in GL_n(\mb{R})|A\ \text{orthogonal} \right\}\]
  \[O_n:=\left\{ A\in GL_n(\mb{C})|A\ \text{unitär} \right\}\]
  Weil
  \[A,B\in O(n)\implies (AB)^t(AB)=B^tA^tAB=B^tB=E_n\implies AB\in O(n)\]
  haben wir $O(n)\subset GL_n(\mb{R})$ ist eine Untergruppe. Ähnlich: $U(n)\subset GL_n(\mb{C})$ ist eine Untergruppe.
\end{Not}
\begin{Not}
  \[SO(n)=O(n)\cap SL_n(\mb{R})\]
  \[SU(n)=U(n)\cap SL_n(\mb{C})\]
\end{Not}
\begin{Not}{ortho. bzw. unitärer Vektorraum}
  \[O(V)=\left\{ F\in GL(V)|\text{ortho.} \right\}\]
  \[U(V)=\left\{ F\in GL(V)|\text{unitär} \right\}\]
\end{Not}
\begin{Bem}
  \[A\in O(n)\implies \det A\in \left\{ \pm 1 \right\}\]
  \[A\in U(n)\implies \det A\in \left\{ \pm z\in\mb{C}: \Abs{Z}=1 \right\}\]
\end{Bem}
\begin{Eig}{Charakterisierungen von ortho. bzw. unitären Matrizen}
  Äquivalente Charakterisierungen von orthogonalen bzw. unitären Matrizen $A\in GL_n(\mb{R})$:\\
  $A$ ist orthogonal $\Lra$ $A^{-1}=A^t$ $\Lra$ $A^tA=E_n$ $\Lra$ $AA^t=E_n$ $\Lra$ die Spalten von $A$ bilden eine Orthonormalbasis von $\mb{R}^n$ $\Lra$ die Zeilen von $A$ bilden eine Orthonormalbasis von $\mb{R}^n$.\\
  Ähnlich:\\
  $A$ ist unitär $\Lra$ $A^{-1}=\bar{A}^t$ $\Lra$ $A^t\bar{A}=E_n$ $\Lra$ $\bar{A}A^t=E_n$ $\Lra$ die Spalten von $A$ bilden eine Orthonormalbasis von $\mb{C}^n$ $\Lra$ die Zeilen von $A$ bilden eine Orthonormalbasis von $\mb{C}^n$.\\
  Für $n=1$
  \begin{align*}
    O(1)=\left\{ \pm 1 \right\}& & U(1)=\left\{ z\in\mb{C}:\Abs{z}=1 \right\}\cong S^1\\
    SO(1)=\left\{ 1 \right\}& &SU(1)=\left\{ 1 \right\}
  \end{align*}
  Für $n=2$: $(a,b)\in\mb{R}^2$, $a^2+b^2=1$
  \begin{align*}
    O(2)=\left\{ \Mx{\cos\theta&-\sin\theta\\ \sin\theta&\cos\theta}|\theta\in\mb{R} \right\}\cup\left\{ \Mx{\cos\theta&\sin\theta\\ \sin\theta&-\cos\theta}|\theta\in\mb{R} \right\} \\
    SO(2)=\left\{ \Mx{\cos\theta&-\sin\theta\\ \sin\theta&\cos\theta}|\theta\in\mb{R} \right\}\cong S^1\\
  \end{align*}
  $(z,w)\in\mb{C}^2$, $\Abs{z}^2+\Abs{w}^2=1$, $(-\bar{w},\bar{z})\perp(z,w)$
  \begin{align*}
    U(2)=\left\{ \Mx{z& -\lambda\bar{w}\\ w &\lambda\bar{z}}|(z,w)\in\mb{C}^2, \Abs{z}^2+\Abs{w}^2=1, \lambda\in\mb{C}, \Abs{\lambda}=1\right\} \cong S^3 \times S^1
  \end{align*}
  \begin{align*}
    SU(2)=\left\{ \Mx{z& -\bar{w}\\ w &\bar{z}}|(z,w)\in\mb{C}^2, \Abs{z}^2+\Abs{w}^2=1\right\}\cong S^3
  \end{align*}
  $SO(3)$ eine explizite Beschreibung ist möglich (später)
\end{Eig}
\begin{Prop}
  Sei $V$ ein endlich dimensionaler $\mb{C}$-Vektorraum mit Skalarprodukt $\left\langle , \right\rangle$, und sei $F:V\to V$ ein unitärer Endomorphismus. Dann besitzt $V$ eine Orthonormalbasis von Eigenvektoren von $F$.
\end{Prop}
\begin{Bew}
  Durch Indunktion nach $\dim V$. $\dim V=0,1$ trivial. $\dim V\geq 2$ Weil $\mb{C}$ algebraisch abgeschlossen ist, gibt es einen Eigenwert $\lambda\in\mb{C}$. Sei $v\in V$ ein Eigenvektor, mit $\Norm{v}=1$. Weil $F$ untär ist, haben wir $F(v^\perp)=v^\perp$. Wir haben $\dim v^\perp =\dim V-1$
  \begin{align*}
    w\in v^\perp \left\langle v,w \right\rangle \implies \left\langle v,w \right\rangle = 0\\
    \lambda\left\langle v,F(w) \right\rangle = \left\langle \lambda v,F(w) \right\rangle=\left\langle F(v),F(w) \right\rangle=0\\
    \implies F(v^\perp)\subset v^\perp
  \end{align*}
  Aus der Induktionsannahme folgt, dass $\exists$ Orthonormalbasis von $v^\perp$ von Eigenvektoren von $F$. Zusammen mit $v$ $\stackrel{V=\Span\bigoplus v^\perp}{\rsa}$ Orthonormalbasis von $V$
\end{Bew}
\begin{Kor}
  Sei $A\in U(n)$. Dann $\exists S\in U(n)$, $\theta_1,\cdots,\theta_n\in\mb{R}$ so dass
  \[SAS^{-1}=\Mx{e^{i\theta_1}&\cdots&0\\ \vdots&\ddots&\vdots\\0&\cdots&e^{i\theta_n}}\]
\end{Kor}
\begin{Prop}
  Sei $V$ ein endlich dimensionaler $\mb{R}$-Vektorraum mit Skalarprodukt $\left\langle , \right\rangle$, und sei $F:V\to V$ ein orthogonaler Endomorphismus. Dann besitzt $V$ eine Orthonormalbasis $\left( v_1^+,\cdots,v_r^+,v_1^-,\cdots,v_s^-,w_1,w_1',\cdots,w_t,w_t' \right)$
  \begin{itemize}
    \item $F(v^+_i)=v_i^+$
    \item $F(v_i^-)=-v_i^-$
    \item $F(w_i)=(\cos\theta_i)w_i+(\sin\theta_i)w_i'$
    \item $F(w_i')=(-\sin\theta w_i)+(\cos\theta_i)w_i'$
  \end{itemize}
  mit $\theta_i\in\mb{R}$, $0<\Abs{\theta}<\phi$, $i=1,\cdots,t$
\end{Prop}
\begin{Bew}
  Durch Induktion nach $\dim V$: $\dim V=0,1,2$ trivial. $\dim >2$ (nächstes mal)
\end{Bew}

\begin{Faz}
  \begin{tabular}[htb]{cc}
    $\dim_\mb{R}V$&$\left\langle .,. \right\rangle$ Skalarprodukt\\
    $F:V\to V$ orthogonaler Endomorphismus\\
    $\implies$ $\exists$ orthogonale Basis& +1 oder -1 Eigenvektoren
  \end{tabular}
    \[ F(\alpha w_i+\beta w_i')=\left( \alpha \cos \Theta_i - \beta\sin\Theta_i \right)w_i+\left( \alpha\sin\Theta_i\beta\cos\Theta_i \right)w_i',\ \Theta_i\in \mb{R}\]
\end{Faz}
\begin{Bew}{Fortsetzung}
  Durch Induktion nach $\dim V$, Induktionsanfang: $\dim V\leq 2$ $\dim V=2$ bezüglich beliebiger Basis ($w_1,w_1'$).
  \[V:\Mx{\cos\Theta&-\sin\Theta\\ \sin\Theta&\cos\Theta} \text{oder} \Mx{\cos\Theta&\sin\Theta\\ -\sin\Theta&\cos\Theta}\]
  Matrix 1: $w_1,w_2$ ist wie oben, Matrix 2: chaakteristisches Polynom $t^2-1=(t-1)(t+1)$ $\to$ (+1-Eigenvektor, -1-Eigenvektor)
  \begin{enumerate}
    \item Fall: $\exists$ reeller Eigenwert
      \[\lambda\in\mb{R},\ \Abs{\lambda}=1\ v\in V\ F(v)=\lambda v\]
      wir zeigen, dass $F(v^\perp)=v^\perp$ genau wie im Fall einees unitären Endomorphismus
      \begin{align*}
        \dim(v^\perp)=\dim V-1 \stackrel{IA}{\rsa} v^\perp: \text{orthonormale Basis}\\
        V=\span(v)\bigoplus v^\perp
      \end{align*}
    \item Fall: $\not\exists$ reeller Eigenwert
      \[\implies P_F(t)=\prod^{(\dim V)/2}_{i=1}Q_i(t)\]
      $Q_i(t)$ irreduzibles quadratisches Polynom. Aus dem Satz von Cayley-Hamilton folgt:
      \[\implies \exists \overbrace{v}^{\neq 0}\in V,\ i\text{mit} Q_i(F)v=0\]
      Sei $0\neq v_0\in V$ beliebigen Vektor $P_F(F)v_0=0$
      \begin{align*}
        &Q_1(F)Q_2(F)\cdots+_{\frac{\dim V}{2}}(F)v_0=0\\
        \implies \exists j:\ & Q_j(F)Q_{j+1}(F)\cdots+_{\frac{\dim V}{2}}(F)v_0=0\\
        \text{aber}\ & Q_{j+1}(F)\cdots Q_{\dim V}{2}(F)v_0\neq 0\\
      \end{align*}
      $\implies$ wir nehmen $i:=j$ und $v:=Q_{j+1}(F)\cdots Q_{\frac{\dim V}{2}(F)v_0}$.\\
      Beh: $U:=\Span(v,F(v))$ ist ein $F$-invariante Vektorraum. $Q_i(F)_v=0$ $\implies$ $\exists a,b\in\mb{R}$ mit $F(F(v))=av + bF(V)$. Es folgt: $U^\perp$ ist auch $F$-invariant. $V=U\bigoplus U^\perp$ $\stackrel{IA}{\rsa}$ Basen von $U$ und von $U^\perp$ wie oben. Die Vereinigung dieser Basen ist wie erwünscht.
  \end{enumerate}
\end{Bew}
\begin{Kor}
  Sei $A\in O(n)$. Dann gibt es ein $S\in O(n)$ und $r,s,t\in \mb{N}$, $\Theta_1,\cdots,\Theta_t\in\mb{R}$ mit
  \[SAS^{-1}=\Mx{E_r&&&&0\\ &-E_s&&&\\&&D_{\Theta_1}&&\\&&&\ddots&\\0&&&&D_{\Theta_t}}\]
  wobei
  \[D_\Theta:=\Mx{\cos\Theta&-\sin \Theta\\ \sin\Theta&\cos\Theta}\]
\end{Kor}
\begin{Bsp}
  \[A:=\Mx{0&1&0&&0\\&0&1&&\\&&\ddots&\ddots&\\&&&0&1\\ 1&&&&0 }\in U(n)\]
  \begin{align*}
    A(z_1,\cdots,z_n)=(z_2,\cdots,z_n,z_1)\\
    A(1,S,S^2,\cdots,S^{n-1})=(S,S^2,\cdots,S^{n-1},1)\\
    S:=e^{2\pi i/n}\ S^n=1
  \end{align*}
  $\implies$ $(1,S,S^2,\cdots,S^{n-1})$ ist Eigenvektor zum Eigenwert $S$. Ähnlich: für $0\leq j \leq n-1$ haben wir $(1,S^j,S^{2j},\cdots,S^{(n-1)j})$ ist Eigenvektor zum Eigenwert $S^j$. $1,S,S^2,\cdots,S^{n-1}$ sind paarweise verschieden $\implies$ $(1,S^j,S^{2j},\cdots,S^{(n-1)j})$ ist eine Basis von Eigenvektoren. Normalisierung:
  \[\left( \frac{1}{\sqrt{n}}\left( 1,S^j,S^{2j},\cdots,S^{(n-1)j} \right) \right)_{j=0,1,\cdots,n-1}\]
  ist eine orthonormale Basis von Eigenvektoren
\end{Bsp}
\begin{Faz}
  \begin{tabular}[htb]{l}
    $(K=\mb{C})$ unitärer Endomorpismus von $V$\\
    $(K=\mb{R})$ orthogonaler Endormophismus von $V$\\
    $\implies$ $V=\bigoplus_{\text{Eigenwerte} \lambda} \eig (F;\lambda)$ orthogonale direkte Summe
  \end{tabular}
\end{Faz}
\begin{Bsp}
  \[A=\Mx{\frac{3}{13}& \frac{4}{5}&\frac{36}{65}\\\frac{4}{13}&-\frac{3}{5}&\frac{48}{65}\\\frac{12}{13}&0&-\frac{5}{13}}\in O(3)\]
  $\det A=1$ 2 komplex konjugierte + 1 reller oder 3 reelle Eigenwerte $\implies$ $+1$ ist ein Eigenwert. \ldots $\rsa$ Eigenvektor $(6,3,4)$ zum Eigenwert 1. $\to$ v mit $\Norm{v}=1$ $v=\frac{1}{\sqrt{61}}(6,3,4)$ | $v^\perp=\Span\left( (1,-2,0),(2,0,-3) \right)$ $\xrightarrow{\text{Gram-Schmidt}}$ \[(1,-2,0),(\frac{8}{5},\frac{4}{5},-3)\] Normalisieren:
  \[\frac{1}{\sqrt{5}}(1,-2,0), \sqrt{1}{\sqrt{305}}(8,4,-15)\]
  Und wir berechnen
  \[S:=\Mx{\frac{6}{\sqrt{61}}&\frac{1}{\sqrt{5}}&\frac{8}{\sqrt{305}}\\\frac{3}{\sqrt{61}}&-\frac{2}{\sqrt{5}}&\frac{4}{\sqrt{305}}\\\frac{4}{\sqrt{61}}&0&-\frac{15}{\sqrt{305}}}\]
  bekommen wir 
  \[\underbrace{S^{-1}}_{=S^t}AS=\Mx{1&0&0\\0&-\frac{57}{65}&\frac{4\sqrt{61}}{65}&\\0&-\frac{4\sqrt{61}}{65}&-\frac{57}{65}}\]
\end{Bsp}
\subsection{Beschreibung von $SO(3)$ und $O(3)$}
\begin{Eig}
  Sei $A\in SO(3)$. Dann: entweder es gibt 1 reelle und 2 komplex konjugierte Eigenwerte oder 3 reelle Eigenwerte. $\lambda\in \mb{C}$ $\implies$ $\lambda\cdot\bar\lambda=1$. Eigenwerte $+1 (\times 3)$ $\Lra$ $A=E_3$ oder $-1(\times 2)$ / $+1$. Wenn $\not\Lra$ $A=E^3$, dann ist $\dim\eig(A,1)=1$.\\
  \[A:\eig(A,1)^\perp \to\eig (A,1)^\perp\]
  ist eine Drehung durch einen Winkel $\Theta\in (0,2\phi)$. Bezüglich Basis $(v_1,v_2,v_3)$, $v_1\in\eig(A,1)$, $\Norm{v_1}=1$ sieht $A$ aus wie
  \[\Mx{1&0&0\\0&\cos\Theta&-\sin\Theta\\0&\sin\Theta&\cos\Theta}\]
\end{Eig}
\begin{Eig}
  Sei $A\in O(3)$
  Falls $\det A=1$, haben wir $A\in SO(3)$\\
  Falls $\det A=-1$, haben wir $-A\in SO(3)$\\
  Dann bekommen wir die folgende Beschreibung von $A\in O(3)$ mit $\det A=-1$:
  \begin{itemize}
    \item $A=-E_3$
    \item oder $\dim \eig(A,-1)=1$\\
      $v_1\in\eig(A,-1)$, $\Norm{v_1}=1$\\
      $A:\eig(A,-1)^\perp\to\eig(A,-1)^\perp$ ist eine Drehung um den Winkel $\Theta-\pi\in(-\pi,\pi)$ (Spiegelung oder Spiegelung mit Drehung)
  \end{itemize}
\end{Eig}

\subsection{Selbstadjugierte Endomorphismen}
$V$,$\left\langle , \right\rangle$, $K$-Vektorraum mit Skalaprodukt. ($K$=$\mb{R}$ oder $\mb{C}$). Ist $F:V\to V$ ein Endomorphismus, so heisst $F^*:V\to V$ \underline{adjugierter Endomorpismus} falls 
\[\left\langle F(v),w \right\rangle = \left\langle v,F^*(w) \right\rangle\ \forall v,w\in V\]
\begin{Def}
  $F:V\to V$ ist \underline{adjugiert} falls
  \[\left\langle F(v),w \right\rangle=\left\langle v,F(w) \right\rangle\ \forall v,w\in V\]
\end{Def}
\begin{Eig}
  Falls $V=\mb{R}^n$ mit Standardskalarprodukt, so zu $F$ ist eine assoziierte Matrix $A\in M(n\times n,\mb{R})$, dann ist $A^t zu F^*$ assoziiert.
  Falls $V=\mb{C}^n$, dann ist 
  \[F\lra A\in M(n\times n,\mb{C})\]
  \[F*\lra \bar{A}^t\in M(n\times n,\mb{C})\]
\end{Eig}
\begin{Bew}
  \[\left\langle Av,w \right\rangle=(Av)^t\bar{w}=v^tA^t\bar{w}=v^t\bar{A}^tw=\left\langle v,\bar{w}^tw \right\rangle\]
\end{Bew}
\begin{Bem}
  $F^*$ ist eindeutig falls für $\tilde F^*$ gilt
  \[\left\langle F(v),w \right\rangle=\left\langle v,\tilde F^*(w) \right\rangle\]
  dann ist
  \[0=\left\langle v,\tilde F^*(w)-F^*(w) \right\rangle\]
  $\implies$
  \begin{align*}
    0=\left\langle \tilde F^*(w)-F^*(w),\tilde F^*(w)-F^*(w) \right\rangle\\
    =\Norm{\tilde F^*(w)-F^*(w)}^2\\
    \implies \tilde F^*(w)=F^*(w)
  \end{align*}
\end{Bem}
\begin{Faz}
  Im Fall $V=\mb{R}^n$ bzw. $\mb{C}^n$ mit Standardskalarprodukt ist ein selbstadjungierter Endomorphismus durch eine symmetrische bzw. hermitische Matrix gegeben.
\end{Faz}
\begin{Lem}
  Jeder Eigenwert eines selbstadjugierten Endomorphismus ist reell.
\end{Lem}
\begin{Bew}
  Ist $F(v)=\lambda v$ mit $v\neq 0$, so gilt
  \[\lambda\left\langle v,v \right\rangle=\left\langle \lambda v,v \right\rangle=\left\langle F(v),v \right\rangle=\left\langle v,F(v) \right\rangle=\left\langle v,\lambda v \right\rangle=\bar \lambda\left\langle v,v \right\rangle \implies \lambda=\bar\lambda \]
\end{Bew}
\begin{Bem}{Prä-Hilbertraum}
  bezeichnet einen $K$-Vektorraum ($K$=$\mb{R}$ oder $\mb{C}$) mit Skalarprodukt. Euklidische bzw. unitäre Vektorräume sind endlichdimensional,
\end{Bem}
\begin{Prop}
  Sei $V$ ein euklidischer bzw. unitärer Vektorraum und $F:V\to V$ ein selbstadjugierter Endomorphismus. Dann gibt es eine orthonormale Basis von Eigenvektoren.
\end{Prop}
\begin{Bew}
  Falls $V$ ein unitärer Vektorraum ist: durch Induktion nach $\dim V$, $\exists$ Eigenwert $\lambda$, Eigenvektor $v$, oBdA haben wir $\Norm{v}=1$. Wir behaupten:
  \[F(v^\perp)\in V^\perp\]
  \[\left\langle v,w \right\rangle =0\implies \left\langle v,F(w) \right\rangle =\left\langle F(v),w \right\rangle =\left\langle \lambda v,w \right\rangle =\lambda\left\langle v,w \right\rangle =0\]
  IA$\implies$ $\exists$ orthonormale Basis von $v^\perp$. Dies, zusammen mit $v$, gibt eine Basis von $V$. Fall eines euklidischen Vektorraums: Das gleiche Argumente ist gültig, sobald wir wissen, dass $F$ einen Eigenwert besitzt. Man wählt eine Basis von $V$, so:
  \[F\lra A\in M(n\times n,\mb{R})\ \ [n=\dim V]\]
  mit $A=A^t$. Wir betrachten $A$ als komplexe Matrix, so dass 
  \[A=\bar A\implies \bar A^t=A^t=A \implies A\ \text{ist hermetisch}\]
  Sei $\lambda$ ein (komplexer) Eigenwert von $A$. Weil $A$ hermetisch ist, haben wir $\lambda\in \mb{R}$. Wir haben 
  \[\det(A-\lambda E_n)=0\]
  Dann:
  \[\det(F-\lambda id_V)=0\]
  also $\lambda$ ist Eigenwert von $F$.
\end{Bew}
\begin{Kor}
  Sei $A\in M(n\times n,\mb{R})$ symmetrisch. Dann $\exists$ $S\in O(n)$ mit
  \[S^tAS=\diag (\lambda_1,\cdots,\lambda_n),\ \lambda_1,\cdots,\lambda_n\in\mb{R}\]
  Sei $A\in M(n\times n,\mb{C})$ hermetisch. Dann $\exists$ $S\in U(n)$ mit
  \[\bar S^tAS=\diag (\lambda_1,\cdots,\lambda_n),\ \lambda_1,\cdots,\lambda_n\in\mb{R}\]
\end{Kor}
\begin{Kor}
  Sei $F:V\to V$ wie in der Proposition oben. Dann ist $V$ die orthogonale direkte Summe von diesen Eigenräumen:
  \[V=\bigoplus_{\text{Eigenwerte} \lambda}\eig(F;\lambda)\]
\end{Kor}
\begin{Faz}
  $\rsa$ Praktisches Verfahren: $A$ symmetrisch bzw. hermetische Matrix\\
  $\hookrightarrow$ berechnen $\eig(A;\lambda)$\\
  $\hookrightarrow$ wählen von jedem eine orthonormale Basis
\end{Faz}
\begin{Bsp}
  \begin{align*}
    A=\Mx{5&3&3+3i\\3&5&-3-3i\\3-3i&-3+3i&2}\\
    P_A(t)=\det(tE_3-A)=(t-5)^2(t-2)+\cdots = t^3-12t^2+256=(t+4)(t-8)^2\\
    \eig(A;-4)=\Ker\Mx{9&3&3+3i\\3&9&-3-3i\\3-3i&-3+3i&6}=\Span \left\{ \Mx{1\\-1\\-1+i} \right\} \\
    \eig(A;\delta)=\Ker\Mx{-3&3&3+3i\\3&-3&-3+3i\\3-3i&-3+3i&-6}=\Span\left\{ \Mx{1\\1\\0} ,\Mx{2\\0\\1-i} \right\} \rsa \Mx{1\\1\\0},\Mx{1\\-1\\1-i}
  \end{align*}
  bzw.
  \[\Mx{\frac{\sqrt{2}}{2}&\frac{\sqrt{2}}{2}&0},\Mx{\frac{1}{2}&-\frac{1}{2}&\frac{1-i}{2}}\]
  Wir bekommen:
  \[S:=\Mx{\frac{1}{2}&\frac{\sqrt{2}}{2}&\frac{1}{2}\\ -\frac{1}{2}&\frac{\sqrt{2}}{2}&-\frac{1}{2}\\ \frac{-1+i}{2}&0&\frac{1-i}{2}}\]
  dan:
  \[\bar S^tAS=\diag(-4,8,8)\]
\end{Bsp}
\begin{Bem}
  Das Resultat von der Proposition oben im Fall eines euklidischen Vektorraums ist klar, auch aus geometrischem Grund.
  \[\text{symm. Matrizen}\setminus \mb{R}\ \rsa\ \text{quadratische Formen}\]
  (Prop aktuelle-7) $S^tAS$ aus der Transformationsformel.\\
  \ldots und man kann auch einen alternativen Beweis in diesem Fall geben.
  \[A\in M(n\times n,\mb{R}), A^t=A\ \rsa\ q:\mb{R}^n\to\mb{R},q(v):=v^tAv\]
  (Faktum aus der Analysis)
  \[\exists x\in\mb{R}^n, \Norm{x}=1\ \text{mit} q(x)\geq g(x')\ \forall x'\in\mb{R}n^n, \Norm{x'}=1\]
  Dann für $v\in\mb{R}^n, v\perp x$ haben wir $Av\perp x$. In der Tat haben wir 
  \[(Av-q(x)v)\perp x\ \forall v\in\mb{R}^n\]
  denn
  \[\left\langle Av-q(x)v, v\right\rangle +2\lambda\left\langle Av-q(x)v,x \right\rangle =(v+\lambda x)^t(A-q(x)E_n)(v+\lambda x)\leq 0\ \forall \lambda\in\mb{R}\]
  (Details im Buch, 5.6.4)
\end{Bem}

\begin{Bem}
  $F$ selbstadjugiert, $\dim V <\infty$ $\implies$ $\exists$ orthonromale Basis von Eigenvektoren $\implies$
  \[V=\bigoplus_\lambda \eig(F;\lambda)\]
  orthogonale direkte Summe $\rsa$ orthogonalte Projektion \[P_\lambda V\to \eig (F,\lambda)\] Dann können wir schreiben
  \[F=\sum_{\text{Eigenwerte} \lambda} \lambda P_\lambda\]
  \[=\left\{ \text{Eigenwerte von } F \right\}=\text{''Spektrum''}\]
  Geschrieben mit Matrizen:
  \[A\in \Mat{R} \text{ symmetrisch} \implies \exists S\in O(n)\]
  so dass $S^{-1}AS$ eine Diagonalmatrix ist.\\
  Interpretation: der zu $A$ assoziierte Endomorphismus ist Diagonalisierbar.\\
  $S^tAS$ ist eine Diagonalmatrix\\
  Interpretation: $A\ \lra\ s:\mb{R}^n\times\mb{R}^n\to\mb{R}$ Bilinearform. 
  \begin{align*}
    \diag(\lambda_1,\cdots,\lambda_n)\ \lra\ (x,y)\mapsto& x^t\diag(\lambda_1,\cdots,\lambda_n)y\\
    \Mx{x_1\\\vdots\\x_n}\Mx{y_1\\\vdots\\y_n}\mapsto&\sum_{i=1}^n\lambda_ix_iy_i
  \end{align*}
  \paragraph{Fragen}
  \begin{itemize}
    \item Zu einer symmetrischen Bilinearform gibt es eine bestimmte Normalform?
    \item Wie kann man das praktisch berechnen?
  \end{itemize}
\end{Bem}
\begin{Prop}{Hauptachsentransofrmation symmetrischer Matrizen}
  Sei $A\in M(n\times n,\mb{R})$ symmetrisch und $s:\mb{R}^n\times\mb{R}^n\to\mb{R}$ die entsprechende symmetrische Bilinearform. Dann:
  \begin{enumerate}
    \item Ist $B=(w_1,\cdots,w_n)$ eine orthonormale Basis von Eigenvektoren von $A$, so ist $M_B(s)=\diag(\lambda_1,\cdots,\lambda)$ wobei $\lambda_1,\cdots,\lambda_n)$ die Eigenwerte von $A$ sind.
    \item Es gibt eine Basis $B'$ mit
      \[M_{B'}(s)=\Mx{E_k & &\\ & -E_l&\\& & 0}\]
      Blockdiagonalmatrix, wobei
      \begin{align*}
        k=\#\left\{ i|\lambda_i>0 \right\}&&\\
        l=\#\left\{ i|\lambda_i<0 \right\}&&
      \end{align*}
  \end{enumerate}
\end{Prop}
\begin{Bew}
  \begin{enumerate}
    \item $\Lra$ $\exists S\in O(n)$ mit $S^tAS=\diag(\lambda_1,\cdots,\lambda_n)$
    \item $\Lra$ $\exists T\in GL_n(\mb{R})$ mit $T^tAT=\Mx{E_k & &\\ & -E_l&\\& & 0}$
  \end{enumerate}
  oBdA habe wir 
  \begin{align*}
    \lambda_1,\cdots,\lambda_k > 0&&\\
    \lambda_{k+1},\cdots,\lambda_{k+l}<0&&\\
    \lambda_{k+l+1}=\cdots=\lambda_n=0&&    
  \end{align*}
  Wir nehmen $B'=\left( w_1',\cdots,w_n' \right)$ mit
  \[w_i'=\begin{cases}
    \frac{w_i}{\sqrt{\Abs{\lambda_i}}}&i\leq k+l\\
    w_i, i>l+l
  \end{cases}\]
  \begin{gather*}
    (w_i')^tAw_i'=\frac{1}{\Abs{\lambda_i}}w_i^tAw_i=\frac{1}{\Abs{\lambda_i}}\lambda_i\ \text{für } i\leq k+l
  \end{gather*}
\end{Bew}
\begin{Bem}
  \begin{gather*}
    T^tAT=\underbrace{\Mx{E_k&&\\&-E_n&\\&&0}}_{\text{Sylvester-Form}}
  \end{gather*}
  \ldots Erklärung zum Namen ``Hauptachsentransformation''\ldots
\end{Bem}
\begin{Kor}
  Sei $s:\mb{R}^n\times\mb{R}^n\to\mb{R}$ eine symmetrische Bilinearform mit entsprechender Matrix $A$. Die folgenden Aussagen sind äquivalent:
  \begin{enumerate}
    \item $s$ ist positiv definit
    \item Alle Eigenwerte von $A$ sind positiv
    \item Die Koeffizienten des charakteristischen Polynoms haben alternierende Vorzeichen
  \end{enumerate}
  Vorzeichenregel von Descartes
\end{Kor}
\begin{Bsp}
  \[A=\Mx{3&1&2\\1&3&-1\\2&-1&2}\]
  \[P_A(t)=\det(tE_3-A)=t^3-ut^2+15t\underbrace{+}3\]
  \[P_A(-1)=-21\ P_a(0)=3 \implies \exists \lambda: -1<\lambda<0\]
\end{Bsp}
\begin{Bew}
  $s$ ist äquivalent zu 
  \[(x,y)\mapsto\sum^n_{i=1}\lambda x_iy_i\]
  $\implies$ $s$ positiv definit $\Lra$ $\lambda_i >0$ $\forall i$
\end{Bew}
\begin{Bem}{Weitere Begriffe}
  $s:V\times V\to\mb{R}$ symmetrische Bilinearform
 \begin{table}[htb]
   \centering
   \begin{tabular}{cc}
     positiv definit & positiv semidefinit \\
     negativ definit & negativ semidefinit \\
     indefinit: & $\exists x\in V: s(x,x)> 0$ und $y\in V: s(y,y)<0$
   \end{tabular}
   \caption{Weitere Begriffe}
 \end{table}
\end{Bem}
\begin{Bem}{Ausartungsraum}
  Ausartungsraum von einer Bilinearform $s:V\times V\to K$ auf einem Vektorraum über einem beliebigen Körper $K$ ist:  
  \[U:=\left\{ v\in V|s(v,w)=0\ \forall w\in V \right\}\]
  und ist ein Untervektorraum. Falls $s$ symmetrisch oder schiefsymmetrisch ist, bekommen wir eine induzierte Bilinearform $\bar s:V/U\times V/U\to K$, gegeben durch
  \[v+U,w+U)\mapsto s(v,w)\]
  und $\bar s$ ist nicht ausgeartet.
  \begin{align*}
    v'=v+u, &\ u\in U\\
    w'=w+\tilde u, &\ \tilde u \in U\\
  \end{align*}
  \begin{gather*}
    s(v',v')=s(v,w)+s(u,w)+s(v,\tilde u)+s(u, \tilde u)=\\
    =s(v,w)+s\underbrace{(u,w)}_{=0} \pm \underbrace{s(\tilde u,v)}_{=0}+s\underbrace{(u,\tilde u)}_{=0}
  \end{gather*}
  \\
  \begin{gather*}
    s(v,w)=0 \implies v\in U \implies v+U
  \end{gather*}
  ist Nullvektor von $V/U$
\end{Bem}
\begin{Kor}
  Sei $n\in\mb{N}_{>0}$ und $s:\mb{R}^n\times\mb{R}^n\to\mb{R}$ eine symmetrische Bilinearform. dann gibt es eine orthogonale Zerlegung
  \[\mb{R}^n=W_+\oplus W_i\oplus W_0\]
  mit
  \[s|_{W_+}>0,\ s|_{W_-}<0\]
  und $W_0=$ Ausartungsraum von $s$
\end{Kor}

\begin{Prop}{Trägheitsgesetz/Signatur von Sylvester}
  Sei $V$ ein endlichdimensionaler reeller Vektorraum und $s:V\times V\to\mb{R}$ eine symmetrische Bilinearform. Sei
  \[V=V_+\oplus V_-\oplus V_0\]
  eine Zerlegung als orthogonale direkte Summe, mit $s|_{V_+}>0$, $s|_{V_-}<0$ und $V_0$=Ausartungsraum von $s$. Dann sind
  \[r_+:=\dim(V_+),\ r_-=\dim(V_-)\ und\ r_0:=\dim(V_0)\]
  Invarianten von $s$, charakterisiert durch
  \[r_+=\max\left\{ \dim W| W\subset V\ \text{Untervektorraum},\ s|_W>0 \right\}\]
  \[r_-=\max\left\{ \dim W| W\subset V\ \text{Untervektorraum},\ s|_W<0 \right\}\]
  Die Invarianten $(r_+,r_-,r_0)$ heisst Trägheitsindex oder Signatur von $s$
\end{Prop}
\begin{Bem}
  Ist $A$ eine $n\times n$ symmetrische reelle Matrix, heisst Signatur die Signatur von der zu $A$ entsprechender Biliniearform.
\end{Bem}
\begin{Bem}
  Auch $r_+-r_-$ heisst Signatur.
  \begin{table*}[htb]
    \centering
    \begin{tabular}{ccc}
      Dimension & $\dim V=r_++r_-+r_0$\\
      Rang & $r_++r_-$ & $\lra$ $(r_+,r_-,r_0)$\\
      Signatur in diesen Sinn & $r_+-r_-$
    \end{tabular}
  \end{table*}
\end{Bem}
\begin{Bew}
  Reduktionsschritt: Es genüngt, das Resultat zu beweisen, im Fall dass $s$ nicht ausgeartete ist.
  \[V\to\bar V=V/V_0\]
  \[V=V_+\oplus V_- \oplus V_0\]
  \[\bar V=\bar V_+\oplus \bar V_-\]
  wobei $\bar V_\pm$ = Bild von $V_\pm$. Bew. $\bar V=\bar V_++\bar V_-$ direkte Summe 
  \[\Lra\ \bar V_+\cap \bar V_-=0\]
  \[\bar v\ \lra\ v\in V_+\oplus V_0\]
  und
  \[v\in V_-\oplus V_0\]
  $\Lra v\in V_0$
  Behauptung:
  $\bar s$ induzierte Bilinearform auf $\bar V$
  \[\max \left\{ \dim W| s|_W>0 \right\} = \max \left\{ \dim U | U\subset \bar V, \bar s|_U>0 \right\}\]
  und
  \[\max \left\{ \dim W| s|_W<0 \right\} = \max \left\{ \dim U | U\subset \bar V, \bar s|_U<0 \right\}\]
  Ist $W\subset V, s|_W>0$, und $\bar W:=$ Bild von $W$, so haben wir
  \[\dim\bar W=\dim W\]
  und
  \[\bar s|\bar W>0\]
  Dimensionsformel: 
  \[\dim \bar W=\dim W-\dim(\underbrace{W\cap V_0}_{=0})=\dim W\]
  und
  \[\bar s(\bar v,\bar v)=s(v,v)\]
  wobei $v\in W \mapsto v\in \bar W$
  Umgekehrt ist
  \[U\subset\bar V,\ \bar s|_U>0,\ \dim U=d\]
  wählen Basis $(\bar v_1,\dots,\bar v_d)$ von $\bar U$, mit $v_i\mapsto\bar v_i$ $\forall i$ dann haben wir $W:=\Span(v_1,\dots,v_d)$ hat die Eigenschaft
  \[\dim W=d\ s|_W>0,\ \Im(W)=U\]

  Beweis im Fall $s$ nicht ausgeartet:\\
  Behauptung: Ist
  \[W_+\subset V,s|_{W_+}>0,\ W_-\subset V,s|_{W_-}<0\]
  so haben wir
  \[W_+\cap W_-=0\]
  Es folgt:
  \[\dim W_- + \dim W_+ \leq \dim V\]
  mit Gleicheit $\Lra$ $V=W_+\oplus W_-$\\
  Deshalb
  \[r_++r_- \leq \dim V\]
  Und wir haben $=$ aus dem Korollar\\
  (alternativer Beweis ohne Quotientenvektorräume sehe Buch)\\
\end{Bew}
\begin{Bem}
  Praktische Fragen:
  \begin{itemize}
    \item Wie berechnet man die Signatur einer symmetrischen Bilinearform?
    \item Wie findet man eine Basis, so dass die darstellden Matrix in Sylversterform ist?
  \end{itemize}
  In Matrixen: $A\in\Mat{R}$ symm.
  \begin{itemize}
    \item Signatur?
    \item Finden $T\in GL_n(\mb{R})$ mit $T^tAT$ in Sylvesterform
  \end{itemize}
  Antwort:\\
  Aus der Hauptachsentrasformation:
  \[\exists\ S\in O(n),\ S^tAS=S^{-1}AS=\diag(\lambda,\dots,\lambda_n)\]
  $\implies$ Signatur
  \[r_+=\#\left\{ i|\lambda_i>0 \right\}\]
  \[r_-=\#\left\{ i|\lambda_i<0 \right\}\]
  \[r_0=\dim\Ker(A)\]
  \[S\stackrel{\text{Normieren der Spaltenvektoren}}{\rsa}S'\]
  mit $S'^tAS$ in Sylvesterform.\\
  Alternatives, oft leicheres Verfahren:
  \begin{itemize}
    \item $\Ker(A)$ = Ausartungsraum berechnen
    \item Vektoren Wählen, wobei $q(v)=s(v,v)$ verschieden von Null ist. $\rsa q(v)\in\left\{ \pm 1 \right\}$ $\rsa v^\perp$
  \end{itemize}
\end{Bem}
\begin{Bsp}{Silvesterform}
  \begin{gather*}
    A=\Mx{5&2&3\\ 2&1&1\\ 3&1&2}\\
    P_A(t)=t^3-8t^2+3t=t\left(t-(4+\sqrt{13})\right)\left( t-(4-\sqrt{13} \right)\\
  \end{gather*}
  Signaturen $(2,0,1)$
  Mit Halbachsentransformation\\
  \begin{tabular}{ccc}
    Eigenwert 0&$\rsa$&Eigenvektor $(1,-1,1)$\\
    Eigenwert $4+\sqrt{13}$ & $\rsa$ & Eigenvektor $(1,4-\sqrt{13},-3+\sqrt{13})$\\
    Eigenwert $4-\sqrt{13}$ & $\rsa$ & Eigenvektor $(1,4+\sqrt{13},-3-\sqrt{13})$
  \end{tabular}\\
  Normieren\dots\\
  $S'$ ausrechnen\dots (ne danke)\\
  haben wir 
  \[S'^t=\Mx{1&&\\&1&\\&&0}\]
\end{Bsp}
\begin{Bsp}{Alternativ}
  \begin{gather*}
    e_2:\ q(e_2)=e^t_2Ae_2=1\\
    e_2^\perp = \left\{ (x,y,z)|2x+y+z=0 \right\}\\
    (-1,2,0)A\Mx{-1\\2\\0}=1\\
    \Ker(A)=\Span\Mx{1&-1&-1}\\
    T:=\Mx{0&-1&1\\1&2&-1\\0&0&-1}
  \end{gather*}
  mit $T$ haben wir $T^tAT=\Mx{1&&\\&1&\\&&0}$
\end{Bsp}
\section{Klassifikation von Bilinearformen auf $\mb{R}^n$ $\lra$ Signatur}

Seien euklidische $(V,\left\langle , \right\rangle )$ und symmetrische Bilinearform $s:V\times V\to\mb{R}$, so können wir die Bilinearform durch die Hauptachsentransormation verstehen.
Seien ein endlichdimensionaler $\mb{R}$-Vektorraum $V$ und die symmetrische Bilinearform $s:V\times V\to\mb{R}$, dann ist $s$ durch die Signatur $(r_+,r_-,r_0)$ klassifiziert.
\begin{Eig}
  $\dim V=2$
  \begin{tabular}[htbp]{rcc}
    Signatur & $(2,0,0)$ & $q(v):=s(v,v)$ Quadratische Schale (positiv)\\
    & $(0,2,0)$ & Quadratische Schale (negativ)\\
    & $(1,1,0)$ & Sattelpunkt\\
    & $(0,2,0)$ & Quadratisches halbes Rohr (positiv)\\
    & $(1,0,1)$ & Quadratisches halbes Rohr (negativ)
  \end{tabular}
\end{Eig}
\begin{Eig}
  $\dim V=3$ $\left\{ q(v)=1 \right\}$
  \begin{tabular}[htbp]{cc}
    (3,0,0) & Sphäre \\
    (2,1,0) & einschaliges Hyperboloid \\
    (1,2,0) & zweischaliges Hyperboloid \\
    (0,3,0) & $\varnothing$
  \end{tabular}
  + Fälle $s$ entartet
\end{Eig}
Der Fall von Bilinearformen über Vektorräumen über $\mb{K}$, $\mb{K}$ beliebiger Körper.
\begin{Prop}{Orthogonalisierungssatz}
  Sei $V$ ein endlichdimensionaler Vektorraum über einem Körper mit $\text{char} (K)\neq 2$. Sei $s$ eine symmetrische Bilinearform über $V$. Dann gibt es eine Basis $B$ von $V$, so dass die $M_B(s)$ eine Diagonalmatrix ist.
\end{Prop}
\begin{Bew}
  Reduktionsschritt zum Fall $s$ nicht ausgeartet. Sei $U=$ Ausartungsraum. $\bar V:=V/U$ und $\bar s:=$ induzierte Bilinearform. Wählen wir ein Komplement $W\subset V$ zu $U$, so haben wir
  \[V=U\oplus W\]
  \[W\xrightarrow{\text{Isomorphismus}}\bar V\]
  \[s|_W \ \text{nicht ausgeartet}\]
  Wir können deshalb behaupten, dass $s$ nicht ausgeartet ist. Dann beweisen wir dies Aussage durch Induktion nach $\dim V$. $\dim V\leq 1$ trivial. Induktionsschritt:
  \[s\ \text{nicht ausgeartet} \xRightarrow{\dim (\mb{K})\neq 2}\ \exists v\in V: s(v,v)\neq 0\]
  Sei $V':=V^\perp$ Wir haben $\dim V'=\dim V-1$, weil $s$ nicht ausgeartet ist.\\
  IA $\rsa$ Basis $B'$ von $V'$ mit $\underbrace{M_B(s|_{V'})}_{\text{auch nicht ausgeartet}}$ diagonal.
  Dann:
  \[B\& v \rsa B \text{ mit } M_B(s) \text{ eine Diagonalmatrix}\]
\end{Bew}
\begin{Kor}
  Ist $\text{char } K\neq 2$, so gibt es zu einer symmetrischen Matrix $A\in\Mat{K}$ ein $S\in GL_n(\mb{K})$ so dass $S^tAS$ eine Diagonalmatrix ist.
\end{Kor}
\begin{Bsp}
  $\mb{K}$ beliebig, $\text{chat}\ (K)\neq 2$ $V=K^2$
  \begin{gather*}
    s(x,y)=x_1y_2+x_2y_1\\
    v_1=\Mx{1\\1} \\ s(v_1,v_1)=2\\
    v_1^\perp = \Span\Mx{1\\-1}\\ v_2=\Mx{1\\-1}\\
    s(v_2,v_2)=-2\\
    B:=(v_1,v_2)\\ M_B(s)=\Mx{2&0\\0&-2}
  \end{gather*}
  $S\ \lra\ \Mx{0&1\\1&0}$ Standardbasis. Mit $S:=\Mx{1&1\\1&-1}$ haben wir
  \[S^tAS=\Mx{1&1\\1&-1}\Mx{0&1\\1&0}\Mx{1&1\\1&-1}=\Mx{2&0\\0&-2}\]
\end{Bsp}
\begin{Bem}
  offen bleibt die Frage: Sind symmetrische Bilinearformen $s$ und $s'$ auf $V$ gegeben $(\dim_\mb{K}<\infty)$, können wir entscheiden ob $s$ und $s'$ äquivalent sind?\\
  Oder, in Matrizen: Sind symmetrische $A,A'\in \Mat{K}$ gegeben, können wir entscheiden, obes ein $S\in GL_N(\mb{K})$ gibt, so dass $S^tAS=A'$?\\
  Die Antwort hängt von $\mb{K}$ ab.
  \begin{itemize}
    \item $\mb{K}=\mb{R}$ durch die Signatur
    \item $\mb{K}=\mb{C}$ durch die Rang
    \item andere $\mb{K}$?
  \end{itemize}
  Im Allgemeinen:
  \begin{itemize}
    \item Rang
    \item Reduktion zum Fall einer nichtausgearteten Form
  \end{itemize}
  Wir behaupten: $s$ ist nicht ausgeartet $\Lra$ eine darstellende Matrix $A$ ist invertierbar.
  \[\det(A)\in \mb{K}^*/( \mb{K}^*)^2\]
  ist eine Invariante von $s$, wegen der Transformationsform.
  \[T\in GL_n(\mb{K})\ \rsa\ T^tAT\]
  ist eine andere darstellende Matrix. Und
  \[\det(T^tAT)=\det(T^t)\det(A)\det(T)=(\det T)^2\det(A)\]
\end{Bem}
\begin{Def}{Diskriminante}
  Sei $s:V\times V\to\mb{K}$ eine symmetrische Bilinearform (mit $\dim_\mb{K}V<\infty$). Die Diskiminante von $s$ ist 0 falls $s$ ausgeartet ist, sonst ist die Klasse ovn $\det(A)$ in $\mb{K}^*/( \mb{K}^*)^2$, wobei $A$ eine darstellende Matrix von $s$ ist. Die Diskriminante ist eine Invariante von $s$
  \begin{itemize}
    \item Rang
    \item Diskriminante
  \end{itemize}
\end{Def}
\begin{Bem}
  Noch offen: sind $s$,$s'$ nicht ausgeartet, mit derselben Diskriminante, zu entscheiden, ob $s$ und $s'$ äquivalent sind.
\end{Bem}
\begin{Bsp}
  $\mb{K}=\mb{Q}$, z.B. $V=\mb{Q}^2$, $s$ Standardskalarprodukt $s(x,y)=x_1,x_1+x_2y_2$ und $s'$ symmetrische Bilinearform mit $\disc(s)=-1$
  \[\Mx{a&0\\0&a'}\stackrel{\text{Basiswechsel}}{\rsa}\Mx{a&0\\0&a}\]
  mit $aa'=b^2, b\in \mb{Q}$
  \[\implies a=\frac{b^2}{a'}=a'\left( \frac{b}{a'} \right)^2\]
  $a=2$
  \[\Mx{\frac{1}{2}&\frac{1}{2}\\\frac{1}{2}&-\frac{1}{2}}\Mx{2&0\\0&2}\Mx{\frac{1}{2}&\frac{1}{2}\\\frac{1}{2}&-\frac{1}{2}}=\Mx{1&0\\0&1}\]
  $a=3$
  \[s'\ \lra\ q'(x)=3x^2_1+3x^2_2\]
  Beh: \[q'(x)\neq 1\ \forall x\in \mb{Q}^2\]
  Konsequenz: $s'$ ist nicht äquivalent zu $s$. Ist $3x^2_1+3x^2_2=1$ so schreiben wir $x_1=\frac{r_1}{s_1},x_2?\frac{r_2}{s_2}$, $r_1,r_2,s_1,s_2\in\mb{Q}$, $s_1,s_2\neq 0$
  \begin{gather*}
    3r_1^2s_2^2+3r_2^2s_1^2=s^2_1s^2_2
  \end{gather*}
  \begin{equation}
    \text{oder }3r^2+3s^2=t^2\ \text{wobei}\ r=r_1s_2,\ s=r_2s_1,\ \overbrace{t}^{\neq 0}=s_1s_2
    \label{bil:1}
  \end{equation}
    \begin{gather*}
    3^\text{ungerade}(3k_1+1)+3^\text{ungerade}(3k_2+1)\\
    \implies \text{Widerspruch zu } (\ref{bil:1})
  \end{gather*}
\end{Bsp}

\begin{Faz}
  $\mb{K}$: Körper, $\text{char}(\mb{K})\neq 2$\\
  $V$:endlichdimensionaler $\mb{K}$-Vektorraum\\
  $s$: $V\times V\to\mb{K}$ symmetrische Bilinearform
  \begin{description}
    \item[Rang] Rang < $\dim V$ $\Lra$ $s$ ist ausgeartet. $U$:=Ausartungsraum. $\bar s$ induzierte Bilinearform auf $\bar V:=V/U$ (nicht ausgeartet)
    \item[Diskriminiante] für $s$ nicht ausgeartet: $\disc(s)\in\mb{K}^*/(K^*)^2$
  \end{description}
\end{Faz}
\begin{Bsp}
  $\mb{K}=\mb{Q}$, $V=\mb{Q}^2$ Bilinear entsprechend zu $\Mx{1&0\\0&1}$ und $\Mx{3&0\\0&3}$
  \begin{itemize}
    \item Beide: Rang 2, Diskriminante 1
    \item nicht äquivalent
  \end{itemize}
\end{Bsp}
\begin{Bem}
  Die Frage, ob eine nicht ausgeartete symmetrische Bilinearform auf $V:=\mb{Q}^2$ der Diskriminante 1 äquivalent zum Standardskalarprodukt ist, können wir nur beantworten mittels einem Resultat aus der Zahlentheorie.
\end{Bem}
\begin{Sat}
  $s$ nicht ausgeartete symmetrische Bilinearform auf $\mb{Q}^2$ $\disc(s)=1$ $\implies$ $\exists B$ Basis mit
  \[M_B(s)=\Mx{a&0\\0&a},\ a\in\mb{Z},a\neq 0\]
  Dann: $s$ ist äquivalent zum Standardskalarprodukt $\Lra$
  \[\exists v\in V,\ s(v,v)=1\text{ d.h. }\exists x,y\in\mb{Q}:\ ax^2+ay^2=1\]
  $\Lra$
  \[\exists x,y\in\mb{Q}\text{ mit }x^2+y^2=a\]
\end{Sat}
\begin{Bem}
  Ein Resultat aus der Zahlentheorie gibt uns eine Charakterisierung von Summen zweier Quadrate in $\mb{Q}$: für $a\in\mb{Z},a\neq 0$:
  \[\exists x,y\in\mb{Q}:x^2+y^2=a\ \Lra x,y\in\mb{Z}:x^2+y^2=a\]
  $\Lra$ $a>0$ und jede Primzahl $p=4k+3$ $(k\in\mb{N})$ kommt mit gerader Vielfachheit in der Primzahlzerlegun von $a$ vorkommen. Der Beweis nutzt
  \[(x_1x_1'-x_2x_2')^2+(x_1x_2'+x_2x_1')^2=(x_1^2+x_2^2)(x_1'^2+x_2'^2)\]
  Satz von Fermat: $p$ Primzahl 
  \[\exists x,y\in\mb{Z},x^2+y^2=p\ \Lra\ p=2\vee 4|(p-1)\]
  Argument vom letzten Mal (auszuschliessen $a=3$)
\end{Bem}
\begin{Faz}
  Zurück zum Fall $\mb{K}=\mb{R}$\\
  Wir wissen: eine symmetrische Bilinearform $s:V\times V\to\mb{R}$ $(\dim_\mb{R}V<\infty)$ ist durch die Signatur $(r_+,r_-,r_0)$ charaktertisiert.
  \[A:M_B(s)\ P_A(t)=\prod^n_{i=1}(t-\lambda_i)\]
  \begin{gather*}
    r_+=\#\left\{ i|\lambda_i>0 \right\}\\
    r_-=\#\left\{ i|\lambda_i<0 \right\}\\ 
    r_0=\#\left\{ i|\lambda_i=0 \right\}
  \end{gather*}
  und
  \[s>0\ \Lra\ \text{Signatur } (n,0,0)\ \Lra\ \lambda_i>0\forall i\ \Lra\ \text{Koeff. }P_A(t)\ \text{hat alternierende Vorzeichen }\]
\end{Faz}
\begin{Def}{Hauptminor}
  Sei $A=(a_{ij})\in\Mat{K}$. Wir schreiben $A_k$ für die Teilmatrix $(A_{ij})_{1\leq i,j\leq k}$, für $1\leq k\leq n$. Der $k$-te Hauptminor von $A$ ist $\det (A_k)$
\end{Def}
\begin{Bem}
  $\dim_\mb{R}<\infty$, $s:V\times V\to\mb{R}$ symmetrische Bilinearform. Wann ist $s$ positiv? Es ist notwendig, aber nicht hinreichend, dass $\det(A)>0$ für eine darstellende Matrix $A$.
\end{Bem}
\begin{Prop}{Hauptminorenkriterium von Jacobi-Sylvester}
  Sei $V$ ein endlichdimensionaler reeller Vektorraum, $s:V\times V\to\mb{R}$ eine symmetrische Bilinearform und $A$ eine darstellende Matrix. Dann ist $s$ positiv definit $\Lra$ $\det(A_K)>0,\ k=1,\cdots,n$ $n=\dim V$
\end{Prop}
\begin{Bew}
  $\Ra$ Ist $s>0$, so ist $s|_W>0$ für alle Untervektorräume $S\subset V$. Sei $B=(v_1,\cdots,v_n)$ eine Basis mit $A=M_B(s)$. Sei $V_k:=\Span(v_1,\cdots,v_k)$ für $1\leq k\leq n$. Dann haben wir:
  \[M_{(v_1,\cdots,v_k)}\left( s|_{V_k} \right)=\left( s(v_i,v_j) \right)_{1\leq i,j\leq k}=\left( A_K \right)\]
  Weil $s|_{V_k}>0$, folgt: $\det(A_k)>0$.\\
  $\La$ Durch eine Induktion nach $n$:\\
  IA: $n=1$\\
  IS: Wir nehmen das Resultat an für einen Vektorraum der Dimension $n-1$. Seien $B=(v_1,\cdots,v_n)$ eine Basis von $V$ und $A=(a_{ij})_{i\leq i,j\leq n}$ $A=M_B(s)$. Wir haben
  \begin{itemize}
    \item aus $\det(A)>0$ folgt: $s$ ist nicht ausgeartet
    \item aus $\det(A_1)>0$ folgt $a_{11}>0$
  \end{itemize}
  Wir haben $V=\Span(v_1)\oplus V^\perp_1$. Es genügt zu zeigen, dass $s|_{v^\perp_1}$ positiv definit ist. Eine Basis von $v^\perp_1$ sieht so aus: Sei $c_i:=\frac{a_{1i}}{a_{11}}$ und $\tilde v_i:=v_i-c_iv_1$ für $i=2,\cdots,n$.
  \begin{gather*}
  s(v_1,\tilde v_1)=s(v_1,v_i)-c_is(v_1,v_1)=a_{1i}-c_ia_{11}=0\\
  (\tilde v_2,\cdots,\tilde v_n)\ \Lra\ \Mx{-c_2&1&&&\\-c_3&&1&&\\&&&\ddots&\\-c_n&&&&1}\\
  (\tilde a_{ij})_{2\leq i,j\leq n}\ \tilde a_{ij}=s(\tilde v_i,\tilde v_j)=a_{ij}-c_ia_{i1}+c_ic_ja_{11}=a_{ij}-c_ia_{1j}\\
  \text{weil }c_ic_ja_{11}=\frac{a_{1i}}{a_{11}}a_{1j}=c_ja_{i1}
  \end{gather*}
  Wir haben für $s\leq k\leq n$:
  \begin{gather*}
    \Mx{-c_2&1&&&\\-c_3&&1&&\\&&&\ddots&\\-c_k&&&&1}\Mx{a_{11}&a_{12}&\cdots&a_{1k}\\a_{21}&&&\\\vdots&&\ddots&\\a_{k1}&&&a_{kk}}=\Mx{a_{11}&a_{12}&\cdots&a_{1k}\\0&a_{22}-c_2a_{12}&\cdots&a_{2k}-c_2a_{1k}\\\vdots&\vdots&\vdots&\vdots\\0&a_{2k}-c_ka_{12}&\cdots&a_{kk}-c_ka_{1k}}\\
    = \Mx{a_{11}&a_{12}&\cdots&a_{1k}\\0&\tilde a_{22}&\cdots&\tilde a_{2k}\\\vdots&\vdots&\vdots&\vdots\\0&\tilde a_{k2}&\cdots&\tilde a_{kk}}\\
    \implies \det(A_k)=a_{11}\det(\tilde a_{ij})_{2\leq i,j\leq k}
  \end{gather*}
  Aus $\det(A_k)>0$ und $a_{11}>0$ folgt:
  \[\det(\tilde a_{ij})_{2\leq i,j\leq k}>0,\ \text{für}\ k=2,\cdots,n\]
  Aus der Induktionsvoraussetzung folgt $s|_{v_1^\perp}>0$
\end{Bew}


\newpage

%= Stichwortverzeichnis ======================================================================
\rhead{}
\addcontentsline{toc}{section}{Stichwortverzeichnis}
\printindex

\end{document}
