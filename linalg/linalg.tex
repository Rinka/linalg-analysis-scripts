
%= Document-Class ==================================================================================
\documentclass[10pt,a4paper]{article}

%= Packages ========================================================================================
\usepackage[utf8]{inputenc}
\usepackage{ngerman,amsmath,amssymb,amsfonts,mathrsfs}
\usepackage{amsthm}
\usepackage{bbm}
\usepackage{epic,eepic,pstricks,pst-node,pst-plot}
\usepackage{pstricks}
\usepackage{colortbl}
\usepackage{graphicx}
\usepackage{makeidx}
\usepackage{fancyhdr}
\usepackage{latexsym}
\usepackage{psfrag}
\usepackage{enumerate}
\usepackage{float}
%\usepackage{mathtext}
\usepackage[all, knot, poly]{xy}
\usepackage{dsfont}
\pagestyle{fancy}
\usepackage{multirow, bigdelim, bigstrut}
\usepackage{rotating}
\usepackage{ifthen}
\usepackage{boxedminipage}
\usepackage{chemarrow}
% \usepackage{mathtools}

%= Seiten-Layout =========================================================================
\voffset-22mm \textheight715pt 

%Seitenbreite==============================================================

%\oddsidemargin=0.4in
%\evensidemargin=0.2in
%\textwidth=5.2in
%\headwidth=5.2in

%= Index-Befehle ========================================================================
\renewcommand{\indexname}{Stichwortverzeichnis}
\makeindex

%= Befehl-Overwriting =======================================================================
\makeatletter
\makeatother

%= Strings ================================================================
\newcommand{\mainfold}{.}
\newcommand{\prefix}{A1-}

%= Eigene Befehle ==========================================================================
\DeclareMathOperator{\id}{Id}
\DeclareMathOperator{\arccot}{arccot}
\DeclareMathOperator{\arsinh}{arsinh}
\DeclareMathOperator{\arcosh}{arcosh}
\DeclareMathOperator{\artanh}{artanh}
\DeclareMathOperator{\md}{d}
\DeclareMathOperator{\Grad}{grad}

\newcommand{\Diff}[2]{\displaystyle\frac{\mathrm{d}#1}{\mathrm{d}#2}}
\newcommand{\End}{\hfill{\hbox{$\Box$}}\par\vspace{2mm}}
\newcommand{\eps}{\varepsilon}
\newcommand{\ePic}[1]{\input{\mainfold/graphics/\prefix#1.eepic}}
\newcommand{\pst}[1]{\input{\mainfold/graphics/\prefix#1.pst}}
\newcommand{\pic}[1]{\input{\mainfold/graphics/\prefix#1.pic}}
\newcommand{\Mx}[1]{\begin{pmatrix}#1\end{pmatrix}}
\newcommand{\im}[1]{\operatorname{Im}(#1)}
%\newcommand{\Include}[4]{\rhead{#2.#3.20#4}\input{\mainfold/lectures/#1-#4-#3-#2.tex}}
\newcommand{\Index}[1]{\emph{#1}\index{#1}}
\newcommand{\Int}[4]{\displaystyle\int\limits_{#1}^{#2}#3\,\mathrm{d}#4}
\newcommand{\diff}[1]{\operatorname{d}\!#1}
\newcommand{\Limi}[1]{\displaystyle\lim_{#1\rightarrow\infty}}
\newcommand{\Limo}[1]{\displaystyle\lim_{#1\rightarrow0}}
\newcommand{\mb}[1]{\mathbb{#1}}
\newcommand{\ds}{\displaystyle}
\newcommand{\ol}[1]{\overline{#1}}
\newcommand{\Part}[2]{\dfrac{\partial #1}{\partial #2}}
\newcommand{\QED}{\hfill{\hbox{(QED)}}\par\vspace{2mm}}
\newcommand{\re}[1]{\operatorname{Re}(#1)}
\newcommand{\s}{\hspace{2mm}}
\newcommand{\vsa}{\vspace{1mm} \\}
\newcommand{\vsb}{\vspace{2mm} \\}
\newcommand{\vsc}{\vspace{3mm} \\}
% \newcommand{\tr}[1]{\textrm{#1}}
\newcommand{\tr}[1]{\text{#1}}
\newcommand{\ra}{\rightarrow}
\newcommand{\Ra}{\Rightarrow}
\newcommand{\Lra}{\Leftrightarrow}
\newcommand{\lra}{\leftrightarrow}
\newcommand{\La}{\Leftarrow}
\newcommand{\ul}[1]{\underline{#1}}
\newcommand{\rsa}{\rightsquigarrow}
\newcommand{\ara}[2]{\autorightarrow{\ensuremath{#1}}{\ensuremath{#2}}} 

%\newcommand{\detmx}{\left| \begin{array} #1 \end{array} \right|}

\newcommand{\grad}[1]{\Grad(#1)}
\newcommand{\fr}[2]{\displaystyle\frac{#1}{#2}} % fertiger bullshit, daf�r gibts \dfrac{}{}
\renewcommand{\Re}{\operatorname{Re}}
\renewcommand{\Im}{\operatorname{Im}}

% ---- DELIMITER PAIRS ----
\def\floor#1{\lfloor #1 \rfloor}
\def\ceil#1{\lceil #1 \rceil}
\def\seq#1{\langle #1 \rangle}
\def\set#1{\{ #1 \}}
\def\abs#1{\mathopen| #1 \mathclose|}	% use instead of $|x|$ 
\def\norm#1{\mathopen\| #1 \mathclose\|}% use instead of $\|x\|$ 

% --- Self-scaling delmiter pairs ---
\def\Floor#1{\left\lfloor #1 \right\rfloor}
\def\Ceil#1{\left\lceil #1 \right\rceil}
\def\Seq#1{\left\langle #1 \right\rangle}
\def\Set#1{\left\{ #1 \right\}}
\def\Abs#1{\left| #1 \right|}
\def\Norm#1{\left\| #1 \right\|}

%Adrians Abbildungs-Environment ==============================================

\newcommand{\Sidein}{\begin{rotate}{90}\small$\in$\end{rotate}}

\newcommand{\Abb}[5][]{\ensuremath{
    \begin{array}{lc}
      \ifthenelse{\equal{#1}{}}{}{#1:}\;\; & 
      \begin{xy}
        \xymatrixrowsep{1em}\xymatrixcolsep{2em}%
        \xymatrix{ #2 \ar[r] \ar@{}[d]^<<<<{\hspace{0.001em} \Sidein}
          & #3  \ar@{}[d]^<<<<{\hspace{0.001em} \Sidein} \\
          #4 \ar@{|->}[r] & #5} \end{xy}
    \end{array}
  }%
}

%= Environments ========================================================================
\def\thechapter{\Roman{chapter}}
\def\thesection{\arabic{section}}
\newtheorem{Bew}{Beweis}
\newtheorem{Lem}{Lemma}
\newtheorem{Kor}{Korollar}
\newtheorem{Sat}{Satz}
\newtheorem{Prop}{Proposition}
\theoremstyle{definition}
\newtheorem{Bsp}{Beispiel}
\newtheorem{Def}{Definition}
\newtheorem{Prob}{Problem}
\theoremstyle{remark}
\newtheorem{Bem}{Bemerkung}
\newtheorem{Eig}{Eigenschaften}
\newtheorem{Faz}{Fazit}
%Der todo: Notation

\def\pstexInput#1{%
  \begin{center}
    \begin{picture}(0,0)%
      \special{psfile=\mainfold/graphics/A2-#1.pstex}%
    \end{picture}%
    \input{\mainfold/graphics/A2-#1.pstex_t}%
  \end{center}
}

%= Titelseite ===========================================================================
\begin{document}
\headheight15pt
\begin{titlepage}
\hfill
\vspace{20mm}
\pagenumbering{roman}
\begin{center}
{\LARGE Lineare Algebra I - Vorlesungs-Script} %\vskip 3em {\large Prof.
%Guido Mislin} \vskip 1.5em
%{\large Basisjahr 06/07}\vspace{30mm}\\
%{\large {\bf Mitschrift:} \vspace{2mm}\\
%Alexander Berthold van der Bourg}\vspace{5mm}\\ %30mm
%{\large {\bf Graphics:} \vspace{2mm}\\
%Pirmin Weigele }\vspace{30mm}\\ %30mm

\end{center}
\vfill

\end{titlepage}


%= Inhaltsverzeichnis ==========================================================================
\lhead{}
\rhead{}
\tableofcontents
\newpage
\pagenumbering{arabic}
\setcounter{page}{1}

%= Vorlesung-Skripts ==========================================================================
\cfoot{\thepage}
\fancyhead[L]{\nouppercase{\leftmark}}
\newpage

%= LinAlg I & & II ==========================================================================

%LinAlg I
\section{Bilinearformen}
Das kanonische Skalarprodukt (oder: Standardskalarprodukt) von $\mb{R}^n$ ist die Abbildung
\begin{align*}<,>: & \mb{R}^nx\mb{R}^n & \to & \mb{R}\\
 & (x,y) & \mapsto & <x,y>\in\mb{R}
\end{align*}
gegeben durch
\[<x,y>:=x_1y_1+\cdots+x_ny_n\]
falls $x=(x_1,\cdots,x_n)$ und $y=(y_1,\cdots,y_n)$ sind.
\begin{Def}[Konvention]
eine $1x1$ Matrix wird mit Eintrag indentifiziert
\[(x)\in{}M(1\times 1,K) \lra x\in{}K\]
Dann können wir schreiben:
\[<x,y>=(x^t)(y)\]
\[x=\Mx{x_1,\vdots,x_n}, y=\Mx{y_1,\vdots,}\]
\[(x_1,\cdots,x_n)\Mx{y_1,\vdots,y_n}=x_1y_1+\cdots+x_ny_n)\]
\end{Def}
\begin{Bem}[$<,>$ ist bilinear]
\begin{align*}
  <x+x',y>&=<x,y>+<x',y>\\
  <\lambda{}x,y>&=\lambda<x,y>\\
  <x,y+y'>&=<x,y>+<x,y'>\\
  <x,\lambda{}y>&=\lambda<x,y>
\end{align*}
symmetrisch:
\[<x,y>=<y,x>\]
positiv defininit:
\begin{align*}
  <x,x> &\geq& 0 \\
  <x,x> &=&= \Lra x=0\in\mb{R}^n
\end{align*}
\[\text{für} \forall x,y,x',y'\in\mb{R}^n, \lambda\in\mb{R}\]
\end{Bem}
\begin{Bem}[Hintergrund: euklidische Geometrie]
\begin{align*}
  \Norm{x}&=&=\sqrt{<x,x>}
  &=&=\sqrt{x^2_1+\cdots+x_n^2}
\end{align*}
\end{Bem}
\begin{Bem}[Eigenschaften von $\Norm{.}$]
\begin{align*}
  \Norm{x}\geq0, \text{mit} \Norm{x}=0 \Lra x=0\\
  \Norm{\lambda{}x}=\Abs{\lambda}\Norm{x}\\
  \Norm{x+y}\leq\Norm{x}+\Norm{y}
\end{align*}
\end{Bem}
Dann definieren wir den Abstand von $x,y\in\mb{R}^n$:
\[d(x,y)\in\mb{R}\]
\[d(x,y):=\Norm{y-x}\]
\begin{Bem}{Eigenschaften}
\begin{align*}
  d(x,y)\geq0, \text{mit}d(x,y)=0 \Lra x=y\\
  d(y,x)=d(x,y)\\
  d(x,z)\leq d(x,y)+d(y,z)
\end{align*}
\[\text{für} x,y,z\in\mb{R}^n\]
\end{Bem}
Wir sind motiviert, Struktiren zu definieren, basierend auf diesen Eigenschaften, so z.B.
\begin{itemize}
  \item Bilineare Formen (symetrisch, positiv definit)
  \item Norme
  \item Metriken
\end{itemize}
\begin{proof} $\Norm{.}$ und $d$: die Dreiecksungleichung folgt aus der Cauchy-Schwarzschen Ungleichung
\begin{align*}
  \Norm{x+y}^2&=&=<x+y,x+y>\\
  &=&=\Norm{x}^2+\Norm{y}^2+2<x,y> \leq (?) \left(\Norm{x}+\Norm{y}\right)^2\\
  &&\Lra <x,y> \leq \Norm{x}\Norm{y}
\end{align*}
Cauchy-Schwarz'sche Ungleichung: für $x,y\in\mb{R}^n$
\[<x,y>^2\leq<x,x><y,y>\]
mit Gleichheit genau dann, wenn $x$ und $y$ linear abhängig sind.
\[\Lra\]
\[A=\Mx{---&x&---\\---&y&---}\in M(2\times n,\mb{R})\]
$A$ hat Rang $\leq 1$
\end{proof}
\begin{proof}
\begin{align*}
  A\cdot{}A^t&=&=\Mx{<x,x>&<x,y>\\<x,y>&<y,y>}\in M(2\times 2),\mb{R}\\
  \det(A\cdot{}A^t)&=&=<x,x><y,y>-<x,y>^2
\end{align*}
Es gibt eine Gleichung von Determinanten:
\begin{align*}
  A,B\in M(k\times n,K)\\
  \det(A\cdot B^t)=\sum_{1\leq s_1 <s_2<\cdots<s_k\leq n} \det(A^{s_1,\cdots,s_k})\det(B^{s_1,\cdots,s_k})\\
  \text{wobei } A^{s_1,\cdots,s_k}:=(a_i,s_j)_{1\leq i,j\leq k}, B^{s_1,\cdots,s_k}=(b_i,s_j)_{1\leq i,j\leq k}\\
\end{align*}
Beweis-Skizze: Reduktion zum Fall, dass die Zeilen von $A$ und $B$ Standardbasiselemente sind; direkte Berechnung in diesem Fall.\\
Es folgt:
\begin{align*}
  \det(A\cdot A^t)=\sum_{1\leq i < j \leq n} \det(A^{i,j})^2 \geq 0
\end{align*}
und ist $=0$ $\Lra$ alle $2\times 2$ Minoren von $A$ sind 0 $\Lra$ $rang(A)\leq 1$
\end{proof}
\begin{Kor} Wir können definieren
\begin{align*}
  \angle(x,y):=\cos^{-1}\underbrace{\frac{<x,y}{\Norm{x}\Norm{y}}}_{\in [-1,1]\in\mb{R}} \in [0,\pi]\in\mb{R}\\
  \text{für}\\
  0\neq x\in\mb{R}^n\\
  0\neq y\in\mb{R}^n
\end{align*}
\end{Kor}
\begin{Kor}
$x,y$ Vektoren, $\theta$ Winkel zwischen den beiden
\[<x,y>=\frac{1}{2}\left(\Norm{x}^2+\Norm{y}^2-\Norm{y-x}^2\right)\]
und deshalb:
\[\cos\theta=\frac{\Norm{x}^2+\Norm{y}^2-\Norm{y-x}^2}{2\Norm{x}\Norm{y}}\]
$\implies$ Winkel eines Dreiecks ist nur von den Seitenlängen abhängig.
\end{Kor}
\begin{Bsp}
\[\angle(x,y)=\frac{\pi}{2}\Lra <x,y>=0\]
\[\underbrace{\{y|<x,y>=0\}={0}}_{\text{Untervektorraum}}\cup\{0\neq y\in\mb{R}^n|\angle(x,y)=\frac{\pi}{2}\}\]
Man nennt $x$ und $y$ \underline{senkrecht} falls $<x,y>=0$
\end{Bsp}

\begin{Faz}
  \begin{align*}
    \Seq{.,.}& \mb{R}^n\times\mb{R}^n\to\mb{R}^n&\text{bilinear form}\\
    \Norm{.}& \mb{R}^n\to \mb{R}_{\geq 0}& \text{Norm}\\
    d(.,.)&\mb{R}^n\times\mb{R}^n\to\mb{R}_{>0}&\text{Metrik}
  \end{align*}
  \begin{align*}
    \Norm{x}&=\sqrt{\Seq{x,x}}\\
    d(x,y)&=\Norm{y-x}\\
    \Seq{x,y}&=\frac{\Norm{x}^2+\Norm{y}^2-\Norm{y-x}^2}{2}
  \end{align*}
\end{Faz}
\subsection{Vektorprodukt in $\mb{R}^3$}
  \begin{align*}
    \mb{R}^3\times\mb{R}^3&\to&\mb{R}^3\\
    (x,y)&\mapsto&x\times y
  \end{align*}
  für $y=(y_1,y_2,y_3)$ und $y=(y_1,y_2,y_3)$ ist 
  \[x\times y=(x_2y_3-x_3y_2, x_3y_1-x_1y_2,x_1y_2-x_2y_1)\]
  oder:
  \[x\times y=\det\Mx{e_1&e_2&e_3\\x_1&x_2&x_3\\y_1&y_2&y_3}\]
  wobei $(e_1,e_2,e_3)$ die Standardbasis ist. Es ist deshalb klar, dass
  \[0=\det\Mx{x_1&x_2&x_3\\x_1&x_2&x_3\\y_1&y_2&y_3}=\Seq{x,x\times y}\]
  \[0=\det\Mx{y_1&y_2&y_3\\x_1&x_2&x_3\\y_1&y_2&y_3}=\Seq{y,x\times y}\]
  $x\times y$ liegt auf der Gerade von Vektoren senkrecht zu $x$ und $y$.
  weiter:
  \[\det\Mx{w_1&w_2&w_3\\x_1&x_2&x_3\\y_1&y_2&y_3}=\Seq{x\times y,x\times y}\]
  \[=\Norm{x\times y}^2=(x_2y_3-x_3y_2)^2+(x_3y_1-x_1y_3)^2+(x_1y_2-x_2y_1)^2\]
  \[=\Norm{x}^2\Norm{y}^2-\Seq{x,y}^2=\Norm{x}^2\Norm{y}^2\left( 1-\frac{\Seq{x,y}^2}{\Norm{x}^2\Norm{y}^2} \right)\]
  \[=\Norm{x}^2\Norm{y}^2(1-\cos^2\angle (x,y) = \Norm{x}^2\Norm{y}^2\sin^2\angle (x,y)\]
\begin{Faz}
  Wenn das Ergebnis $=0$, folgt daraus, dass $x$ und $y$ linear abhängig sind. Falls $x$ und $y$ linear unabhängig sind, dann folgt dass $(x\times y,x,y)$ zu derselben Orientierungsklasse gehört wie $(e_1,e_2,e_3)$. Insgesamt bedeutet dies, dass $x\times y$ folgende Eigenschaften hat:
  \begin{itemize}
    \item ist senkrecht zu $x$ und $y$
    \item ist 0 $\Lra$ $x$ und $y$ sind linear abhängig
    \item hat Länge $\Norm{x}\Norm{y}\sin \angle (x,y)$
    \item und hat die Richtung, die mit $x$ und $y$ die gleiche Orientierungsklassse wie die Standardbasis hat.
  \end{itemize}
\end{Faz}
\subsection{Skalarprodukt über $\mb{C}^n$}
Sei $z=(z_1,\cdots,z_n)$ und $w=(w_1,\cdots,w_n)\in\mb{C}^n$
\begin{Bem}
  Der Ausdruck macht Sinn.
  \begin{align*}
    \Seq{z,w}&:=z_1w_1+\cdots+z_nw_n\\
    \Seq{z,z}&:=z_1^2+\cdots+z_n^2\\
  \end{align*}
  Dann kann die Länge nicht mehr interpretiert werden, z.B. für $z=(1,i,0,\cdots,0)$ haben wir $\Seq{z,z}=1^2+i^2=0$. Isotropische Untervektorräume von $\mb{C}^n$ werden nicht in in diesem Kurs behandelt. ($V\subset\mb{C}^n$ s.d. $\Seq{v,w}=0 \ \forall v,w\in V$). Für die Physik, die Geometrie usw. ist eine Interpretation in Zusammenhang mit Länge wichtig, deshalb brauchen wir eine neue Definition.
\end{Bem}
\begin{Def}[Das kanonische Skalarprodukt]
  von $\mb{C}^n$ ist gegeben durch
  \begin{align*}
    \Seq{.,.}_c:&\ \mb{C}^n\mb{C}^n\to\mb{C}\\
    &\ (z,w)\mapsto z_1\bar{w_1}+\cdots+z_n\bar{w_n}
  \end{align*}
\end{Def}
\begin{Eig}[von $\Seq{.,}_c$]
  \begin{align*}
    \Seq{z+z',w}&= \Seq{z,w}_c+\Seq{z',w}_c\\
    \Seq{\lambda z,w}_c&= \lambda\Seq{z,w}_c\\
    \Seq{z,w+w'}_c&= \Seq{z,w}_c+\Seq{z,w'}_c\\
    \Seq{z,\lambda w}_c&= \bar{\lambda}\Seq{z,w}_c
  \end{align*}
  für $z,z',w,w'\in\mb{C}^n$, $\lambda\in\mb{C}$\\
  $\Seq{.,.}_c$ ist sesquilinear
  \begin{align*}
    \Seq{w,z}_c&= \overline{\Seq{z,w}_c}& \text{hermitisch}\\
    \Seq{z,z}_c&\in\mb{R}_{\geq 0}& \text{positiv definit}\\
    \Seq{z,z}=0 &\Lra z=0
  \end{align*}
\end{Eig}
\begin{Faz}
  $\Seq{.,.}_c$ ist sesquilinear, hermitisch und positiv definit.
\end{Faz}
\begin{Bew}
  Bei Bedarf sonstwo nachschauen (Zu viele Zeichen und zu wenig Sinn). Es läuft auf eine Sammlung von Quadraten heraus.
\end{Bew}
\begin{Def}[Norm von $\mb{C}^n$]
  \[\Norm{z}=\sqrt{\Seq{z,z}_c}\]
\end{Def}
\begin{Bem}
  Sei $w=(x_1'+xy_1',\cdots,x_n'+iy_n')$. Dann:
  \begin{align*}
    \Seq{z,w}_c=(x_1+iy_1)(x_1'-iy_1')+\cdots+(x_n+iy_n)(x_n'-iy_n')\\
    =(x_1x_1'+y_1y_1'+\cdots+x_nx_n'+y_ny_n')+i(x_1'y_1-x_1y_1'+\cdots+x_n'x_y-x_ny_n')
  \end{align*}
  Auf diese Weise ist $\Seq{.,.}_c$ eine Erweiterung von reellen Skalarprodukt.
  \begin{align*}
    \mb{R}^{2n}&\xrightarrow{~}\mb{C}^n&\mb{R}\text{-linear}\\
    e_1\mapsto&(1,0,\cdots,0)\\
    e_2\mapsto&(i,0,\cdots,0\\
    \cdots\\
    e_{2n}&\mapsto(0,\cdots,0,i)
  \end{align*}
  \[\Seq{.,.}_c=\left( \Seq{.,.} \text{von}\ \mb{R}^{2n} \right) + i(\text{neues})\]
  $\Re\Seq{.,.}_c=\Seq{.,.}$ von $\mb{R}^{2n}$ unter diesem Isomorpismus.\\
  Sei $\omega:=\Im\Seq{.,.}$:
  \begin{align*}
    \omega:&\mb{C}^n\times \mb{C}^n\to\mb{R}\\
    \text{oder}\ & \mb{R}^{2n}\times\mb{R}^{2n}\to\mb{R}
  \end{align*}
\end{Bem}
\begin{Eig}[von $\omega$ (Imaginärteil des kanonischen Skalarproduktes)]
  \begin{description}
    \item[bilinear]
    \item[schiefsymmetrisch] $\omega(w,z)=-\omega(z,w)$
    \item[] $\omega(z,z)=0\ \forall z\in\mb{C}^n$ (oder $\mb{R}^{2n}$)
  \end{description}
\end{Eig}
\subsection{Bilinearform}
Sei $K$ ein Körper und $V$ ein $K$-Vektorraum.
\begin{Def}[Bilinearform]
  Eine bilineare Form auf $V$ ist eine Abbildung
  \[s:V\times V\to K\]
  so dass:
  \begin{align*}
    s(v+v',w)&=s(v,w)+s(v',w)\\
    s(\lambda v,w)&=\lambda s(v,w)\\
    s(v,w+w')&=s(v,w)+s(v,w')\\
    s(v,\lambda w)&= \lambda s(v,w)
  \end{align*}
  $\forall v,v',w,w'\in V, \lambda \in K$\\
  Und: $s$ heisst \underline{symmetrisch}, falls $s(w,v)=s(v,w)$ und \underline{schiefsymmetrisch}, falls $s(w,v)=-s(v,w)$.
\end{Def}
\begin{Bsp}
  \begin{itemize}
    \item $\Seq{.,.}:=\mb{R}^n\times \mb{R}^n\to\mb{R}$ ist eine symmetrische bilineare Form
    \item $\omega$ ist eine schiefsymmetrisch bilineare Form
    \item ($\Seq{.,.}_c$ nicht)
    \item $V=\{\text{stetige Abbildung} [0,1] \to\mb{R}\}$ über $\mb{R}$: $f,g\in V$
      \[s(f,g)=\int^1_0 f(x)g(x)dx\]
      ist eine symmetrisch bilineare Form auf $V$
  \end{itemize}
\end{Bsp}

Sei $K$ ein Körper, $V$ ein $K$-Vektorraum, mit $\dim_KV<\infty$, und $s:V\times V\to K$ eine bilineare Form.
\begin{Def}{darstellende Matrix}
  Ist $B=(v_i)_{1\leq i \leq n}$ eine Basis von $V$, so setzen wir 
  \[M_B(s):=\left( s(v_i,v_j) \right)_{1\leq i, j\leq n} \in M(n\times n,K)\]
  die darstellende Matrix
\end{Def}
\begin{Kor}
  für $x,y\in V$
  \begin{align*}
    x&= x_1v_1+\cdots+x_nv_n\\
    y&= y_1v_1,+\cdots+y_nv_n
  \end{align*}
  und
  \[M_B(s)=(a_{ij})_{1\leq i, j\leq n}, \text{d.h.} a_{ij}=s(v_i,v_j)\]
  haben wir:
  \begin{align*}
    s(x,y)&= \sum^n_{i,j=1}x_iy_ja_{ij}\\
    &= (x_1\cdots x_n)\cdot\Mx{a_{11}&\cdots&a_{1n}\\ \vdots&&\vdots\\a_{n1}&\cdots&a_{nn}}\cdot \Mx{y_1\\ \vdots\\ y_n}\\
    &= x^tM_B(s)\cdot y
  \end{align*}
\end{Kor}
\begin{Prop}
  Sei $V$ ein endlich-dim. Vektorraum über $K$ mit Basis $B=(v_i)_{1\leq i \leq n}$. Es gibt eine Bijektion zwischen der Menge von Bilinearformen und $M(n\times n,K)$, gegeben durch
  \[\left( s:V\times V\to K \right)\mapsto M_B(s)\]
\end{Prop}
\begin{Bew}
  Wir schreiben einen Vektor $x\in V$ als $(x_1,\cdots,x_n)$ falls $x=x_1v_1+\cdots+x_nv_n$. Ähnlich für $y$. Dann ist
  \begin{align*}
   A\in M(n\times n,K)\mapsto & V\times V \to K\\
   & (x,y)\mapsto x^t\cdot A \cdot y
  \end{align*}
  inverses zu der obigen Abbildung.
\end{Bew}
\begin{Bem}
  Sei $\left( s:V\times V\to K \right)$ eine bilineare Forum und $A=(a_{ij})_{1\leq i,j\leq n}$ die darstellende Matrix. Wir erinnern uns an die Notation
  \begin{align*}
    \Phi_B:&K^n\to V\\
    &e_1 \mapsto v_i
  \end{align*}
  Dann:
  % use diagram package
  \begin{align*}
    K^n\times K^n&\ara{\Phi_B\times\Phi_B}{}V\times V\ara{s}{}K\\
    \text{ist gegeben durch}&\\
    (x,y)&\xmapsto{\phantom{CCCCCCCCCCCCCCCCC}} t_xA\cdot y
  \end{align*}
  Sei $A=(u_i)_{1\leq i\leq n}$ eine andere Basis.  
  \[\begindc{\commdiag}[30]
  \obj(0,2)[K1]{$K^n$}
  \obj(0,0)[K2]{$K^n$}
  \obj(2,1)[V]{$V$}
  \mor{K1}{K2}{$T=\Phi_B^{-1}\circ\Phi_A$}[-1,0]
  \mor{K1}{V}{$\Phi_A$}
  \mor{K2}{V}{$\Phi_B$}
  \enddc\]
\end{Bem}
\begin{Prop}{Transforationsformel}
  Mit dieser Notation haben wir:
  \[M_A(s)=T^t\cdot M_B(s)\cdot T\]
\end{Prop}
\begin{Bew}
  \begin{align*}
    K^n\times K^n &\ara{\Phi_B\times\Phi_B}{} V\times V\ara{s}{}K\\
    (x,y)&\xmapsto{\phantom{CCCCCCCCCCCCCCCCC}} t_x\cdot M_B(s)\cdot y
  \end{align*}
  Es folgt:
  \[\begindc{\commdiag}[4]
  \obj(5,20)[xy]{$(x,y)$}
  \obj(20,25)[KK1]{$K\times K$}
  \obj(30,15)[KK2]{$K\times K$}
  \obj(40,25)[V]{$V\times V$}
  \obj(55,25)[K]{$K$}
  \obj(30,7)[T]{$(T_x,T_y)$}
  \obj(70,20)[x]{$x^tM_A(s)y=$}
  \mor{KK1}{KK2}{$T\times T$}[-1,0]
  \mor{KK1}{V}{$\Phi_A\times\Phi_A$}
  \mor{KK2}{V}{$\Phi_B\times\Phi_B$}[-1,0]
  \mor{V}{K}{$s$}
  \mor{KK1}{xy}{$\in$}[1,2]
  \mor{KK2}{T}{$\in$}[1,2]
  \mor{K}{x}{$\in$}[1,2]
  \cmor((5,22)(20,30)(55,30)(67,27)(70,22)) \pdown(40,33){$\mapsto$}
  \mor{xy}{T}{}[1,6]
  \mor{T}{x}{}[1,6]
  \enddc\]
  \[=x^tT^tM_B(s)Ty=(T_x)^tM_B(s)(T_y)\]
  Es folgt aus der oberen Proposition (Vor der Transf.):
  \[T^tM_B(s)T=M_a(s)\]
\end{Bew}
\begin{Bsp}
  $V=K^n$, mit Standardskalaprodukt $<.,.>$. Ist $B=(e_1,\cdots,e_n)$, so ist
  \[\Mx{1&&0\\&\ddots&\\0&&1}=M_{\text{Standardbasis}}(<.,.>)\]
  Sei
  \begin{align*}
    A=& (e_1,e_2-e_1,e_3-e_2,\cdots,e_n-e_{n-1})\\
    =:& (u_1,u_2,\cdots,u_n)
  \end{align*}
  Direkt aus der Definition:
  \[<u_i,u_j>=\begin{cases} 1& i=j=1 \\ 2&i=j>1\\ -1& \Abs{i-j}=1 \\ 0&\text{sonst}\end{cases} \]
  oder mit der Transformationsformel
  \[T=\Mx{1&1&&0\\\cdots\\\cdots&&&1\\0&&&1} \text{und}\ T^tE_N T''\]
\end{Bsp}
\begin{Bem}
  Ist $A$ die darstellende Matrix bezügloich einer Basis, so haben wir:
  \begin{itemize}
    \item symmetrisch $\Lra$ $A=A^t$
    \item schiefsymmetrisch $\Lra$ $A=-A^t$
  \end{itemize}
  Das stimmt überein mit (vgl. Übungsblatt 3): $A\in M(n\times n)$ ist symmetrisch $\Lra$ $A=A^t$. $A$ ist schiefsymmetrisch oder antisymmetrisch (oder alternierend wenn $\text{char}(K)\neq 2$) $\Lra$ $A=-A^t$
\end{Bem}
\subsection{Bilineare und quadratische Formen}
Eine quadratische Form $V\to K$ wird zu einer Bilinearform assoziert. Falls $\dim_KV<\infty$: ``quadratische Form'' bedeutet $q:V\to K$ bezüglich einem Koordinatensystem gegeben als homogenes quadratisches Polynom. Ist $s:V\times V\to K$ eine bilineare Form, dann heisst
\begin{align*}
  q:&V\to K\\
  &v \mapsto q(v)=s(v,v)
\end{align*}
die zu $s$ gehörige quadratische Form.
\begin{Bsp}
  $<v,v>=v_1^2+\cdots+v_n^2$ für $v\in K^n$\\
  Für $A=(a_{ij})_{1\leq i, j\leq n}$ eine symmetrische Matrix mit $s:V\times V\to K$, $(x,y)\mapsto x^tAy$, haben wir
  \begin{align*}
    s(x,x)&= x^tAx\\
    &= \Mx{x_1&\cdots&x_n}\Mx{a_{11}&\cdots&a_{1n}\\ \vdots&&\vdots\\ a_{n1}&\cdots&a_{nn}}\Mx{x_1\\ \vdots \\ s_n}\\
    &= \sum^n_{i,j=1}a_{ij}x_ix_j\\
    &= \sum^n_{i=1}a_{ii}x_i^2+2\sum_{1\leq i<j\leq n}a_{ij}x_ix_j
  \end{align*}
\end{Bsp}
Ist $\text{char}(K)\neq 2$, so haben wir:
\begin{align*}
  \{\text{symm. bilineare Formen in $K^n$}\}&\lra\{\text{quadr. Formen auf}\ K^n\}\\
  s&\mapsto q(v):=s(v,v)\\
  &\mapsfrom (\text{Polarisierungsformel}) 
\end{align*}
\subsubsection{Polarisierungsformel}
Ist $s$ eine symmetrische Bilinearform und $q$ die zu $s$ gehörende quadratische Form über einem Vektorraum $V$ über $K$ mit $\text{char}(K)\neq 2$, dann gilt: 
\begin{align*}
  s(v,w)&= \frac{1}{2}\left( q(v+w) - q(v) - q(w) \right)\\
  &= \frac{1}{2}\left( q(v)+q(w)-q(v+w) \right)\\
  &= \frac{1}{4}\left( q(v+w)-q(v-w) \right)
\end{align*}
\subsection{Sesquilineare Form}
\begin{Def}
  Sei $V$ ein komplexer Vektorraum. Eine Abbildung
  \[s:V\times V\to \mb{C}\]
  heisst sesquilinear falls:
  \begin{align*}
    s(v+v',w)&= s(v,w)+s(v',w)\\
    s(\lambda v,w)&= \lambda s(v,w)\\
    s(v,w+w')&= s(v,w)+s(v,w')\\
    s(v,\lambda w)&= \bar{\lambda} s(v,w)
  \end{align*}
  für $v,v',w,w'\in V$, $\lambda\in\mb{C}$
\end{Def}
\begin{Bsp}
  $\Seq{.,.}$ auf $\mb{C}^n$
  \begin{align*}
    s(f,g)&= \int^1_0f(x)g(x)dx\\
    \text{auf}\ V&:= \{\text{stetige Abb.}[0,1]\to\mb{C}\}
  \end{align*}
\end{Bsp}


\newpage

%= Stichwortverzeichnis ======================================================================
\rhead{}
\addcontentsline{toc}{section}{Stichwortverzeichnis}
\printindex

\end{document}
