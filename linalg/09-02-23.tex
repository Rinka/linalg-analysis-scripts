Sei $K$ ein Körper, $V$ ein $K$-Vektorraum, mit $\dim_KV<\infty$, und $s:V\times V\to K$ eine bilineare Form.
\begin{Def}
  Ist $B=(v_i)_{1\leq i \leq n}$ eine Basis von $V$, so setzen wir 
  \[M_B(s):=\left( s(v_i,v_j) \right)_{1\leq i, j\leq n} \in M(n\times n,K)\]
  die \underline{darstellende Matrix}
\end{Def}
\begin{Kor}
  für $x,y\in V$
  \begin{align*}
    x&=& x_1v_1+\cdots+x_nv_n\\
    y&=& y_1v_1,+\cdots+y_nv_n
  \end{align*}
  und
  \[M_B(s)=(a_{ij})_{1\leq i, j\leq n}, \text{d.h.} a_{ij}=s(v_i,v_j)\]
  haben wir:
  \begin{align*}
    s(x,y)&=& \sum^n_{i,j=1}x_iy_ja_{ij}\\
    &=& (x_1\cdots x_n)\cdot\Mx{a_{11}&\cdots&a_{1n}\\ \vdots&&\vdots\\a_{n1}&\cdots&a_{nn}}\cdot \Mx{y_1\\ \vdots\\ y_n}\\
    &=& x^tM_B(s)\cdot y
  \end{align*}
\end{Kor}
\begin{Prop}
  Sei $V$ ein endlich-dim. Vektorraum über $K$ mit Basis $B=(v_i)_{1\leq i \leq n}$. Es gibt eine Bijektion zwischen der Menge von Bilinearformen und $M(n\times n,K)$, gegeben durch
  \[\left( s:V\times V\to K \right)\mapsto M_B(s)\]
\end{Prop}
\begin{Bew}
  Wir schreiben einen Vektor $x\in V$ als $(x_1,\cdots,x_n)$ falls $x=x_1v_1+\cdots+x_nv_n$. Ähnlich für $y$. Dann ist
  \begin{align*}
   A\in M(n\times n,K)\mapsto & V\times V \to K\\
   & (x,y)\mapsto x^t\cdot A \cdot y
  \end{align*}
  inverses zu der obigen Abbildung.
\end{Bew}
\begin{Bem}
  Sei $\left( s:V\times V\to K \right)$ eine bilineare Forum und $A=(a_{ij})_{1\leq i,j\leq n}$ die darstellende Matrix. Wir erinnern uns an die Notation
  \begin{align*}
    \Phi_B:&K^n\to V\\
    &e_1 \mapsto v_i
  \end{align*}
  Dann:
  % use diagram package
  \begin{align*}
    K^n\times K^n&\ara{\Phi_B\times\Phi_B}{}V\times V\ara{s}{}&K\\
    \text{ist gegeben durch}\\
    (x,y)&\mapsto&t_xA\cdot y
  \end{align*}
  Sei $A=(u_i)_{1\leq i\leq n}$ eine andere Basis.  
  \begin{align*}
    K^n&\ara{\Phi_A}{}&V\\
    \ara{T}{}=\Phi^{-1}_B\circ \Phi_A&\\
    K^n&\ara{\Phi_B}{}&
  \end{align*}
\end{Bem}
\begin{Prop}{Transforationsformel}
  Mit dieser Notation haben wir:
  \[M_A(s)=T^t\cdot M_B(s)\cdot T\]
\end{Prop}
\begin{Bew}
  \begin{align*}
    K^n\times K^n &\ara{\Phi_B\times\Phi_B}{} V\times V\ara{s}{}&K\\
    (x,y)&\mapsto&t_x\cdot M_B(s)\cdot y
  \end{align*}
  Es folgt: (eine Bastelei\ldots)
  \begin{align*}
    K^n\times K^n&\ara{\Phi_A\times\Phi_A}{}&V\times V&\ara{s}{}&K\\
    \ara{T\times T}{}&K^n\times K^n&\ara{\Phi_B\Phi_B}{}\uparrow&&
  \end{align*}
  \begin{align*}
    K^n\times K^n (x,y)&\mapsto& K(x^tM_a(s)\cdot y)=t_x^tM_B(s)T_y = (T_x)^tM_B(s)(T_y)\\
    \mapsto&K^n\times K^n(T_x,T_y)&\xmapsto{(T_x)^tM_B(s)(T_y)}\uparrow
  \end{align*}
  Es folgt aus der oberen Proposition (Vor der Transf.):
  \[T^tM_B(s)T=M_a(s)\]
\end{Bew}
\begin{Bsp}
  $V=K^n$, mit Standardskalaprodukt $<.,.>$. Ist $B=(e_1,\cdots,e_n)$, so ist
  \[\Mx{1&&0\\&\ddots&\\0&&1}=M_{\text{Standardbasis}}(<.,.>)\]
  Sei
  \begin{align*}
    A&=& (e_1,&e_2-e_1,&e_3-e_2,\cdots,&e_n-e_{n-1})\\
    &=:& (u_1,&u_2,&\cdots,&u_n
  \end{align*}
  Direkt aus der Definition:
  \[<u_i,u_j>=\begin{cases} 1& i=j=1 \\ 2&i=j>1\\ -1& \Abs{i-j}=1 \\ 0&\text{sonst}\end{cases} \]
  oder mit der Transformationsformel
  \[T=\Mx{1&1&&0\\\cdots\\\cdots&&&1\\0&&&1} \text{und} T^tE_N T''\]
\end{Bsp}
\begin{Bem}
  Ist $A$ die darstellende Matrix bezügloich einer Basis, so haben wir:
  \begin{itemize}
    \item symmetrisch $\Lra$ $A=A^t$
    \item schiefsymmetrisch $\Lra$ $A=-A^t$
  \end{itemize}
  Das stimmt überein mit (vgl. Übungblatt 3): $A\in M(n\times n)$ ist symmetrisch $\Lra$ $A=A^t$. $A$ ist schiefsymmetrisch oder antisymmetrisch (oder alternierend wenn $\text{char}(K)\neq 2$) $\Lra$ $A=-A^t$
\end{Bem}
\subsection{Bilineare und quadratische Formen}
Eine quadratische Form $V\to K$ wird zu einer Bilinearform assoziert. Falls $\dim_KV<\infty$: ``quadratische Form'' bedeutet $q:V\to K$ bezüglich einem Koordinatensystem gegeben als homogenes quadratisches Polynom. Ist $s:V\times V\to K$ eine bilineare Form, dann heisst
\begin{align*}
  q:&V&\to&K\\
  &v& \mapsto &q(v)=s(v,v)
\end{align*}
die zu $s$ gehörige quadratische Form.
\begin{Bsp}
  $<v,v>=v_1^2+\cdots+v_n^2$ für $v\in K^n$\\
  Für $A=(a_{ij})_{1\leq i, j\leq n}$ eine symmetrische Matrix mit $s:V\times V\to K$, $(x,y)\mapsto x^tAy$, haben wir
  \begin{align*}
    s(x,x)&=& x^tAx\\
    &=& \Mx{x_1&\cdots&x_n}\Mx{a_{11}&\cdots&a_{1n}\\ \vdots&&\vdots\\ a_{n1}&\cdots&a_{nn}}\Mx{x_1\\ \vdots \\ s_n}\\
    &=& \sum^n_{i,j=1}a_{ij}x_ix_j\\
    &=& \sum^n_{i=1}a_{ii}x_i^2+2\sum_{1\leq i<j\leq n}a_{ij}x_ix_j
  \end{align*}
\end{Bsp}
Ist $\text{char}(K)\neq 2$, so haben wir:
\begin{align*}
  \{\text{symm. bilineare Formen in $K^n$}\}&\lra&\{\text{quadr. Formen auf} K^n\}\\
  s&\mapsto &q(v):=s(v,v)\\
  &\mapsfrom (\text{Polarisierungsformel}) &
\end{align*}
\subsubsection{Polarisierungsformel}
Ist $s$ eine symmetrische Bilinearform und $q$ die zu $s$ gehörende quadratische Form über einem Vektorraum $V$ über $K$ mit $\text{char}(K)\neq 2$, dann gilt: 
\begin{align*}
  s(v,w)&=& \frac{1}{2}\left( q(v+w) - q(v) - q(w) \right)\\
  &=& \frac{1}{2}\left( q(v)+q(w)-q(v+w) \right)\\
  &=& \frac{1}{4}\left( q(v+w)-q(v-w) \right)
\end{align*}
\subsection{Sesquilineare Form}
\begin{Def}
  Sei $V$ ein komplexer Vektorraum. Eine Abbildung
  \[s:V\times V\to \mb{C}\]
  heisst sesquilinear falls:
  \begin{align*}
    s(v+v',w)&=& s(v,w)+s(v',w)\\
    s(\lambda v,w)&=& \lambda s(v,w)\\
    s(v,w+w')&=& s(v,w)+s(v,w')\\
    s(v,\lambda w)&=& \bar{\lambda} s(v,w)
  \end{align*}
  für $v,v',w,w'\in V$, $\lambda\in\mb{C}$
\end{Def}
\begin{Bsp}
  $<.,.>$ auf $\mb{C}^n$
  \begin{align*}
    s(f,g)&=& \int^1_0f(x)g(x)dx\\
    \text{auf} V&:=& \{\text{stetige Abb.}\}[0,1]\to\mb{C}\}
  \end{align*}
\end{Bsp}
